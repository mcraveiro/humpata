%Document generated by wvWare/wvWare version 1.2.9
%wvWare written by Caolan.McNamara@ul.ie
%http://wvware.sourceforge.net/
%
\documentclass[12pt]{article}
\usepackage[dvips]{color}
\usepackage[dvips]{graphics}
\usepackage[T1]{fontenc}
\usepackage[latin1]{inputenc}
\usepackage{longtable}
\usepackage{times}
\usepackage{enumerate}

\usepackage[normalem]{ulem}
\usepackage{geometry}
\geometry{lmargin=2cm,rmargin=2cm}
\newcommand\suppress[1]{}
\newcommand\deleted[1]{\xout{#1}}
\newcommand\revised[1]{\uline{#1}}
\newlength\wvtextpercent
\setlength\wvtextpercent{0.009\textwidth}

\newbox\strikebox
\def\strike#1{\setbox\strikebox \hbox{<#1>}\hbox{\raise0.5ex\hbox to 0pt{\vrule height 0.4pt width \wd\strikebox\hss}\copy\strikebox}}

\setlength\parindent{0pt}
\setlength{\parskip}{\smallskipamount}
\begin{document}
\sloppy


\vspace{4.17mm}
\setlength{\parindent}{0.00mm}
\setlength{\leftskip}{0.00mm}
\setlength{\rightskip}{0.00mm}
\raggedright
\textbf{Table of Contents\newpage
}
\vspace{2.08mm}



\vspace{0.00mm}
\setlength{\parindent}{0.00mm}
\setlength{\leftskip}{0.00mm}
\setlength{\rightskip}{0.00mm}
\raggedright
Table of Cases \hfill{}3
\vspace{0.00mm}

\vspace{0.00mm}
\setlength{\parindent}{0.00mm}
\setlength{\leftskip}{0.00mm}
\setlength{\rightskip}{0.00mm}
\raggedright
Table of Treaties\hfill{}4
\vspace{0.00mm}

\vspace{0.00mm}
\setlength{\parindent}{0.00mm}
\setlength{\leftskip}{0.00mm}
\setlength{\rightskip}{0.00mm}
\raggedright
Table of Other Documents \hfill{}5
\vspace{0.00mm}

\vspace{0.00mm}
\setlength{\parindent}{0.00mm}
\setlength{\leftskip}{0.00mm}
\setlength{\rightskip}{0.00mm}
\raggedright
Introduction:\hfill{}7
\vspace{0.00mm}

\vspace{0.00mm}
\setlength{\parindent}{0.00mm}
\setlength{\leftskip}{0.00mm}
\setlength{\rightskip}{0.00mm}
\raggedright
1. Introducing the Right to Development\hfill{}10
\vspace{0.00mm}

\vspace{0.00mm}
\setlength{\parindent}{0.00mm}
\setlength{\leftskip}{4.91mm}
\setlength{\rightskip}{0.00mm}
\raggedright
1.1. The Emergence of the Right to Development\hfill{}10
\vspace{0.00mm}

\vspace{0.00mm}
\setlength{\parindent}{0.00mm}
\setlength{\leftskip}{9.83mm}
\setlength{\rightskip}{0.00mm}
\raggedright
1.1.1. Towards Integrating Development and Human Rights\hfill{}14
\vspace{0.00mm}

\vspace{0.00mm}
\setlength{\parindent}{0.00mm}
\setlength{\leftskip}{9.83mm}
\setlength{\rightskip}{0.00mm}
\raggedright
1.1.2. From the Declaration on the Right to Development and Beyond. \hfill{}16
\vspace{0.00mm}

\vspace{0.00mm}
\setlength{\parindent}{0.00mm}
\setlength{\leftskip}{0.00mm}
\setlength{\rightskip}{0.00mm}
\raggedright
2. The Definition and Content of TRTD\hfill{}19
\vspace{0.00mm}

\vspace{0.00mm}
\setlength{\parindent}{0.00mm}
\setlength{\leftskip}{4.91mm}
\setlength{\rightskip}{0.00mm}
\raggedright
2.1. The Non-Controversial Aspects of the Right to Development\hfill{}19
\vspace{0.00mm}

\vspace{0.00mm}
\setlength{\parindent}{0.00mm}
\setlength{\leftskip}{9.83mm}
\setlength{\rightskip}{0.00mm}
\raggedright
2.1.1. Universality and Indivisibility\hfill{}19
\vspace{0.00mm}

\vspace{0.00mm}
\setlength{\parindent}{0.00mm}
\setlength{\leftskip}{9.83mm}
\setlength{\rightskip}{0.00mm}
\raggedright
2.1.2. Non-Discrimination and Equality of Opportunity\hfill{}21
\vspace{0.00mm}

\vspace{0.00mm}
\setlength{\parindent}{0.00mm}
\setlength{\leftskip}{9.83mm}
\setlength{\rightskip}{0.00mm}
\raggedright
2.1.3. The Right to Development as a Right to Participate in Development\hfill{}22
\vspace{0.00mm}

\vspace{0.00mm}
\setlength{\parindent}{0.00mm}
\setlength{\leftskip}{9.83mm}
\setlength{\rightskip}{0.00mm}
\raggedright
2.1.4. Accountability and Transparency \hfill{}23
\vspace{0.00mm}

\vspace{0.00mm}
\setlength{\parindent}{0.00mm}
\setlength{\leftskip}{9.83mm}
\setlength{\rightskip}{0.00mm}
\raggedright
2.1.5. Equity and Justice \hfill{}24
\vspace{0.00mm}

\vspace{0.00mm}
\setlength{\parindent}{0.00mm}
\setlength{\leftskip}{4.91mm}
\setlength{\rightskip}{0.00mm}
\raggedright
2.2. The Controversies Surrounding the Right to Development\hfill{}25
\vspace{0.00mm}

\vspace{0.00mm}
\setlength{\parindent}{0.00mm}
\setlength{\leftskip}{9.83mm}
\setlength{\rightskip}{0.00mm}
\raggedright
2.2.1. The Right to Development as a Human Right\hfill{}25
\vspace{0.00mm}

\vspace{0.00mm}
\setlength{\parindent}{0.00mm}
\setlength{\leftskip}{9.83mm}
\setlength{\rightskip}{0.00mm}
\raggedright
2.2.2. Collective Rights\hfill{}28
\vspace{0.00mm}

\vspace{0.00mm}
\setlength{\parindent}{0.00mm}
\setlength{\leftskip}{9.83mm}
\setlength{\rightskip}{0.00mm}
\raggedright
2.2.3. The Right to Development as a Synthesis of All Rights \hfill{}35
\vspace{0.00mm}

\vspace{0.00mm}
\setlength{\parindent}{0.00mm}
\setlength{\leftskip}{9.83mm}
\setlength{\rightskip}{0.00mm}
\raggedright
2.2.4. The Right to Development as a Right to a Process of Development\hfill{}38
\vspace{0.00mm}

\vspace{0.00mm}
\setlength{\parindent}{0.00mm}
\setlength{\leftskip}{9.83mm}
\setlength{\rightskip}{0.00mm}
\raggedright
2.2.5. International Co-operation v National Implementation\hfill{}43
\vspace{0.00mm}

\vspace{0.00mm}
\setlength{\parindent}{0.00mm}
\setlength{\leftskip}{14.74mm}
\setlength{\rightskip}{0.00mm}
\raggedright
All Human Beings as the Duty-Bearers of the Right to Development \hfill{}43
\vspace{0.00mm}

\vspace{0.00mm}
\setlength{\parindent}{0.00mm}
\setlength{\leftskip}{14.74mm}
\setlength{\rightskip}{0.00mm}
\raggedright
States as the Duty-Bearers of the Right to Development\hfill{}44
\vspace{0.00mm}

\vspace{0.00mm}
\setlength{\parindent}{0.00mm}
\setlength{\leftskip}{14.74mm}
\setlength{\rightskip}{0.00mm}
\raggedright
The International Community as the Duty-Bearer of the Right to Development\hfill{}46
\vspace{0.00mm}

\vspace{0.00mm}
\setlength{\parindent}{0.00mm}
\setlength{\leftskip}{14.74mm}
\setlength{\rightskip}{0.00mm}
\raggedright
International Organisations as the Duty-Bearers of the Right to Development\hfill{}50
\vspace{0.00mm}

\vspace{0.00mm}
\setlength{\parindent}{0.00mm}
\setlength{\leftskip}{9.83mm}
\setlength{\rightskip}{0.00mm}
\raggedright
2.2.6. Security, Peace and the Right to Development\hfill{}51
\vspace{0.00mm}

\vspace{0.00mm}
\setlength{\parindent}{0.00mm}
\setlength{\leftskip}{4.91mm}
\setlength{\rightskip}{0.00mm}
\raggedright
3.1. International Treaties\hfill{}52
\vspace{0.00mm}

\vspace{0.00mm}
\setlength{\parindent}{0.00mm}
\setlength{\leftskip}{9.83mm}
\setlength{\rightskip}{0.00mm}
\raggedright
3.1.1. The Right to Development and The African Charter on Human and Peoples' Rights\hfill{}53
\vspace{0.00mm}

\vspace{0.00mm}
\setlength{\parindent}{0.00mm}
\setlength{\leftskip}{4.91mm}
\setlength{\rightskip}{0.00mm}
\raggedright
3.2. Customary International Law  \hfill{}60
\vspace{0.00mm}

\vspace{0.00mm}
\setlength{\parindent}{0.00mm}
\setlength{\leftskip}{9.83mm}
\setlength{\rightskip}{0.00mm}
\raggedright
3.2.1. Generality of the Practice\hfill{}62
\vspace{0.00mm}

\vspace{0.00mm}
\setlength{\parindent}{0.00mm}
\setlength{\leftskip}{9.83mm}
\setlength{\rightskip}{0.00mm}
\raggedright
3.2.2. The Persistent Objector \hfill{}63
\vspace{0.00mm}

\vspace{0.00mm}
\setlength{\parindent}{0.00mm}
\setlength{\leftskip}{9.83mm}
\setlength{\rightskip}{0.00mm}
\raggedright
3.2.3. General Assembly Resolutions and International Conferences \hfill{}63
\vspace{0.00mm}

\vspace{0.00mm}
\setlength{\parindent}{0.00mm}
\setlength{\leftskip}{9.83mm}
\setlength{\rightskip}{0.00mm}
\raggedright
3.2.4. The Right to Development in Customary International Law\hfill{}65
\vspace{0.00mm}

\vspace{0.00mm}
\setlength{\parindent}{0.00mm}
\setlength{\leftskip}{4.91mm}
\setlength{\rightskip}{0.00mm}
\raggedright
3.3. The Right to Development in other Legal Instruments\hfill{}79
\vspace{0.00mm}

\vspace{0.00mm}
\setlength{\parindent}{0.00mm}
\setlength{\leftskip}{0.00mm}
\setlength{\rightskip}{0.00mm}
\raggedright
Conclusion \hfill{}81
\vspace{0.00mm}

\vspace{0.00mm}
\setlength{\parindent}{0.00mm}
\setlength{\leftskip}{0.00mm}
\setlength{\rightskip}{0.00mm}
\raggedright
Bibliography\hfill{}82
\vspace{0.00mm}

\vspace{0.00mm}
\setlength{\parindent}{0.00mm}
\setlength{\leftskip}{0.00mm}
\setlength{\rightskip}{0.00mm}
\raggedright
Annex 1: Declaration on the Right to Development\hfill{}98\newpage

\vspace{0.00mm}



\vspace{0.00mm}
\setlength{\parindent}{0.00mm}
\setlength{\leftskip}{0.00mm}
\setlength{\rightskip}{0.00mm}
\raggedright
\newpage

\vspace{0.00mm}
\begin{itemize}

\item
\vspace{4.17mm}
\setlength{\parindent}{0.00mm}
\setlength{\leftskip}{0.00mm}
\setlength{\rightskip}{0.00mm}

\textbf{Table of Cases }
\vspace{2.08mm}

\end{itemize}
\vspace{0.00mm}
\setlength{\parindent}{0.00mm}
\setlength{\leftskip}{0.00mm}
\setlength{\rightskip}{0.00mm}

International Court of Justice
\vspace{0.00mm}

\vspace{0.00mm}
\setlength{\parindent}{0.00mm}
\setlength{\leftskip}{0.00mm}
\setlength{\rightskip}{0.00mm}
\raggedright

\vspace{0.00mm}

\vspace{0.00mm}
\setlength{\parindent}{0.00mm}
\setlength{\leftskip}{0.00mm}
\setlength{\rightskip}{0.00mm}

Anglo-Norwegian Fisheries Case: United Kingdom v Norway (1951) ICJ Rep 116.
\vspace{0.00mm}

\vspace{0.00mm}
\setlength{\parindent}{0.00mm}
\setlength{\leftskip}{0.00mm}
\setlength{\rightskip}{0.00mm}

Asylum Case: Columbia v Peru (1950) ICJ Rep 266
\vspace{0.00mm}

\vspace{0.00mm}
\setlength{\parindent}{0.00mm}
\setlength{\leftskip}{0.00mm}
\setlength{\rightskip}{0.00mm}

Legality of the Threat or Use of Nuclear Weapons (Request for Advisory Opinion by the World Health Organisation) (1996) ICJ Rep 90
\vspace{0.00mm}

\vspace{0.00mm}
\setlength{\parindent}{0.00mm}
\setlength{\leftskip}{0.00mm}
\setlength{\rightskip}{0.00mm}

Military and Paramilitary Activities in and Against Nicaragua: Nicaragua v United States of America, (Merits), (1986) ICJ Rep 14
\vspace{0.00mm}

\vspace{0.00mm}
\setlength{\parindent}{0.00mm}
\setlength{\leftskip}{0.00mm}
\setlength{\rightskip}{0.00mm}

North Sea Continental  Shelf Cases: Federal Republic of Germany v Denmark; Federal Republic of Germany v Netherlands (1969) ICJ Rep 3. 
\vspace{0.00mm}

\vspace{0.00mm}
\setlength{\parindent}{-4.91mm}
\setlength{\leftskip}{4.91mm}
\setlength{\rightskip}{0.00mm}


\vspace{0.00mm}

\vspace{0.00mm}
\setlength{\parindent}{0.00mm}
\setlength{\leftskip}{0.00mm}
\setlength{\rightskip}{0.00mm}

African Commission on Human and Peoples' Rights
\vspace{0.00mm}

\vspace{0.00mm}
\setlength{\parindent}{-4.91mm}
\setlength{\leftskip}{4.91mm}
\setlength{\rightskip}{0.00mm}


\vspace{0.00mm}

\vspace{0.00mm}
\setlength{\parindent}{0.00mm}
\setlength{\leftskip}{0.00mm}
\setlength{\rightskip}{0.00mm}

Malawi African Association etc v Mauritania (2000) AHRLR 149 (ACHPR 2000).
\vspace{0.00mm}

\vspace{0.00mm}
\setlength{\parindent}{0.00mm}
\setlength{\leftskip}{0.00mm}
\setlength{\rightskip}{0.00mm}

Social and Economic Rights Action Center and the Center for Economic and Social Rights v Nigeria Communication 155/96, 27 October 2001.
\vspace{0.00mm}

\vspace{0.00mm}
\setlength{\parindent}{0.00mm}
\setlength{\leftskip}{0.00mm}
\setlength{\rightskip}{0.00mm}


\vspace{0.00mm}
\begin{itemize}

\item
\vspace{4.17mm}
\setlength{\parindent}{0.00mm}
\setlength{\leftskip}{0.00mm}
\setlength{\rightskip}{0.00mm}

\textbf{}
\vspace{2.08mm}

\end{itemize}
\vspace{0.00mm}
\setlength{\parindent}{0.00mm}
\setlength{\leftskip}{0.00mm}
\setlength{\rightskip}{0.00mm}
\raggedright
\newpage

\vspace{0.00mm}
\begin{itemize}

\item
\vspace{4.17mm}
\setlength{\parindent}{0.00mm}
\setlength{\leftskip}{0.00mm}
\setlength{\rightskip}{0.00mm}

\textbf{Table of Treaties}
\vspace{2.08mm}

\end{itemize}
\vspace{0.00mm}
\setlength{\parindent}{0.00mm}
\setlength{\leftskip}{0.00mm}
\setlength{\rightskip}{0.00mm}

Including Covenants, Treaties, Conventions, ICJ Statute and UN Charter etc. 
\vspace{0.00mm}

\vspace{0.00mm}
\setlength{\parindent}{0.00mm}
\setlength{\leftskip}{0.00mm}
\setlength{\rightskip}{0.00mm}


\vspace{0.00mm}

\vspace{0.00mm}
\setlength{\parindent}{0.00mm}
\setlength{\leftskip}{0.00mm}
\setlength{\rightskip}{0.00mm}
\raggedright
Additional Protocol to the American Convention on Human Rights in the Area of Economic, social and cultural rights Protocol of San Salvador 1988
\vspace{0.00mm}

\vspace{0.00mm}
\setlength{\parindent}{0.00mm}
\setlength{\leftskip}{0.00mm}
\setlength{\rightskip}{0.00mm}
\raggedright
Additional Protocol to the European Social Charter Providing for a System of Collective Complaints 1995
\vspace{0.00mm}

\vspace{0.00mm}
\setlength{\parindent}{0.00mm}
\setlength{\leftskip}{0.00mm}
\setlength{\rightskip}{0.00mm}

African Charter on Human and People's Rights 1981
\vspace{0.00mm}

\vspace{0.00mm}
\setlength{\parindent}{0.00mm}
\setlength{\leftskip}{0.00mm}
\setlength{\rightskip}{0.00mm}
\raggedright
African Charter on the Rights and Welfare of the Child 1990
\vspace{0.00mm}

\vspace{0.00mm}
\setlength{\parindent}{0.00mm}
\setlength{\leftskip}{0.00mm}
\setlength{\rightskip}{0.00mm}
\raggedright
American Convention on Human Rights 1969
\vspace{0.00mm}

\vspace{0.00mm}
\setlength{\parindent}{0.00mm}
\setlength{\leftskip}{0.00mm}
\setlength{\rightskip}{0.00mm}
\raggedright
American Declaration of Human Rights and Duties of  Man 1948
\vspace{0.00mm}

\vspace{0.00mm}
\setlength{\parindent}{0.00mm}
\setlength{\leftskip}{0.00mm}
\setlength{\rightskip}{0.00mm}

Arab Charter on Human Rights 1994
\vspace{0.00mm}

\vspace{0.00mm}
\setlength{\parindent}{0.00mm}
\setlength{\leftskip}{0.00mm}
\setlength{\rightskip}{0.00mm}

Charter of the Organisation of African Unity 1963
\vspace{0.00mm}

\vspace{0.00mm}
\setlength{\parindent}{0.00mm}
\setlength{\leftskip}{0.00mm}
\setlength{\rightskip}{0.00mm}
\raggedright
Convention on the Elimination of All Forms of Discrimination Against Women 1979 
\vspace{0.00mm}

\vspace{0.00mm}
\setlength{\parindent}{0.00mm}
\setlength{\leftskip}{0.00mm}
\setlength{\rightskip}{0.00mm}
\raggedright
Convention on the Elimination of All Forms of Racial Discrimination 1965
\vspace{0.00mm}

\vspace{0.00mm}
\setlength{\parindent}{0.00mm}
\setlength{\leftskip}{0.00mm}
\setlength{\rightskip}{0.00mm}
\raggedright
Convention on the Rights of the Child 1989
\vspace{0.00mm}

\vspace{0.00mm}
\setlength{\parindent}{0.00mm}
\setlength{\leftskip}{0.00mm}
\setlength{\rightskip}{0.00mm}
\raggedright
Draft Protocol to the African Charter on Human and Peoples' Rights on the Rights of Women  
\vspace{0.00mm}

\vspace{0.00mm}
\setlength{\parindent}{0.00mm}
\setlength{\leftskip}{0.00mm}
\setlength{\rightskip}{0.00mm}

European Convention for the Protection of Fundamental Human Rights and Fundamental Freedoms 1950
\vspace{0.00mm}

\vspace{0.00mm}
\setlength{\parindent}{0.00mm}
\setlength{\leftskip}{0.00mm}
\setlength{\rightskip}{0.00mm}

European Social Charter 1961
\vspace{0.00mm}

\vspace{0.00mm}
\setlength{\parindent}{0.00mm}
\setlength{\leftskip}{0.00mm}
\setlength{\rightskip}{0.00mm}
\raggedright
European Social Charter 1961
\vspace{0.00mm}

\vspace{0.00mm}
\setlength{\parindent}{0.00mm}
\setlength{\leftskip}{0.00mm}
\setlength{\rightskip}{0.00mm}

Inter-American Convention on Human Rights 1969
\vspace{0.00mm}

\vspace{0.00mm}
\setlength{\parindent}{0.00mm}
\setlength{\leftskip}{0.00mm}
\setlength{\rightskip}{0.00mm}
\raggedright
International Convention on the Protection of the Rights of All Migrant Workers and Members of their Families 1990
\vspace{0.00mm}

\vspace{0.00mm}
\setlength{\parindent}{0.00mm}
\setlength{\leftskip}{0.00mm}
\setlength{\rightskip}{0.00mm}

International Covenant on Civil and Political Rights 1966
\vspace{0.00mm}

\vspace{0.00mm}
\setlength{\parindent}{0.00mm}
\setlength{\leftskip}{0.00mm}
\setlength{\rightskip}{0.00mm}

International Covenant on Economic, Social and Cultural Rights 1966
\vspace{0.00mm}

\vspace{0.00mm}
\setlength{\parindent}{0.00mm}
\setlength{\leftskip}{0.00mm}
\setlength{\rightskip}{0.00mm}
\raggedright
Revised European Social Charter 1996 
\vspace{0.00mm}

\vspace{0.00mm}
\setlength{\parindent}{0.00mm}
\setlength{\leftskip}{0.00mm}
\setlength{\rightskip}{0.00mm}

Statute of the International Court of Justice 1945 
\vspace{0.00mm}

\vspace{0.00mm}
\setlength{\parindent}{0.00mm}
\setlength{\leftskip}{0.00mm}
\setlength{\rightskip}{0.00mm}

Statute of the Permanent Court of International Justice 1920
\vspace{0.00mm}

\vspace{0.00mm}
\setlength{\parindent}{0.00mm}
\setlength{\leftskip}{0.00mm}
\setlength{\rightskip}{0.00mm}

United Nations Charter 1945
\vspace{0.00mm}

\vspace{0.00mm}
\setlength{\parindent}{0.00mm}
\setlength{\leftskip}{0.00mm}
\setlength{\rightskip}{0.00mm}


\vspace{0.00mm}
\begin{itemize}

\item
\vspace{4.17mm}
\setlength{\parindent}{0.00mm}
\setlength{\leftskip}{0.00mm}
\setlength{\rightskip}{0.00mm}

\textbf{}
\vspace{2.08mm}

\end{itemize}
\vspace{0.00mm}
\setlength{\parindent}{0.00mm}
\setlength{\leftskip}{0.00mm}
\setlength{\rightskip}{0.00mm}
\raggedright
\newpage

\vspace{0.00mm}
\begin{itemize}

\item
\vspace{4.17mm}
\setlength{\parindent}{0.00mm}
\setlength{\leftskip}{0.00mm}
\setlength{\rightskip}{0.00mm}

\textbf{Table of Other Documents }
\vspace{2.08mm}

\end{itemize}
\vspace{0.00mm}
\setlength{\parindent}{0.00mm}
\setlength{\leftskip}{0.00mm}
\setlength{\rightskip}{0.00mm}

Declarations, Resolutions etc.
\vspace{0.00mm}

\vspace{0.00mm}
\setlength{\parindent}{0.00mm}
\setlength{\leftskip}{0.00mm}
\setlength{\rightskip}{0.00mm}


\vspace{0.00mm}

\vspace{0.00mm}
\setlength{\parindent}{0.00mm}
\setlength{\leftskip}{0.00mm}
\setlength{\rightskip}{0.00mm}

International Labour Organisation Philadelphia Declaration 1944  
\vspace{0.00mm}

\vspace{0.00mm}
\setlength{\parindent}{0.00mm}
\setlength{\leftskip}{0.00mm}
\setlength{\rightskip}{0.00mm}


\vspace{0.00mm}

\vspace{0.00mm}
\setlength{\parindent}{0.00mm}
\setlength{\leftskip}{0.00mm}
\setlength{\rightskip}{0.00mm}

General Assembly of the United Nations Resolutions and and Declarations.
\vspace{0.00mm}

\vspace{0.00mm}
\setlength{\parindent}{0.00mm}
\setlength{\leftskip}{0.00mm}
\setlength{\rightskip}{0.00mm}


\vspace{0.00mm}

\vspace{0.00mm}
\setlength{\parindent}{0.00mm}
\setlength{\leftskip}{0.00mm}
\setlength{\rightskip}{0.00mm}

Charter of Economic Rights and Duties of States 1974, Resolution 3281 (XXIX).
\vspace{0.00mm}

\vspace{0.00mm}
\setlength{\parindent}{0.00mm}
\setlength{\leftskip}{0.00mm}
\setlength{\rightskip}{0.00mm}

Declaration on Principles of International Law Concerning Friendly Relations and Co-operation among States in Accordance with the Charter of the United Nations 1970, Resolution 2625 (XXV).
\vspace{0.00mm}

\vspace{0.00mm}
\setlength{\parindent}{0.00mm}
\setlength{\leftskip}{0.00mm}
\setlength{\rightskip}{0.00mm}

Declaration on Social Progress and Development 1969, Resolution 2542 (XXIV)
\vspace{0.00mm}

\vspace{0.00mm}
\setlength{\parindent}{0.00mm}
\setlength{\leftskip}{0.00mm}
\setlength{\rightskip}{0.00mm}

Declaration on the Permanent Sovereignty over Natural Resources 1962, Resolution 1803 (XVII)
\vspace{0.00mm}

\vspace{0.00mm}
\setlength{\parindent}{0.00mm}
\setlength{\leftskip}{0.00mm}
\setlength{\rightskip}{0.00mm}

Declaration on the Preparation of Societies for Life in Peace 1978, Resolution 33/73 (1978)
\vspace{0.00mm}

\vspace{0.00mm}
\setlength{\parindent}{0.00mm}
\setlength{\leftskip}{0.00mm}
\setlength{\rightskip}{0.00mm}

Declaration on the Right to Development 1986, Resolution 41/128 (1986). 
\vspace{0.00mm}

\vspace{0.00mm}
\setlength{\parindent}{0.00mm}
\setlength{\leftskip}{0.00mm}
\setlength{\rightskip}{0.00mm}

Definition of Aggression 1974, Resolution 3314 (XXIX).
\vspace{0.00mm}

\vspace{0.00mm}
\setlength{\parindent}{0.00mm}
\setlength{\leftskip}{0.00mm}
\setlength{\rightskip}{0.00mm}

General Assembly Resolutions on the Right to development:
\vspace{0.00mm}

\vspace{0.00mm}
\setlength{\parindent}{0.00mm}
\setlength{\leftskip}{0.00mm}
\setlength{\rightskip}{0.00mm}

\hfill{}Resolution 50/184 (1996) Right to development 
\vspace{0.00mm}

\vspace{0.00mm}
\setlength{\parindent}{0.00mm}
\setlength{\leftskip}{0.00mm}
\setlength{\rightskip}{0.00mm}

\hfill{}Resolution 51/99 (1997) Right to development 
\vspace{0.00mm}

\vspace{0.00mm}
\setlength{\parindent}{0.00mm}
\setlength{\leftskip}{0.00mm}
\setlength{\rightskip}{0.00mm}

\hfill{}Resolution 52/136 (1998) Right to development 
\vspace{0.00mm}

\vspace{0.00mm}
\setlength{\parindent}{0.00mm}
\setlength{\leftskip}{0.00mm}
\setlength{\rightskip}{0.00mm}

\hfill{}Resolution 53/155 (1999) Right to development 
\vspace{0.00mm}

\vspace{0.00mm}
\setlength{\parindent}{0.00mm}
\setlength{\leftskip}{0.00mm}
\setlength{\rightskip}{0.00mm}

\hfill{}Resolution 54/175 (2000) The right to development 
\vspace{0.00mm}

\vspace{0.00mm}
\setlength{\parindent}{0.00mm}
\setlength{\leftskip}{0.00mm}
\setlength{\rightskip}{0.00mm}

\hfill{}Resolution 55/108 (2001) The right to development 
\vspace{0.00mm}

\vspace{0.00mm}
\setlength{\parindent}{0.00mm}
\setlength{\leftskip}{0.00mm}
\setlength{\rightskip}{0.00mm}

\hfill{}Resolution 56/150 (2002) The right to development.
\vspace{0.00mm}

\vspace{0.00mm}
\setlength{\parindent}{0.00mm}
\setlength{\leftskip}{0.00mm}
\setlength{\rightskip}{0.00mm}

Millennium Declaration 2000, Resolution 55/2 (2000) 
\vspace{0.00mm}

\vspace{0.00mm}
\setlength{\parindent}{0.00mm}
\setlength{\leftskip}{0.00mm}
\setlength{\rightskip}{0.00mm}

Universal Declaration of Human Rights 1948, Resolution 217A (III).
\vspace{0.00mm}

\vspace{0.00mm}
\setlength{\parindent}{0.00mm}
\setlength{\leftskip}{0.00mm}
\setlength{\rightskip}{0.00mm}


\vspace{0.00mm}

\vspace{0.00mm}
\setlength{\parindent}{0.00mm}
\setlength{\leftskip}{0.00mm}
\setlength{\rightskip}{0.00mm}

International Conferences
\vspace{0.00mm}

\vspace{0.00mm}
\setlength{\parindent}{0.00mm}
\setlength{\leftskip}{0.00mm}
\setlength{\rightskip}{0.00mm}


\vspace{0.00mm}

\vspace{0.00mm}
\setlength{\parindent}{0.00mm}
\setlength{\leftskip}{0.00mm}
\setlength{\rightskip}{0.00mm}

International Conference on Population and Development, Cairo 1994
\vspace{0.00mm}

\vspace{0.00mm}
\setlength{\parindent}{0.00mm}
\setlength{\leftskip}{0.00mm}
\setlength{\rightskip}{0.00mm}

International Conference on Human Rights, Tehran 1969
\vspace{0.00mm}

\vspace{0.00mm}
\setlength{\parindent}{0.00mm}
\setlength{\leftskip}{0.00mm}
\setlength{\rightskip}{0.00mm}

Second UN Conference on Human Settlements (Habitat II)1996
\vspace{0.00mm}

\vspace{0.00mm}
\setlength{\parindent}{0.00mm}
\setlength{\leftskip}{0.00mm}
\setlength{\rightskip}{0.00mm}

The Platform for Action 1995 Fourth World Conference on Women
\vspace{0.00mm}

\vspace{0.00mm}
\setlength{\parindent}{0.00mm}
\setlength{\leftskip}{0.00mm}
\setlength{\rightskip}{0.00mm}

The World Food Summit 1996
\vspace{0.00mm}

\vspace{0.00mm}
\setlength{\parindent}{0.00mm}
\setlength{\leftskip}{0.00mm}
\setlength{\rightskip}{0.00mm}

World Conference against Racism, Racial Discrimination, Xenophobia and Related Intolerance 2001
\vspace{0.00mm}

\vspace{0.00mm}
\setlength{\parindent}{0.00mm}
\setlength{\leftskip}{0.00mm}
\setlength{\rightskip}{0.00mm}

World Conference on Human Rights, Vienna, Vienna Declaration and Programme of Action 1993
\vspace{0.00mm}

\vspace{0.00mm}
\setlength{\parindent}{0.00mm}
\setlength{\leftskip}{0.00mm}
\setlength{\rightskip}{0.00mm}

World Summit for Social Development, Copenhagen 1995
\vspace{0.00mm}

\vspace{0.00mm}
\setlength{\parindent}{0.00mm}
\setlength{\leftskip}{0.00mm}
\setlength{\rightskip}{0.00mm}

World Summit on Sustainable Development 2002\newpage

\vspace{0.00mm}
\begin{itemize}

\item
\vspace{4.17mm}
\setlength{\parindent}{0.00mm}
\setlength{\leftskip}{0.00mm}
\setlength{\rightskip}{0.00mm}
\raggedright
\textbf{Introduction:}
\vspace{2.08mm}

\end{itemize}
\vspace{0.00mm}
\setlength{\parindent}{0.00mm}
\setlength{\leftskip}{0.00mm}
\setlength{\rightskip}{0.00mm}


\vspace{0.00mm}

\vspace{0.00mm}
\setlength{\parindent}{0.00mm}
\setlength{\leftskip}{0.00mm}
\setlength{\rightskip}{0.00mm}

Traditionally, international law regulated the relations between states, based on the principle of sovereign equality of each state and consent. The 20$^{th}$ century saw the birth of international human rights law.  In contrast to international law, international human rights law regulates state behaviour toward its citizens and the citizens of other states on their territory. International human rights law is also based on the principle of state sovereignty and consent and, as such, each state determines to what extent it wishes to be bound by it. Under the principles of non-intervention and state sovereignty, states do not have to ensure the protection of human rights in any other territory but their own. 
\vspace{0.00mm}

\vspace{0.00mm}
\setlength{\parindent}{0.00mm}
\setlength{\leftskip}{0.00mm}
\setlength{\rightskip}{0.00mm}


\vspace{0.00mm}

\vspace{0.00mm}
\setlength{\parindent}{0.00mm}
\setlength{\leftskip}{0.00mm}
\setlength{\rightskip}{0.00mm}

Whilst international human rights law emerged, in a parallel setting international economic law also materialised as a separate branch of international law, dealing with development issues. A states development was measured by its economic growth in terms of its Gross National Product (GNP) or Gross Domestic Product (GDP) and focused on policies such as 'basic needs' and 'structural adjustment'. This branch of law predominately dealt with issues regarding poverty, primarily seen as an economic concern. However, it became apparent that the definition of development within this branch of law was essentially concerned with development as economic growth and material wealth and omitted concerns for human rights. With the need to recognise fundamental human rights, such as the rights to food, housing, basic health care and education, as concerns which affect the development discourse, the human right to development (TRTD) emerged, forging a link between the human rights discourse and the development discourse. 
\vspace{0.00mm}

\vspace{0.00mm}
\setlength{\parindent}{0.00mm}
\setlength{\leftskip}{0.00mm}
\setlength{\rightskip}{0.00mm}


\vspace{0.00mm}

\vspace{0.00mm}
\setlength{\parindent}{0.00mm}
\setlength{\leftskip}{0.00mm}
\setlength{\rightskip}{0.00mm}

The concept of the right to development is fairly new within international law. It was endorsed as a human right by the General Assembly Declaration on the Right to Development$^{}$ (TDRTD) by a majority, with only the USA casting a dissenting vote. The link between human rights and economic and social development was recognised in the United Nations Charter$^{}$. The Universal Declaration of Human Rights$^{}$, declared that human rights  encompassed both civil and political rights (CPR's)$^{}$ and Economic, social and cultural rights (ESCR's)$^{}$. During the cold war, disagreement occurred regarding the unity of such rights, which resulted in the adoption of two separate covenants, the International Covenant on Civil and Political Rights (ICCPR) and the International Covenant on Economic, Social and Cultural rights (ICESCR). This created a hierarchy between rights which was not envisaged by the Universal Declaration and as such, CPR's were given priority over ESCR's, by many states$^{}$. Although neither covenant specifically address TRTD, TDRTD re-integrates CPR's and ESCR's as indivisible and interdependent. 
\vspace{0.00mm}

\vspace{0.00mm}
\setlength{\parindent}{0.00mm}
\setlength{\leftskip}{0.00mm}
\setlength{\rightskip}{0.00mm}

 
\vspace{0.00mm}

\vspace{0.00mm}
\setlength{\parindent}{0.00mm}
\setlength{\leftskip}{0.00mm}
\setlength{\rightskip}{0.00mm}

This paper focuses upon the TRTD as an emerging concept in international law and attempts to determine if there is a legal obligation for the international community to realise the right. TRTD has been framed as a human right on numerous occasions. This suggests that it falls within international human rights law. I will first clarify the status of TRTD within the human rights framework. This is of vital significance as the international community does not have a legal obligation to realise TRTD as a human right -- this is the duty of individual states. TRTD is then analysed in the context of international law, which serves to clarify the rules that denote legal obligations on states. 
\vspace{0.00mm}

\vspace{0.00mm}
\setlength{\parindent}{0.00mm}
\setlength{\leftskip}{0.00mm}
\setlength{\rightskip}{0.00mm}


\vspace{0.00mm}

\vspace{0.00mm}
\setlength{\parindent}{0.00mm}
\setlength{\leftskip}{0.00mm}
\setlength{\rightskip}{0.00mm}

This research uses several methodologies. An historical methodology is used in numerous sections to identify the background of the issues which are discussed and to place TRTD into context. Stemming from this, a descriptive methodology is used to provide information on the key issues, this serves to set the scene for a critical analysis. In many of the sections, correlational and explanatory methodologies are employed to identify, discuss and explain the relationships between TRTD and the issues which have been identified.   
\vspace{0.00mm}

\vspace{0.00mm}
\setlength{\parindent}{0.00mm}
\setlength{\leftskip}{0.00mm}
\setlength{\rightskip}{0.00mm}


\vspace{0.00mm}

\vspace{0.00mm}
\setlength{\parindent}{0.00mm}
\setlength{\leftskip}{0.00mm}
\setlength{\rightskip}{0.00mm}

TRTD remains contentious and as such, this research serves to provide a synthesis on the current debate of the subject. While undertaking this investigation, it was discovered that, although there are primary materials specifically addressing TRTD, there is a limited range of secondary and tertiary material. Journal articles are scarce to find and appear sporadically, with only a few academic scholars commenting on the right. In addition, many of the articles have been written specifically by United Nations experts and other governmental organisations; there is very little Non-Governmental Organisations (NGO) feedback on the RTD and thus there is a risk of bias in this research. Furthermore, this chosen area of study can encompass a wide variety of issues and can become too dense and complicated. It is for this reason that this paper excludes any in depth analysis on economic policies, economic law and international environmental law, making it more suitable both in terms of the time given to undertake it and also the strict word limit.   
\vspace{0.00mm}

\vspace{0.00mm}
\setlength{\parindent}{0.00mm}
\setlength{\leftskip}{0.00mm}
\setlength{\rightskip}{0.00mm}


\vspace{0.00mm}

\vspace{0.00mm}
\setlength{\parindent}{0.00mm}
\setlength{\leftskip}{0.00mm}
\setlength{\rightskip}{0.00mm}

Chapter 1 lays the foundations and introduces the emergence of TRTD. It provides a brief historical account of international human rights law and identifies the implicit references to TRTD by the international community, to place TRTD into the framework of human rights and international law as a whole.  Chapter 1 also introduces the political factors which affect TRTD and identifies the explicit references to TRTD made by the international community. 
\vspace{0.00mm}

\vspace{0.00mm}
\setlength{\parindent}{0.00mm}
\setlength{\leftskip}{0.00mm}
\setlength{\rightskip}{0.00mm}


\vspace{0.00mm}

\vspace{0.00mm}
\setlength{\parindent}{0.00mm}
\setlength{\leftskip}{0.00mm}
\setlength{\rightskip}{0.00mm}

Chapter 2, summarises the non-controversial aspects of TRTD, which have been agreed by the international community. It then focuses on the controversies surrounding TRTD in order to establish whether or not these can clarify the nature of the right, its relationship with different aspects of human rights and where TRTD belongs in international law.   
\vspace{0.00mm}

\vspace{0.00mm}
\setlength{\parindent}{0.00mm}
\setlength{\leftskip}{0.00mm}
\setlength{\rightskip}{0.00mm}


\vspace{0.00mm}

\vspace{0.00mm}
\setlength{\parindent}{0.00mm}
\setlength{\leftskip}{0.00mm}
\setlength{\rightskip}{0.00mm}

Chapter 3 examines the legal status of TRTD in international treaties and in customary international law, in order to establish whether or not there is a legal obligation for the international community to realise TRTD. This chapter provides information on the background of specific legal and political instruments.  
\vspace{0.00mm}

\vspace{0.00mm}
\setlength{\parindent}{0.00mm}
\setlength{\leftskip}{0.00mm}
\setlength{\rightskip}{0.00mm}


\vspace{0.00mm}

\vspace{0.00mm}
\setlength{\parindent}{0.00mm}
\setlength{\leftskip}{0.00mm}
\setlength{\rightskip}{0.00mm}


\vspace{0.00mm}

\vspace{0.00mm}
\setlength{\parindent}{0.00mm}
\setlength{\leftskip}{0.00mm}
\setlength{\rightskip}{0.00mm}
\raggedright
\newpage

\vspace{0.00mm}
\begin{itemize}

\item
\vspace{4.17mm}
\setlength{\parindent}{0.00mm}
\setlength{\leftskip}{0.00mm}
\setlength{\rightskip}{0.00mm}
\raggedright
\textbf{1. Introducing the Right to Development}
\vspace{2.08mm}

\end{itemize}
\vspace{0.00mm}
\setlength{\parindent}{0.00mm}
\setlength{\leftskip}{0.00mm}
\setlength{\rightskip}{0.00mm}


\vspace{0.00mm}

\vspace{0.00mm}
\setlength{\parindent}{0.00mm}
\setlength{\leftskip}{0.00mm}
\setlength{\rightskip}{0.00mm}

This chapter provides a brief historical account of international human rights law and identifies the implicit references made to the right. It provides an insight into the political factors surrounding the emergence of international human rights law. This chapter then focuses on the background of TRTD which essentially deals with the explicit references to the right and the political backdrop in which it emerged.  
\vspace{0.00mm}

\vspace{0.00mm}
\setlength{\parindent}{0.00mm}
\setlength{\leftskip}{0.00mm}
\setlength{\rightskip}{0.00mm}


\vspace{0.00mm}
\begin{itemize}
\begin{itemize}

\item
\vspace{4.17mm}
\setlength{\parindent}{0.00mm}
\setlength{\leftskip}{0.00mm}
\setlength{\rightskip}{0.00mm}
\raggedright
\textbf{1.1. The Emergence of the Right to Development}
\vspace{2.08mm}

\end{itemize}
\end{itemize}
\vspace{0.00mm}
\setlength{\parindent}{0.00mm}
\setlength{\leftskip}{0.00mm}
\setlength{\rightskip}{0.00mm}


\vspace{0.00mm}

\vspace{0.00mm}
\setlength{\parindent}{0.00mm}
\setlength{\leftskip}{0.00mm}
\setlength{\rightskip}{0.00mm}

The end of the First World War saw the creation of the League of Nations with its subsidiary organs, the International Labour Organisation (ILO) and the Permanent Court of International Justice (PCIJ). The ILO was essentially created to improve people's working conditions and promote freedom of association. It was the first international organisation to address social inequality and social justice, a fundamental maneuver for the recognition of social rights as human rights. 
\vspace{0.00mm}

\vspace{0.00mm}
\setlength{\parindent}{0.00mm}
\setlength{\leftskip}{0.00mm}
\setlength{\rightskip}{0.00mm}


\vspace{0.00mm}

\vspace{0.00mm}
\setlength{\parindent}{0.00mm}
\setlength{\leftskip}{0.00mm}
\setlength{\rightskip}{0.00mm}

The 1944 ILO Philadelphia Declaration contained implicit references to the TRTD, affirming that every individual has ``\textit{the right to pursue both their material well-being and their spiritual development in conditions of freedom and dignity, of economic security and equal opportunity}.$^{}$''   
\vspace{0.00mm}

\vspace{0.00mm}
\setlength{\parindent}{0.00mm}
\setlength{\leftskip}{0.00mm}
\setlength{\rightskip}{0.00mm}


\vspace{0.00mm}

\vspace{0.00mm}
\setlength{\parindent}{0.00mm}
\setlength{\leftskip}{0.00mm}
\setlength{\rightskip}{0.00mm}

As a result of the atrocities witnessed during the Second World War, the distinct branch of international human rights law evolved significantly within international public law. The abuses perpetrated by Nazi Germany stimulated the international community to create a framework in which fundamental rights could be protected without infringing state sovereignty and thus emanating the foundations of what is now modern international human rights law. Although the notion of human rights developed from 19$^{th}$ century western thought and philosophy, the foundations of social justice, liberty and freedom can be traced back over centuries, in many different parts of the world; numerous religions; cultures; philosophies, to name but a few. 
\vspace{0.00mm}

\vspace{0.00mm}
\setlength{\parindent}{0.00mm}
\setlength{\leftskip}{0.00mm}
\setlength{\rightskip}{0.00mm}


\vspace{0.00mm}

\vspace{0.00mm}
\setlength{\parindent}{0.00mm}
\setlength{\leftskip}{0.00mm}
\setlength{\rightskip}{0.00mm}

The United Nations (UN) was established in the aftermath of the Second World War. Created to maintain international peace and security through cooperation, the UN charter underlined the necessity to promote human rights. The UN declared one of its main aims was ``\textit{to reaffirm faith in fundamental human rights, in the dignity and worth of the human person, in the equal rights of men and women and of nations large and small}''$^{}$. Article 55$^{}$ crucially links international stability alongside social and economic development and human rights, suggesting their interdependence. Article 56$^{}$ pledges all members to take both joint and individual action to achieve the purposes set forth in Article 55. Furthermore, Article 68$^{}$ stipulated that the Economic and Social Council (ECOSOC) shall set up a Commission on Human Rights (CHR) . These articles could be used to indirectly affirm the existence of TRTD. 
\vspace{0.00mm}

\vspace{0.00mm}
\setlength{\parindent}{0.00mm}
\setlength{\leftskip}{0.00mm}
\setlength{\rightskip}{0.00mm}


\vspace{0.00mm}

\vspace{0.00mm}
\setlength{\parindent}{0.00mm}
\setlength{\leftskip}{0.00mm}
\setlength{\rightskip}{0.00mm}

The General Assembly of the United Nations (GA) adopted the Universal Declaration of Human Rights (UDHR) in 1948. The UDHR was hailed as a ``\textit{a declaration of the basic principles to serve as a common standard for all nations}$^{}$''. The Declaration begins with declaring that ``all human beings are born free and equal in dignity and rights''$^{}$$^{ }$and that everyone is entitled to the rights set forth in the declaration$^{}$. Articles 3-20 concern the variety of CPR's$^{}$ contained in the UDHR while ESCR's were set forth in Articles 23-28$^{}$. 
\vspace{0.00mm}

\vspace{0.00mm}
\setlength{\parindent}{0.00mm}
\setlength{\leftskip}{0.00mm}
\setlength{\rightskip}{0.00mm}


\vspace{0.00mm}

\vspace{0.00mm}
\setlength{\parindent}{0.00mm}
\setlength{\leftskip}{0.00mm}
\setlength{\rightskip}{0.00mm}

The legal status of the UDHR has been questioned as the Declaration was not intended to be a binding legal document instead, it was envisaged at the time that the Conventions which were to follow the UDHR, would be of a binding legal character. Although the UDHR was not legally binding, the impact it had on the international law of human rights was phenomenal. The Declaration was adopted into the written constitutions of many states, it was also referred to in a number of other instruments including, other specific human rights agreements, regional agreements created to protect human rights and GA resolutions.  Many commentators have suggested that the principles contained in the UDHR have become customary international law and has been raised to the status of \textit{sui generis}$^{}$. 
\vspace{0.00mm}

\vspace{0.00mm}
\setlength{\parindent}{0.00mm}
\setlength{\leftskip}{0.00mm}
\setlength{\rightskip}{0.00mm}
\raggedright

\vspace{0.00mm}

\vspace{0.00mm}
\setlength{\parindent}{0.00mm}
\setlength{\leftskip}{0.00mm}
\setlength{\rightskip}{0.00mm}

The importance of the UDHR cannot be underestimated. It has provided the world with a common standard of rights afforded to all people, taking into account the diversity of opinions and thoughts throughout the globe. It placed international human rights law firmly on the agenda, alongside the specific rights and principles which TRTD would later encompass.
\vspace{0.00mm}

\vspace{0.00mm}
\setlength{\parindent}{0.00mm}
\setlength{\leftskip}{0.00mm}
\setlength{\rightskip}{0.00mm}


\vspace{0.00mm}

\vspace{0.00mm}
\setlength{\parindent}{0.00mm}
\setlength{\leftskip}{0.00mm}
\setlength{\rightskip}{0.00mm}

The GA indirectly made reference to the contents of what would later become TRTD. In 1957 it issued a resolution which endorseded the notion that integrated economic and social development contributed to the promotion of peace, security, social progress, better standards of living throughout the world and respect for human rights$^{}$. A crucial moment to the integration of development objectives and human rights occurred when a link between the two was identified in a UN report$^{}$.
\vspace{0.00mm}

\vspace{0.00mm}
\setlength{\parindent}{0.00mm}
\setlength{\leftskip}{0.00mm}
\setlength{\rightskip}{0.00mm}


\vspace{0.00mm}

\vspace{0.00mm}
\setlength{\parindent}{0.00mm}
\setlength{\leftskip}{0.00mm}
\setlength{\rightskip}{0.00mm}

The GA declared the 1960's as the \textit{United Nations Development Decade }and in 1965 adopted a resolution which recognised the need to promote human rights within the context of the development decade. These recognitions made by the GA are of significant importance as they lay the foundations for TRTD. 
\vspace{0.00mm}

\vspace{0.00mm}
\setlength{\parindent}{0.00mm}
\setlength{\leftskip}{0.00mm}
\setlength{\rightskip}{0.00mm}
\raggedright

\vspace{0.00mm}

\vspace{0.00mm}
\setlength{\parindent}{0.00mm}
\setlength{\leftskip}{0.00mm}
\setlength{\rightskip}{0.00mm}

The different ideologies of States were enhanced during the cold war and evidenced in discussions held by the CHR regarding the content of a covenant, which would have legally binding obligations and monitoring mechanisms for the human rights already established in the UDHR. During the discussions it became apparent that the ideologies of different nations would compromise the outcomes for the human rights discourse. Many of the western delegates argued that the inclusion of economic and social rights in one single legal covenant would lead to many countries restricting CPR's in the pursuit of economic stability. Whilst on the other side, delegates argued that a covenant only addressing CPR's completely disregarded economic exploitation and extreme poverty. Even the juridical nature of ESCR's was questioned, leading to arguments regarding justiciability and enforcement. Some delegates believed that ESCR's were not justiciable and were not capable of immediate implementation as they were a set of 'positive' obligations, requiring the state to 'to do something' in order to realise such rights. Whereas CPR's were seen as 'negative' obligations, only requiring the state to abstain from 'doing something' to realise CPR's. To solve such differences the original idea of one covenant was divided into two, one concerning CPR's and the other on ESCR's, a division not envisaged by the UDHR as all rights were seen as indivisible and interdependent. Both the ICCPR and ICESCR are afforded the status of an international treaty and thus all states who ratify the covenants are legally bound by their obligations under international law. 
\vspace{0.00mm}

\vspace{0.00mm}
\setlength{\parindent}{0.00mm}
\setlength{\leftskip}{0.00mm}
\setlength{\rightskip}{0.00mm}


\vspace{0.00mm}

\vspace{0.00mm}
\setlength{\parindent}{0.00mm}
\setlength{\leftskip}{0.00mm}
\setlength{\rightskip}{0.00mm}

The ICCPR expands on the rights provided for in the UDHR$^{}$ and provides mechanisms for enforcement. Article 28 established the Human Rights Committee, the body responsible for the enforcement of the covenant. Furthermore, Article 40 requires states to submit a report after one year of ratification and then every five years, regarding the progress made on realising the CPR's contained in the covenant. Article 1 of the first Optional Protocol 1966 provided individual victims with measures to seek redress for violations of CPR's from the HRC$^{}$. Moreover, the HRC can also seek information from other sources including Non- Governmental Organisations (NGO's). 
\vspace{0.00mm}

\vspace{0.00mm}
\setlength{\parindent}{0.00mm}
\setlength{\leftskip}{0.00mm}
\setlength{\rightskip}{0.00mm}


\vspace{0.00mm}

\vspace{0.00mm}
\setlength{\parindent}{0.00mm}
\setlength{\leftskip}{0.00mm}
\setlength{\rightskip}{0.00mm}

The ICESCR provides numerous provisions$^{}$ for the advancement of ESCR's broadening the scope of the UDHR. Articles 16 and 17 provide obligations for states to submit reports which will be considered by ECOSOC, reports can be obtained by other specialised agencies$^{}$ and can also be examined by the Commission on Human Rights$^{}$.  However, the ICESCR has been severely criticised for a number of reasons including criticisms regarding the 'vague' language used throughout the document; implementation differs from one covenant to the other thus seeming to afford more protection to one set of rights over another in contradiction with the aims of the human rights discourse; state obligations differ in the covenants; progressive implementation of provisions makes it difficult to determine who has fulfilled their obligations and the covenants provisions are too 'aspirational' and can never be achieved in full $^{}$. Nevertheless, the ICESCR provides the only framework in the international arena in which ESCR's are declared and protected human rights$^{}$. Furthermore it provides indirect references to the contents of TRTD.
\vspace{0.00mm}

\vspace{0.00mm}
\setlength{\parindent}{0.00mm}
\setlength{\leftskip}{0.00mm}
\setlength{\rightskip}{0.00mm}


\vspace{0.00mm}

\vspace{0.00mm}
\setlength{\parindent}{0.00mm}
\setlength{\leftskip}{0.00mm}
\setlength{\rightskip}{0.00mm}

The consequences of dividing CPR's and ESCR's was seen throughout the subsequent decades and in other human rights instruments. Three generations of rights emerged$^{}$. First generation rights consisted of CPR's, 'freedom' rights; second generation rights were ESR's, 'equality' rights and a further generation emerged as third generation rights, rights of 'solidarity'$^{}$ encompassing both CPR's and ESCR's, which included amongst others the right to a healthy environment, the right to peace and the TRTD.      
\vspace{0.00mm}

\vspace{0.00mm}
\setlength{\parindent}{0.00mm}
\setlength{\leftskip}{0.00mm}
\setlength{\rightskip}{0.00mm}


\vspace{0.00mm}
\begin{itemize}
\begin{itemize}
\begin{itemize}

\item
\vspace{4.17mm}
\setlength{\parindent}{0.00mm}
\setlength{\leftskip}{0.00mm}
\setlength{\rightskip}{0.00mm}
\raggedright
\textbf{1.1.1. Towards Integrating Development and Human Rights}
\vspace{2.08mm}

\end{itemize}
\end{itemize}
\end{itemize}
\vspace{0.00mm}
\setlength{\parindent}{0.00mm}
\setlength{\leftskip}{0.00mm}
\setlength{\rightskip}{0.00mm}


\vspace{0.00mm}

\vspace{0.00mm}
\setlength{\parindent}{0.00mm}
\setlength{\leftskip}{0.00mm}
\setlength{\rightskip}{0.00mm}

Many factors were accountable to the increase of attention to rights within the development discourse.  The end of both the world wars gave rise to an increase of States, either newly established or the result of decolonisation, who's primary concerns included economic development  in the international sphere.  Developing countries contended that it was necessary to afford more attention to economic and social rights and furthermore, that colonialism and neocolonialism constituted gross violations of international law and as such development cooperation was required between the developed and developing countries$^{}$. Moreover, the developing countries suggested that developed countries had a legal obligation to make reparations, either by transfers of technology, capital or other forms of goods and services$^{}$. They believed they were entitled to such transfers and cooperation and they should not be perceived as charity, aid or welfare$^{}$. The Arab oil embargo of 1973 was another factor which increased attention as it further intensified North-South relations. Such events led to the search for a 'New International Economic Order' (NIEO), which according to the developing nations was a search for equality and justice for all states and its peoples.  
\vspace{0.00mm}

\vspace{0.00mm}
\setlength{\parindent}{0.00mm}
\setlength{\leftskip}{0.00mm}
\setlength{\rightskip}{0.00mm}


\vspace{0.00mm}

\vspace{0.00mm}
\setlength{\parindent}{0.00mm}
\setlength{\leftskip}{0.00mm}
\setlength{\rightskip}{0.00mm}

It also became evident that the international community was dissatisfied with the separation of human rights with the adoption of two covenants. In 1968 the Proclamation of Tehran  expressed that ``\textit{since all human rights and fundamental freedoms are indivisible, the full realisation of civil and political rights without the enjoyment of ESCR's is impossible.}\textit{$^{}$}'' The Declaration on Social Progress and Development 1969$^{}$ emphasised the interdependence of all human rights and by the 1970's, the notion of a 'right to development' surfaced.  
\vspace{0.00mm}

\vspace{0.00mm}
\setlength{\parindent}{0.00mm}
\setlength{\leftskip}{0.00mm}
\setlength{\rightskip}{0.00mm}


\vspace{0.00mm}

\vspace{0.00mm}
\setlength{\parindent}{0.00mm}
\setlength{\leftskip}{0.00mm}
\setlength{\rightskip}{0.00mm}

The first direct references to TRTD can be attributed to the developing countries in the context of the NIEO. In 1966 the Senegalese Foreign Minister at a GA meeting expressed,
\vspace{0.00mm}

\vspace{0.00mm}
\setlength{\parindent}{0.00mm}
\setlength{\leftskip}{0.00mm}
\setlength{\rightskip}{0.00mm}


\vspace{0.00mm}

\vspace{0.00mm}
\setlength{\parindent}{0.00mm}
\setlength{\leftskip}{9.84mm}
\setlength{\rightskip}{9.84mm}

``Not only must we affirm our right to development, but we must also take the steps which will enable this right to become a reality. We must build a new system, based not only on the theoretical affirmation of the sacred rights of people and nations but on the actual enjoyment of these rights.$^{}$'' 
\vspace{4.91mm}

\vspace{0.00mm}
\setlength{\parindent}{0.00mm}
\setlength{\leftskip}{9.84mm}
\setlength{\rightskip}{9.84mm}


\vspace{4.91mm}

\vspace{0.00mm}
\setlength{\parindent}{0.00mm}
\setlength{\leftskip}{0.00mm}
\setlength{\rightskip}{0.00mm}

In 1972 at the UN Conference for Trade and Development (UNCTAD III), in Chile, TRTD was claimed by many of the developing nations. This notion was elaborated by the Sengalese jurist Keba M'baye who advocated extensively on the existence of the right and Karel Vasak who advocated on behalf of third generation solidarity rights. By 1977 the CHR issued a resolution calling upon the Secretary General to examine and prepare a study on the human right to development and its relationship with other human rights and the NIEO. It was evident that the study was not to see whether the right existed, this was implicitly recognised in the resolution. The Declaration on Race and Racial Prejudice adopted in 1978 by UNESCO referred to ``\textit{the right of every human being and group to full development''}$^{}$. The GA Declaration on the Preparation of Societies for Life in Peace 1978 also stated that everyone has the right ``\textit{to determine the road of their development''}$^{}$. By 1979, having considered the suggestions made by the Secretary General, the CHR issued a further resolution recognising TRTD$^{}$. Whilst a working group was established to prepare a draft declaration on TRTD, The Banjul Charter$^{}$gave TRTD legal recognition for the first time. In 1986, TDRTD was adopted by the GA. 
\vspace{0.00mm}

\vspace{0.00mm}
\setlength{\parindent}{0.00mm}
\setlength{\leftskip}{0.00mm}
\setlength{\rightskip}{0.00mm}


\vspace{0.00mm}
\begin{itemize}
\begin{itemize}
\begin{itemize}

\item
\vspace{4.17mm}
\setlength{\parindent}{0.00mm}
\setlength{\leftskip}{0.00mm}
\setlength{\rightskip}{0.00mm}
\raggedright
\textbf{1.1.2. From the Declaration on the Right to Development and Beyond. }
\vspace{2.08mm}

\end{itemize}
\end{itemize}
\end{itemize}
\vspace{0.00mm}
\setlength{\parindent}{0.00mm}
\setlength{\leftskip}{0.00mm}
\setlength{\rightskip}{0.00mm}


\vspace{0.00mm}

\vspace{0.00mm}
\setlength{\parindent}{0.00mm}
\setlength{\leftskip}{0.00mm}
\setlength{\rightskip}{0.00mm}

While TDRTD framed TRTD as a human right, this did not bring a closure to the controversies. TDRTD was adopted after several years of negotiations between the North and the South - or developed and developing countries - and as such the text of the declaration was constructed with heavy compromises. TDRTD was unable to escape the uneasy political atmosphere which was present during its negotiations and, although it was adopted by consensus, with 146 votes for the declaration, there was a notable negative vote made by the United States. There were also eight abstentions made by Denmark, Finland, The Federal Republic of Germany, Iceland, Israel, Japan, Sweden and the United Kingdom. The underlying reasons for the vote against TDRTD and the abstentions reveals the discontentment surrounding the nature of TRTD. Objections to TDRTD included: the possibility that the declaration would lead to an erosion of individual rights$^{}$; TRTD could be used to legitimise violations of the rights of citizens$^{}$;  TDRTD did not make a distinction between individual rights and group rights$^{}$; the human rights discourse tended to become diluted and confused with the addition of TRTD$^{}$; the declaration provided a mistaken link between human rights and a NIEO, and it did not take into account the complex relationship between development, security and disarmament - but rather provided a very simplistic view of the relationship$^{}$.     
\vspace{0.00mm}

\vspace{0.00mm}
\setlength{\parindent}{0.00mm}
\setlength{\leftskip}{0.00mm}
\setlength{\rightskip}{0.00mm}
\raggedright

\vspace{0.00mm}

\vspace{0.00mm}
\setlength{\parindent}{0.00mm}
\setlength{\leftskip}{0.00mm}
\setlength{\rightskip}{0.00mm}

As a result of such objections, the definition and the content of TRTD have been the crux of debates surrounding the right. The vague language adopted by TDRTD reflects the complexity of TRTD and the demands for political compromises.$^{}$
\vspace{0.00mm}

\vspace{0.00mm}
\setlength{\parindent}{0.00mm}
\setlength{\leftskip}{0.00mm}
\setlength{\rightskip}{0.00mm}
\raggedright

\vspace{0.00mm}

\vspace{0.00mm}
\setlength{\parindent}{0.00mm}
\setlength{\leftskip}{0.00mm}
\setlength{\rightskip}{0.00mm}

A Working Group of Governmental Experts$^{}$ on TRTD failed to provide any concrete recommendations with regards to the nature of the right and its implementation. Thus in 1989 the UN Commission on Human Rights$^{}$ asked the Secretary-General to organise a 'Global Consultation' involving representatives from all arenas including the UN, its specialised agencies, governments and non-governmental organisations, to ``\textit{focus on the fundamental problems faced by the implementation of the Declaration, the criteria which might be used to identify progress and the possible mechanisms for evaluating such progress}\textit{$^{}$}\textit{.'' }
\vspace{0.00mm}

\vspace{0.00mm}
\setlength{\parindent}{0.00mm}
\setlength{\leftskip}{0.00mm}
\setlength{\rightskip}{0.00mm}


\vspace{0.00mm}

\vspace{0.00mm}
\setlength{\parindent}{0.00mm}
\setlength{\leftskip}{0.00mm}
\setlength{\rightskip}{0.00mm}

The affirmation of TRTD in the 1993 Vienna Declaration and Programme of action as ``\textit{a universal an inalienable human right and an integral part of fundamental human rights}'' was of vital significance. Although the international community had not fully reached consensus on the actual definition of the right and its content, it nonetheless affirmed its existence as a  human right, with the USA in favour. But the problem remained and the need to elaborate on TDRTD proved to be the catalyst for the creation of further working groups on the matter$^{}$. 
\vspace{0.00mm}

\vspace{0.00mm}
\setlength{\parindent}{0.00mm}
\setlength{\leftskip}{0.00mm}
\setlength{\rightskip}{0.00mm}


\vspace{0.00mm}

\vspace{0.00mm}
\setlength{\parindent}{0.00mm}
\setlength{\leftskip}{0.00mm}
\setlength{\rightskip}{0.00mm}

In 1997, the GA endorseded that the inclusion of TRTD in the International Bill of Rights would be an appropriate means of celebrating the 50$^{th}$ anniversary of the UDHR$^{}$. The CHR later recognised that the celebrations were an important opportunity for all human rights, and in particular TRTD, to be placed at the top of the agenda for the whole international community. After many years of discussions and debate, it was evident that, even at this stage, consensus had not been completely reached. Many States still had their reservations while others purported TRTD to be something which it was not. Thus, the Commission recommended the establishment of a further working group, the Open Ended Working Group (OEWG) and an Independent Expert (IE). The OEWG was to monitor and review the progress made by the IE and report its conclusions and findings to the commission. The mandate of the IE, Arjun Sengupta$^{}$, was to ``\textit{present to the working group at each of its sessions a study on the current state of progress in the implementation of the TRTD as a basis for a focused discussion, taking into account, inter alia, the deliberations and suggestions of the working group''.}$^{}$ One of the first tasks faced by the IE was to clarify the definition and content of TRTD, a highly controversial area in which many different nations had their own interpretation. 
\vspace{0.00mm}

\vspace{0.00mm}
\setlength{\parindent}{0.00mm}
\setlength{\leftskip}{0.00mm}
\setlength{\rightskip}{0.00mm}


\vspace{0.00mm}

\vspace{0.00mm}
\setlength{\parindent}{0.00mm}
\setlength{\leftskip}{0.00mm}
\setlength{\rightskip}{0.00mm}

The precise nature of the right has proved elusive. Many issues have been brought forward such as; what kind of right it is; who are the right holders and duty bearers;  and what forms of obligations are imposed and to whom. These issues are the crux of contention surrounding TRTD and once resolved, if they can be resolved, will identify the true nature of TRTD and contribute to the identification of whether there exists a legal obligation for the international community to realise TRTD. 
\vspace{0.00mm}

\vspace{0.00mm}
\setlength{\parindent}{0.00mm}
\setlength{\leftskip}{0.00mm}
\setlength{\rightskip}{0.00mm}

\textbf{}
\vspace{0.00mm}
\begin{itemize}
\begin{itemize}

\item
\vspace{4.17mm}
\setlength{\parindent}{0.00mm}
\setlength{\leftskip}{0.00mm}
\setlength{\rightskip}{0.00mm}
\raggedright
\textbf{\textit{}}
\vspace{2.08mm}

\end{itemize}
\end{itemize}
\vspace{0.00mm}
\setlength{\parindent}{0.00mm}
\setlength{\leftskip}{0.00mm}
\setlength{\rightskip}{0.00mm}
\raggedright
\newpage

\vspace{0.00mm}
\begin{itemize}

\item
\vspace{4.17mm}
\setlength{\parindent}{0.00mm}
\setlength{\leftskip}{0.00mm}
\setlength{\rightskip}{0.00mm}
\raggedright
\textbf{2. The Definition and Content of TRTD}
\vspace{2.08mm}

\end{itemize}
\vspace{0.00mm}
\setlength{\parindent}{0.00mm}
\setlength{\leftskip}{0.00mm}
\setlength{\rightskip}{0.00mm}


\vspace{0.00mm}

\vspace{0.00mm}
\setlength{\parindent}{0.00mm}
\setlength{\leftskip}{0.00mm}
\setlength{\rightskip}{0.00mm}

The non-controversial aspects of TRTD\textbf{ }are easily established from TDRTD and these are generally agreed upon by the international community. This chapter will first summarise the non-controversial aspects of TRTD. The chapter will then address each of the controversies surrounding TRTD. This chapter aims to determine the contours of  TRTD and what it entails. 
\vspace{0.00mm}

\vspace{0.00mm}
\setlength{\parindent}{0.00mm}
\setlength{\leftskip}{0.00mm}
\setlength{\rightskip}{0.00mm}


\vspace{0.00mm}
\begin{itemize}
\begin{itemize}

\item
\vspace{4.17mm}
\setlength{\parindent}{0.00mm}
\setlength{\leftskip}{0.00mm}
\setlength{\rightskip}{0.00mm}
\raggedright
\textbf{2.1. The Non-Controversial Aspects of the Right to Development}
\vspace{2.08mm}

\end{itemize}
\end{itemize}
\vspace{0.00mm}
\setlength{\parindent}{0.00mm}
\setlength{\leftskip}{0.00mm}
\setlength{\rightskip}{0.00mm}
\raggedright
\textbf{}
\vspace{2.08mm}

\vspace{0.00mm}
\setlength{\parindent}{0.00mm}
\setlength{\leftskip}{0.00mm}
\setlength{\rightskip}{0.00mm}

This section will summarise the non-controversial aspects of TRTD including the principles of universality and indivisibility; non-discrimination and equality of opportunity; TRTD as a right to participate in development; accountability and transparency and finally equity and justice.   
\vspace{0.00mm}

\vspace{0.00mm}
\setlength{\parindent}{0.00mm}
\setlength{\leftskip}{0.00mm}
\setlength{\rightskip}{0.00mm}


\vspace{0.00mm}
\begin{itemize}
\begin{itemize}
\begin{itemize}

\item
\vspace{4.17mm}
\setlength{\parindent}{0.00mm}
\setlength{\leftskip}{0.00mm}
\setlength{\rightskip}{0.00mm}
\raggedright
\textbf{2.1.1. Universality and }\textbf{Indivisibility}
\vspace{2.08mm}

\end{itemize}
\end{itemize}
\end{itemize}
\vspace{0.00mm}
\setlength{\parindent}{0.00mm}
\setlength{\leftskip}{0.00mm}
\setlength{\rightskip}{0.00mm}


\vspace{0.00mm}

\vspace{0.00mm}
\setlength{\parindent}{0.00mm}
\setlength{\leftskip}{0.00mm}
\setlength{\rightskip}{0.00mm}

The principles of universality$^{}$ and indivisibility$^{}$ are of crucial significance to TRTD. Universality is a key principle since, if TRTD is a human right, it must be applicable to everyone. The notion of universality has been questioned in a much larger debate concerning universalism and cultural relativism$^{}$. Nonetheless, it is evident that TRTD can be applicable to everyone regardless of cultural background, gender etc. Article 1(1) defines the right as a human right belonging to every person and peoples, who are all entitled to participate in, contribute to, and enjoy TRTD. It is not just a right of individuals and peoples in developing countries; it can easily be applied to people in developed countries as well.    
\vspace{0.00mm}

\vspace{0.00mm}
\setlength{\parindent}{0.00mm}
\setlength{\leftskip}{0.00mm}
\setlength{\rightskip}{0.00mm}


\vspace{0.00mm}

\vspace{0.00mm}
\setlength{\parindent}{0.00mm}
\setlength{\leftskip}{0.00mm}
\setlength{\rightskip}{0.00mm}

Indivisibility is also crucial as TDRTD emphasises that all rights are interrelated since one right cannot be enjoyed in isolation; its enjoyment is dependent on other rights and thus all rights should be given equal attention. This principle is concisely articulated in Article 6(2) TDRTD and leaves no room for interpretation. 
\vspace{0.00mm}

\vspace{0.00mm}
\setlength{\parindent}{0.00mm}
\setlength{\leftskip}{0.00mm}
\setlength{\rightskip}{0.00mm}


\vspace{0.00mm}

\vspace{0.00mm}
\setlength{\parindent}{0.00mm}
\setlength{\leftskip}{0.00mm}
\setlength{\rightskip}{0.00mm}

The IE clearly articulates the principles of universality and indivisibility when he states;  
\vspace{0.00mm}

\vspace{0.00mm}
\setlength{\parindent}{0.00mm}
\setlength{\leftskip}{0.00mm}
\setlength{\rightskip}{0.00mm}


\vspace{0.00mm}

\vspace{0.00mm}
\setlength{\parindent}{0.00mm}
\setlength{\leftskip}{9.84mm}
\setlength{\rightskip}{9.84mm}

``[H]uman rights must conform to the principles of \textit{universality} and \textit{indivisibility. }Universality implies that every individual is endowed with human rights, by virtue of being human, irrespective of their cultural background or citizenship.[...] Indivisibility is associated with the principle of interdependence. Two rights are indivisible if one cannot be enjoyed if the other is violated. Two rights are interdependent if the level of enjoyment of one is dependent on the level of enjoyment of the other.''$^{}$ 
\vspace{4.91mm}

\vspace{0.00mm}
\setlength{\parindent}{0.00mm}
\setlength{\leftskip}{9.84mm}
\setlength{\rightskip}{9.84mm}


\vspace{4.91mm}

\vspace{0.00mm}
\setlength{\parindent}{0.00mm}
\setlength{\leftskip}{-0.35mm}
\setlength{\rightskip}{-0.17mm}

The IE's articulation of the principles of universality and indivisibility is highly commendable as it's direct and precise, clearly identifying what the principles are and how they apply to TRTD. However, Sengupta may be accused of inferring far more depth into the principles than those agreed within TDRTD and perhaps exaggerates their meanings as he continues:
\vspace{0.00mm}

\vspace{0.00mm}
\setlength{\parindent}{0.00mm}
\setlength{\leftskip}{-0.35mm}
\setlength{\rightskip}{-0.17mm}


\vspace{0.00mm}

\vspace{0.00mm}
\setlength{\parindent}{0.00mm}
\setlength{\leftskip}{9.84mm}
\setlength{\rightskip}{9.84mm}

``The obligations related to such rights are also universal, to be implemented to the best of their possibility by all agents who are in a position to help, whether they are State authorities or others belonging to the same country or other States and international organisations.$^{}$.''
\vspace{4.91mm}

\vspace{0.00mm}
\setlength{\parindent}{0.00mm}
\setlength{\leftskip}{9.84mm}
\setlength{\rightskip}{9.84mm}


\vspace{4.91mm}

\vspace{0.00mm}
\setlength{\parindent}{0.00mm}
\setlength{\leftskip}{-0.35mm}
\setlength{\rightskip}{-0.17mm}

Sengupta suggests that, the implications of the principle of universality make the obligations related to such rights also universal. As will be discussed below, the issue of duty-bearers is still very controversial. Thus Sengupta's notion of ``\textit{all agents in a position to help}'' has not been agreed upon by the international community. Likewise, the obligations of international organisations and individuals relating to TRTD are also still unclear, and thus the IE has perhaps interpreted the notion too widely. Having noted this, the IE's mandate was to clarify the content of TRTD and in this light the concepts identified by him provide a useful background to what the right aims towards. 
\vspace{0.00mm}

\vspace{0.00mm}
\setlength{\parindent}{0.00mm}
\setlength{\leftskip}{-0.35mm}
\setlength{\rightskip}{-0.17mm}


\vspace{0.00mm}

\vspace{0.00mm}
\setlength{\parindent}{0.00mm}
\setlength{\leftskip}{0.00mm}
\setlength{\rightskip}{0.00mm}

  
\vspace{0.00mm}
\begin{itemize}
\begin{itemize}
\begin{itemize}

\item
\vspace{4.17mm}
\setlength{\parindent}{0.00mm}
\setlength{\leftskip}{0.00mm}
\setlength{\rightskip}{0.00mm}
\raggedright
\textbf{2.1.2. Non-Discrimination and Equality of Opportunity}
\vspace{2.08mm}

\end{itemize}
\end{itemize}
\end{itemize}
\vspace{0.00mm}
\setlength{\parindent}{0.00mm}
\setlength{\leftskip}{0.00mm}
\setlength{\rightskip}{0.00mm}


\vspace{0.00mm}

\vspace{0.00mm}
\setlength{\parindent}{0.00mm}
\setlength{\leftskip}{0.00mm}
\setlength{\rightskip}{0.00mm}

A common feature of all human rights documents is the emphasis placed on the principle of non-discrimination. This is a fairly straightforward principle, intended to protect people from various forms of discrimination. TDRTD is no exception and  the principle of non-discrimination is clearly conceptualised in Article 6$^{}$. Nevertheless, the formulation of the principle in TDRTD could have carried greater force, as Article 5 of TDRTD uses the term ``shall'' implying that states must act, the term ``should'' used in Article 6, by comparison, could now be construed as less powerful and less obligatory. Furthermore, both the ICCPR and the ICESCR convey the principle in more obligatory forms$^{}$.
\vspace{0.00mm}

\vspace{0.00mm}
\setlength{\parindent}{0.00mm}
\setlength{\leftskip}{0.00mm}
\setlength{\rightskip}{0.00mm}

\textcolor{red}{{}}
\vspace{0.00mm}

\vspace{0.00mm}
\setlength{\parindent}{0.00mm}
\setlength{\leftskip}{0.00mm}
\setlength{\rightskip}{0.00mm}

\textcolor{red}{{}}
\vspace{0.00mm}

\vspace{0.00mm}
\setlength{\parindent}{0.00mm}
\setlength{\leftskip}{0.00mm}
\setlength{\rightskip}{0.00mm}

TDRTD affirms the principle of equality for all its beneficiaries by stating that  development policies in line with TRTD are aimed ``\textit{at the constant improvement of the well-being of}'' not only ``\textit{individuals}'' but also the ``\textit{entire population''}. Article 8(1) underscores the need for states to ensure equal opportunity for everyone to access ``\textit{basic resources, education, health services, food, housing, employment, and the fair distribution of income}''. Women have been highlighted specifically and they should be able to actively participate in the development process. The role of women has been emphasised in subsequent GA resolutions as ``\textit{the empowerment of women and their full participation on a basis of equality in all spheres of society is fundamental for development}''$^{}$.
\vspace{0.00mm}

\vspace{0.00mm}
\setlength{\parindent}{0.00mm}
\setlength{\leftskip}{0.00mm}
\setlength{\rightskip}{0.00mm}

\textbf{}
\vspace{0.00mm}
\begin{itemize}
\begin{itemize}
\begin{itemize}

\item
\vspace{4.17mm}
\setlength{\parindent}{0.00mm}
\setlength{\leftskip}{0.00mm}
\setlength{\rightskip}{0.00mm}
\raggedright
\textbf{2.1.3. The Right to Development as a Right to Participate in Development}
\vspace{2.08mm}

\end{itemize}
\end{itemize}
\end{itemize}
\vspace{0.00mm}
\setlength{\parindent}{0.00mm}
\setlength{\leftskip}{0.00mm}
\setlength{\rightskip}{0.00mm}

\textbf{}
\vspace{0.00mm}

\vspace{0.00mm}
\setlength{\parindent}{0.00mm}
\setlength{\leftskip}{0.00mm}
\setlength{\rightskip}{0.00mm}

One of the main themes of TRTD as formulated in TDRTD is that of `participation'. There is a distinct emphasis on the principle of participation throughout TDRTD and it is mentioned in Articles 1, 2(1), 2(3) and Article 8. The interpretation of TRTD as a right to participate in development has been described by the Global Consultation as ``\textit{the right through which all other rights in the declaration on the right to development are exercised and protected''}\textit{$^{}$}.  
\vspace{0.00mm}

\vspace{0.00mm}
\setlength{\parindent}{0.00mm}
\setlength{\leftskip}{0.00mm}
\setlength{\rightskip}{0.00mm}


\vspace{0.00mm}

\vspace{0.00mm}
\setlength{\parindent}{0.00mm}
\setlength{\leftskip}{0.00mm}
\setlength{\rightskip}{0.00mm}

The principle of participation has been described by Bunn as;
\vspace{0.00mm}

\vspace{0.00mm}
\setlength{\parindent}{0.00mm}
\setlength{\leftskip}{0.00mm}
\setlength{\rightskip}{0.00mm}


\vspace{0.00mm}

\vspace{0.00mm}
\setlength{\parindent}{0.00mm}
\setlength{\leftskip}{9.84mm}
\setlength{\rightskip}{9.84mm}

``an on-going process at the local, regional, national, and international levels. It is the primary mechanism for setting goals toward the realisation of the right to development, and for assuring the compatibility of development activities with human rights and cultural values. Participation is also important in evaluating progress toward the realisation of the right.$^{}$''
\vspace{4.91mm}

\vspace{0.00mm}
\setlength{\parindent}{0.00mm}
\setlength{\leftskip}{9.84mm}
\setlength{\rightskip}{9.84mm}


\vspace{4.91mm}

\vspace{0.00mm}
\setlength{\parindent}{0.00mm}
\setlength{\leftskip}{0.00mm}
\setlength{\rightskip}{0.00mm}

As noted by Bunn, participation is of crucial importance to the realisation of TRTD. People who are affected by any development processes must be able to participate in the decisions which affect their lives. As the individual is the central subject of development, all policies concerning such development should be accessible to that individual. The type of participation is also of relevance. It is not sufficient inform people of the policies being implemented; participation, in TRTD perspective, requires that people must have control over the process. This is clearly articulated by the Human Rights Council of Australia, who extrapolate the human rights meaning of participation as; 
\vspace{0.00mm}

\vspace{0.00mm}
\setlength{\parindent}{0.00mm}
\setlength{\leftskip}{0.00mm}
\setlength{\rightskip}{0.00mm}


\vspace{0.00mm}

\vspace{0.00mm}
\setlength{\parindent}{0.00mm}
\setlength{\leftskip}{9.84mm}
\setlength{\rightskip}{9.84mm}

``Participation understood as control cannot easily be confused with `involved', `consulted', `empowered' or even `ownership'. To ask who has control, authority, direction over a particular aspect of the development program is a much tougher question than to ask who is involved or empowered by it. It also leads to significantly more meaningful answers.$^{}$''
\vspace{4.91mm}

\vspace{0.00mm}
\setlength{\parindent}{0.00mm}
\setlength{\leftskip}{0.00mm}
\setlength{\rightskip}{0.00mm}

In contrast, the IE view of participation is that it empowers people;  
\vspace{0.00mm}

\vspace{0.00mm}
\setlength{\parindent}{0.00mm}
\setlength{\leftskip}{0.00mm}
\setlength{\rightskip}{0.00mm}


\vspace{0.00mm}

\vspace{0.00mm}
\setlength{\parindent}{0.00mm}
\setlength{\leftskip}{9.84mm}
\setlength{\rightskip}{9.84mm}

``Participation is concerned with access to decision-making and the exercise of power in the execution of projects which lead to the programme of development. This means that citizens have to be empowered and have ownership of the programme.$^{}$''
\vspace{4.91mm}

\vspace{0.00mm}
\setlength{\parindent}{0.00mm}
\setlength{\leftskip}{9.84mm}
\setlength{\rightskip}{9.84mm}


\vspace{4.91mm}

\vspace{0.00mm}
\setlength{\parindent}{0.00mm}
\setlength{\leftskip}{0.00mm}
\setlength{\rightskip}{0.00mm}

At first sight it seems as though both interpretations of the meaning of participation are similar. Rather, on close analysis, the Human Rights Council of Australia provide a more complex view which in fact develops that of the IE's. Perhaps the view of the Human Rights Council is more appropriate for TRTD empowerment of people to be involved - to have their say - is relevant, but to actually have control over the decision making brings a new dimension to the notion. This  gives people the \textit{authority} to act according to their judgment on a given situation. For the purpose of illustration lets say a development project has aspects which are incompatible with their views; in this case they will be able to act to remedy the situation with relative ease. Thus participation is vital to the realisation of TRTD, particularly if it is seen as a form of control. 
\vspace{0.00mm}

\vspace{0.00mm}
\setlength{\parindent}{0.00mm}
\setlength{\leftskip}{0.00mm}
\setlength{\rightskip}{0.00mm}


\vspace{0.00mm}
\begin{itemize}
\begin{itemize}
\begin{itemize}

\item
\vspace{4.17mm}
\setlength{\parindent}{0.00mm}
\setlength{\leftskip}{0.00mm}
\setlength{\rightskip}{0.00mm}
\raggedright
\textbf{2.1.4. Accountability and Transparency }
\vspace{2.08mm}

\end{itemize}
\end{itemize}
\end{itemize}
\vspace{0.00mm}
\setlength{\parindent}{0.00mm}
\setlength{\leftskip}{0.00mm}
\setlength{\rightskip}{0.00mm}


\vspace{0.00mm}

\vspace{0.00mm}
\setlength{\parindent}{0.00mm}
\setlength{\leftskip}{0.00mm}
\setlength{\rightskip}{0.00mm}

The concepts of accountability and transparency are not specifically addressed by TDRTD. However, the General Secretary has identified a connection between accountability and the assessment of activities with respect to promoting human rights within development as early as 1980$^{}$.  
\vspace{0.00mm}

\vspace{0.00mm}
\setlength{\parindent}{0.00mm}
\setlength{\leftskip}{0.00mm}
\setlength{\rightskip}{0.00mm}


\vspace{0.00mm}

\vspace{0.00mm}
\setlength{\parindent}{0.00mm}
\setlength{\leftskip}{0.00mm}
\setlength{\rightskip}{0.00mm}

The concept of accountability infers notions of responsibility and answerability and is thus quite important for the realisation of TRTD. The duty holder of the right will be responsible to realise their obligations; if they do not fulfill their duties they can be identified and held accountable. The IE's descriptions of accountability and transparency convey the vital role they play toward the realisation of TRTD;    
\vspace{0.00mm}

\vspace{0.00mm}
\setlength{\parindent}{0.00mm}
\setlength{\leftskip}{0.00mm}
\setlength{\rightskip}{0.00mm}


\vspace{0.00mm}

\vspace{0.00mm}
\setlength{\parindent}{0.00mm}
\setlength{\leftskip}{9.84mm}
\setlength{\rightskip}{9.84mm}

``Accountability involves establishing responsibility for the non-fulfillment of a right and applying corrective measures. Any development programme must identify duty-holders and their obligations.
\vspace{4.91mm}

\vspace{0.00mm}
\setlength{\parindent}{0.00mm}
\setlength{\leftskip}{9.84mm}
\setlength{\rightskip}{9.84mm}

Transparency is essential for ensuring accountability because programmes must be designed in such a manner as to bring out openly all the interrelations and linkages between different actions and actors.''$^{}$ 
\vspace{4.91mm}

\vspace{0.00mm}
\setlength{\parindent}{0.00mm}
\setlength{\leftskip}{9.84mm}
\setlength{\rightskip}{9.84mm}


\vspace{4.91mm}

\vspace{0.00mm}
\setlength{\parindent}{0.00mm}
\setlength{\leftskip}{0.00mm}
\setlength{\rightskip}{0.00mm}

Unfortunately, identifying the right holders and duty bearers of TRTD remains controversial as will be seen below.\textbf{ }
\vspace{0.00mm}

\vspace{0.00mm}
\setlength{\parindent}{0.00mm}
\setlength{\leftskip}{0.00mm}
\setlength{\rightskip}{0.00mm}

\textbf{}
\vspace{0.00mm}
\begin{itemize}
\begin{itemize}
\begin{itemize}

\item
\vspace{4.17mm}
\setlength{\parindent}{0.00mm}
\setlength{\leftskip}{0.00mm}
\setlength{\rightskip}{0.00mm}
\raggedright
\textbf{2.1.5. Equity and Justice }
\vspace{2.08mm}

\end{itemize}
\end{itemize}
\end{itemize}
\vspace{0.00mm}
\setlength{\parindent}{0.00mm}
\setlength{\leftskip}{0.00mm}
\setlength{\rightskip}{0.00mm}

\textbf{}
\vspace{0.00mm}

\vspace{0.00mm}
\setlength{\parindent}{0.00mm}
\setlength{\leftskip}{0.00mm}
\setlength{\rightskip}{0.00mm}

The principles of equity and justice are quite complex and lie beyond the scope of this paper, but a brief summary of the principles which are contained in TDRTD will now be presented. TDRTD stipulates that all individuals are entitled to ``the fair distribution of the benefits$^{}$'' resulting from development. According to Article 8(1) states must also ensure equal access to basic resources, education and health services, amongst others$^{}$. The IE suggests that;
\vspace{0.00mm}

\vspace{0.00mm}
\setlength{\parindent}{0.00mm}
\setlength{\leftskip}{0.00mm}
\setlength{\rightskip}{0.00mm}


\vspace{0.00mm}

\vspace{0.00mm}
\setlength{\parindent}{0.00mm}
\setlength{\leftskip}{9.84mm}
\setlength{\rightskip}{9.84mm}

``Equity is reflected in policies that are aimed at ensuring the equitable distribution of benefits and that target the most vulnerable and marginalised segments of society.$^{}$''
\vspace{4.91mm}

\vspace{0.00mm}
\setlength{\parindent}{0.00mm}
\setlength{\leftskip}{9.84mm}
\setlength{\rightskip}{9.84mm}


\vspace{4.91mm}

\vspace{0.00mm}
\setlength{\parindent}{0.00mm}
\setlength{\leftskip}{0.00mm}
\setlength{\rightskip}{0.00mm}

Article 8(1) specifically addresses the notion of ``social injustices'', focusing on their eradication, and stipulates that states should carry out both economic and social reforms. This Article is of considerable importance as some of the key values of TRTD can be extrapolated from it. In the most concise terms, Bunn describes;
\vspace{0.00mm}

\vspace{0.00mm}
\setlength{\parindent}{0.00mm}
\setlength{\leftskip}{0.00mm}
\setlength{\rightskip}{0.00mm}


\vspace{0.00mm}

\vspace{0.00mm}
\setlength{\parindent}{0.00mm}
\setlength{\leftskip}{9.84mm}
\setlength{\rightskip}{9.84mm}

``Although there is only one reference to ``injustice'' within the text of the UNDR[T]D (Article 8.1), it is nevertheless clear that much of the Declaration's underlying intention is aimed precisely at eliminating it. While international social and economic justice is an entire topic in itself, the right to development can help provide a vision for the international order. The right seeks to address complex issues of equality, fairness, distribution of benefits and burdens within and among societies. The right focuses on solving problems of exploitation and oppression, fulfilling basis human needs, and maintaining respect for all persons. It serves as a reminder of our responsibility to future generations. Finally, on a global scale, it raises the classic question of justice: how to render to each person what is due.$^{}$''    
\vspace{4.91mm}

\vspace{0.00mm}
\setlength{\parindent}{0.00mm}
\setlength{\leftskip}{0.00mm}
\setlength{\rightskip}{0.00mm}


\vspace{0.00mm}

\vspace{0.00mm}
\setlength{\parindent}{0.00mm}
\setlength{\leftskip}{0.00mm}
\setlength{\rightskip}{0.00mm}

Thus, TDRTD encompasses an array of different principles, each one as important as the next, and all are crucial to the effective realisation of TRTD, which seeks to address a variety of concerns at the international and national levels and, ultimately, eradicate injustices of different kinds. 
\vspace{0.00mm}

\vspace{0.00mm}
\setlength{\parindent}{0.00mm}
\setlength{\leftskip}{0.00mm}
\setlength{\rightskip}{0.00mm}


\vspace{0.00mm}
\begin{itemize}
\begin{itemize}

\item
\vspace{4.17mm}
\setlength{\parindent}{0.00mm}
\setlength{\leftskip}{0.00mm}
\setlength{\rightskip}{0.00mm}
\raggedright
\textbf{2.2. The Controversies Surrounding the Right to Development}
\vspace{2.08mm}

\end{itemize}
\end{itemize}
\vspace{0.00mm}
\setlength{\parindent}{0.00mm}
\setlength{\leftskip}{0.00mm}
\setlength{\rightskip}{0.00mm}


\vspace{0.00mm}

\vspace{0.00mm}
\setlength{\parindent}{0.00mm}
\setlength{\leftskip}{0.00mm}
\setlength{\rightskip}{0.00mm}

The legal obligation for the international community to realise TRTD is greatly affected by its clarity. The controversies surrounding TRTD and the ambiguous language used in TDRTD obscure the substance and meaning of the right. In order to evaluate if TRTD legally obligates the international community to realise TRTD, such controversies are addressed in this section to examine if these can be reconciled to denote specific legal obligations on the international community. This chapter addresses the different views and interpretations regarding the TRTD as a human right; collective rights and TRTD; TRTD as a synthesis of all rights; TRTD as a right to a process of development and the duty bearers of TRTD.  Finally, a brief introduction of the link between security, peace and development is provided at the end of the chapter. 
\vspace{0.00mm}

\vspace{0.00mm}
\setlength{\parindent}{0.00mm}
\setlength{\leftskip}{0.00mm}
\setlength{\rightskip}{0.00mm}


\vspace{0.00mm}
\begin{itemize}
\begin{itemize}
\begin{itemize}

\item
\vspace{4.17mm}
\setlength{\parindent}{0.00mm}
\setlength{\leftskip}{0.00mm}
\setlength{\rightskip}{0.00mm}
\raggedright
\textbf{2.2.1. The Right to Development as a Human Right}
\vspace{2.08mm}

\end{itemize}
\end{itemize}
\end{itemize}
\vspace{0.00mm}
\setlength{\parindent}{0.00mm}
\setlength{\leftskip}{0.00mm}
\setlength{\rightskip}{0.00mm}


\vspace{0.00mm}

\vspace{0.00mm}
\setlength{\parindent}{0.00mm}
\setlength{\leftskip}{0.00mm}
\setlength{\rightskip}{0.00mm}

Of crucial importance to the understanding of TRTD is the emphasis placed on the principle of respect for human rights. The word 'right' or 'rights' has been used in TDRTD at least forty-five times, denoting several different aspects of TRTD. Firstly, TDRTD proclaimed it as an ``\textit{inalienable human right}''$^{}$. This is of fundamental importance as it is the additional value of claiming a human right which has significant consequences for many people. Secondly, TRTD is formulated as a right in which ``\textit{economic, social, cultural and political development}'' will accommodate the realisation of all human rights and fundamental freedoms$^{}$. 
\vspace{0.00mm}

\vspace{0.00mm}
\setlength{\parindent}{0.00mm}
\setlength{\leftskip}{0.00mm}
\setlength{\rightskip}{0.00mm}


\vspace{0.00mm}

\vspace{0.00mm}
\setlength{\parindent}{0.00mm}
\setlength{\leftskip}{0.00mm}
\setlength{\rightskip}{0.00mm}

All human beings have the responsibility for development, including the promotion and protection of ``\textit{an }\textit{appropriate political, social and economic order for development}''$^{}$. However, a limitation on this is provided from the principle of respect for human rights which states that they must take into account ``\textit{the need for full respect for their human rights and fundamental freedoms as well as their duties to the community}''. Such a limitation is provided in order to protect human rights against those willing to repress them in the pursuit of development. Furthermore, states should fulfill their duties in such a manner as to ``\textit{encourage the observance and realisation of human rights}'' and they are required to take steps to eliminate the massive and flagrant violations of human rights$^{}$. Article 5 expressly uses the term ``shall'' whereas in other articles ``should'' is used instead. This implies that states must act in order to eliminate massive and flagrant violations of human rights. In comparison, the term ``should'' , used in other Articles, could now be construed as less powerful and less obligatory. For example, states ``should'' take steps or make considerations, but it could imply that it is of no great consequence if they are unable to.
\vspace{0.00mm}

\vspace{0.00mm}
\setlength{\parindent}{0.00mm}
\setlength{\leftskip}{0.00mm}
\setlength{\rightskip}{0.00mm}


\vspace{0.00mm}

\vspace{0.00mm}
\setlength{\parindent}{0.00mm}
\setlength{\leftskip}{0.00mm}
\setlength{\rightskip}{0.00mm}

TDRTD reaffirms the principle contained in previous human rights documents, that all states should co-operate to strengthen the observance of all human rights and fundamental freedoms without any distinction as to race, sex, language or religion etc. It further emphasises that equal attention must be given to all rights, be they civil, political, economic, social or cultural$^{}$. States should also take ``\textit{take steps to eliminate obstacles to development resulting from failure to observe civil and political rights, as well as economic social and cultural rights''}$^{}$.
\vspace{0.00mm}

\vspace{0.00mm}
\setlength{\parindent}{0.00mm}
\setlength{\leftskip}{0.00mm}
\setlength{\rightskip}{0.00mm}


\vspace{0.00mm}

\vspace{0.00mm}
\setlength{\parindent}{0.00mm}
\setlength{\leftskip}{0.00mm}
\setlength{\rightskip}{0.00mm}

Thus it is made evidently clear that the concept of human rights is intrinsically linked with development. Nevertheless, the proclamation of TRTD as a human right has by no means settled the debate on whether TRTD is a human right, a moral claim, a synthesis of all human rights, a collective right or a right to a process of development which could also be framed as a human right. Neither has it been determined which theory of rights it stems from, whether the naturalist or positivist.  
\vspace{0.00mm}

\vspace{0.00mm}
\setlength{\parindent}{0.00mm}
\setlength{\leftskip}{0.00mm}
\setlength{\rightskip}{0.00mm}


\vspace{0.00mm}

\vspace{0.00mm}
\setlength{\parindent}{0.00mm}
\setlength{\leftskip}{0.00mm}
\setlength{\rightskip}{0.00mm}

Sengupta recognises that the traditional notion of positive rights assumes that, in order to recognise a right, not only must the right-holder be identified, but also the duty-bearer with certain precision and clarity. He asserts that according to this view, 
\vspace{0.00mm}

\vspace{0.00mm}
\setlength{\parindent}{0.00mm}
\setlength{\leftskip}{0.00mm}
\setlength{\rightskip}{0.00mm}


\vspace{0.00mm}

\vspace{0.00mm}
\setlength{\parindent}{0.00mm}
\setlength{\leftskip}{9.84mm}
\setlength{\rightskip}{9.84mm}

``to have a right means to have a claim to something of value on other people, institutions, a state, or the international community, who in turn have the obligation of providing or helping to provide something of value.$^{}$''
\vspace{4.91mm}

\vspace{0.00mm}
\setlength{\parindent}{0.00mm}
\setlength{\leftskip}{9.84mm}
\setlength{\rightskip}{9.84mm}


\vspace{4.91mm}

\vspace{0.00mm}
\setlength{\parindent}{0.00mm}
\setlength{\leftskip}{0.00mm}
\setlength{\rightskip}{0.00mm}

 Sen articulates positivist human rights with correlated duties, he states that; 
\vspace{0.00mm}

\vspace{0.00mm}
\setlength{\parindent}{0.00mm}
\setlength{\leftskip}{0.00mm}
\setlength{\rightskip}{0.00mm}


\vspace{0.00mm}

\vspace{0.00mm}
\setlength{\parindent}{0.00mm}
\setlength{\leftskip}{9.84mm}
\setlength{\rightskip}{9.84mm}

``Rights are entitlements that require, in this view, correlated duties. If person A has a right to some x, then there is some agency, say B, that has a duty to provide A with x.$^{}$''
\vspace{4.91mm}

\vspace{0.00mm}
\setlength{\parindent}{0.00mm}
\setlength{\leftskip}{9.84mm}
\setlength{\rightskip}{9.84mm}


\vspace{4.91mm}

\vspace{0.00mm}
\setlength{\parindent}{0.00mm}
\setlength{\leftskip}{0.00mm}
\setlength{\rightskip}{0.00mm}

Within the human rights regime, each person within the state is deemed as the right holder and the duty bearer corresponds to that of the State, thus the correlated entitlements and duties are easily established. Unfortunately, as TDRTD identifies several possible right-holders and also several duty bearers, it has thus been argued by the positivist school that these entities are too vague, making it impossible to accept TRTD as a human right$^{}$. 
\vspace{0.00mm}

\vspace{0.00mm}
\setlength{\parindent}{0.00mm}
\setlength{\leftskip}{0.00mm}
\setlength{\rightskip}{0.00mm}


\vspace{0.00mm}

\vspace{0.00mm}
\setlength{\parindent}{0.00mm}
\setlength{\leftskip}{0.00mm}
\setlength{\rightskip}{0.00mm}

Moreover, according to the positivist school, rights which are not legally enforceable, cannot be regarded as human rights. Thus any entitlements which are derived from a human right must be sanctioned by a legal authority and appropriate legislative instruments. Disagreeing with this, Sengupta states that;
\vspace{0.00mm}

\vspace{0.00mm}
\setlength{\parindent}{0.00mm}
\setlength{\leftskip}{0.00mm}
\setlength{\rightskip}{0.00mm}
\raggedright

\vspace{0.00mm}

\vspace{0.00mm}
\setlength{\parindent}{0.00mm}
\setlength{\leftskip}{9.84mm}
\setlength{\rightskip}{9.84mm}

``This view, however, confuses human rights with legal rights. Human rights proceed law and are derived not from law but from the concept of human dignity. There is nothing in principle to prevent a right being an internationally recognised human right even if it is not justiciable.$^{}$''
\vspace{4.91mm}

\vspace{0.00mm}
\setlength{\parindent}{0.00mm}
\setlength{\leftskip}{9.84mm}
\setlength{\rightskip}{9.84mm}


\vspace{4.91mm}

\vspace{0.00mm}
\setlength{\parindent}{0.00mm}
\setlength{\leftskip}{0.00mm}
\setlength{\rightskip}{0.00mm}

Naturalist doctrines identify human rights as inherent to the individual$^{}$. But as TRTD has been formulated as a right which also belongs to groups of people it is incompatible with this notion of human rights. This means that the IE has not identified TRTD as belonging to either of school of thought, as something within each, can conflict with the notion of TRTD as a human right. Consequently, TRTD can not be easily ``\textit{pigeon-holed, analogized or catalogued in terms of one or another theory of moral, neutral or legal rights}\textit{$^{}$}''. The above statements lead one to question what exactly human rights \textit{are}, in order to clarify what kind of right TRTD \textit{is}. However, the debate surrounding the nature of human rights, including its foundations and sources, is still ongoing. As Alston clearly notes; 
\vspace{0.00mm}

\vspace{0.00mm}
\setlength{\parindent}{0.00mm}
\setlength{\leftskip}{0.00mm}
\setlength{\rightskip}{0.00mm}
\raggedright

\vspace{0.00mm}

\vspace{0.00mm}
\setlength{\parindent}{0.00mm}
\setlength{\leftskip}{9.84mm}
\setlength{\rightskip}{9.84mm}

``[I]t must be acknowledged that there is substantial disagreement among philosophers as to the precise nature of human rights, their sources or foundations and their practical implications.$^{}$''
\vspace{4.91mm}

\vspace{0.00mm}
\setlength{\parindent}{0.00mm}
\setlength{\leftskip}{9.84mm}
\setlength{\rightskip}{9.84mm}


\vspace{4.91mm}

\vspace{0.00mm}
\setlength{\parindent}{0.00mm}
\setlength{\leftskip}{0.00mm}
\setlength{\rightskip}{0.00mm}

He then quotes ``\textit{Philosophers disagree on almost every aspect of the theory of human rights\ldots}\textit{$^{}$}''. From this it can be concluded that an in depth analysis on the precise nature of human rights, the theoretical and philosophical background of these and the issue of justiciability, are beyond the scope of this paper.
\vspace{0.00mm}

\vspace{0.00mm}
\setlength{\parindent}{0.00mm}
\setlength{\leftskip}{0.00mm}
\setlength{\rightskip}{0.00mm}

\textbf{}
\vspace{0.00mm}
\begin{itemize}
\begin{itemize}
\begin{itemize}

\item
\vspace{4.17mm}
\setlength{\parindent}{0.00mm}
\setlength{\leftskip}{0.00mm}
\setlength{\rightskip}{0.00mm}
\raggedright
\textbf{2.2.2. Collective Rights}
\vspace{2.08mm}

\end{itemize}
\end{itemize}
\end{itemize}
\vspace{0.00mm}
\setlength{\parindent}{0.00mm}
\setlength{\leftskip}{0.00mm}
\setlength{\rightskip}{0.00mm}


\vspace{0.00mm}

\vspace{0.00mm}
\setlength{\parindent}{0.00mm}
\setlength{\leftskip}{0.00mm}
\setlength{\rightskip}{0.00mm}

During the last few decades, increased attention has been drawn to the subject of 'collective rights', and TRTD has played a prominent role throughout the discussions, alongside the right to self-determination$^{}$, the rights of minorities and the right to a healthy environment, to name a few.
\vspace{0.00mm}

\vspace{0.00mm}
\setlength{\parindent}{0.00mm}
\setlength{\leftskip}{0.00mm}
\setlength{\rightskip}{0.00mm}


\vspace{0.00mm}

\vspace{0.00mm}
\setlength{\parindent}{0.00mm}
\setlength{\leftskip}{0.00mm}
\setlength{\rightskip}{0.00mm}

The introduction of these rights has led to some confusion surrounding human rights and their collective nature. The confusion begins from the very terminology used to refer to these rights as not only the term 'collective rights' is used, but also a whole array of expressions including, 'solidarity rights', 'communitarian rights', 'group rights' and 'third generation rights'. Collective rights have been advanced as third generation, solidarity rights, distinguishing them from other rights. The concept of generations of human rights first emerged in the 1970's by Karel Vasak. Vasak proposed third generation rights as rights of solidarity, and this generation of human rights was said to include TRTD. According to him, first generation rights are CPR's, which impose negative obligations on the state to refrain from infringing a citizen's liberty. Second generation rights are ESCR's which require positive obligations for the state to act in some way to enforce these rights. Third generation rights are rights which have emerged from global interdependence and problems of a global nature requiring international solidarity$^{}$.
\vspace{0.00mm}

\vspace{0.00mm}
\setlength{\parindent}{0.00mm}
\setlength{\leftskip}{0.00mm}
\setlength{\rightskip}{0.00mm}


\vspace{0.00mm}

\vspace{0.00mm}
\setlength{\parindent}{0.00mm}
\setlength{\leftskip}{0.00mm}
\setlength{\rightskip}{0.00mm}

There have been many arguments put forward which have opposed the notion of collective rights or third generation rights and also many which have been strongly in favour of their emergence. These arguments will be described in turn, starting from the point of view of the critics$^{}$ and then from the proponents$^{}$ of collective rights. I will then seek to analyse these, to see if such arguments contribute to the clarification of TRTD or result in further confusion.   
\vspace{0.00mm}

\vspace{0.00mm}
\setlength{\parindent}{0.00mm}
\setlength{\leftskip}{0.00mm}
\setlength{\rightskip}{0.00mm}


\vspace{0.00mm}

\vspace{0.00mm}
\setlength{\parindent}{0.00mm}
\setlength{\leftskip}{0.00mm}
\setlength{\rightskip}{0.00mm}

The predominately western view is that the notion of collective rights, and the tendency to proclaim them and afford them the status of 'human rights', is very dangerous ground for a number of reasons and should be approached with extreme caution.
\vspace{0.00mm}

\vspace{0.00mm}
\setlength{\parindent}{0.00mm}
\setlength{\leftskip}{0.00mm}
\setlength{\rightskip}{0.00mm}


\vspace{0.00mm}

\vspace{0.00mm}
\setlength{\parindent}{0.00mm}
\setlength{\leftskip}{0.00mm}
\setlength{\rightskip}{0.00mm}

Firstly, with regards to the definition and concept of human rights, it has been argued that human rights in their very nature are rights which have been afforded to human beings by virtue of being human, and, thus, by definition, cannot be afforded to a collective. Consequently, it has been argued that specific conditions must be met in order for a right to be classified as a human right and such conditions have not been met by collective rights including TRTD. 
\vspace{0.00mm}

\vspace{0.00mm}
\setlength{\parindent}{0.00mm}
\setlength{\leftskip}{0.00mm}
\setlength{\rightskip}{0.00mm}


\vspace{0.00mm}

\vspace{0.00mm}
\setlength{\parindent}{0.00mm}
\setlength{\leftskip}{0.00mm}
\setlength{\rightskip}{0.00mm}

Secondly, the nature of collective rights has been criticised as it is unclear who the right holders and duty-bearers are and also who would represent the collective. Thus it has been argued that the traditional notion of human rights could be eroded if collective rights which are vague and ambiguous are afforded the status of human rights. 
\vspace{0.00mm}

\vspace{0.00mm}
\setlength{\parindent}{0.00mm}
\setlength{\leftskip}{0.00mm}
\setlength{\rightskip}{0.00mm}


\vspace{0.00mm}

\vspace{0.00mm}
\setlength{\parindent}{0.00mm}
\setlength{\leftskip}{0.00mm}
\setlength{\rightskip}{0.00mm}

Thirdly, there is concern that the existence and proclamation of collective rights may be politically abused by states and groups in order to justify violations of individual rights. Overall, the critics of collective rights argue that because of their very nature they are fundamentally incompatible with the human rights discourse.  Finally, the notion of positive and negative rights in the first and second generation of right, as postulated in Vasak's argument, have been heavily criticised. It has now been widely established that both generations require positive and negative actions by States$^{}$.  
\vspace{0.00mm}

\vspace{0.00mm}
\setlength{\parindent}{0.00mm}
\setlength{\leftskip}{0.00mm}
\setlength{\rightskip}{0.00mm}

\textbf{}
\vspace{0.00mm}

\vspace{0.00mm}
\setlength{\parindent}{0.00mm}
\setlength{\leftskip}{0.00mm}
\setlength{\rightskip}{0.00mm}

In defending collective rights, the proponents have argued that, firstly, those who believe that by definition collective rights cannot be human rights are mostly influenced by the natural law doctrines, which are in itself incompatible with the collective rights approach. But, if human rights are universal and based on universal values, then the natural law doctrine should not be the only one consulted when establishing human rights. Thus accordingly, human rights are dynamic, have universal values and are not solely based on natural law theories and therefore collective rights can be human rights. 
\vspace{0.00mm}

\vspace{0.00mm}
\setlength{\parindent}{0.00mm}
\setlength{\leftskip}{0.00mm}
\setlength{\rightskip}{0.00mm}


\vspace{0.00mm}

\vspace{0.00mm}
\setlength{\parindent}{0.00mm}
\setlength{\leftskip}{0.00mm}
\setlength{\rightskip}{0.00mm}

In rebutting the criticism regarding the vagueness of collective rights, it has been argued that some individual rights are also vague and thus such a reason cannot be used to deny the emergence of collective rights. Additionally, some collective rights have been evolving for many decades, including the rights of minorities, self determination and people who are protected from genocide. It has been claimed that some collective rights are \textit{de lege lata} or at least \textit{de lege ferenda}\textit{$^{}$}.   
\vspace{0.00mm}

\vspace{0.00mm}
\setlength{\parindent}{0.00mm}
\setlength{\leftskip}{0.00mm}
\setlength{\rightskip}{0.00mm}


\vspace{0.00mm}

\vspace{0.00mm}
\setlength{\parindent}{0.00mm}
\setlength{\leftskip}{0.00mm}
\setlength{\rightskip}{0.00mm}

Tensions between collective rights and individual rights cannot be used as an argument against collective rights as such tensions exist between rights anyhow and collective rights are not exceptional in this regard. Furthermore, all rights are liable to be politically abused and thus to reject collective rights from this angle cannot be justifiable.  
\vspace{0.00mm}

\vspace{0.00mm}
\setlength{\parindent}{0.00mm}
\setlength{\leftskip}{0.00mm}
\setlength{\rightskip}{0.00mm}


\vspace{0.00mm}

\vspace{0.00mm}
\setlength{\parindent}{0.00mm}
\setlength{\leftskip}{0.00mm}
\setlength{\rightskip}{0.00mm}

With regards to the beneficiaries of collective rights, at least four can be identified; 
\vspace{0.00mm}

\vspace{0.00mm}
\setlength{\parindent}{0.00mm}
\setlength{\leftskip}{0.00mm}
\setlength{\rightskip}{0.00mm}


\vspace{0.00mm}
\begin{enumerate}~[1.]


\item
\vspace{0.00mm}
\setlength{\parindent}{-4.91mm}
\setlength{\leftskip}{4.91mm}
\setlength{\rightskip}{0.00mm}

the global community, e.g. the right to peace
\vspace{0.00mm}

\item
\vspace{0.00mm}
\setlength{\parindent}{-4.91mm}
\setlength{\leftskip}{4.91mm}
\setlength{\rightskip}{0.00mm}

the state, e.g. TRTD, the right to self determination
\vspace{0.00mm}

\item
\vspace{0.00mm}
\setlength{\parindent}{-4.91mm}
\setlength{\leftskip}{4.91mm}
\setlength{\rightskip}{0.00mm}

a group within a state, e.g. TRTD,  indigenous rights
\vspace{0.00mm}

\item
\vspace{0.00mm}
\setlength{\parindent}{-4.91mm}
\setlength{\leftskip}{4.91mm}
\setlength{\rightskip}{0.00mm}

the individual who belongs to a group, e.g. TRTD, minority rights including cultural identity 
\vspace{0.00mm}

\end{enumerate}
\vspace{0.00mm}
\setlength{\parindent}{0.00mm}
\setlength{\leftskip}{0.00mm}
\setlength{\rightskip}{0.00mm}
\raggedright

\vspace{0.00mm}

\vspace{0.00mm}
\setlength{\parindent}{0.00mm}
\setlength{\leftskip}{0.00mm}
\setlength{\rightskip}{0.00mm}

As human rights have traditionally been regarded as belonging to each individual of the human race, every person, individually, is the beneficiary of those rights. However, with the emergence of the concept of collective rights, the beneficiary of the right can be any of the entities described above and thus conflicts between the entities claiming such a right can emerge. It is in such a situation that the rights of individuals can be curtailed for the benefit of one of the other beneficiaries within the collective rights domain. Not only is it possible for an individual's rights to be curtailed but it is also possible that there could be conflicts between each of the beneficiaries within the collective right in question. For example, the right to retain one's own cultural identity can conflict with TRTD, as Prott$^{}$clearly articulates; 
\vspace{0.00mm}

\vspace{0.00mm}
\setlength{\parindent}{0.00mm}
\setlength{\leftskip}{0.00mm}
\setlength{\rightskip}{0.00mm}
\raggedright

\vspace{0.00mm}

\vspace{0.00mm}
\setlength{\parindent}{0.00mm}
\setlength{\leftskip}{9.84mm}
\setlength{\rightskip}{9.84mm}

``[T]he assertion of the right to development may run directly counter to the right to preserve or develop a culture. Economic development may obliterate or mutilate important cultural sites and destroy social structures which are essential for the survival of traditional arts and other cultural activities. Pressures to exploit tourism as a source of foreign revenue often lead to degradation of traditional crafts to cater with the increased demand, to the provision of services for tourists inimical to the local environment, and to damage to static life-styles from the constant stamp of tourist feet. Economic development may mean cultural stagnation: cultural development may mean economic stagnation''.$^{}$  
\vspace{4.91mm}

\vspace{0.00mm}
\setlength{\parindent}{0.00mm}
\setlength{\leftskip}{9.84mm}
\setlength{\rightskip}{9.84mm}


\vspace{4.91mm}

\vspace{0.00mm}
\setlength{\parindent}{0.00mm}
\setlength{\leftskip}{0.00mm}
\setlength{\rightskip}{0.00mm}

It seems evident that, although Prott's criticism on the conflicting nature between cultural rights and TRTD has some ground, he has failed to note that TRTD is not only an economic right but is also a social and cultural right as well as reinforcing CPR's. Thus in the scenarios proposed, economic factors will not be the sole indicators of whether TRTD has been fulfilled or not; if economic factors are the only indicator of fulfilment, then TRTD itself has not been realised. Furthermore, the Vienna Declaration underlined the indivisibility of all human rights.
\vspace{0.00mm}

\vspace{0.00mm}
\setlength{\parindent}{0.00mm}
\setlength{\leftskip}{0.00mm}
\setlength{\rightskip}{0.00mm}


\vspace{0.00mm}

\vspace{0.00mm}
\setlength{\parindent}{0.00mm}
\setlength{\leftskip}{0.00mm}
\setlength{\rightskip}{0.00mm}

From this analysis of collective rights as a whole, they seem to be completely irreconcilable with the concept of human rights as individual rights. Sieghart's description of the process of first claiming to be acting on behalf of the 'people' and then leading to the violation and abuse of exactly those 'peoples' rights, cannot be framed any clearer$^{}$. The danger is evident. It is for this precise reason that the nature of the beneficiaries of collective rights have clearly caused controversy within the human rights domain. Nonetheless, the concept of 'collective rights' cannot be disregarded; each right which emerges as a collective right needs to be analysed on its own. 
\vspace{0.00mm}

\vspace{0.00mm}
\setlength{\parindent}{0.00mm}
\setlength{\leftskip}{0.00mm}
\setlength{\rightskip}{0.00mm}


\vspace{0.00mm}

\vspace{0.00mm}
\setlength{\parindent}{0.00mm}
\setlength{\leftskip}{0.00mm}
\setlength{\rightskip}{0.00mm}

TRTD as a collective right, as both Prott and Sieghart have suggested, may conflict with not only individual rights but also with other collective rights. TDRTD recognises that 'every human person and all peoples' are the beneficiaries of TRTD, which in itself is ambiguous. The concept of 'every human person' is already firmly established in human rights as meaning every individual belonging to the human race, but the concept of 'all peoples' is clearly uncertain; it could mean all peoples in the world community, all peoples belonging to a State, all peoples belonging to a group within a state and also all peoples as individuals within a group, within a state. The reasons for criticising TRTD become evident.
\vspace{0.00mm}

\vspace{0.00mm}
\setlength{\parindent}{0.00mm}
\setlength{\leftskip}{0.00mm}
\setlength{\rightskip}{0.00mm}
\raggedright

\vspace{0.00mm}

\vspace{0.00mm}
\setlength{\parindent}{0.00mm}
\setlength{\leftskip}{0.00mm}
\setlength{\rightskip}{0.00mm}

The collective dimension of TRTD has been claimed by some states as a right of states. The basis for this claim comes from the preamble of TDRTD;  
\vspace{0.00mm}

\vspace{0.00mm}
\setlength{\parindent}{0.00mm}
\setlength{\leftskip}{0.00mm}
\setlength{\rightskip}{0.00mm}


\vspace{0.00mm}

\vspace{0.00mm}
\setlength{\parindent}{0.00mm}
\setlength{\leftskip}{9.84mm}
\setlength{\rightskip}{9.84mm}

``the right to development is an inalienable human right and that equality of opportunity for development is a prerogative both of nations and of individuals who make up nations'' 
\vspace{4.91mm}

\vspace{0.00mm}
\setlength{\parindent}{0.00mm}
\setlength{\leftskip}{9.84mm}
\setlength{\rightskip}{9.84mm}


\vspace{4.91mm}

\vspace{0.00mm}
\setlength{\parindent}{0.00mm}
\setlength{\leftskip}{0.00mm}
\setlength{\rightskip}{0.00mm}

The content of this article confirms TRTD as a human right, and then states that equality of opportunity for development is a prerogative of nations. The exact substance of this article and what it denotes is unclear. Two very different concepts are merged in one sentence and thus its meaning becomes obscured. It has been argued by many states that this article identifies TRTD as a right of states alongside other articles.  
\vspace{0.00mm}

\vspace{0.00mm}
\setlength{\parindent}{0.00mm}
\setlength{\leftskip}{0.00mm}
\setlength{\rightskip}{0.00mm}


\vspace{0.00mm}

\vspace{0.00mm}
\setlength{\parindent}{0.00mm}
\setlength{\leftskip}{0.00mm}
\setlength{\rightskip}{0.00mm}

Article 3(3), in conjunction with the above preambular sentence, can be interpreted to give substance to such as a claim as it stated that ``\textit{States should realise their rights and fulfil their duties in such a manner as to promote a new international economic order}''. The words '\textit{States should realise their rights}' infers the possibility that states hold TRTD but is far too ambiguous and fails to identify what kind of rights, belonging to the state, are to be realised in the promotion of a new economic order. Although TDRTD is vague, in absolute terms, it does not explicitly or implicitly recognise TRTD as a right of states, and the formulation of the right as a right of states has been heavily criticised. Ghai has suggested that some Asian governments have advocated on behalf of TRTD for the purposes of seeking ``\textit{to promote the ideology of developmentalism which justifies repression at home and the evasion of responsibility abroad}\textit{$^{}$}\textit{''}. He further suggests that governments are supporting TRTD as part of a wider goal of establishing ``\textit{the primacy of economic development over human rights}\textit{$^{}$}.'' Furthermore, Donnelly argues that if TRTD can be claimed as a right of states,  ``\textit{[t]he danger here is that the state is thereby placed in a position to use its human rights to deny the human rights of individuals, while still plausibly claiming to be pursuing human rights.''}\textit{$^{}$}\textit{ }
\vspace{0.00mm}

\vspace{0.00mm}
\setlength{\parindent}{0.00mm}
\setlength{\leftskip}{0.00mm}
\setlength{\rightskip}{0.00mm}


\vspace{0.00mm}

\vspace{0.00mm}
\setlength{\parindent}{0.00mm}
\setlength{\leftskip}{0.00mm}
\setlength{\rightskip}{0.00mm}

Nonetheless, from the offset, TDRTD declared that all aspects of the declaration ``\textit{are indivisible and interdependent and each of them should be considered in the context of the whole}\textit{$^{}$}''. Thus although TRTD  undoubtedly has a collective dimension, this is to be construed in the light that ``\textit{The human person is the central subject of development and should be the active participant and beneficiary of the right to development''}\textit{$^{}$}, and this aspect was confirmed by the Vienna Declaration$^{}$. 
\vspace{0.00mm}

\vspace{0.00mm}
\setlength{\parindent}{0.00mm}
\setlength{\leftskip}{0.00mm}
\setlength{\rightskip}{0.00mm}


\vspace{0.00mm}

\vspace{0.00mm}
\setlength{\parindent}{0.00mm}
\setlength{\leftskip}{0.00mm}
\setlength{\rightskip}{0.00mm}

Article 2(3) sheds light on how TRTD can be conceived as a right of states. This article identifies the state as a holder of TRTD, but as Orford suggests the state is acting; 
\vspace{0.00mm}

\vspace{0.00mm}
\setlength{\parindent}{0.00mm}
\setlength{\leftskip}{0.00mm}
\setlength{\rightskip}{0.00mm}


\vspace{0.00mm}

\vspace{0.00mm}
\setlength{\parindent}{0.00mm}
\setlength{\leftskip}{9.84mm}
\setlength{\rightskip}{9.84mm}

``as the agent 'of the entire population and of all individuals'. By Implication, the right is exercisable by the state against those with the power to deny or constrain the capacity of the state to formulate national development policies that benefit the entire population and all individuals.$^{}$'' 
\vspace{4.91mm}

\vspace{0.00mm}
\setlength{\parindent}{0.00mm}
\setlength{\leftskip}{9.84mm}
\setlength{\rightskip}{9.84mm}


\vspace{4.91mm}

\vspace{0.00mm}
\setlength{\parindent}{0.00mm}
\setlength{\leftskip}{0.00mm}
\setlength{\rightskip}{0.00mm}

Consequently, Alston suggests that, in this context, TRTD is a right of states insofar as the state is a ``\textit{medium through which the rights of individuals are effectively asserted vis-a-vis the international} \textit{community}\textit{$^{}$}\textit{''}. Such a formulation of TRTD is the most comprehensible and one which poses little threat to the human rights field. Thus, although there are valid criticisms regarding such a controversial aspect of TRTD, these comments are unnecessary if TRTD is seen as a right of states acting as agents for the well-being of the whole population and all individuals. It then becomes apparent that what was meant by TRTD as a right of states is  that the state can act as agent on behalf of its people but it cannot claim rights for itself.  
\vspace{0.00mm}

\vspace{0.00mm}
\setlength{\parindent}{0.00mm}
\setlength{\leftskip}{0.00mm}
\setlength{\rightskip}{0.00mm}


\vspace{0.00mm}

\vspace{0.00mm}
\setlength{\parindent}{0.00mm}
\setlength{\leftskip}{0.00mm}
\setlength{\rightskip}{0.00mm}

TDRTD affirms that every human person and all peoples are the right holders of TRTD in Article 1(1). To date, this aspect of TRTD is still very contentious. Not only is there disagreement amongst scholars surrounding this dimension, but, more importantly, consensus has still not been reached amongst states. Thus, without the resolution or a compromise as to the meaning and definition of the collective dimension of TRTD, uncertainty regarding it will remain. 
\vspace{0.00mm}

\vspace{0.00mm}
\setlength{\parindent}{0.00mm}
\setlength{\leftskip}{0.00mm}
\setlength{\rightskip}{0.00mm}


\vspace{0.00mm}

\vspace{0.00mm}
\setlength{\parindent}{0.00mm}
\setlength{\leftskip}{0.00mm}
\setlength{\rightskip}{0.00mm}

Having noted the dangers posed by the emergence of the concept of collective rights, it would be a gross misconception to identify the whole concept as a fallacy. The right-holders, albeit sometimes uncertain  are equally in some other cases quite determinable. For example, the right-holder of a right to peace, which has also been suggested as belonging to the third generation of human rights, is every person within the world community, and the duty-bearer is thus evidently the people in power -- ordinarily the governments of each and every state in the world. Thus the conflict of rights as described by Sieghart will not occur$^{}$. With regards to TRTD, such a problem would not be posed if the right holders were to be all peoples belonging to the world community at large. The problem lies in the fact that TRTD has also been claimed as a right of states, and it is this claim which can be seen as irreconcilable with the whole human rights framework. As Galenkemp states, commenting on collective  rights as a whole; 
\vspace{0.00mm}

\vspace{0.00mm}
\setlength{\parindent}{0.00mm}
\setlength{\leftskip}{0.00mm}
\setlength{\rightskip}{0.00mm}


\vspace{0.00mm}

\vspace{0.00mm}
\setlength{\parindent}{0.00mm}
\setlength{\leftskip}{9.84mm}
\setlength{\rightskip}{9.84mm}

``it seems fair to conclude that the debate on collective rights to date has been conducted in rather extreme terms, resulting in polarisation. The ordinary reader is left with the impression of two irreconcilable camps.$^{}$''
\vspace{4.91mm}

\vspace{0.00mm}
\setlength{\parindent}{0.00mm}
\setlength{\leftskip}{9.84mm}
\setlength{\rightskip}{9.84mm}


\vspace{4.91mm}

\vspace{0.00mm}
\setlength{\parindent}{0.00mm}
\setlength{\leftskip}{0.00mm}
\setlength{\rightskip}{0.00mm}

This denotes that because of the collective dimension of TRTD, the legal obligation for the international community to realise TRTD is highly questionable and further clarification of this dimension of the right, is necessary before any legal obligation can be assumed.  Nevertheless, TRTD is not only seen as a collective right, it is also a right of every person and, even though it is a right of all peoples, the human person is the central subject of TRTD. Thus the individualistic component of TRTD cannot be disregarded, but this still does not denote that the international community is legally obligated to realise TRTD. 
\vspace{0.00mm}

\vspace{0.00mm}
\setlength{\parindent}{0.00mm}
\setlength{\leftskip}{0.00mm}
\setlength{\rightskip}{0.00mm}


\vspace{0.00mm}
\begin{itemize}
\begin{itemize}
\begin{itemize}

\item
\vspace{4.17mm}
\setlength{\parindent}{0.00mm}
\setlength{\leftskip}{0.00mm}
\setlength{\rightskip}{0.00mm}
\raggedright
\textbf{2.2.3. The Right to Development as a Synthesis of All Rights }
\vspace{2.08mm}

\end{itemize}
\end{itemize}
\end{itemize}
\vspace{0.00mm}
\setlength{\parindent}{0.00mm}
\setlength{\leftskip}{0.00mm}
\setlength{\rightskip}{0.00mm}

\textbf{}
\vspace{0.00mm}

\vspace{0.00mm}
\setlength{\parindent}{0.00mm}
\setlength{\leftskip}{0.00mm}
\setlength{\rightskip}{0.00mm}

TDRTD clearly articulates that all human rights are indivisible and confirms that priority must not be given to TRTD over other human rights. This was reiterated with a stronger sense of affirmation in the Vienna Declaration. TRTD as a synthesis of human rights has been proclaimed on a number of instances and suggested by scholars such as Alston$^{}$. The concept was evident throughout the early working group discussions and many of the experts considered TRTD as embodying all rights$^{}$ and as the ``\textit{synthesis of already defined rights}''$^{}$. This led to the Secretary-General to affirm ``\textit{[t]he approach by which the right to development is viewed as a synthesis of a large number of human rights has found favor with a number of commentators''}$^{}$. Reiterating this concept, it was stated that ``\textit{in the view of most experts, this right is a combination of existing rights recognised by the international community and contributing to the development of peoples and states''}$^{}$.  Thus, the relationship between TRTD and other human rights can be deemed as crucial to the understanding of TRTD and what it constitutes. 
\vspace{0.00mm}

\vspace{0.00mm}
\setlength{\parindent}{0.00mm}
\setlength{\leftskip}{0.00mm}
\setlength{\rightskip}{0.00mm}


\vspace{0.00mm}

\vspace{0.00mm}
\setlength{\parindent}{0.00mm}
\setlength{\leftskip}{0.00mm}
\setlength{\rightskip}{0.00mm}

Having noted this, it has been expressed by a number of commentators that perceiving TRTD as synthesis of all other rights adds nothing new to the human rights discourse and ``t\textit{akes attention away from the specific rights, for example, speech, assembly, social welfare, to an ambiguous portmanteau right to development}\textit{$^{}$}''. Donnelly expresses the same sentiment but further adds;
\vspace{0.00mm}

\vspace{0.00mm}
\setlength{\parindent}{0.00mm}
\setlength{\leftskip}{0.00mm}
\setlength{\rightskip}{0.00mm}


\vspace{0.00mm}

\vspace{0.00mm}
\setlength{\parindent}{0.00mm}
\setlength{\leftskip}{9.84mm}
\setlength{\rightskip}{9.84mm}

``Discussion of the right to development as a ``synthesis of all human rights'' shifts the focus from particular rights to the package as whole. This is most likely to lead not to a broader and more comprehensive view but rather an increasing detachment from the realities of implementation in particular cases; that is, it is likely to obscure the central fact that progress in realising the whole must be achieved by hard work in implementing the ``parts'', the separate civil, political, Economic, social and cultural rights already recognised internationally.$^{}$''
\vspace{4.91mm}

\vspace{0.00mm}
\setlength{\parindent}{0.00mm}
\setlength{\leftskip}{9.84mm}
\setlength{\rightskip}{9.84mm}


\vspace{4.91mm}

\vspace{0.00mm}
\setlength{\parindent}{0.00mm}
\setlength{\leftskip}{0.00mm}
\setlength{\rightskip}{0.00mm}

While Donnelly and Ghai make some valid points regarding the need to implement recognised human rights and questioning the value of TRTD as a synthesis of all rights, both detract from the notion of indivisibility of all human rights which has been unequivocally affirmed in a variety of international human rights instruments. Orford argues ``\textit{that the focus on the indivisibility of rights is an important aspect''} of TRTD$^{}$. 
\vspace{0.00mm}

\vspace{0.00mm}
\setlength{\parindent}{0.00mm}
\setlength{\leftskip}{-9.90mm}
\setlength{\rightskip}{-13.02mm}


\vspace{0.00mm}

\vspace{0.00mm}
\setlength{\parindent}{0.00mm}
\setlength{\leftskip}{0.00mm}
\setlength{\rightskip}{0.00mm}

While Orford clearly identifies the role of the concept of indivisibility of all human rights and does so in a very concise manner, she fails to address the notion of TRTD as a synthesis of all rights. Stating that all human rights are indivisible is a very valid point - but it falls short of arguing that TRTD entails all these rights, and Orford fails to grasp it. It is clear that Donelly is not arguing the value of the concept of indivisibility of all rights but a new notion that TRTD encompasses all those specific rights. There is no attempt within his text to undermine the principle of indivisibility of all rights when he states that attention should be given ``\textit{to implementing the ``parts'', the separate political, civil, political, Economic, social and cultural rights already recognised internationally.}\textit{$^{}$}'' He does not infer in any way that such ``\textit{parts}'' are to be implemented unequally or in some form of hierarchical order. Having noted this, Donnelly's view that TRTD as a synthesis of all rights may obscure individual rights is a very restrictive  one. TRTD aims at complementing all rights and strengthening their implementation through a rights-based approach to development.
\vspace{0.00mm}

\vspace{0.00mm}
\setlength{\parindent}{0.00mm}
\setlength{\leftskip}{0.00mm}
\setlength{\rightskip}{0.00mm}


\vspace{0.00mm}

\vspace{0.00mm}
\setlength{\parindent}{0.00mm}
\setlength{\leftskip}{0.00mm}
\setlength{\rightskip}{0.00mm}

However, the IE has notably disagreed with the concept that TRTD is just a ``\textit{synthesis of all human rights}''$^{}$. Rather, he describes;
\vspace{0.00mm}

\vspace{0.00mm}
\setlength{\parindent}{0.00mm}
\setlength{\leftskip}{0.00mm}
\setlength{\rightskip}{0.00mm}


\vspace{0.00mm}

\vspace{0.00mm}
\setlength{\parindent}{0.00mm}
\setlength{\leftskip}{9.84mm}
\setlength{\rightskip}{9.84mm}

``The right to development is a composite right to a process of development; it is not just an ``umbrella'' right, or the sum of a set of rights. The integrity of these rights implies that if any of them is violated, the whole composite right to development is also violated. The independent expert describes this in terms of a ``vector'' of human rights composed of various elements that represent the various economic, social and cultural as we as the civil and political rights. The realisation of the right to development requires an improvement of some, or at least one of those rights without violating any other.$^{}$''   
\vspace{4.91mm}

\vspace{0.00mm}
\setlength{\parindent}{0.00mm}
\setlength{\leftskip}{9.84mm}
\setlength{\rightskip}{9.84mm}


\vspace{4.91mm}

\vspace{0.00mm}
\setlength{\parindent}{0.00mm}
\setlength{\leftskip}{0.00mm}
\setlength{\rightskip}{0.00mm}

TRTD as a ``\textit{composite right to a process of development}'' is much more than TRTD as a ``\textit{synthesis of all human rights}'', and this has been clearly articulated by the IE. The ``\textit{composite right}'' element, identified as a vector, adds substance and value to the right - which the synthesis approach did not - and clearly identifies the purpose of TRTD as a human right. It would now be beneficial to understand what the process of development entails in the context of TRTD, since one of its key features is the wide definition of development.    
\vspace{0.00mm}

\vspace{0.00mm}
\setlength{\parindent}{0.00mm}
\setlength{\leftskip}{0.00mm}
\setlength{\rightskip}{0.00mm}

              
\vspace{0.00mm}
\begin{itemize}
\begin{itemize}
\begin{itemize}

\item
\vspace{4.17mm}
\setlength{\parindent}{0.00mm}
\setlength{\leftskip}{0.00mm}
\setlength{\rightskip}{0.00mm}
\raggedright
\textbf{2.2.4. The Right to Development as a Right to a Process of Development}
\vspace{2.08mm}

\end{itemize}
\end{itemize}
\end{itemize}
\vspace{0.00mm}
\setlength{\parindent}{0.00mm}
\setlength{\leftskip}{0.00mm}
\setlength{\rightskip}{0.00mm}


\vspace{0.00mm}

\vspace{0.00mm}
\setlength{\parindent}{0.00mm}
\setlength{\leftskip}{0.00mm}
\setlength{\rightskip}{0.00mm}

The task to define development is, \textit{per se}, not the purpose of this paper, and many will acknowledge that it has been the subject of much scholarly attention$^{}$. This notwithstanding, since development is a core element of TRTD, its definition is of crucial significance. TDRTD defines development in the preamble of TDRTD. This is a wide interpretation, comprised not just of economic aspects but also social, cultural and politic elements, and thus it varies fundamentally from traditional notions of economic development. What development \textit{is}, is crucial to understanding what people are claiming when they assert they have the right to \textit{development}. It is unfortunate that this aspect of TRTD has gone relatively unnoticed in the literature available. It can be noted that whilst writers have acknowledged that the definition of development in TRTD is uncertain and vague, any in-depth analysis or even suggestion as to what it might constitute have been notably absent. Thus,  in the context of TRTD, the question `\textit{what is development?'} remains at large open. 
\vspace{0.00mm}

\vspace{0.00mm}
\setlength{\parindent}{0.00mm}
\setlength{\leftskip}{0.00mm}
\setlength{\rightskip}{0.00mm}


\vspace{0.00mm}

\vspace{0.00mm}
\setlength{\parindent}{0.00mm}
\setlength{\leftskip}{0.00mm}
\setlength{\rightskip}{0.00mm}

Development in its purely economic sense is ``\textit{seen as simultaneously the vision of a better life, a life materially richer, institutionally more modern and technologically more efficient and an array of means to achieve that vision''}$^{}$. This definition of development is typically associated with the capitalist model and seems to imply that material wealth is of key importance in order to have a better life, but fails to take into account that the individual may be able to develop himself through social, cultural and political means, and, in doing so, contribute to the development of society or the state as a whole. In addition it does not address cases where such a  vision may not be possible as a reality due to the circumstances surrounding that individual, as for example in cases of extreme poverty. 
\vspace{0.00mm}

\vspace{0.00mm}
\setlength{\parindent}{0.00mm}
\setlength{\leftskip}{0.00mm}
\setlength{\rightskip}{0.00mm}


\vspace{0.00mm}

\vspace{0.00mm}
\setlength{\parindent}{0.00mm}
\setlength{\leftskip}{0.00mm}
\setlength{\rightskip}{0.00mm}

Having noted this, Rodney$^{}$, dedicating a whole book on \textit{How Europe Underdeveloped Africa}, suggests what development could mean, in the context of identifying what underdevelopment constitutes. While offering different definitions of development and noting that ``\textit{development in human society is a many-sided process'', Rodney states that development, ``[a]t the level of the individual, implies increased skill and capacity, greater freedom, creativity, self-discipline, responsibility, and material well-being''}. With regards to groups, Rodney states that; ``\textit{development implies an increasing capacity to regulate both internal and external relationships}''. Hence it can be easily established that development in TRTD is something more than merely economic development$^{}$. This is not to say that economic development is not important; quite on the contrary. Development in the social, cultural and political sense will be curtailed without the economic component. Economic development is an indivisible and interdependent aspect of the `whole' of development, in a similar way that all human rights are interdependent and indivisible: one human right cannot be fulfilled at the expense of another and all are interconnected, as are the different facets of development. 
\vspace{0.00mm}

\vspace{0.00mm}
\setlength{\parindent}{0.00mm}
\setlength{\leftskip}{0.00mm}
\setlength{\rightskip}{0.00mm}


\vspace{0.00mm}

\vspace{0.00mm}
\setlength{\parindent}{0.00mm}
\setlength{\leftskip}{0.00mm}
\setlength{\rightskip}{0.00mm}

From this perspective it becomes clear why traditional notions of economic development are not adequate to determine whether or not the people within a state have developed. The social, cultural and political aspects of development have emerged to bridge this gap, and take into account not just the state as a whole but also the development of the people -- as opposed to the traditional notion where the individual is not of paramount significance, as evidenced by the indices of pure economic development such as GDP and GNP, which focus only on economic growth and ignore the individual. The need for a human rights approach to development, or at least a more people-centred approach, became apparent. 
\vspace{0.00mm}

\vspace{0.00mm}
\setlength{\parindent}{0.00mm}
\setlength{\leftskip}{0.00mm}
\setlength{\rightskip}{0.00mm}
\raggedright

\vspace{0.00mm}

\vspace{0.00mm}
\setlength{\parindent}{0.00mm}
\setlength{\leftskip}{0.00mm}
\setlength{\rightskip}{0.00mm}

Many institutions already incorporated human rights or a people-centred approach into their policies, but it has been argued that all such incorporations were merely rhetoric, and were making very little difference for people at the grass-roots level. Uvin felt that;
\vspace{0.00mm}

\vspace{0.00mm}
\setlength{\parindent}{0.00mm}
\setlength{\leftskip}{0.00mm}
\setlength{\rightskip}{0.00mm}
\raggedright

\vspace{0.00mm}

\vspace{0.00mm}
\setlength{\parindent}{0.00mm}
\setlength{\leftskip}{9.84mm}
\setlength{\rightskip}{9.84mm}

``During the 1990s, bilateral and multilateral aid agencies published a slew of policy statements, guidelines and documents on the incorporation of human rights in their mandate. An enormous amount of this work was little more than thinly disguised repackaging of old wine in new bottles.$^{}$'' 
\vspace{4.91mm}

\vspace{0.00mm}
\setlength{\parindent}{0.00mm}
\setlength{\leftskip}{9.84mm}
\setlength{\rightskip}{9.84mm}


\vspace{4.91mm}

\vspace{0.00mm}
\setlength{\parindent}{0.00mm}
\setlength{\leftskip}{0.00mm}
\setlength{\rightskip}{0.00mm}

Similarly, Frankovits is also highly critical of the incorporation of human rights terminology by donors and also the World Bank$^{}$.
\vspace{0.00mm}

\vspace{0.00mm}
\setlength{\parindent}{0.00mm}
\setlength{\leftskip}{0.00mm}
\setlength{\rightskip}{0.00mm}


\vspace{0.00mm}

\vspace{0.00mm}
\setlength{\parindent}{0.00mm}
\setlength{\leftskip}{0.00mm}
\setlength{\rightskip}{0.00mm}

Taking all this into account, Allot's General Felicity Index (GFI) was one of the first indices to recognise that, to measure development, one must measure ``\textit{not just the increase in the number of factories or expansion of services, but basically whether life is happier and more fruitful and enjoyable for the individual. In doing this, one has to balance one factor against another''}$^{}$. Such an approach can be seen in the UNDP Human Development Reports, where a wide range of indices are used to measure development$^{}$.
\vspace{0.00mm}

\vspace{0.00mm}
\setlength{\parindent}{0.00mm}
\setlength{\leftskip}{0.00mm}
\setlength{\rightskip}{0.00mm}


\vspace{0.00mm}

\vspace{0.00mm}
\setlength{\parindent}{0.00mm}
\setlength{\leftskip}{0.00mm}
\setlength{\rightskip}{0.00mm}

TRTD has not escaped being criticised for its economic bias, as eloquently argued by Charlesworth$^{}$. Her criticism is highly valuable because it points out that the ``\textit{international law of development}'' has assumed development to mean industrialisation and westernisation, and, although TRTD contains social, cultural and political elements, the foundations on which these have been erected were markedly economic. Underdevelopment has been defined by the model of a capitalist economy and thus any state attempting to recover from such `underdevelopment' will, by proxy, be basing the evaluation of that recovery with what development is, as dictated by the model of a capitalist economy. This undermines a human rights approach to development and to TRTD. Assuming that development means industrialisation and westernisation, the suggestion made by Charlesworth could imply that in asserting one's TRTD, one is asserting the right to industrialisation and westernisation. This is a bit far fetched since   there are many states and peoples that support TRTD -- or at any rate, their understanding of it - and have no desire to be westernised or industrialised. The reality is in fact that the world community is culturally diverse and one model of development cannot possibly fit all nations.  Furthermore, such an approach to development would not be acceptable to many developing countries, considering the fear of being controlled by developed states, in the form of neo-colonialism.  Charlesworth fails to take into account that in fact many developing countries do not envisage TRTD as a right to westernisation or industrialisation; although they may assert they are entitled to some economic assistance, it has never been suggested that this assistance is for westernisation.   
\vspace{0.00mm}

\vspace{0.00mm}
\setlength{\parindent}{0.00mm}
\setlength{\leftskip}{0.00mm}
\setlength{\rightskip}{0.00mm}


\vspace{0.00mm}

\vspace{0.00mm}
\setlength{\parindent}{0.00mm}
\setlength{\leftskip}{0.00mm}
\setlength{\rightskip}{0.00mm}

With the approach to development expanding, the notion of TRTD as a right to a process of development emerged. Article 1(1) of TDRTD provides a definition of TRTD, while this does not explicitly mention the word 'process', this is implied in the preamble of TDRTD which provides a definition of development and specifically addresses the notion of development as a process. Sieghart, expressed this notion as early as 1986. He stated that;
\vspace{0.00mm}

\vspace{0.00mm}
\setlength{\parindent}{0.00mm}
\setlength{\leftskip}{0.00mm}
\setlength{\rightskip}{0.00mm}


\vspace{0.00mm}

\vspace{0.00mm}
\setlength{\parindent}{0.00mm}
\setlength{\leftskip}{9.84mm}
\setlength{\rightskip}{9.84mm}

``If one defines development as a process designed, progressively, to create conditions in which \textit{every} individual can enjoy, exercise, and utilise, under the rule of law, \textit{all} his human rights ... then the individual's right to development can be simply defined as his right to participate in, benefit from, that process without any of the prohibited forms of discrimination$^{}$''. 
\vspace{4.91mm}

\vspace{0.00mm}
\setlength{\parindent}{0.00mm}
\setlength{\leftskip}{9.84mm}
\setlength{\rightskip}{9.84mm}


\vspace{4.91mm}

\vspace{0.00mm}
\setlength{\parindent}{0.00mm}
\setlength{\leftskip}{0.00mm}
\setlength{\rightskip}{0.00mm}

However, it is clear that the IE expands on the notion of a process to development, from that described by Sieghart above, as the IE contends that the actual process is a human right in itself. Distinguishing the outcomes of a process of development from the means of how that process is implemented. This approach to development is very similar to that advocated by Sen in his work \textit{Development} \textit{as} \textit{Freedom}.$^{}$ The justification of claiming the actual process of development as a human right and not only the outcomes of that process was put forward in terms of the concept of a \textit{metaright} by Sen.  A clear conceptualisation of the \textit{metaright} is contained in the following quote, which states that; 
\vspace{0.00mm}

\vspace{0.00mm}
\setlength{\parindent}{0.00mm}
\setlength{\leftskip}{0.00mm}
\setlength{\rightskip}{0.00mm}

\textbf{}
\vspace{0.00mm}

\vspace{0.00mm}
\setlength{\parindent}{0.00mm}
\setlength{\leftskip}{9.84mm}
\setlength{\rightskip}{9.84mm}

``a \textit{metaright} to something \textit{x} can be defined as the right to have policies \textit{p(x)} that genuinely pursue the objective of making the right to \textit{x} realisable ... So a right to \textit{x} -- such as the right to food or the right to livelihood -- may be an abstract, background right, but to give a person the right to demand that policy be directed towards securing objectives of making the right to food or the right to livelihood a realisable right is a \textit{right to p(x)}, which is a \textit{metaright to x} a real right. So, even if the right to \textit{x} remains unfulfilled or immediately unrealisable, the metaright to x, p(x) can be a fully valid right if all the obligations associated with p(x) can be clearly specified. Therefore, if the programme of actions and measures that is associated with realising TRTD are designed in a way that all the obligations ... have been clearly specified, that programme implemented as a process of development can be regarded as a \textit{metaright}. That programme can be claimed as a right entailing enforceable obligations on the duty-bearers and can be realisable, with a high probability of leading to the realisation of the background rights.''$^{}$ 
\vspace{4.91mm}

\vspace{0.00mm}
\setlength{\parindent}{0.00mm}
\setlength{\leftskip}{9.84mm}
\setlength{\rightskip}{9.84mm}


\vspace{4.91mm}

\vspace{0.00mm}
\setlength{\parindent}{0.00mm}
\setlength{\leftskip}{0.00mm}
\setlength{\rightskip}{0.00mm}

This extension of the right to a process of development cannot be articulated better in any other form. The concept of a process of development entailing specified obligations on different agencies, as a \textit{metaright}, to realise TRTD, is of significant value as it clarifies the contours of the right and adds substance to TRTD as a whole.  
\vspace{0.00mm}

\vspace{0.00mm}
\setlength{\parindent}{0.00mm}
\setlength{\leftskip}{0.00mm}
\setlength{\rightskip}{0.00mm}


\vspace{0.00mm}

\vspace{0.00mm}
\setlength{\parindent}{0.00mm}
\setlength{\leftskip}{0.00mm}
\setlength{\rightskip}{0.00mm}

Overall, the definition of TRTD as a right to a process of development elaborated by the IE is certainly the most appealing. It identifies the contours of TRTD, although some may be critical in accepting the \textit{metaright} as a human right. Piron suggests that the IE has defined this notion of development as a compromise, acceptable to developing and developed states$^{}$. Whilst this definition is clear and concise, it can be noted that disagreement still continues, which overall undermines TRTD and the prospects of its implementation and creates uncertainty regarding any legal obligation to realise TRTD.  
\vspace{0.00mm}

\vspace{0.00mm}
\setlength{\parindent}{0.00mm}
\setlength{\leftskip}{0.00mm}
\setlength{\rightskip}{0.00mm}

\textbf{}
\vspace{0.00mm}
\begin{itemize}
\begin{itemize}
\begin{itemize}

\item
\vspace{4.17mm}
\setlength{\parindent}{0.00mm}
\setlength{\leftskip}{0.00mm}
\setlength{\rightskip}{0.00mm}
\raggedright
\textbf{2.2.5. International Co-operation v National Implementation}
\vspace{2.08mm}

\end{itemize}
\end{itemize}
\end{itemize}
\vspace{0.00mm}
\setlength{\parindent}{0.00mm}
\setlength{\leftskip}{0.00mm}
\setlength{\rightskip}{0.00mm}

\textbf{}
\vspace{0.00mm}

\vspace{0.00mm}
\setlength{\parindent}{0.00mm}
\setlength{\leftskip}{0.00mm}
\setlength{\rightskip}{0.00mm}

Traditionally, international law regulates the relations between states and international human rights law does not envisage a state having a duty to realise human rights in any other state but its own.  Increasingly, however, with the emergence of globalisation, international law is attempting to go beyond the realms of regulating state actions and into those of International Organisations, including international financial institutions such as the World Bank, International Monetary Fund (IMF) and the Word Trade Organisation (WTO), as the actions of these actors are increasingly affecting human rights across the globe. TDRTD identifies several actors as having the responsibility for realising TRTD, including all human beings, States and the international community. There is a suggestion that International Organisations also have the responsibility to cooperate in the realisation of TRTD. However, as has been noted, the concept of all these actors as the duty-bearers of TRTD in essence could dilute the right. 
\vspace{0.00mm}

\vspace{0.00mm}
\setlength{\parindent}{0.00mm}
\setlength{\leftskip}{0.00mm}
\setlength{\rightskip}{0.00mm}


\vspace{0.00mm}

\vspace{0.00mm}
\setlength{\parindent}{0.00mm}
\setlength{\leftskip}{0.00mm}
\setlength{\rightskip}{0.00mm}

Although many states have acknowledged the principle of respect for human rights, they do not have positive obligations to provide assistance to another state's citizens' to realise those citizens human rights. Under international human rights law they are only obliged to respect, promote and protect the rights of their own citizens. Thus the legal obligation to realise TRTD by the international community is questionable. Nevertheless, the duty to cooperate with a state who is attempting to realise the rights of its citizens is another issue, but this is also complicated by the fact that some developing countries have interpreted TRTD as a right of states.  
\vspace{0.00mm}

\vspace{0.00mm}
\setlength{\parindent}{0.00mm}
\setlength{\leftskip}{0.00mm}
\setlength{\rightskip}{0.00mm}


\vspace{0.00mm}

\vspace{0.00mm}
\setlength{\parindent}{0.00mm}
\setlength{\leftskip}{0.00mm}
\setlength{\rightskip}{0.00mm}

The two key political sides view the provisions on State implementation and international cooperation in two different lights. Developing states argue that there is a duty for international cooperation for the realisation of TRTD, while developed countries argue that the focus of TDRTD is on state implementation and TDRTD in no way articulates there is a legal duty to cooperate. Thus the extent of which the international community has a duty to cooperate in realising TRTD is still unresolved. This issue will be addressed briefly in this section and analysed further in the next chapter. 
\vspace{0.00mm}

\vspace{0.00mm}
\setlength{\parindent}{0.00mm}
\setlength{\leftskip}{0.00mm}
\setlength{\rightskip}{0.00mm}
\raggedright

\vspace{0.00mm}

\vspace{0.00mm}
\setlength{\parindent}{0.00mm}
\setlength{\leftskip}{0.00mm}
\setlength{\rightskip}{0.00mm}


\vspace{0.00mm}
\begin{itemize}
\begin{itemize}
\begin{itemize}
\begin{itemize}

\item
\vspace{4.17mm}
\setlength{\parindent}{0.00mm}
\setlength{\leftskip}{0.00mm}
\setlength{\rightskip}{0.00mm}
\raggedright
All Human Beings as the Duty-Bearers of the Right to Development 
\vspace{2.08mm}

\end{itemize}
\end{itemize}
\end{itemize}
\end{itemize}
\vspace{0.00mm}
\setlength{\parindent}{0.00mm}
\setlength{\leftskip}{0.00mm}
\setlength{\rightskip}{0.00mm}


\vspace{0.00mm}

\vspace{0.00mm}
\setlength{\parindent}{0.00mm}
\setlength{\leftskip}{0.00mm}
\setlength{\rightskip}{0.00mm}

TDRTD identifies all human beings, individually and collectively, as having the responsibility for development$^{}$. Such responsibility entails that they take into account respect for their human rights and fundamental freedoms and also their duties to the community. Such consideration will ensure their fulfilment as a human being and, for this reason, all human beings are responsible ``\textit{to promote and protect an }\textit{appropriate political, social and economic order for development}''. This article fundamentally recognises that although all human beings are entitled to enjoy all human rights, their responsibilities extend to the whole community and thus, whilst enjoying their human rights, they must respect those belonging to others. According to the analysis of the IE, the obligations of all human beings are then those considered as `\textit{imperfect obligations}' and all human beings in a position to help are obligated to realise the right$^{}$. 
\vspace{0.00mm}

\vspace{0.00mm}
\setlength{\parindent}{0.00mm}
\setlength{\leftskip}{0.00mm}
\setlength{\rightskip}{0.00mm}


\vspace{0.00mm}

\vspace{0.00mm}
\setlength{\parindent}{0.00mm}
\setlength{\leftskip}{0.00mm}
\setlength{\rightskip}{0.00mm}

The obligation of all human beings to respect and promote human rights was further recognised by the Commission on Global Governance, which, in 1995, called for recognition that ``\textit{governments are only one source of threats to human rights}'' and that ``\textit{all citizens \ldots should accept that obligation to recognise and help promote the rights of others''}\textit{$^{}$}. However, the legal obligation for all people to realise TRTD is questionable as international law does not regulate the behaviour of people towards each other. 
\vspace{0.00mm}

\vspace{0.00mm}
\setlength{\parindent}{0.00mm}
\setlength{\leftskip}{0.00mm}
\setlength{\rightskip}{0.00mm}


\vspace{0.00mm}
\begin{itemize}
\begin{itemize}
\begin{itemize}
\begin{itemize}

\item
\vspace{4.17mm}
\setlength{\parindent}{0.00mm}
\setlength{\leftskip}{0.00mm}
\setlength{\rightskip}{0.00mm}
\raggedright
States as the Duty-Bearers of the Right to Development
\vspace{2.08mm}

\end{itemize}
\end{itemize}
\end{itemize}
\end{itemize}
\vspace{0.00mm}
\setlength{\parindent}{0.00mm}
\setlength{\leftskip}{0.00mm}
\setlength{\rightskip}{0.00mm}
\raggedright

\vspace{2.08mm}

\vspace{0.00mm}
\setlength{\parindent}{0.00mm}
\setlength{\leftskip}{0.00mm}
\setlength{\rightskip}{0.00mm}

International human rights law does not envisage a state having a duty to realise human rights in any other state but its own.  
\vspace{0.00mm}

\vspace{0.00mm}
\setlength{\parindent}{0.00mm}
\setlength{\leftskip}{0.00mm}
\setlength{\rightskip}{0.00mm}


\vspace{0.00mm}

\vspace{0.00mm}
\setlength{\parindent}{0.00mm}
\setlength{\leftskip}{0.00mm}
\setlength{\rightskip}{0.00mm}

TDRTD emphasises that states have the primary duty to create national and international conditions to realise TRTD$^{}$. Although this can be construed as meaning that states have the primary duty to realise TRTD, it does however cast doubts to this element as it could imply that states have the duty only to create favorable international and national conditions for the realisation of TRTD, but they can stop short there and not actually implement anything which would realise the right. Furthermore, the term `favorable conditions' is very ambiguous as it could either be interpreted subjectively, in which case each state determines what the favorable conditions are. Alternatively, `favorable conditions' could imply that they have to be `favorable' with a certain model of development in mind. For example, the capitalist model could be advanced by many of the developed states as the appropriate method to achieve `favorable conditions' for the realisation of TRTD, which may not be acceptable to some of the developing countries.      
\vspace{0.00mm}

\vspace{0.00mm}
\setlength{\parindent}{0.00mm}
\setlength{\leftskip}{0.00mm}
\setlength{\rightskip}{0.00mm}


\vspace{0.00mm}

\vspace{0.00mm}
\setlength{\parindent}{0.00mm}
\setlength{\leftskip}{0.00mm}
\setlength{\rightskip}{0.00mm}

However, Article 2(3) clearly stipulates that states have the right and duty to formulate national policies, implying that the state itself decides what is appropriate or not -- although, the policies must be aimed at the constant improvement of the entire population and all individuals -- for the realisation of TRTD. The Working Group on TRTD firmly stated that;
\vspace{0.00mm}

\vspace{0.00mm}
\setlength{\parindent}{0.00mm}
\setlength{\leftskip}{0.00mm}
\setlength{\rightskip}{0.00mm}
\raggedright

\vspace{0.00mm}

\vspace{0.00mm}
\setlength{\parindent}{0.00mm}
\setlength{\leftskip}{9.84mm}
\setlength{\rightskip}{9.84mm}

``Sates have the primary responsibility to ensure the conditions necessary for the enjoyment of the right to development, as both an individual and a collective right. Development cannot be seen as an imported phenomenon or one that is based on the charity of developed countries.$^{}$''  
\vspace{4.91mm}

\vspace{0.00mm}
\setlength{\parindent}{0.00mm}
\setlength{\leftskip}{9.84mm}
\setlength{\rightskip}{9.84mm}


\vspace{4.91mm}

\vspace{0.00mm}
\setlength{\parindent}{0.00mm}
\setlength{\leftskip}{0.00mm}
\setlength{\rightskip}{0.00mm}

This reinforces that TRTD is to be realised by states for their people, and implemented through appropriate national policies. Udumbana further suggests that developing countries must adopt an appropriate definition of development which suits their own needs$^{}$. This in accordance with the view that states themselves should ultimately decide what appropriate national policies they will need to implement to create conditions which are favourable for the realisation of TRTD. Furthermore, states should not attempt to import foreign notions of development$^{}$ which may not work for their own state and should re-evaluate what is meant by development in TRTD in the context of their own circumstances. 
\vspace{0.00mm}

\vspace{0.00mm}
\setlength{\parindent}{0.00mm}
\setlength{\leftskip}{0.00mm}
\setlength{\rightskip}{0.00mm}


\vspace{0.00mm}

\vspace{0.00mm}
\setlength{\parindent}{0.00mm}
\setlength{\leftskip}{0.00mm}
\setlength{\rightskip}{0.00mm}

The IE describes TRTD as a right to a particular process of development. This particular process -- and thus TRTD -- would expand the basic capabilities of the individual to enjoy their human rights and also enables the realisation of all human rights and fundamental freedoms.  The IE suggests that states adopt a development plan which incorporates the principles of equity, non-discrimination, transparency, accountability and participation. He considers a step-by-step approach appropriate in realising TRTD as a concrete programme. From this, the IE identifies that immediate action in realising the rights to food, primary education and health is required as part of the process for the realisation of TRTD. Although the IE expresses that this does not imply that any right takes precedence over another, it seems as though such an approach to the realisation of TRTD is controversial as the international community had finally, with the adoption of TDRTD accepted that all human rights were indivisible and interdependent$^{}$.  
\vspace{0.00mm}

\vspace{0.00mm}
\setlength{\parindent}{0.00mm}
\setlength{\leftskip}{0.00mm}
\setlength{\rightskip}{0.00mm}

  
\vspace{0.00mm}

\vspace{0.00mm}
\setlength{\parindent}{0.00mm}
\setlength{\leftskip}{0.00mm}
\setlength{\rightskip}{0.00mm}

The duties of states to realise human rights have been enumerated by Steiner and Alston  to include several obligations$^{}$. The Maastricht Guidelines on Economic, social and cultural rights identify the obligations of states to respect, protect and fulfill those rights. Marks$^{}$ further suggests that states have at least four obligations to realise TRTD, two 'perfect' and two 'imperfect'$^{}$. The two perfect obligations, include those to respect and to protect, which are justiciable and capable of being enforced by a judicial process. While the obligations to provide and promote are ``\textit{general commitments to pursue a certain policy or achieve certain results}\textit{$^{}$}\textit{''}. Marks considers these obligations as non-justiciable but nevertheless as legal obligations, as states are required to take the necessary steps ``\textit{in the direction of sound progressive realisation of the right''}$^{}$.        
\vspace{0.00mm}

\vspace{0.00mm}
\setlength{\parindent}{0.00mm}
\setlength{\leftskip}{0.00mm}
\setlength{\rightskip}{0.00mm}


\vspace{0.00mm}

\vspace{0.00mm}
\setlength{\parindent}{0.00mm}
\setlength{\leftskip}{0.00mm}
\setlength{\rightskip}{0.00mm}

However, the difficulty still remains as to who decides what the \textit{international} `favourable conditions' for the realisation of TRTD are, as it is already established that each state has the duty to create such conditions but, as the whole international community is involved, the notion can differ from state to state. This is primarily the reason why the provisions on international cooperation are highly controversial. 
\vspace{0.00mm}

\vspace{0.00mm}
\setlength{\parindent}{0.00mm}
\setlength{\leftskip}{0.00mm}
\setlength{\rightskip}{0.00mm}
\raggedright
\textbf{}
\vspace{0.00mm}
\begin{itemize}
\begin{itemize}
\begin{itemize}
\begin{itemize}

\item
\vspace{4.17mm}
\setlength{\parindent}{0.00mm}
\setlength{\leftskip}{0.00mm}
\setlength{\rightskip}{0.00mm}
\raggedright
The International Community as the Duty-Bearer of the Right to Development
\vspace{2.08mm}

\end{itemize}
\end{itemize}
\end{itemize}
\end{itemize}
\vspace{0.00mm}
\setlength{\parindent}{0.00mm}
\setlength{\leftskip}{0.00mm}
\setlength{\rightskip}{0.00mm}


\vspace{0.00mm}

\vspace{0.00mm}
\setlength{\parindent}{0.00mm}
\setlength{\leftskip}{0.00mm}
\setlength{\rightskip}{0.00mm}


\vspace{0.00mm}

\vspace{0.00mm}
\setlength{\parindent}{0.00mm}
\setlength{\leftskip}{0.00mm}
\setlength{\rightskip}{0.00mm}

TDRTD declares that States have a duty to co-operate with each other$^{}$$^{ }$. This has caused some contention in the international community as under traditional international law, there is no such duty$^{}$.
\vspace{0.00mm}

\vspace{0.00mm}
\setlength{\parindent}{0.00mm}
\setlength{\leftskip}{0.00mm}
\setlength{\rightskip}{0.00mm}

It has been argued that the legal obligation for international cooperation stems from the UN Charter and the International Bill of Rights. This issue will be analysed in the following chapter to see if there is any foundation to such a claim and whether or not there is a legal obligation for international cooperation to fulfill TRTD.  
\vspace{0.00mm}

\vspace{0.00mm}
\setlength{\parindent}{0.00mm}
\setlength{\leftskip}{0.00mm}
\setlength{\rightskip}{0.00mm}


\vspace{0.00mm}

\vspace{0.00mm}
\setlength{\parindent}{0.00mm}
\setlength{\leftskip}{0.00mm}
\setlength{\rightskip}{0.00mm}

TDRTD stipulates that it is the primary duty of the state to realise TRTD, however in some circumstances it may not be possible for the state to implement the right and for this reason the DRTD calls upon the international community to cooperate to enable states to fulfill their obligations on TRTD. TRTD does not explicitly recognise a legal obligation for the international community to realise TRTD.   
\vspace{0.00mm}

\vspace{0.00mm}
\setlength{\parindent}{0.00mm}
\setlength{\leftskip}{0.00mm}
\setlength{\rightskip}{0.00mm}


\vspace{0.00mm}

\vspace{0.00mm}
\setlength{\parindent}{0.00mm}
\setlength{\leftskip}{0.00mm}
\setlength{\rightskip}{0.00mm}

The notions of a duty to cooperate contained in TDRTD stipulate that this duty exists in ensuring development and eliminating obstacles to development. Furthermore, based on principles such as sovereign equality, interdependence, mutual interest and co-operation, States should realise their rights and duties in such manner which promotes a new international order. Moreover, States should encourage observance to and realisation of, human rights. Article 4$^{}$ stipulates that states have a duty to formulate international policies, both individually and collectively. It also underlines that effective international cooperation is essential to provide developing countries with the appropriate means and facilities to foster their comprehensive development. The IE suggests that in order to fully appreciate the emphasis TDRTD places on international cooperation, Article 4 should be read alongside the preamble$^{}$ -- which refers to Article 1 of the UN Charter $^{}$. Subsequently, when this is all read in conjunction with Articles 55$^{}$ and 56$^{}$ of the Charter, the commitment of international cooperation to realise TRTD is further strengthened$^{}$.   
\vspace{0.00mm}

\vspace{0.00mm}
\setlength{\parindent}{0.00mm}
\setlength{\leftskip}{0.00mm}
\setlength{\rightskip}{0.00mm}


\vspace{0.00mm}

\vspace{0.00mm}
\setlength{\parindent}{0.00mm}
\setlength{\leftskip}{0.00mm}
\setlength{\rightskip}{0.00mm}

The IE has expressed the view that TRTD should be realised through coordinated efforts, including developing states, developed states, international organisations, international financial institutions and NGO's. He advocates on behalf of a `\textit{Development Compact}' (DC) between developing countries and donor countries or international financial institutions$^{}$. This underscores the importance of international cooperation in realising TRTD. Although the DC would be specific to each state, the DC envisages that developing countries undertake to fulfill the realisation of human rights within their state while the international community provides certain resources and shares the cost burden. Furthermore, at the international level, states should act on measures which would allow;
\vspace{0.00mm}

\vspace{0.00mm}
\setlength{\parindent}{0.00mm}
\setlength{\leftskip}{0.00mm}
\setlength{\rightskip}{0.00mm}
\raggedright

\vspace{0.00mm}
\begin{itemize}

\item
\vspace{0.00mm}
\setlength{\parindent}{-6.25mm}
\setlength{\leftskip}{12.50mm}
\setlength{\rightskip}{0.00mm}

``trade and access to markets
\vspace{0.00mm}

\item
\vspace{0.00mm}
\setlength{\parindent}{-6.25mm}
\setlength{\leftskip}{12.50mm}
\setlength{\rightskip}{0.00mm}

debt adjustment for the poorest countries 
\vspace{0.00mm}

\item
\vspace{0.00mm}
\setlength{\parindent}{-6.25mm}
\setlength{\leftskip}{12.50mm}
\setlength{\rightskip}{0.00mm}

transfer of resources and technology 
\vspace{0.00mm}

\item
\vspace{0.00mm}
\setlength{\parindent}{-6.25mm}
\setlength{\leftskip}{12.50mm}
\setlength{\rightskip}{0.00mm}

protection of migrants and labour standards
\vspace{0.00mm}

\item
\vspace{0.00mm}
\setlength{\parindent}{-6.25mm}
\setlength{\leftskip}{12.50mm}
\setlength{\rightskip}{0.00mm}

restructuring of the international financial system to give the developing countries a greater share in power and decision-making and to increase the flow of private capital to their economies.'' $^{}$ 
\vspace{0.00mm}

\end{itemize}
\vspace{0.00mm}
\setlength{\parindent}{0.00mm}
\setlength{\leftskip}{0.00mm}
\setlength{\rightskip}{0.00mm}


\vspace{0.00mm}

\vspace{0.00mm}
\setlength{\parindent}{0.00mm}
\setlength{\leftskip}{0.00mm}
\setlength{\rightskip}{0.00mm}

Although TDRTD declares that states have a duty to cooperate with each other, it is difficult to establish to what extent international cooperation is required and what kind of international cooperation. The above suggestions serve to clarify the notion  but  it is still clear that this aspect has not been resolved by the international community.  Nevertheless, it is clear that TDRTD denotes some obligations on the international community to realise TRTD but whether or not these are legal, is highly questionable.  
\vspace{0.00mm}

\vspace{0.00mm}
\setlength{\parindent}{0.00mm}
\setlength{\leftskip}{0.00mm}
\setlength{\rightskip}{0.00mm}

 
\vspace{0.00mm}

\vspace{0.00mm}
\setlength{\parindent}{0.00mm}
\setlength{\leftskip}{0.00mm}
\setlength{\rightskip}{0.00mm}

Donnelly has heavily criticised the notion that obligations of realising TRTD can extend to actors other than states. According to him, human rights have traditionally been conceived as held only against a state by its citizens and ``\textit{are essentially instruments to protect the individual against the state or to assure that the state guarantees to each individual certain minimum gods, services and opportunities}''$^{}$. Donnelly argues that the focus of TRTD on violators other than states functions as a tool to avoid state responsibility for human rights violations and thus the emergence of TRTD has allowed repressive governments ``\textit{to shift attention from particular rights to general issues, and from the primary role of the state as a violator of human rights to external forces that also contribute to human rights violations}$^{}$''. Donnelly envisages TRTD as a tool ``\textit{to use as an excuse not to act on human rights now}\textit{$^{}$}'' and because of this, he urges that people must; 
\vspace{0.00mm}

\vspace{0.00mm}
\setlength{\parindent}{0.00mm}
\setlength{\leftskip}{0.00mm}
\setlength{\rightskip}{0.00mm}


\vspace{0.00mm}

\vspace{0.00mm}
\setlength{\parindent}{0.00mm}
\setlength{\leftskip}{9.84mm}
\setlength{\rightskip}{9.84mm}

``not lose sight of the fact that most human rights violations are directly perpetrated on people by the governments of their own countries. Discussions of the right to development, however, seem to have the effect, and perhaps even the intent, of obscuring this central point.$^{}$''   
\vspace{4.91mm}

\vspace{0.00mm}
\setlength{\parindent}{0.00mm}
\setlength{\leftskip}{9.84mm}
\setlength{\rightskip}{9.84mm}


\vspace{4.91mm}

\vspace{0.00mm}
\setlength{\parindent}{0.00mm}
\setlength{\leftskip}{0.00mm}
\setlength{\rightskip}{0.00mm}

Ghai takes a similar stance as Donnelly and believes that the obligation to fulfill TRTD by actors other than states will ``\textit{provide an alternative framework for the international discourse on human rights}$^{}$'' and it ``\textit{shifts the focus from domestic arenas (where most violations of rights take place) to the international}\textit{$^{}$}''. 
\vspace{0.00mm}

\vspace{0.00mm}
\setlength{\parindent}{0.00mm}
\setlength{\leftskip}{0.00mm}
\setlength{\rightskip}{0.00mm}


\vspace{0.00mm}

\vspace{0.00mm}
\setlength{\parindent}{0.00mm}
\setlength{\leftskip}{0.00mm}
\setlength{\rightskip}{0.00mm}

Some of the arguments put forward by both Ghai and Donnelly seem very persuasive at first glance. Orford acknowledges that some governments have attempted to claim that the realisation of TRTD takes precedence over other human rights and have restricted human rights because of this. Nevertheless, Orford suggests that arguments put forward by such critics fundamentally ignore the influence and power which international institutions have and their affects on human rights globally. Orford argues that states no longer have the sovereign power to determine the ``\textit{economic, social and cultural conditions in which people live}\textit{$^{}$}\textit{''} or power over ``\textit{policy decisions that shape access to resources and services and determine the nature of constitutional and governmental systems}\textit{$^{}$}\textit{''}. Agreeing with Alston$^{}$ that globalisation has had serious effects to state sovereignty, Orford states that; 
\vspace{0.00mm}

\vspace{0.00mm}
\setlength{\parindent}{0.00mm}
\setlength{\leftskip}{0.00mm}
\setlength{\rightskip}{0.00mm}


\vspace{0.00mm}

\vspace{0.00mm}
\setlength{\parindent}{0.00mm}
\setlength{\leftskip}{9.84mm}
\setlength{\rightskip}{9.84mm}

``Economic globalisation has made the fictitious nature of state sovereignty apparent to all but the most myopic observer of international relations and international law.$^{}$''
\vspace{4.91mm}

\vspace{0.00mm}
\setlength{\parindent}{0.00mm}
\setlength{\leftskip}{9.84mm}
\setlength{\rightskip}{9.84mm}


\vspace{4.91mm}

\vspace{0.00mm}
\setlength{\parindent}{0.00mm}
\setlength{\leftskip}{0.00mm}
\setlength{\rightskip}{0.00mm}

Moreover, Orford argues that;
\vspace{0.00mm}

\vspace{0.00mm}
\setlength{\parindent}{0.00mm}
\setlength{\leftskip}{0.00mm}
\setlength{\rightskip}{0.00mm}


\vspace{0.00mm}

\vspace{0.00mm}
\setlength{\parindent}{0.00mm}
\setlength{\leftskip}{9.84mm}
\setlength{\rightskip}{9.84mm}

``The arguments of critics like Donnelly fail to address situations where individuals or peoples do not need protection only or primarily against the state, but also from other powerful states, transnational corporations (TNC's) or international institutions. Nor do such arguments address situations where actors other than states make decisions about the provision of goods, services and opportunities, and thus about the protection of Economic, social and cultural rights. In many cases in the globalised economy, other states, international organisations or foreign investors will be in a position to deny or effect those protections or guarantees.$^{}$'' 
\vspace{4.91mm}

\vspace{0.00mm}
\setlength{\parindent}{0.00mm}
\setlength{\leftskip}{9.84mm}
\setlength{\rightskip}{9.84mm}


\vspace{4.91mm}

\vspace{0.00mm}
\setlength{\parindent}{0.00mm}
\setlength{\leftskip}{0.00mm}
\setlength{\rightskip}{0.00mm}

As international law concerns relations between states and human rights law concerns state behavior toward its citizens, it is difficult to apprehend how non-state actors can be held accountable for direct and indirect violations of human rights. States are responsible for the activities of non-state actors on their territories but many actions of organisations which are likely to violate human rights may not happen on the territory of a state - and thus that state is not accountable to prevent any violations by that particular organisation.  It is for these reasons that states are not the only violators of human rights and the notion of international cooperation contained in TDRTD extends to the behavior of states in their international relations. This concludes that the element of international cooperation is significant in realising TRTD by both states and non-state actors in an increasingly globalised world. 
\vspace{0.00mm}

\vspace{0.00mm}
\setlength{\parindent}{0.00mm}
\setlength{\leftskip}{0.00mm}
\setlength{\rightskip}{0.00mm}

\textcolor{red}{{}}
\vspace{0.00mm}
\begin{itemize}
\begin{itemize}
\begin{itemize}
\begin{itemize}

\item
\vspace{4.17mm}
\setlength{\parindent}{0.00mm}
\setlength{\leftskip}{0.00mm}
\setlength{\rightskip}{0.00mm}
\raggedright
International Organisations as the Duty-Bearers of the Right to Development
\vspace{2.08mm}

\end{itemize}
\end{itemize}
\end{itemize}
\end{itemize}
\vspace{0.00mm}
\setlength{\parindent}{0.00mm}
\setlength{\leftskip}{0.00mm}
\setlength{\rightskip}{0.00mm}
\raggedright

\vspace{0.00mm}

\vspace{0.00mm}
\setlength{\parindent}{0.00mm}
\setlength{\leftskip}{0.00mm}
\setlength{\rightskip}{0.00mm}

The concept that international organisations have a duty to realise TRTD is not without some controversy. Nevertheless, an an increasingly globalised world, international actors other than states, play a pivotal role in international relations. As institutions like the WB, IMF and WTD consist of states, it has been argued that the obligations of states -- as part of these organisations -- include the need to make the activities of these institutions, compatible with human rights and TRTD$^{}$. For example, international cooperation from this perspective includes the removal of trade barriers. 
\vspace{0.00mm}

\vspace{0.00mm}
\setlength{\parindent}{0.00mm}
\setlength{\leftskip}{0.00mm}
\setlength{\rightskip}{0.00mm}
\raggedright

\vspace{0.00mm}
\begin{itemize}
\begin{itemize}
\begin{itemize}

\item
\vspace{4.17mm}
\setlength{\parindent}{0.00mm}
\setlength{\leftskip}{0.00mm}
\setlength{\rightskip}{0.00mm}
\raggedright
\textbf{2.2.6. Security, Peace and the Right to Development}
\vspace{2.08mm}

\end{itemize}
\end{itemize}
\end{itemize}
\vspace{0.00mm}
\setlength{\parindent}{0.00mm}
\setlength{\leftskip}{0.00mm}
\setlength{\rightskip}{0.00mm}
\raggedright

\vspace{2.08mm}

\vspace{0.00mm}
\setlength{\parindent}{0.00mm}
\setlength{\leftskip}{0.00mm}
\setlength{\rightskip}{0.00mm}

The link between security, peace and development is made in Article 7 of the DRTD$^{}$. Along the lines of the UN charter, Article 7 stipulated that states ``\textit{should promote the establishment, maintenance and strengthening of international peace}''. Moreover, to achieve this end states are required to do their ``\textit{utmost to achieve general and complete disarmament}''. These measures will benefit the development process as, states, in achieving general and complete disarmament should ensure that the resources released by such measures are used for comprehensive development. Unfortunately this is not without controversy as Article 7 could be seen to imply differentiated responsibilities for developing and developed countries: one can interpret it to mean that the latter should allocate part of these revenues to help the development process in developing countries$^{}$.
\vspace{0.00mm}

\vspace{0.00mm}
\setlength{\parindent}{-4.91mm}
\setlength{\leftskip}{4.91mm}
\setlength{\rightskip}{0.00mm}
\raggedright

\vspace{0.00mm}

\vspace{0.00mm}
\setlength{\parindent}{0.00mm}
\setlength{\leftskip}{0.00mm}
\setlength{\rightskip}{0.00mm}
\raggedright
\newpage

\vspace{0.00mm}

\vspace{0.00mm}
\setlength{\parindent}{-4.91mm}
\setlength{\leftskip}{4.91mm}
\setlength{\rightskip}{0.00mm}
\raggedright
3. The Legal Status of the Right to Development in International Law
\vspace{0.00mm}

\vspace{0.00mm}
\setlength{\parindent}{0.00mm}
\setlength{\leftskip}{0.00mm}
\setlength{\rightskip}{0.00mm}


\vspace{0.00mm}

\vspace{0.00mm}
\setlength{\parindent}{0.00mm}
\setlength{\leftskip}{0.00mm}
\setlength{\rightskip}{0.00mm}

This chapter attempts to clarify the current status of TRTD in international law. In order to establish whether TRTD has any binding legal force in international law, it must be established if it falls into any of the categories of the sources of international law as stipulated by the International Court of Justice (ICJ). Article 38$^{}$ of the Statute of the Court reveals five distinct sources;$^{ }$
\vspace{0.00mm}

\vspace{0.00mm}
\setlength{\parindent}{0.00mm}
\setlength{\leftskip}{0.00mm}
\setlength{\rightskip}{0.00mm}


\vspace{0.00mm}
\begin{enumerate}~[1.]


\item
\vspace{0.00mm}
\setlength{\parindent}{-4.91mm}
\setlength{\leftskip}{4.91mm}
\setlength{\rightskip}{0.00mm}
\raggedright
International conventions (or treaties)
\vspace{0.00mm}

\end{enumerate}\begin{enumerate}~[1.]


\item
\vspace{0.00mm}
\setlength{\parindent}{-4.91mm}
\setlength{\leftskip}{4.91mm}
\setlength{\rightskip}{0.00mm}
\raggedright
International custom
\vspace{0.00mm}

\end{enumerate}\begin{enumerate}~[1.]


\item
\vspace{0.00mm}
\setlength{\parindent}{-4.91mm}
\setlength{\leftskip}{4.91mm}
\setlength{\rightskip}{0.00mm}

General principles of law
\vspace{0.00mm}

\end{enumerate}\begin{enumerate}~[1.]


\item
\vspace{0.00mm}
\setlength{\parindent}{-4.91mm}
\setlength{\leftskip}{4.91mm}
\setlength{\rightskip}{0.00mm}
\raggedright
Judicial decisions
\vspace{0.00mm}

\item
\vspace{0.00mm}
\setlength{\parindent}{-4.91mm}
\setlength{\leftskip}{4.91mm}
\setlength{\rightskip}{0.00mm}
\raggedright
Writings of publicists
\vspace{0.00mm}

\end{enumerate}
\vspace{0.00mm}
\setlength{\parindent}{0.00mm}
\setlength{\leftskip}{0.00mm}
\setlength{\rightskip}{0.00mm}
\raggedright

\vspace{0.00mm}

\vspace{0.00mm}
\setlength{\parindent}{0.00mm}
\setlength{\leftskip}{0.00mm}
\setlength{\rightskip}{0.00mm}

This chapter provides a brief introduction on international treaties and focuses on the legal status of TRTD in the African Charter on Human and Peoples' Rights. It then introduces the notion of customary international law and the elements required to prove that a norm has been accepted as customary international law. The chapter deals with the legal status of General Assembly Resolutions and international conferences and then finally analyses the TRTD as customary international law. 
\vspace{0.00mm}

\vspace{0.00mm}
\setlength{\parindent}{0.00mm}
\setlength{\leftskip}{0.00mm}
\setlength{\rightskip}{0.00mm}


\vspace{0.00mm}
\begin{itemize}
\begin{itemize}

\item
\vspace{4.17mm}
\setlength{\parindent}{0.00mm}
\setlength{\leftskip}{0.00mm}
\setlength{\rightskip}{0.00mm}
\raggedright
\textbf{3.1. International Treaties}
\vspace{2.08mm}

\end{itemize}
\end{itemize}
\vspace{0.00mm}
\setlength{\parindent}{0.00mm}
\setlength{\leftskip}{0.00mm}
\setlength{\rightskip}{0.00mm}
\raggedright

\vspace{0.00mm}

\vspace{0.00mm}
\setlength{\parindent}{0.00mm}
\setlength{\leftskip}{0.00mm}
\setlength{\rightskip}{0.00mm}

The fundamental character of international law-making treaties is consent. It is the mutual consent of each of the parties to the treaty which creates its binding force. International human rights treaties, also fundamentally based on consent, have one vital difference. As a legislature or constitution does not exist within the international community, many states perform their legal obligations in 'good faith'$^{}$. In most treaties, the reciprocal obligations and duties of each state party makes it easier to confront a state who is not performing their legal obligations. For example, Sieghart compares most international treaties as ``\textit{commercial contracts, involving some exchange from which both (or all) the state parties will benefit''}. Sieghart explains that when formulating international human rights treaties ``\textit{there are no such commercial incentives or sanctions... Although it is the governments of states which enter these treaties, the trouble is that the beneficiaries are not those governments but their subjects, who are not themselves parties to the treaties''}$^{}$. 
\vspace{0.00mm}

\vspace{0.00mm}
\setlength{\parindent}{0.00mm}
\setlength{\leftskip}{0.00mm}
\setlength{\rightskip}{0.00mm}


\vspace{0.00mm}

\vspace{0.00mm}
\setlength{\parindent}{0.00mm}
\setlength{\leftskip}{0.00mm}
\setlength{\rightskip}{0.00mm}

Form this light, it seems that many states may enter a human rights treaty without necessarily having to fulfill their obligations in 'good faith'. Although they may be legally obliged to do so, other parties may not have the incentive to sanction a 'rogue' state. However, to assume that other incentives are not present will be to deny the value of human rights treaties$^{}$.
\vspace{0.00mm}

\vspace{0.00mm}
\setlength{\parindent}{0.00mm}
\setlength{\leftskip}{0.00mm}
\setlength{\rightskip}{0.00mm}


\vspace{0.00mm}

\vspace{0.00mm}
\setlength{\parindent}{0.00mm}
\setlength{\leftskip}{0.00mm}
\setlength{\rightskip}{0.00mm}

For the purpose of this section, the provisions in international treaties which directly refer to TRTD will be discussed here$^{}$. 
\vspace{0.00mm}

\vspace{0.00mm}
\setlength{\parindent}{0.00mm}
\setlength{\leftskip}{0.00mm}
\setlength{\rightskip}{0.00mm}
\raggedright
\textbf{}
\vspace{0.00mm}
\begin{itemize}
\begin{itemize}
\begin{itemize}

\item
\vspace{4.17mm}
\setlength{\parindent}{0.00mm}
\setlength{\leftskip}{0.00mm}
\setlength{\rightskip}{0.00mm}
\raggedright
\textbf{3.1.1. The Right to Development and The African Charter on Human and Peoples' Rights}
\vspace{2.08mm}

\end{itemize}
\end{itemize}
\end{itemize}
\vspace{0.00mm}
\setlength{\parindent}{0.00mm}
\setlength{\leftskip}{0.00mm}
\setlength{\rightskip}{0.00mm}
\raggedright

\vspace{0.00mm}

\vspace{0.00mm}
\setlength{\parindent}{0.00mm}
\setlength{\leftskip}{0.00mm}
\setlength{\rightskip}{0.00mm}

The African Charter on Human and Peoples' Rights$^{}$, also known as the Banjul Charter, was adopted by the Organisation of African Unity (OAU) in 1981. It entered into force in 1986, and is, to date, the only treaty which recognises TRTD. As a regional treaty, it only binds members of the OAU who have ratified it. To understand the substance and the content of TRTD enshrined in the Banjul Charter, it would be beneficial to provide a brief account on the influential regional factors which were present at the time of its adoption. It can be noted that although many discussions on TRTD acknowledge the legal recognition given to the right in the Banjul Charter, many commentators fail to provide any analysis on the concept -- except to mention that TRTD is a very controversial subject at the international level. This is very unfortunate as an analysis on the very concept of TRTD contained in the Banjul Charter could, by proxy, clarify the substance of the concept at the international level. 
\vspace{0.00mm}

\vspace{0.00mm}
\setlength{\parindent}{0.00mm}
\setlength{\leftskip}{0.00mm}
\setlength{\rightskip}{0.00mm}
\raggedright

\vspace{0.00mm}

\vspace{0.00mm}
\setlength{\parindent}{0.00mm}
\setlength{\leftskip}{0.00mm}
\setlength{\rightskip}{0.00mm}

The OAU was created in 1963 as a regional agreement between African States. Although the primary focus of the treaty was not on human rights, it reaffirmed the principles contained in the UN Charter and the UDHR. Significant attention was placed on the right to self-determination and the eradication of colonialism. It also referred to the welfare and the well being of African people. The OAU actively supported liberation movements in Mozambique and Angola, and was committed to the achievement of self-determination and human rights in South Africa and Namibia. However, the fundamental principle which underlined the OAU was that of state sovereignty and non-interference, due to the effects of colonialism and the establishment of new independent states. Principles such as respect for human rights, paramount to the international law of human rights, were, to a large extent, deemed as domestic concerns of a state. To this end, Umozurike$^{ }$states that;
\vspace{0.00mm}

\vspace{0.00mm}
\setlength{\parindent}{0.00mm}
\setlength{\leftskip}{0.00mm}
\setlength{\rightskip}{0.00mm}
\raggedright

\vspace{0.00mm}

\vspace{0.00mm}
\setlength{\parindent}{0.00mm}
\setlength{\leftskip}{9.84mm}
\setlength{\rightskip}{9.84mm}

``[T]he massacres of thousands of Hutu in Burundi in 1972 and 1973 were neither discussed nor condemned by the OAU, which regarded them as matters of internal affairs\textcolor{blue}{{\uline{.}}} The notorious regimes of Idi Amin of Uganda (1971-1979), Marcias Nguema of Equatorial Guinea (1969-1979), and Jean-Bedel Bokassa of the Central African Republic (1966-1979) escaped the criticism of the OAU and most of its members, Tanzania, Zambia, and Mozambique being the exceptions.$^{}$''
\vspace{4.91mm}

\vspace{0.00mm}
\setlength{\parindent}{0.00mm}
\setlength{\leftskip}{9.84mm}
\setlength{\rightskip}{9.84mm}


\vspace{4.91mm}

\vspace{0.00mm}
\setlength{\parindent}{0.00mm}
\setlength{\leftskip}{0.00mm}
\setlength{\rightskip}{0.00mm}

According to Umozurike, the OAU regarded matters relating to human rights as the domestic concern of states fundamentally because the principle of state sovereignty enshrined in the OAU charter was ``\textit{unduly emphasised}$^{}$'' and as a consequence the state parties were able to maintain an ``\textit{indifferent attitude to the suppression of human rights}$^{}$''. Thus although on the international plane the protection of human rights was accepted as a concern for the whole international community, alongside the adoption of other regional agreements, African states seemed a long distance away from achieving respect for human rights in their territories when their primary concern was to establish themselves as independent nation states. 
\vspace{0.00mm}

\vspace{0.00mm}
\setlength{\parindent}{0.00mm}
\setlength{\leftskip}{0.00mm}
\setlength{\rightskip}{0.00mm}
\raggedright

\vspace{0.00mm}

\vspace{0.00mm}
\setlength{\parindent}{0.00mm}
\setlength{\leftskip}{0.00mm}
\setlength{\rightskip}{0.00mm}

The Banjul Charter was adopted with the principle of respect for human rights as a focal point. It included several provisions on CPR's as well as ESCR's. It recognised the indivisibility of all rights. Furthermore, it emphasised certain rights as belonging to the 'peoples'. However, the principle of state sovereignty was still as paramount and Slimane$^{}$ suggests that; 
\vspace{0.00mm}

\vspace{0.00mm}
\setlength{\parindent}{0.00mm}
\setlength{\leftskip}{0.00mm}
\setlength{\rightskip}{0.00mm}
\raggedright

\vspace{0.00mm}

\vspace{0.00mm}
\setlength{\parindent}{0.00mm}
\setlength{\leftskip}{9.84mm}
\setlength{\rightskip}{9.84mm}

``The African Charter on Human and Peoples' Rights could not but reflect the fundamental objectives of state sovereignty and territorial integrity of each state and integrity of borders declared in the 1963 Organisation of African Unity (OAU) Charter, which was basically concerned with relations between states. As such, it is safe to say that it was not the intent of the drafters of the 1981 Charter to equate the term 'peoples' with the notion of minorities or ethnic groups. Rather, the concept of 'peoples' was identified with the African nation-state.$^{}$''
\vspace{4.91mm}

\vspace{0.00mm}
\setlength{\parindent}{0.00mm}
\setlength{\leftskip}{9.84mm}
\setlength{\rightskip}{9.84mm}


\vspace{4.91mm}

\vspace{0.00mm}
\setlength{\parindent}{0.00mm}
\setlength{\leftskip}{0.00mm}
\setlength{\rightskip}{0.00mm}

Thus once again the notion of a state right to development emerges. Noting that the principle of non-intervention in the domestic affairs of a member state ``\textit{had been the bane of the continent}''$^{}$, Umozurike argues that the adoption of the Banjul Charter ``\textit{was arguably the single most important development in the field of human rights in Africa''}\textit{$^{}$}. He implies that the principle of non-intervention had to some extent, been superseded by that of the principle of respect for human rights, with the adoption of the Banjul Charter. Thus, although critics may argue that the Banjul Charter was only ``\textit{basically concerned with relations between states}\textit{$^{}$}'', the fundamental nature of the Banjul Charter, with all its faults, still attempts to build a bridge between the legitimate principle of state sovereignty -- the bedrock of international law -- and the principle of respect for human rights, needed to protect African people from their own state leaders.  
\vspace{0.00mm}

\vspace{0.00mm}
\setlength{\parindent}{0.00mm}
\setlength{\leftskip}{0.00mm}
\setlength{\rightskip}{0.00mm}
\raggedright

\vspace{0.00mm}

\vspace{0.00mm}
\setlength{\parindent}{0.00mm}
\setlength{\leftskip}{0.00mm}
\setlength{\rightskip}{0.00mm}

The Banjul Charter, unlike its regional counterparts -- the European Convention$^{}$ and the American Convention$^{}$ -- did not create a regional Court$^{}$. Instead, the African Commission on Human and Peoples' Rights was established, with a mandate to promote and protect the rights enshrined in the Banjul Charter and also to interpret its provisions. 
\vspace{0.00mm}

\vspace{0.00mm}
\setlength{\parindent}{0.00mm}
\setlength{\leftskip}{0.00mm}
\setlength{\rightskip}{0.00mm}
\raggedright

\vspace{0.00mm}

\vspace{0.00mm}
\setlength{\parindent}{0.00mm}
\setlength{\leftskip}{0.00mm}
\setlength{\rightskip}{0.00mm}

The Banjul Charter recognises TRTD twice, once in its preamble and then in Article 22. The preamble states; 
\vspace{0.00mm}

\vspace{0.00mm}
\setlength{\parindent}{0.00mm}
\setlength{\leftskip}{0.00mm}
\setlength{\rightskip}{0.00mm}
\raggedright

\vspace{0.00mm}

\vspace{0.00mm}
\setlength{\parindent}{0.00mm}
\setlength{\leftskip}{9.84mm}
\setlength{\rightskip}{9.84mm}

\textit{``}that it is henceforth essential to pay a particular attention to the right to development and that civil and political rights cannot be dissociated from Economic, social and cultural rights in their conception as well as universality and that the satisfaction of Economic, social and cultural rights ia (\textit{sic}) a guarantee for the enjoyment of civil and political rights;''
\vspace{4.91mm}

\vspace{0.00mm}
\setlength{\parindent}{0.00mm}
\setlength{\leftskip}{9.84mm}
\setlength{\rightskip}{9.84mm}


\vspace{4.91mm}

\vspace{0.00mm}
\setlength{\parindent}{0.00mm}
\setlength{\leftskip}{0.00mm}
\setlength{\rightskip}{0.00mm}

This paragraph of the preamble unequivocally reinforces the indivisibility and universality of all human rights. Moreover, it expressly identifies that the enjoyment of ESCR's guarantees the enjoyment of CPR's. The meaning of this sentiment has often been criticised as it could suggest that ESC rights are a priority and thus CPR could be curtailed in the achievement of them. However, this point has been taken out of context from the whole paragraph. It is clear that this paragraph, read as a whole, denotes that all human rights are connected and are associated with each other. 
\vspace{0.00mm}

\vspace{0.00mm}
\setlength{\parindent}{0.00mm}
\setlength{\leftskip}{0.00mm}
\setlength{\rightskip}{0.00mm}


\vspace{0.00mm}

\vspace{0.00mm}
\setlength{\parindent}{0.00mm}
\setlength{\leftskip}{0.00mm}
\setlength{\rightskip}{0.00mm}
\raggedright
Nevertheless, the substance of TRTD is unclear from the preamble; the only clear aspect, is that the Banjul Charter recognises the existence of a 'right to development'. To elaborate on the substance of TRTD, Article 22, states that; 
\vspace{0.00mm}

\vspace{0.00mm}
\setlength{\parindent}{0.00mm}
\setlength{\leftskip}{9.84mm}
\setlength{\rightskip}{9.84mm}


\vspace{4.91mm}

\vspace{0.00mm}
\setlength{\parindent}{0.00mm}
\setlength{\leftskip}{9.84mm}
\setlength{\rightskip}{9.84mm}

``1. All peoples shall have the right to their economic, social and cultural development with due regard to their freedom and identity and in the equal enjoyment of the common heritage of mankind. 2. States shall have the duty, individually or collectively, to ensure the exercise of the right to development.''
\vspace{4.91mm}

\vspace{0.00mm}
\setlength{\parindent}{0.00mm}
\setlength{\leftskip}{9.84mm}
\setlength{\rightskip}{9.84mm}


\vspace{4.91mm}

\vspace{0.00mm}
\setlength{\parindent}{0.00mm}
\setlength{\leftskip}{0.00mm}
\setlength{\rightskip}{0.00mm}

The formulation of TRTD in Article 22(2) identifies that states have the duty to realise TRTD both individually and collectively. This is in line with TDRTD which also expresses that states have this duty. It is unclear what the term ``all peoples'' denotes. It could once again be construed in number of different ways$^{}$. The identification of what the term `peoples' means in the Banjul Charter is further complicated by the effects of colonialism and the subsequent need of States to establish themselves as an independent nation. This is illustrated clearly by Morel, who describes that;
\vspace{0.00mm}

\vspace{0.00mm}
\setlength{\parindent}{0.00mm}
\setlength{\leftskip}{0.00mm}
\setlength{\rightskip}{0.00mm}


\vspace{0.00mm}

\vspace{0.00mm}
\setlength{\parindent}{0.00mm}
\setlength{\leftskip}{9.84mm}
\setlength{\rightskip}{9.84mm}

``African States have traditionally viewed the recognition of the distinct identities of minorities as an element that poses a threat to national unity and undermines the objective of nation building.$^{}$'' 
\vspace{4.91mm}

\vspace{0.00mm}
\setlength{\parindent}{0.00mm}
\setlength{\leftskip}{9.84mm}
\setlength{\rightskip}{9.84mm}


\vspace{4.91mm}

\vspace{0.00mm}
\setlength{\parindent}{0.00mm}
\setlength{\leftskip}{0.00mm}
\setlength{\rightskip}{0.00mm}

Thus, taking into account the history of the continent and of its people, it is evident that the term `peoples' contained in the charter could display elements which can be identified with the objective of  nation building. This supports the notion that TRTD enshrined in the Banjul Charter is a right of states. 
\vspace{0.00mm}

\vspace{0.00mm}
\setlength{\parindent}{0.00mm}
\setlength{\leftskip}{0.00mm}
\setlength{\rightskip}{0.00mm}


\vspace{0.00mm}

\vspace{0.00mm}
\setlength{\parindent}{0.00mm}
\setlength{\leftskip}{0.00mm}
\setlength{\rightskip}{0.00mm}

In order to clarify the term `peoples' and recognise that the term also included groups of people within state, the African Commission stated that;
\vspace{0.00mm}

\vspace{0.00mm}
\setlength{\parindent}{0.00mm}
\setlength{\leftskip}{0.00mm}
\setlength{\rightskip}{0.00mm}


\vspace{0.00mm}

\vspace{0.00mm}
\setlength{\parindent}{0.00mm}
\setlength{\leftskip}{9.84mm}
\setlength{\rightskip}{9.84mm}

``Although the terms 'peoples rights' have not been defined in the Charter, these rights generally refer to the rights of a community (be it ethnic or national) to determine how they should be governed, how their economies and cultures should develop; they include other rights such as the right to national and international peace and security, the right to a clean and satisfactory environment$^{}$''.
\vspace{4.91mm}

\vspace{0.00mm}
\setlength{\parindent}{0.00mm}
\setlength{\leftskip}{9.84mm}
\setlength{\rightskip}{9.84mm}


\vspace{4.91mm}

\vspace{0.00mm}
\setlength{\parindent}{0.00mm}
\setlength{\leftskip}{0.00mm}
\setlength{\rightskip}{0.00mm}

Furthermore, the African Commission took the opportunity in the case of \textit{Social and Economic Rights Action Center and the Center for Economic and Social Rights v Nigeria},$^{}$ to reaffirm the term 'peoples' in the charter included ethnic minorities and it referred to 'peoples' as groups of peoples within a state. This case reinforced that 'peoples' had the collective right, under article 24, ``\textit{to a general satisfactory environment favorable to their development}''$^{}$.  This case also highlights the importance of the principle of participation. The African Commission stated ``\textit{In all their dealings with the oil consortiums, the Government did not involve the Ogoni communities in the decisions that affected the development of Ogoniland}$^{}$''. This is in accordance to the principle of participation contained in TDRTD.  
\vspace{0.00mm}

\vspace{0.00mm}
\setlength{\parindent}{0.00mm}
\setlength{\leftskip}{0.00mm}
\setlength{\rightskip}{0.00mm}


\vspace{0.00mm}

\vspace{0.00mm}
\setlength{\parindent}{0.00mm}
\setlength{\leftskip}{0.00mm}
\setlength{\rightskip}{0.00mm}

The addition of the text ``\textit{be it ethnic or national}'' stated by the African Commission above,  suggests that the term peoples used in the context of the charter can be also be identified with the right of the national community, which is equated with the state. Accordingly, TRTD contained in the Banjul Charter is also a right of states as well as a right of peoples as groups within the state. Thus, there are dangers of states curtailing the rights of people within their jurisdiction. Furthermore, the absence of 'individuals' in Article 22, suggests that TRTD is only a rights of 'peoples' and not also of individuals. This element is in clear contravention with TDRTD which states unequivocally that TRTD is a right of individuals and peoples. Furthermore, TDRTD clearly stipulates that the individual is central to development, whereas such a notion is notably absent in the Banjul Charter. Thus, the concept of TRTD enshrined in the Banjul Charter contradicts the concept of TRTD in TDRTD. The right which has been postivised in the Banjul  Charter seems, in essence, to be a right of peoples and states to development, excluding the individual, which can have serious consequences for the achievement of all human rights, whether individual or collective rights.  However, if article 22 is read alongside article 2 of the charter, the individual is entitled to enjoy all the rights guaranteed by the Banjul Charter. Article 2 reads; 
\vspace{0.00mm}

\vspace{0.00mm}
\setlength{\parindent}{0.00mm}
\setlength{\leftskip}{0.00mm}
\setlength{\rightskip}{0.00mm}


\vspace{0.00mm}

\vspace{0.00mm}
\setlength{\parindent}{0.00mm}
\setlength{\leftskip}{9.84mm}
\setlength{\rightskip}{9.84mm}

``Every individual shall be entitled to the enjoyment of the rights and freedoms recognised and guaranteed in the present Charter without distinction of any kind such as race, ethnic group, colour, sex, language, religion, political or any other opinion, national and social origin, fortune, birth or any status.''
\vspace{4.91mm}

\vspace{0.00mm}
\setlength{\parindent}{0.00mm}
\setlength{\leftskip}{9.84mm}
\setlength{\rightskip}{9.84mm}


\vspace{4.91mm}

\vspace{0.00mm}
\setlength{\parindent}{0.00mm}
\setlength{\leftskip}{0.00mm}
\setlength{\rightskip}{0.00mm}

Thus the principle of non-discrimination contained in article 2, used in tandem with article 22, indicates that TRTD in the Banjul charter is a right of individuals as well as a right of collectives and of states -- however controversial that concept may be. It is evident in the jurisprudence of the African Commission that the Articles in the Charter can be used together. For example, Article 2 was used in conjunction with other articles to protect minority rights in the case of \textit{Malawi African Association etc v Mauritania}\textit{$^{}$}\textit{. }Also, the case concerning the Ogoni Community$^{}$ identifies that the rights contained in the Banjul Charter are individual rights as well$^{}$, it then becomes clear that the `peoples' rights contained in the Charter can also be individual rights. 
\vspace{0.00mm}

\vspace{0.00mm}
\setlength{\parindent}{0.00mm}
\setlength{\leftskip}{0.00mm}
\setlength{\rightskip}{0.00mm}
\raggedright
\textit{}
\vspace{0.00mm}

\vspace{0.00mm}
\setlength{\parindent}{0.00mm}
\setlength{\leftskip}{0.00mm}
\setlength{\rightskip}{0.00mm}

The African Commission has not yet come to hear a case concerning Article 22, but one of the key NGO's working in the field of minority rights, Minority Rights Group International (MRG) have initiated a communication to the African Commission alleging a violation of article 22 of the Banjul Charter -- alongside other articles -- of the Endorois Community in Kenya. The Endorois community will be the first to claim their TRTD. As Morel notes;
\vspace{0.00mm}

\vspace{0.00mm}
\setlength{\parindent}{0.00mm}
\setlength{\leftskip}{0.00mm}
\setlength{\rightskip}{0.00mm}
\raggedright

\vspace{0.00mm}

\vspace{0.00mm}
\setlength{\parindent}{0.00mm}
\setlength{\leftskip}{9.84mm}
\setlength{\rightskip}{9.84mm}

``While the right to development has not been formally adjudicated as such by any human rights judicial or quasi-judicial body\ldots [T]he African Commission made clear in the Ogoni communication that there was no right in under the African Charter that was not able to be made effective. The conclusion can therefore be drawn that the right to development can be given practical and effective meaning in Africa.''$^{}$ 
\vspace{4.91mm}

\vspace{0.00mm}
\setlength{\parindent}{0.00mm}
\setlength{\leftskip}{9.84mm}
\setlength{\rightskip}{9.84mm}


\vspace{4.91mm}

\vspace{0.00mm}
\setlength{\parindent}{0.00mm}
\setlength{\leftskip}{0.00mm}
\setlength{\rightskip}{0.00mm}

Thus, the African Commission will have the opportunity to give substance to the concept of TRTD contained in the Banjul Charter, which could also serve to clarify the concept at the international level. It can therefore be noted that although the legal force of TRTD and the duty to cooperate in the Banjul Charter concerns only the African State parties who have ratified it, its clarification within this area could serve as a significant step to uncover the substance of the right at the international level. From the analysis above, it can be noted that the concept of TRTD enshrined in the Banjul Charter differs from that contained in TDRTD. The Banjul Charter recognises the right primarily as a right of peoples and states, which is fundamentally different from the concept at the international level, as the latter underscores the individual. This point is crucial in understanding why there is a political deadlock concerning TRTD at the international level. However, most principles in both concepts are the same, including the importance given to the principle of participation, the principle of non-discrimination, among others. From this we can extrapolate that the concept of TRTD in the Banjul Charter and the one at the international level can be reconciled to some extent.     
\vspace{0.00mm}

\vspace{0.00mm}
\setlength{\parindent}{0.00mm}
\setlength{\leftskip}{0.00mm}
\setlength{\rightskip}{0.00mm}
\raggedright

\vspace{0.00mm}

\vspace{0.00mm}
\setlength{\parindent}{0.00mm}
\setlength{\leftskip}{0.00mm}
\setlength{\rightskip}{0.00mm}
\raggedright

\vspace{0.00mm}
\begin{itemize}
\begin{itemize}

\item
\vspace{4.17mm}
\setlength{\parindent}{0.00mm}
\setlength{\leftskip}{0.00mm}
\setlength{\rightskip}{0.00mm}
\raggedright
\textbf{3.2. Customary International Law  }
\vspace{2.08mm}

\end{itemize}
\end{itemize}
\vspace{0.00mm}
\setlength{\parindent}{0.00mm}
\setlength{\leftskip}{0.00mm}
\setlength{\rightskip}{0.00mm}


\vspace{0.00mm}

\vspace{0.00mm}
\setlength{\parindent}{0.00mm}
\setlength{\leftskip}{0.00mm}
\setlength{\rightskip}{0.00mm}

Despite customary law being identified as one of the principal sources of international law, one is faced with serious difficulties when attempting to establish a rule of customary law for a number of reasons. By virtue of its very nature, customary law is created through a very informal law-making process, which in turn reveals that the precision and clarity required through other 'formal' law-making methods will not necessarily be evident when attempting to discover the existence of a customary rule$^{}$. 
\vspace{0.00mm}

\vspace{0.00mm}
\setlength{\parindent}{0.00mm}
\setlength{\leftskip}{0.00mm}
\setlength{\rightskip}{0.00mm}


\vspace{0.00mm}

\vspace{0.00mm}
\setlength{\parindent}{0.00mm}
\setlength{\leftskip}{0.00mm}
\setlength{\rightskip}{0.00mm}

International customary law comprises two elements, state practice and the belief by states that such practice is carried out because it is the law -- otherwise known as the \textit{opinio juris sive necessitates (opinio juris)}. State practice can be evidenced by both actions and abstentions from certain practices. A non-exhaustive list of material which could be used as evidence of state practice includes, \textit{inter alia}, actions of governments, statements made by governmental persons to their respective Parliaments, or to foreign governments, or in intergovernmental conferences. Custom was described as ``\textit{a constant and uniform usage practiced by states''}\textit{$^{}$}\textit{ }in the \textit{Asylum Case: Columbia v Peru (1950) ICJ Rep 266}. Thus in order to establish a customary law, state practice must be uniform and consistent. The extent of uniformity of state practice required and the duration of time that would be sufficient to be classified as `\textit{constant and uniform}' was not considered in this case, thus subsequent cases$^{}$ determined that there was no requirement for excessive repetition of the state practice concerned nor was there a requirement for absolute rigorous conformity$^{}$ of a certain practice for it to be deemed as a customary rule; it would be sufficient for the practice to be generally consistent. The duration of state practice was discussed in the \textit{North Sea Continental Shelf Cases}$^{}$, where the ICJ indicated that;
\vspace{0.00mm}

\vspace{0.00mm}
\setlength{\parindent}{0.00mm}
\setlength{\leftskip}{0.00mm}
\setlength{\rightskip}{0.00mm}


\vspace{0.00mm}

\vspace{0.00mm}
\setlength{\parindent}{0.00mm}
\setlength{\leftskip}{9.84mm}
\setlength{\rightskip}{9.84mm}

``Although the passage of only a short period of time is not necessarily, or of itself, a bar to the formation of a new rule of customary international law on the basis of what was originally a purely conventional rule, an indispensable requirement would be that within the period in question, short though it may be, state practice, including that of states whose interests are specially affected, should have been both extensive and virtually uniform in the sense of the provision invoked -- and should moreover have occurred in such a way as to show a general recognition that a rule of law or legal obligation is involved.''$^{}$ 
\vspace{4.91mm}

\vspace{0.00mm}
\setlength{\parindent}{0.00mm}
\setlength{\leftskip}{0.00mm}
\setlength{\rightskip}{0.00mm}


\vspace{0.00mm}

\vspace{0.00mm}
\setlength{\parindent}{0.00mm}
\setlength{\leftskip}{0.00mm}
\setlength{\rightskip}{0.00mm}

Furthermore, where there are significant inconsistencies between different State practices, the rule will not be deemed as customary international law. The ICJ in the \textit{Fisheries Case}$^{}$ noted that although a ten-mile limit for bays had;
\vspace{0.00mm}

\vspace{0.00mm}
\setlength{\parindent}{0.00mm}
\setlength{\leftskip}{0.00mm}
\setlength{\rightskip}{0.00mm}


\vspace{0.00mm}

\vspace{0.00mm}
\setlength{\parindent}{0.00mm}
\setlength{\leftskip}{9.84mm}
\setlength{\rightskip}{9.84mm}

``been adopted by certain States both in their national law and in their treaties and conventions , and although certain arbitral decisions have applied it as between these states, other States have adopted a different limit. Consequently, the ten-mile rule has not acquired the authority of a general rule of international law.$^{}$''  
\vspace{4.91mm}

\vspace{0.00mm}
\setlength{\parindent}{0.00mm}
\setlength{\leftskip}{9.84mm}
\setlength{\rightskip}{9.84mm}


\vspace{4.91mm}

\vspace{0.00mm}
\setlength{\parindent}{0.00mm}
\setlength{\leftskip}{0.00mm}
\setlength{\rightskip}{0.00mm}

The need for consistency in state practice was further underscored in the Asylum Case$^{}$, which although concerned a regional customary rule, serves to clarify the rules on consistent state practice. The ICJ held in this case;
\vspace{0.00mm}

\vspace{0.00mm}
\setlength{\parindent}{0.00mm}
\setlength{\leftskip}{0.00mm}
\setlength{\rightskip}{0.00mm}


\vspace{0.00mm}

\vspace{0.00mm}
\setlength{\parindent}{0.00mm}
\setlength{\leftskip}{9.84mm}
\setlength{\rightskip}{9.84mm}

``The facts brought to the knowledge of the Court disclose so much uncertainty and contradiction, so much fluctuation and discrepancy in the exercise of diplomatic asylum and in official views expressed on various occasions, there has been so much inconsistency in the rapid succession of conventions on asylum, ratified by some States and rejected by others, and the practice has been much influenced by considerations of political expediency in the various cases, that it is not possible to discern in all this any constant and uniform usage, accepted as law \ldots$^{}$''
\vspace{4.91mm}

\vspace{0.00mm}
\setlength{\parindent}{0.00mm}
\setlength{\leftskip}{0.00mm}
\setlength{\rightskip}{0.00mm}

State practice alone will not suffice to establish a customary law as in order to distinguish binding legal norms with those of mere comity, states must also demonstrate that they were obliged by law to behave in such a manner, in other words, states have to accept such practice as a legal obligation and believe they are obliged, by law to conform with its requirements. The ICJ has indicated that the \textit{opinio juris} of states must be strictly proven$^{}$, however in the \textit{Nicaragua Case}$^{}$ the court changed its stance and accepted that \textit{opinio juris} may be inferred$^{}$.   
\vspace{0.00mm}

\vspace{0.00mm}
\setlength{\parindent}{0.00mm}
\setlength{\leftskip}{0.00mm}
\setlength{\rightskip}{0.00mm}


\vspace{0.00mm}
\begin{itemize}
\begin{itemize}
\begin{itemize}

\item
\vspace{4.17mm}
\setlength{\parindent}{0.00mm}
\setlength{\leftskip}{0.00mm}
\setlength{\rightskip}{0.00mm}
\raggedright
\textbf{3.2.1. Generality of the Practice}
\vspace{2.08mm}

\end{itemize}
\end{itemize}
\end{itemize}
\vspace{0.00mm}
\setlength{\parindent}{0.00mm}
\setlength{\leftskip}{0.00mm}
\setlength{\rightskip}{0.00mm}


\vspace{0.00mm}

\vspace{0.00mm}
\setlength{\parindent}{0.00mm}
\setlength{\leftskip}{0.00mm}
\setlength{\rightskip}{0.00mm}

The recognition of a particular customary rule by the majority of states infers that the custom is generally recognised as a binding international norm. It will be legally binding on states generally as long as the state practice is extensive and representative, although an individual state may not be bound if they have persistently objected to the rule. State practice does not need to be universal. A customary international rule can be formed if the state practice has only been followed by a small number of states, providing that those states are significantly representative in the field where the norm is to be created and that there is no significant dissent. Such a principle was highlighted in the \textit{North Sea Continental Shelf Cases}$^{}$ where the ICJ held that providing the state practice included participation by ``\textit{states whose interests are specially affected''}\textit{$^{}$}, the requirement of representative state practice would be fulfilled. The International Law Association (ILA) further elaborates upon this principle and suggests that;
\vspace{0.00mm}

\vspace{0.00mm}
\setlength{\parindent}{0.00mm}
\setlength{\leftskip}{0.00mm}
\setlength{\rightskip}{0.00mm}


\vspace{0.00mm}

\vspace{0.00mm}
\setlength{\parindent}{0.00mm}
\setlength{\leftskip}{9.84mm}
\setlength{\rightskip}{9.84mm}

``The Criterion of representativeness has in fact a dual aspect -- negative and positive. The positive aspect is that, if all major interests (``specially affected States'') are represented, it is not essential for a majority of States to have participated (still less a great majority, or all of them). The negative aspect is that if important actors do not accept the practice, it cannot mature into a rule of general customary law.$^{}$''  
\vspace{4.91mm}

\vspace{0.00mm}
\setlength{\parindent}{0.00mm}
\setlength{\leftskip}{0.00mm}
\setlength{\rightskip}{0.00mm}


\vspace{0.00mm}

\vspace{0.00mm}
\setlength{\parindent}{0.00mm}
\setlength{\leftskip}{0.00mm}
\setlength{\rightskip}{0.00mm}

From this, it can be established that if a state or a group of states --  who are important in the field in which the norm is to be created --  significantly dissent and do not accept the rule, they can stop the rule from becoming customary international law as it would not conform to adequate representation required in the state practice. According to the ILA, only one important state in the field can by its opposition, stop the rule from being formed in customary international law, as the rule would not conform to the requirement of sufficiently representative participation of the practice. 
\vspace{0.00mm}

\vspace{0.00mm}
\setlength{\parindent}{0.00mm}
\setlength{\leftskip}{0.00mm}
\setlength{\rightskip}{0.00mm}


\vspace{0.00mm}
\begin{itemize}
\begin{itemize}
\begin{itemize}

\item
\vspace{4.17mm}
\setlength{\parindent}{0.00mm}
\setlength{\leftskip}{0.00mm}
\setlength{\rightskip}{0.00mm}
\raggedright
\textbf{3.2.2. The Persistent Objector }
\vspace{2.08mm}

\end{itemize}
\end{itemize}
\end{itemize}
\vspace{0.00mm}
\setlength{\parindent}{0.00mm}
\setlength{\leftskip}{0.00mm}
\setlength{\rightskip}{0.00mm}


\vspace{0.00mm}

\vspace{0.00mm}
\setlength{\parindent}{0.00mm}
\setlength{\leftskip}{0.00mm}
\setlength{\rightskip}{0.00mm}

The principle that an individual state may not be bound if they have persistently objected to a rule has been long established. It was confirmed in the \textit{Anglo-Norwegian Fisheries case}$^{}$. The ICJ stated that even if the rule had acquired the status of customary international law ``\textit{[i]n any event the \ldots rule would appear to be inapplicable as against Norway, in as much as she has always opposed any attempt to apply it''}\textit{$^{}$}.  The State must object to the rule in public and do so throughout the emergence of the rule in order to be a persistent objector.  
\vspace{0.00mm}

\vspace{0.00mm}
\setlength{\parindent}{0.00mm}
\setlength{\leftskip}{0.00mm}
\setlength{\rightskip}{0.00mm}


\vspace{0.00mm}
\begin{itemize}
\begin{itemize}
\begin{itemize}

\item
\vspace{4.17mm}
\setlength{\parindent}{0.00mm}
\setlength{\leftskip}{0.00mm}
\setlength{\rightskip}{0.00mm}
\raggedright
\textbf{3.2.3. General Assembly Resolutions and International Conferences }
\vspace{2.08mm}

\end{itemize}
\end{itemize}
\end{itemize}
\vspace{0.00mm}
\setlength{\parindent}{0.00mm}
\setlength{\leftskip}{0.00mm}
\setlength{\rightskip}{0.00mm}

\textbf{\textit{}}
\vspace{0.00mm}

\vspace{0.00mm}
\setlength{\parindent}{0.00mm}
\setlength{\leftskip}{0.00mm}
\setlength{\rightskip}{0.00mm}

There has often been confusion surrounding the nature of GA resolutions and their legal effects. Some writers have suggested that GA resolutions are a direct source of international law, either categorising them into international treaties or attempting to establish that they are customary international law$^{}$. The United Nations General Assembly (GA) is not an established law-making body and do not have the legal authority to bind states. However, the declarations and resolutions of the GA may have some legal effect to the extent that they have been used as evidence of state practice towards establishing emerging customary law or the content of an existing customary rule. 
\vspace{0.00mm}

\vspace{0.00mm}
\setlength{\parindent}{0.00mm}
\setlength{\leftskip}{0.00mm}
\setlength{\rightskip}{0.00mm}

\textbf{\textit{}}
\vspace{0.00mm}

\vspace{0.00mm}
\setlength{\parindent}{0.00mm}
\setlength{\leftskip}{0.00mm}
\setlength{\rightskip}{0.00mm}

In the \textit{Nicaragua case}, the International Court of Justice (ICJ) accepted that Article 2(4) of the United Nations Charter, regarding the prohibition of the use of force, with other instruments such as the Declaration on Principles of International Law Concerning Friendly Relations and Co-operation among States$^{}$ (Resolution 2625) had generated a rule of customary law, which is similar to the content of Article 2(4), but which exists alongside treaty provisions stating the same principle. The \textit{Nicaragua case} highlighted that GA resolutions may be used as evidence of state practice . This can be seen by the use of the GA's Friendly Relations Resolution 2625 by the ICJ in the \textit{Nicaragua case} as support for its ruling in customary international law. The ICJ has used this resolution alongside others, to establish the existence of a customary rule and the \textit{opinio} \textit{juris} of the states. By looking at the circumstances, conditions and attitude of the States at the time the resolution was adopted, the content of the declaration was enough to satisfy the court that this was a significant statement which was reflective of the development of customary law alongside treaty law. The ICJ held that the participating states acceptance of resolution 2625, had confirmed the existence of an \textit{opinio juris} in customary law of the prohibition of the use of force. The effect of consent to such GA resolutions ``\textit{may be understood as acceptance of the validity of the rule or set of rules declared by the resolutions themselves}\textit{$^{}$}''. 
\vspace{0.00mm}

\vspace{0.00mm}
\setlength{\parindent}{0.00mm}
\setlength{\leftskip}{0.00mm}
\setlength{\rightskip}{0.00mm}


\vspace{0.00mm}

\vspace{0.00mm}
\setlength{\parindent}{0.00mm}
\setlength{\leftskip}{0.00mm}
\setlength{\rightskip}{0.00mm}

As the GA consists of representatives throughout the world, its resolutions and declarations are of particular value when attempting to establish state practice which could possibly lead to binding customary international law. The ICJ in the \textit{Legality of the Threat or Use of Nuclear Weapons advisory} \textit{opinion}\textit{$^{}$} clearly articulated this point;
\vspace{0.00mm}

\vspace{0.00mm}
\setlength{\parindent}{0.00mm}
\setlength{\leftskip}{0.00mm}
\setlength{\rightskip}{0.00mm}


\vspace{0.00mm}

\vspace{0.00mm}
\setlength{\parindent}{0.00mm}
\setlength{\leftskip}{9.84mm}
\setlength{\rightskip}{9.84mm}

``The Court notes that General Assembly resolutions, even if they are not binding, may sometimes have normative value. They can, in certain circumstances, provide evidence important for establishing the existence of a rule or the emergence of an \textit{opinio juris}. To establish whether this is true of a General Assembly resolution, it is necessary to look at its content and the conditions of its adoption; it is also necessary to see whether an \textit{opinio juris} exists as to its normative character. Or a series of resolutions may show the gradual evolution of the \textit{opinio juris} required for the establishment of a new rule.$^{}$''   
\vspace{4.91mm}

\vspace{0.00mm}
\setlength{\parindent}{0.00mm}
\setlength{\leftskip}{0.00mm}
\setlength{\rightskip}{0.00mm}

   
\vspace{0.00mm}

\vspace{0.00mm}
\setlength{\parindent}{0.00mm}
\setlength{\leftskip}{0.00mm}
\setlength{\rightskip}{0.00mm}

Moreover, when a GA resolution or declaration refers to subjects which are contained in the UN Charter, they can be seen as an authoritative interpretation of the Charter itself.$^{}$
\vspace{0.00mm}

\vspace{0.00mm}
\setlength{\parindent}{0.00mm}
\setlength{\leftskip}{0.00mm}
\setlength{\rightskip}{0.00mm}


\vspace{0.00mm}

\vspace{0.00mm}
\setlength{\parindent}{0.00mm}
\setlength{\leftskip}{0.00mm}
\setlength{\rightskip}{0.00mm}

Resolutions and declarations arising from international conferences of a universal character, can also in the same manner as GA resolutions and declarations, contribute to the evidence of state practice of a new customary rule or can be used as evidence of the content of existing customary law.
\vspace{0.00mm}

\vspace{0.00mm}
\setlength{\parindent}{0.00mm}
\setlength{\leftskip}{0.00mm}
\setlength{\rightskip}{0.00mm}


\vspace{0.00mm}
\begin{itemize}
\begin{itemize}
\begin{itemize}

\item
\vspace{4.17mm}
\setlength{\parindent}{0.00mm}
\setlength{\leftskip}{0.00mm}
\setlength{\rightskip}{0.00mm}
\raggedright
\textbf{3.2.4. The Right to Development in Customary International Law}
\vspace{2.08mm}

\end{itemize}
\end{itemize}
\end{itemize}
\vspace{0.00mm}
\setlength{\parindent}{0.00mm}
\setlength{\leftskip}{0.00mm}
\setlength{\rightskip}{0.00mm}


\vspace{0.00mm}

\vspace{0.00mm}
\setlength{\parindent}{0.00mm}
\setlength{\leftskip}{0.00mm}
\setlength{\rightskip}{0.00mm}

As has been established above, in order for TRTD to gain the status of customary international law, state practice must be demonstrated. Such practice can be evidenced in the form of resolutions, declarations and international conferences$^{}$. I will first analyse declarations, resolutions and conferences, which explicitly recognise TRTD to establish if it has gained the status of an international customary norm, which would legally bind the international community to realise the right. To analyse these instruments, McDougal recognises that a number of factors must be considered. He suggests; 
\vspace{0.00mm}

\vspace{0.00mm}
\setlength{\parindent}{0.00mm}
\setlength{\leftskip}{0.00mm}
\setlength{\rightskip}{0.00mm}


\vspace{0.00mm}

\vspace{0.00mm}
\setlength{\parindent}{0.00mm}
\setlength{\leftskip}{9.84mm}
\setlength{\rightskip}{9.84mm}

``In order to decide whether a UN statement reflected an accurate description of what peoples' expectations were concerning the law, one needed to know several facts: Who voted for the statement? Who voted against it? What was the relative and effective power of these votes? How compatible is the asserted policy with past expectations? What followed from the resolution? What were the expectations coming from other sources? And so on.$^{}$''   
\vspace{4.91mm}

\vspace{0.00mm}
\setlength{\parindent}{0.00mm}
\setlength{\leftskip}{9.84mm}
\setlength{\rightskip}{9.84mm}


\vspace{4.91mm}

\vspace{0.00mm}
\setlength{\parindent}{0.00mm}
\setlength{\leftskip}{0.00mm}
\setlength{\rightskip}{0.00mm}

The implicit recognition of TRTD in instruments such as the UN charter, UDHR, ICCPR and ICESR are surrounded by controversy. Some writers have suggested that the elements which are contained in these instruments and in TDRTD -- for example the elements concerning international cooperation -- legally obligate the international community, although others suggest that this is not so. Such controversies will be discussed toward the end of this chapter.  
\vspace{0.00mm}

\vspace{0.00mm}
\setlength{\parindent}{0.00mm}
\setlength{\leftskip}{0.00mm}
\setlength{\rightskip}{0.00mm}


\vspace{0.00mm}

\vspace{0.00mm}
\setlength{\parindent}{0.00mm}
\setlength{\leftskip}{0.00mm}
\setlength{\rightskip}{0.00mm}

TRTD was first explicitly recognised in the Commission of Human Rights Resolution 4 (XXXIII) 1977. This resolution recognised TRTD as a human right and asked the UN Secretary General to conduct a study on the ``\textit{international dimensions of the right to development as a human right in relation with other human rights based on international co-operation, including the right to peace, taking into account the requirements of the NIEO and the fundamental human needs}''$^{}$. What is notable in this resolution is the fact that it immediately expresses TRTD as a human right without any questions. Thus it would imply that such a concept has been discussed to see whether or not TRTD is a human right. However, it can be noted that although other statements have contained some of the principles of TRTD, the concept itself was not discussed within the international community. Thus the explicit recognition by the CHR of TRTD as a human right raises questions as to how the right was elevated to this status. Evidence of state practice toward the recognition of the rule could include the following resolutions, although each must be read in light of the circumstances in which they arose.
\vspace{0.00mm}

\vspace{0.00mm}
\setlength{\parindent}{0.00mm}
\setlength{\leftskip}{0.00mm}
\setlength{\rightskip}{0.00mm}


\vspace{0.00mm}

\vspace{0.00mm}
\setlength{\parindent}{0.00mm}
\setlength{\leftskip}{0.00mm}
\setlength{\rightskip}{0.00mm}

\textbf{General Assembly Resolution 1161 (XII) 1957} contains some of the principles which are inherent to TRTD$^{}$ but it does not mention TRTD explicitly. This would imply that TRTD has not yet come into existence although the foundations for the emergence of the right are being made. As this resolution does not explicitly mention TRTD, it can not be used as evidence of state practice toward the emergence of TRTD, \textit{per se}. However, as noted, it can contribute as evidence of the emergence of certain principles contained in TRTD. 
\vspace{0.00mm}

\vspace{0.00mm}
\setlength{\parindent}{0.00mm}
\setlength{\leftskip}{0.00mm}
\setlength{\rightskip}{0.00mm}


\vspace{0.00mm}

\vspace{0.00mm}
\setlength{\parindent}{0.00mm}
\setlength{\leftskip}{0.00mm}
\setlength{\rightskip}{0.00mm}

The\textbf{ GA Resolution on Permanent Sovereignty over Natural Resources, 1962}, asserts that states have the right to control their natural resources, including their development. In the \textbf{UN Conference on Trade and Development (UNCTAD I) 1964}, more equitable distribution of the world's resources was demanded by developing countries.  Neither of these directly recognise TRTD so it cannot be assumed that they are evidence of state practice although the principles which resulted from these are relevant to the emergence of TRTD and can be taken as evidence of that. Both the resolution and the conference underline the principles of State control over natural resources and self-determination. Such principles are evident in TDRTD, for example in Article 1(2). Thus both the resolution and the conference can be used as evidence that certain principles which make up TRTD were emerging, although it cannot be said that TRTD itself had been agreed.       
\vspace{0.00mm}

\vspace{0.00mm}
\setlength{\parindent}{0.00mm}
\setlength{\leftskip}{0.00mm}
\setlength{\rightskip}{0.00mm}


\vspace{0.00mm}

\vspace{0.00mm}
\setlength{\parindent}{0.00mm}
\setlength{\leftskip}{0.00mm}
\setlength{\rightskip}{0.00mm}

The\textbf{ Tehran International Conference on Human Rights 1968} inextricably linked human rights with the development discourse, without specifically mentioning TRTD. The Conference emphasised that, ``\textit{[t]he widening gap between the economically developed and developing countries impedes the realisation of human rights in the international community}$^{}$''. It further stated: ``\textit{The achievement of lasting progress in the implementation of human rights is dependent upon sound and effective national and international policies of economic and social development}\textit{$^{}$}\textit{''}. Such notions were reinforced by the \textbf{Declaration on Social Progress and Development 1969}, which recognised that certain conditions for social progress and development were necessary$^{}$.
\vspace{0.00mm}

\vspace{0.00mm}
\setlength{\parindent}{0.00mm}
\setlength{\leftskip}{0.00mm}
\setlength{\rightskip}{0.00mm}
\raggedright

\vspace{0.00mm}

\vspace{0.00mm}
\setlength{\parindent}{0.00mm}
\setlength{\leftskip}{0.00mm}
\setlength{\rightskip}{0.00mm}

Both the Tehran Conference and the Declaration on Social Progress and Development reflect the essence of TRTD without explicitly mentioning it and can thus be used as evidence of the emergence of the concept rather than as evidence of the right itself as only specific elements are mentioned and not all.   This can also be said for the \textbf{Charter of Economic Rights and Duties of States 1974}\textbf{$^{}$}\textbf{,} which provided a framework for the establishment of the NIEO, envisaging more equality between nations and expressed several elements of TRTD. In 1972 at the \textbf{UN Conference for Trade and Development (UNCTAD III)}, in Chile, TRTD was claimed by many of the developing nations. Although developing countries explicitly recognised TRTD, such recognition could be said as not representative of the whole international community and thus it could not constitute toward evidence of state practice. The\textbf{ Declaration on Race and Racial Prejudice} adopted in 1978 by UNESCO referred to ``\textit{the right of every human being and group to full development}''$^{}$. The GA \textbf{Declaration on the Preparation of Societies for Life in Peace 1978} also stated that everyone has the right ``\textit{to determine the road of their development}''$^{}$. All of these reveal that the concept of TRTD was emerging but still did not exist and had not been endorsed by the international community.  
\vspace{0.00mm}

\vspace{0.00mm}
\setlength{\parindent}{0.00mm}
\setlength{\leftskip}{0.00mm}
\setlength{\rightskip}{0.00mm}


\vspace{0.00mm}

\vspace{0.00mm}
\setlength{\parindent}{0.00mm}
\setlength{\leftskip}{0.00mm}
\setlength{\rightskip}{0.00mm}

All the above resolutions, declarations and the outcomes of the different international conferences serve to highlight specific elements of TRTD and provide evidence of consensus for those particular aspects. The fact that TRTD had not been expressly recognised reveals that TRTD, as a concept in itself, had not emerged, thus casting doubts over the validity of the above mentioned instruments as evidence of state practice for the whole TRTD, although they can be accepted as evidence of certain principles. Furthermore, as TRTD had not been specifically addressed, it could not possibly be argued that states had the necessary \textit{opinio juris} to be legally bound by the right. One can also criticise the CHR on its formulation of TRTD in \textbf{Resolution }\textbf{4 (XXXIII) 1977}, as they expressly state that TRTD is a human right without conducting any prior investigation as to this aspect. \textbf{Resolution }\textbf{4 (XXXIII) 1977} assumes that this aspect is certain and then asks the SG to conduct a study on the obstacles which the right may come across. It was evident that the study was not to see whether the right existed; this was implicitly recognised in the resolution. Such a presumption has ultimately contributed to the controversies surrounding TRTD as a human right, and the CHR should have undertaken a study on this aspect before raising TRTD to the status of a human right.     
\vspace{0.00mm}

\vspace{0.00mm}
\setlength{\parindent}{0.00mm}
\setlength{\leftskip}{0.00mm}
\setlength{\rightskip}{0.00mm}


\vspace{0.00mm}

\vspace{0.00mm}
\setlength{\parindent}{0.00mm}
\setlength{\leftskip}{0.00mm}
\setlength{\rightskip}{0.00mm}

By 1979, having considered the suggestions made by the Secretary General, the CHR issued a further resolution recognising TRTD$^{}$. Whilst a working group was established to prepare a draft declaration on TRTD$^{}$, \textbf{The Banjul Charter}$^{}$$^{ }$gave TRTD legal recognition for the first time. Such activities suggest that TRTD had already been established rather than considering it as an emerging concept, without consistent, uniform or extensive state practice to reflect the views of the international community at large.    
\vspace{0.00mm}

\vspace{0.00mm}
\setlength{\parindent}{0.00mm}
\setlength{\leftskip}{0.00mm}
\setlength{\rightskip}{0.00mm}


\vspace{0.00mm}

\vspace{0.00mm}
\setlength{\parindent}{0.00mm}
\setlength{\leftskip}{0.00mm}
\setlength{\rightskip}{0.00mm}

 In 1986 TDRTD was adopted by the GA with 146 votes for the declaration, 8 abstentions and 1 negative vote. At first sight it would seem that such consensus on the declaration would contribute to evidence of state practice in favour of recognising TRTD as an emerging rule, as the majority of states voted for the declaration. The fact that TRTD at this point was still a relatively new concept, would not itself bar the right from becoming an emerging international customary norm. However, the key to understanding whether or not TDRTD can be used as evidence of state practice comes from analysing the negative vote and the abstentions. 
\vspace{0.00mm}

\vspace{0.00mm}
\setlength{\parindent}{0.00mm}
\setlength{\leftskip}{0.00mm}
\setlength{\rightskip}{0.00mm}


\vspace{0.00mm}

\vspace{0.00mm}
\setlength{\parindent}{0.00mm}
\setlength{\leftskip}{0.00mm}
\setlength{\rightskip}{0.00mm}

The USA cast the only negative vote. State practice must include participation of those states who are likely to be the most affected by the new rule. It can be taken for granted$^{}$ that the USA is a state whose ``\textit{interests are specially affected}'' and thus its negative vote reveals that participation in the state practice is not representative of the whole international community, concluding that, at this stage, TRTD could not evolve into a rule of general international customary law. Without even analysing the abstentions$^{}$, TDRTD received a significant blow with the negative vote cast by the USA. For this reason, it can not be used as evidence of state practice as it effectively reveals the contrary -- that of discontent. 
\vspace{0.00mm}

\vspace{0.00mm}
\setlength{\parindent}{0.00mm}
\setlength{\leftskip}{0.00mm}
\setlength{\rightskip}{0.00mm}


\vspace{0.00mm}

\vspace{0.00mm}
\setlength{\parindent}{0.00mm}
\setlength{\leftskip}{0.00mm}
\setlength{\rightskip}{0.00mm}

The Working Group of governmental experts on the right to development continued to clarify the content of TRTD and although consensus had not been reached regarding certain aspects of the right, it was endorsed at the \textbf{Rio Conference on Environment and Development 1992 }in the \textbf{Rio Declaration on Environment and Development}\textbf{ }(Rio Declaration). This was the first occasion in which TRTD was fully affirmed by the international community. At first sight, it can be used as evidence of state practice, which is uniform and representative of the whole international community as it was adopted unanimously. The Rio Declaration is a significant statement of the general rights and obligations of states which affect the environment and constitutes a statement of existing customary rules and newly formulated principles. It has, however, been described as somewhat of a `\textit{package deal}'$^{}$, balancing the interests of developing and developed states. This can be seen throughout the declaration as some provisions reflect the interests of the developed states$^{}$, while others reflect those of the developing countries$^{}$. TRTD was strongly supported by the developing states, and Principle 3 stipulated that; \textit{``The right to development must be fulfilled so as to equitably meet developmental and environmental needs of present and future generations''}. Principle 4$^{}$ was supported by developed states and what is striking about the two principles is their relationship to the principle of sustainable development. As Birnie and Boyle noted;
\vspace{2.08mm}

\vspace{0.00mm}
\setlength{\parindent}{0.00mm}
\setlength{\leftskip}{9.84mm}
\setlength{\rightskip}{9.84mm}


\vspace{4.91mm}

\vspace{0.00mm}
\setlength{\parindent}{0.00mm}
\setlength{\leftskip}{9.84mm}
\setlength{\rightskip}{9.84mm}

``One illustration of the declarations package-deal character is the conjunction of Principles 3 and 4, which together form the core of the principle of sustainable development.$^{}$'' 
\vspace{4.91mm}

\vspace{0.00mm}
\setlength{\parindent}{0.00mm}
\setlength{\leftskip}{9.84mm}
\setlength{\rightskip}{9.84mm}


\vspace{4.91mm}

\vspace{0.00mm}
\setlength{\parindent}{0.00mm}
\setlength{\leftskip}{0.00mm}
\setlength{\rightskip}{0.00mm}

The Rio Declaration not only linked the development discourse with human rights but also linked these with environmental protection. This is of considerable importance as such links had not been previously acknowledged by the international community. 
\vspace{2.08mm}

\vspace{0.00mm}
\setlength{\parindent}{0.00mm}
\setlength{\leftskip}{0.00mm}
\setlength{\rightskip}{0.00mm}


\vspace{2.08mm}

\vspace{0.00mm}
\setlength{\parindent}{0.00mm}
\setlength{\leftskip}{0.00mm}
\setlength{\rightskip}{0.00mm}

Birnie and Boyle criticise TRTD on the grounds that it does not employ any language appropriate to environmental protection or sustainability and state that ``\textit{[t}]\textit{his is a classical illustration of the UN's inability to do joined up thinking}$^{}$''. This is a valid comment as TRTD, had up until Rio, concentrated on integrating the development discourse and the human rights discourse and did not specifically relate to environmental protection and sustainable development.  Nevertheless, the affirmation of TRTD in the Rio Declaration is of extreme significance as it was adopted unanimously and expresses the will of the international community. A shadow of doubt can be cast over such unanimity as evidence of state practice as the USA made several reservations to some of the provisions in the declaration, which included a reservation on Principle 3$^{}$. However, Birnie and Boyle state that despite these reservations, the principles and rules contained in the Rio Declaration ``\textit{have a universal significance and cannot be dismissed as the work of one segment of international society}\textit{$^{}$}'' and can thus be used as evidence of state practice toward the establishment of TRTD as an emerging customary rule, whereby the international community would be obliged to realise the right.  However as state practice must be consistent, more evidence is required to conclude this. Further evidence of state practice could include GA resolution 47/123$^{}$ which reaffirmed TRTD as a human right and recalled the principles of the Rio Declaration. 
\vspace{2.08mm}

\vspace{0.00mm}
\setlength{\parindent}{0.00mm}
\setlength{\leftskip}{0.00mm}
\setlength{\rightskip}{0.00mm}


\vspace{2.08mm}

\vspace{0.00mm}
\setlength{\parindent}{0.00mm}
\setlength{\leftskip}{0.00mm}
\setlength{\rightskip}{0.00mm}

The\textbf{ Vienna World Conference on Human Rights 1993 }resulted in the adoption of the Vienna Declaration and Programme of Action (Vienna Declaration). This dealt with TRTD extensively$^{}$, reaffirming it as a ``\textit{universal and inalienable human right''} as well as recognising it as ``\textit{an integral part of fundamental human rights''}. Sengupta argues that the affirmation of TRTD by consensus in the Vienna Declaration settles the debate surrounding whether or not TRTD is a human right, to the affirmative$^{}$. While Sengupta may believe this to be true, I would argue that the debate is far from settled.
\vspace{2.08mm}

\vspace{0.00mm}
\setlength{\parindent}{0.00mm}
\setlength{\leftskip}{0.00mm}
\setlength{\rightskip}{0.00mm}

The Vienna declaration recognised the universality, indivisibility and interdependency of all human rights which was also significant in TDRTD. It emphasised the human person is the central subject of development and underscored that; 
\vspace{2.08mm}

\vspace{0.00mm}
\setlength{\parindent}{0.00mm}
\setlength{\leftskip}{0.00mm}
\setlength{\rightskip}{0.00mm}


\vspace{2.08mm}

\vspace{0.00mm}
\setlength{\parindent}{0.00mm}
\setlength{\leftskip}{9.84mm}
\setlength{\rightskip}{9.84mm}
\raggedright
``While development facilitates the enjoyment of all human rights, the lack of development may not be invoked to justify the abridgement of internationally recognised human rights.''
\vspace{4.91mm}

\vspace{0.00mm}
\setlength{\parindent}{0.00mm}
\setlength{\leftskip}{9.84mm}
\setlength{\rightskip}{9.84mm}
\raggedright

\vspace{4.91mm}

\vspace{0.00mm}
\setlength{\parindent}{0.00mm}
\setlength{\leftskip}{0.00mm}
\setlength{\rightskip}{0.00mm}

This element was added to the Declaration on behalf of states who feared that development would be used as an excuse by repressive governments to limit other human rights. Boven$^{}$ has suggested that the above statement questions the notion of indivisibility and interdependency of all human rights, which the Vienna Declaration notably promotes as it ``\textit{conveys a dual notion of human rights}$^{}$''. He elaborates stating that;
\vspace{2.08mm}

\vspace{0.00mm}
\setlength{\parindent}{0.00mm}
\setlength{\leftskip}{0.00mm}
\setlength{\rightskip}{0.00mm}


\vspace{2.08mm}

\vspace{0.00mm}
\setlength{\parindent}{0.00mm}
\setlength{\leftskip}{9.84mm}
\setlength{\rightskip}{9.84mm}

 ``With `all human rights' the whole range of human rights is obviously included but it appears from the context of the statement that the term `internationally recognised human rights' basically relates to civil and political rights only.$^{}$''  
\vspace{4.91mm}

\vspace{0.00mm}
\setlength{\parindent}{0.00mm}
\setlength{\leftskip}{0.00mm}
\setlength{\rightskip}{0.00mm}


\vspace{2.08mm}

\vspace{0.00mm}
\setlength{\parindent}{0.00mm}
\setlength{\leftskip}{0.00mm}
\setlength{\rightskip}{0.00mm}

Such an analysis of the text does question what the Vienna Declaration meant by using the phrase `\textit{internationally recognised human rights'} rather than just using `\textit{human rights'} which would undeniably denote indivisibility and interdependency of all human rights. Nevertheless, the Vienna Declaration also recognised that democracy, development and respect for human rights and fundamental freedoms are not only interdependent but also mutually reinforcing. It also recognised, \textit{``[r]espect for human rights and for fundamental freedoms without distinction of any kind is a fundamental rule of international human rights law}\textit{$^{}$}\textit{''}. This reaffirms the principle of non-discrimination contained in previous human rights instruments and TDRTD as fundamental and arguably legally binding in itself$^{}$. 
\vspace{2.08mm}

\vspace{0.00mm}
\setlength{\parindent}{0.00mm}
\setlength{\leftskip}{0.00mm}
\setlength{\rightskip}{0.00mm}


\vspace{2.08mm}

\vspace{0.00mm}
\setlength{\parindent}{0.00mm}
\setlength{\leftskip}{0.00mm}
\setlength{\rightskip}{0.00mm}

The Vienna Declaration clearly stipulates that extreme poverty affects the capabilities of people to enjoy their human rights. Paragraph 14 unequivocally states that;
\vspace{2.08mm}

\vspace{0.00mm}
\setlength{\parindent}{0.00mm}
\setlength{\leftskip}{0.00mm}
\setlength{\rightskip}{0.00mm}


\vspace{2.08mm}

\vspace{0.00mm}
\setlength{\parindent}{0.00mm}
\setlength{\leftskip}{9.84mm}
\setlength{\rightskip}{9.84mm}

 ``The existence of widespread extreme poverty inhibits the full and effective enjoyment of human rights; its immediate alleviation and eventual elimination must remain a high priority for the international community.''
\vspace{4.91mm}

\vspace{0.00mm}
\setlength{\parindent}{0.00mm}
\setlength{\leftskip}{9.84mm}
\setlength{\rightskip}{9.84mm}


\vspace{4.91mm}

\vspace{0.00mm}
\setlength{\parindent}{0.00mm}
\setlength{\leftskip}{0.00mm}
\setlength{\rightskip}{0.00mm}

Thus although TDRTD did not expressly mention extreme poverty, references made in the Vienna Declaration expressly link poverty and human rights together and in turn as it affirms TRTD as a human right, exteme poverty affects the realisation of TRTD. 
\vspace{2.08mm}

\vspace{0.00mm}
\setlength{\parindent}{0.00mm}
\setlength{\leftskip}{0.00mm}
\setlength{\rightskip}{0.00mm}


\vspace{2.08mm}

\vspace{0.00mm}
\setlength{\parindent}{0.00mm}
\setlength{\leftskip}{0.00mm}
\setlength{\rightskip}{0.00mm}

The declaration linked environmental protection with the TRTD by including; ``\textit{[t]he right to development should be fulfilled so as to meet equitably the developmental and environmental needs of present and future generations}\textit{$^{}$}\textit{''}. Thus once again the notion that the environment, development and human rights are interrelated emerged, as it had done so in the Rio Declaration. 
\vspace{2.08mm}

\vspace{0.00mm}
\setlength{\parindent}{0.00mm}
\setlength{\leftskip}{0.00mm}
\setlength{\rightskip}{0.00mm}

The Vienna Declaration -- similar to TDRTD -- stipulates that effective international cooperation should be promoted by the international community and it emphasises that states should cooperate with each other to ensure development and eliminate the obstacles to development. However, the Vienna Declaration went beyond TDRTD, linking the external debt problems of states to the effective realisation of ESCR's. The Vienna Declaration;
\vspace{2.08mm}

\vspace{0.00mm}
\setlength{\parindent}{0.00mm}
\setlength{\leftskip}{0.00mm}
\setlength{\rightskip}{0.00mm}


\vspace{2.08mm}

\vspace{0.00mm}
\setlength{\parindent}{0.00mm}
\setlength{\leftskip}{9.84mm}
\setlength{\rightskip}{9.84mm}
\raggedright
 ``[C]alls upon the international community to make all efforts to help alleviate the external debt burden of developing countries, in order to supplement the efforts of the Governments of such countries to attain the full realisation of the Economic, social and cultural rights of their people.$^{}$''   
\vspace{4.91mm}

\vspace{0.00mm}
\setlength{\parindent}{0.00mm}
\setlength{\leftskip}{9.84mm}
\setlength{\rightskip}{9.84mm}
\raggedright

\vspace{4.91mm}

\vspace{0.00mm}
\setlength{\parindent}{0.00mm}
\setlength{\leftskip}{0.00mm}
\setlength{\rightskip}{0.00mm}

This paragraph can be interpreted to imply that states external debt problems are hindering the realisation of all ESCR's, which in turn would affect the realisation of TRTD. Thus, if it is found that the international community is legally obligated to realise TRTD, the issue of alleviating external debt burdens, could fall under this obligation.  
\vspace{2.08mm}

\vspace{0.00mm}
\setlength{\parindent}{0.00mm}
\setlength{\leftskip}{0.00mm}
\setlength{\rightskip}{0.00mm}

 
\vspace{2.08mm}

\vspace{0.00mm}
\setlength{\parindent}{0.00mm}
\setlength{\leftskip}{0.00mm}
\setlength{\rightskip}{0.00mm}

As consensus was reached by the international community on the Vienna Declaration, it could be used as evidence of state practice toward establishing TRTD as a customary rule. Furthermore, the obligation to realise the right by the international community, through effective policies, elimination of obstacles and an appropriate international environment which is conducive to the realisation of TRTD seems to be strongly supported by the Vienna Declaration$^{}$. This could be used to suggest that the obligation for international cooperation for the realisation of TRTD was emerging as a customary norm, although the Vienna Declaration is not legally binding in itself. However, the provisions of the declaration relating to cooperation only infer that states `should' cooperate, implying that it is a goal rather than a legal obligation. This casts doubts over the relevance of the Vienna Declaration as evidence of state practice on the emergence of TRTD, including the element of international cooperation, as legally binding customary rules. Moreover, what is notable about the Vienna Declaration is that although it expressly reaffirms TRTD, it does not attempt to define development or contribute to the clarification of this aspect of TRTD. 
\vspace{2.08mm}

\vspace{0.00mm}
\setlength{\parindent}{0.00mm}
\setlength{\leftskip}{0.00mm}
\setlength{\rightskip}{0.00mm}

The Vienna Declaration and its provisions on TRTD have been readily endorsed in subsequent GA resolutions$^{}$, offering some evidence of state practice towards TRTD being accepted by the international community as binding customary law - or at least reveals the way in which international law is developing on TRTD. 
\vspace{2.08mm}

\vspace{0.00mm}
\setlength{\parindent}{0.00mm}
\setlength{\leftskip}{0.00mm}
\setlength{\rightskip}{0.00mm}


\vspace{2.08mm}

\vspace{0.00mm}
\setlength{\parindent}{0.00mm}
\setlength{\leftskip}{0.00mm}
\setlength{\rightskip}{0.00mm}

The 1990's was a decade in which a number of international conferences endorsed TRTD as a human right. These are as follows; 
\vspace{0.00mm}

\vspace{0.00mm}
\setlength{\parindent}{0.00mm}
\setlength{\leftskip}{0.00mm}
\setlength{\rightskip}{0.00mm}


\vspace{0.00mm}

\vspace{0.00mm}
\setlength{\parindent}{0.00mm}
\setlength{\leftskip}{0.00mm}
\setlength{\rightskip}{0.00mm}

\textbf{International Conference on Population and Development, Cairo, 1994}\textbf{$^{}$}\textbf{}
\vspace{0.00mm}

\vspace{0.00mm}
\setlength{\parindent}{0.00mm}
\setlength{\leftskip}{0.00mm}
\setlength{\rightskip}{0.00mm}

\textbf{World Summit for Social Development, Copenhagen, 1995}\textbf{$^{}$} 
\vspace{0.00mm}

\vspace{0.00mm}
\setlength{\parindent}{0.00mm}
\setlength{\leftskip}{0.00mm}
\setlength{\rightskip}{0.00mm}

\textbf{The Platform for Action 1995 Fourth World Conference on Women}\textbf{$^{}$} 
\vspace{0.00mm}

\vspace{0.00mm}
\setlength{\parindent}{0.00mm}
\setlength{\leftskip}{0.00mm}
\setlength{\rightskip}{0.00mm}

\textbf{The World Food Summit 1996}\textbf{$^{}$}\textbf{}
\vspace{0.00mm}

\vspace{0.00mm}
\setlength{\parindent}{0.00mm}
\setlength{\leftskip}{0.00mm}
\setlength{\rightskip}{0.00mm}

\textbf{Second UN Conference on Human Settlements (Habitat II) 1996}\textbf{$^{}$} 
\vspace{0.00mm}

\vspace{0.00mm}
\setlength{\parindent}{0.00mm}
\setlength{\leftskip}{0.00mm}
\setlength{\rightskip}{0.00mm}


\vspace{0.00mm}

\vspace{0.00mm}
\setlength{\parindent}{0.00mm}
\setlength{\leftskip}{0.00mm}
\setlength{\rightskip}{0.00mm}

Although they did not deal with TRTD as extensively as the Vienna Declaration, some elements of each conference are relevant to realising TRTD and can be used as evidence of state practice. Furthermore, GA resolution 50/184$^{}$affirmed this notion and expressed that each of these conferences contained aspects which are relevant to the realisation of TRTD. The Office of the High Commissioner on Human Rights (OHCHR) concludes that; 
\vspace{0.00mm}

\vspace{0.00mm}
\setlength{\parindent}{0.00mm}
\setlength{\leftskip}{0.00mm}
\setlength{\rightskip}{0.00mm}


\vspace{0.00mm}

\vspace{0.00mm}
\setlength{\parindent}{0.00mm}
\setlength{\leftskip}{9.84mm}
\setlength{\rightskip}{9.84mm}

``The collective message of all the UN summits and conferences of the 1990's may be summed up as a call for greater recognition of human rights in development$^{}$''
\vspace{4.91mm}

\vspace{0.00mm}
\setlength{\parindent}{0.00mm}
\setlength{\leftskip}{9.84mm}
\setlength{\rightskip}{9.84mm}


\vspace{4.91mm}

\vspace{0.00mm}
\setlength{\parindent}{0.00mm}
\setlength{\leftskip}{0.00mm}
\setlength{\rightskip}{0.00mm}

GA resolution 53/155$^{}$is of considerable importance as it reveals the discontent still present regarding the realisation of TRTD. The resolution expresses concerns over the implementation of the right and states that;
\vspace{0.00mm}

\vspace{0.00mm}
\setlength{\parindent}{0.00mm}
\setlength{\leftskip}{0.00mm}
\setlength{\rightskip}{0.00mm}


\vspace{0.00mm}

\vspace{0.00mm}
\setlength{\parindent}{0.00mm}
\setlength{\leftskip}{9.84mm}
\setlength{\rightskip}{9.84mm}

``[O]bstacles to the realisation of the right to development still persist at both national and international levels, that new obstacles to the rights stated therein have emerged and that the progress made in removing these obstacles remains precarious''. 
\vspace{4.91mm}

\vspace{0.00mm}
\setlength{\parindent}{0.00mm}
\setlength{\leftskip}{9.84mm}
\setlength{\rightskip}{9.84mm}


\vspace{4.91mm}

\vspace{0.00mm}
\setlength{\parindent}{0.00mm}
\setlength{\leftskip}{0.00mm}
\setlength{\rightskip}{0.00mm}

It further expresses concern over the insufficient dissemination of TDRTD and states that TDRTD ``\textit{should be taken into account, as appropriate, in bilateral and multilateral cooperation programmes, national development strategies and policies and activities of international organisations''}. Nevertheless, this resolution affirms TRTD as stipulated in TDRTD. 
\vspace{0.00mm}

\vspace{0.00mm}
\setlength{\parindent}{0.00mm}
\setlength{\leftskip}{0.00mm}
\setlength{\rightskip}{0.00mm}
\raggedright

\vspace{0.00mm}

\vspace{0.00mm}
\setlength{\parindent}{0.00mm}
\setlength{\leftskip}{0.00mm}
\setlength{\rightskip}{0.00mm}

By 1998, TRTD had been endorsed in numerous international conferences, GA resolutions and in resolutions of the CHR. Taken from this alone, TRTD would seem to be emerging as a customary international norm. However, the key to understanding the legal status of TRTD, is in an assessment of the concerns which states have expressed, these demonstrate that the opinio juris of states to be legally bound to realise the TRTD, is not present. Some of these will now be discussed. 
\vspace{0.00mm}

\vspace{0.00mm}
\setlength{\parindent}{0.00mm}
\setlength{\leftskip}{0.00mm}
\setlength{\rightskip}{0.00mm}


\vspace{0.00mm}

\vspace{0.00mm}
\setlength{\parindent}{0.00mm}
\setlength{\leftskip}{0.00mm}
\setlength{\rightskip}{0.00mm}

In the view of the Government of Switzerland the implementation and realisation of TRTD had been hindered because of the controversies surrounding the very nature of TRTD and thus; 
\vspace{0.00mm}

\vspace{0.00mm}
\setlength{\parindent}{0.00mm}
\setlength{\leftskip}{0.00mm}
\setlength{\rightskip}{0.00mm}


\vspace{0.00mm}

\vspace{0.00mm}
\setlength{\parindent}{0.00mm}
\setlength{\leftskip}{9.84mm}
\setlength{\rightskip}{9.84mm}

``In order to facilitate its implementation, certain controversial points, such as the very concept of the right to development, need to be clarified... Despite many years of discussion and debate, there is still uncertainty over the Declaration's content.... The only certainty is that ``the human person is the central subject of development''.$^{}$
\vspace{4.91mm}

\vspace{0.00mm}
\setlength{\parindent}{0.00mm}
\setlength{\leftskip}{9.84mm}
\setlength{\rightskip}{9.84mm}


\vspace{4.91mm}

\vspace{0.00mm}
\setlength{\parindent}{0.00mm}
\setlength{\leftskip}{0.00mm}
\setlength{\rightskip}{0.00mm}

Such a statement is a grave blow towards the recognition of TRTD as emerging customary international law. It identifies that rather than clarifying TRTD, the events which have taken place since TDRTD have contributed very little to solving the problems inherent in the right. At this point, it would seem very relevant that Switzerland identifies that the very concept of TRTD is still very controversial as although the right has been endorsed on many occasions, none of these have elaborated on the actual concept. What seems to have happened is that one declaration or conference relies upon another as evidence. In turn, this falls back on to TDRTD, which is fundamentally the source of the controversies because of it's ambiguity. These notions are further expressed by Ghana. Underlying the need to elaborate on the content of TRTD, the Government of Ghana notes that the definition of development needs clarification$^{}$ as well as ``\textit{The expressions ``individually'' and ``collectively'' should be better defined so as to distinguish individual and collective rights, as well as States' rights''}$^{}$. At first sight it would seem that this was a straight forward criticism concerning the uncertainty surrounding TRTD in general. However, read more closely, it is evident that even at this stage, the fact that Ghana expressly mentions `\textit{State's rights}' , reveals that the controversies surrounding TRTD are far from being solved. In fact it could be said that the efforts made during the 1990's to recognise TRTD were made in vain and were merely rhetoric as the Government of India noted$^{}$$^{.}$ While taking this account, it seems impossible to conclude -- at this point -- that there is uniform and consistent state practice alongside \textit{opinio juris}, which are necessary to establish international customary law. 
\vspace{0.00mm}

\vspace{0.00mm}
\setlength{\parindent}{0.00mm}
\setlength{\leftskip}{0.00mm}
\setlength{\rightskip}{0.00mm}
\raggedright

\vspace{0.00mm}

\vspace{0.00mm}
\setlength{\parindent}{0.00mm}
\setlength{\leftskip}{0.00mm}
\setlength{\rightskip}{0.00mm}

Although the resolutions and declarations on TRTD are not completely invalid as the ``\textit{mere recognition of a rule and the conditions for its execution in a resolution give it the beginning of legal force}''$^{}$.  From the above analysis it seems certain that TRTD has not yet received the status of `hard law' or customary international law. However, Gutto argues that as TDRTD directly links to the UN Charter and can be seen as an authoritative interpretation of the Charter itself and confidently asserts;
\vspace{0.00mm}

\vspace{0.00mm}
\setlength{\parindent}{0.00mm}
\setlength{\leftskip}{0.00mm}
\setlength{\rightskip}{0.00mm}


\vspace{0.00mm}

\vspace{0.00mm}
\setlength{\parindent}{0.00mm}
\setlength{\leftskip}{9.84mm}
\setlength{\rightskip}{9.84mm}

``It can therefore be safely stated that General Assembly resolutions, especially those that directly link to the letter and spirit of the Charter of the United Nations, like resolution 41/28, have some appreciable legal authority that ``bind'' States. The fact that even countries that had abstained from voting for the resolution in 1986, like Sweden, can today openly align themselves with the instrument and attempt to integrate its core elements into the national development policies, plans, and practices is very instructive.''$^{}$
\vspace{4.91mm}

\vspace{0.00mm}
\setlength{\parindent}{0.00mm}
\setlength{\leftskip}{9.84mm}
\setlength{\rightskip}{9.84mm}


\vspace{4.91mm}

\vspace{0.00mm}
\setlength{\parindent}{0.00mm}
\setlength{\leftskip}{0.00mm}
\setlength{\rightskip}{0.00mm}

In principle, what Gutto asserts is correct and theoretically, resolutions and declarations of the GA can elucidate and develop customary law. However, his assertion is far from revealing the truth. Although he notes the case of Sweden, he fails to address the views of states like Switzerland, Ghana, the USA and India, which were discussed above. Taking all these views into account and the controversies surrounding the right, TRTD is not sufficiently clear or precise to denote any form of a legal obligation. 
\vspace{0.00mm}

\vspace{0.00mm}
\setlength{\parindent}{0.00mm}
\setlength{\leftskip}{0.00mm}
\setlength{\rightskip}{0.00mm}
\raggedright

\vspace{0.00mm}

\vspace{0.00mm}
\setlength{\parindent}{0.00mm}
\setlength{\leftskip}{0.00mm}
\setlength{\rightskip}{0.00mm}

Despite the relative discontent surrounding TRTD, it was subsequently endorsed in the \textbf{Millennium Declaration 2000}\textbf{$^{}$}, the \textbf{World Conference against Racism, Racial Discrimination, Xenophobia and Related Intolerance 2001}\textbf{$^{}$}\textbf{ }and the \textbf{World Summit on Sustainable Development 2002 }(WSSD) held in Johannesburg. Furthermore, the GA issued an annual resolution on the right$^{}$. By 2003, some states regarded TRTD as an inalienable human right$^{}$ but, as before, confusion surrounding the right was ever present. Japan confirmed its view that the concept of TRTD was not clarified and that it ``\textit{disagreed with the legal obligation of developed countries to render assistance to developing countries}$^{}$''. Furthermore, to Japan TRTD is an individual right and could not be a collective right of any sort$^{}$. 
\vspace{0.00mm}

\vspace{0.00mm}
\setlength{\parindent}{0.00mm}
\setlength{\leftskip}{0.00mm}
\setlength{\rightskip}{0.00mm}
\raggedright

\vspace{0.00mm}

\vspace{0.00mm}
\setlength{\parindent}{0.00mm}
\setlength{\leftskip}{0.00mm}
\setlength{\rightskip}{0.00mm}

In opposing resolution 58/172 the US representative stated that the resolution implies that ``\textit{that lack of development justifies the denial of internationally recognised human rights}\textit{$^{}$}''. This statement is of crucial significance as the Vienna Declaration -- which is recalled in this resolution -- explicitly states the opposite$^{}$. Furthermore, in line with the thoughts of Boven$^{}$, it infers that the USA do not recognise ESC's as human rights as they have not ratified the ICESCR; and, to them, ``\textit{internationally recognised human rights}'' are only CPR's. This is just one more factor which undermines TRTD as the USA do not recognise ESC rights as human rights at all. 
\vspace{0.00mm}

\vspace{0.00mm}
\setlength{\parindent}{0.00mm}
\setlength{\leftskip}{0.00mm}
\setlength{\rightskip}{0.00mm}
\raggedright

\vspace{0.00mm}

\vspace{0.00mm}
\setlength{\parindent}{0.00mm}
\setlength{\leftskip}{0.00mm}
\setlength{\rightskip}{0.00mm}

Furthermore the USA disagree with the resolution's call to ``\textit{prepare a concept document on a legally binding instrument on the Right to Development}\textit{$^{}$}'' and is of the opinion that ``\textit{such an effort is unwarranted in view of the fact that any such legally binding instrument is unlikely to ever garner significant support}\textit{$^{}$}\textit{''}. While hesitating to agree with this, the controversies surrounding the right reveal that it may not be possible to create a document which contains legally binding obligations on TRTD. 
\vspace{0.00mm}

\vspace{0.00mm}
\setlength{\parindent}{0.00mm}
\setlength{\leftskip}{0.00mm}
\setlength{\rightskip}{0.00mm}
\raggedright

\vspace{0.00mm}

\vspace{0.00mm}
\setlength{\parindent}{0.00mm}
\setlength{\leftskip}{0.00mm}
\setlength{\rightskip}{0.00mm}
\raggedright
In concluding their views, the USA representative states that; 
\vspace{0.00mm}

\vspace{0.00mm}
\setlength{\parindent}{0.00mm}
\setlength{\leftskip}{0.00mm}
\setlength{\rightskip}{0.00mm}
\raggedright
 
\vspace{0.00mm}

\vspace{0.00mm}
\setlength{\parindent}{0.00mm}
\setlength{\leftskip}{9.84mm}
\setlength{\rightskip}{9.84mm}

``We cannot support the call to make progress on realising the Right to Development. There is no internationally accepted definition of such a right. Making such a call is premature and irrelevant.$^{}$''   
\vspace{4.91mm}

\vspace{0.00mm}
\setlength{\parindent}{0.00mm}
\setlength{\leftskip}{9.84mm}
\setlength{\rightskip}{9.84mm}


\vspace{4.91mm}

\vspace{0.00mm}
\setlength{\parindent}{0.00mm}
\setlength{\leftskip}{0.00mm}
\setlength{\rightskip}{0.00mm}

Consequently, there is significant evidence that states have not agreed to be legally obligated to realising TRTD. 
\vspace{0.00mm}
\begin{itemize}
\begin{itemize}

\item
\vspace{4.17mm}
\setlength{\parindent}{0.00mm}
\setlength{\leftskip}{0.00mm}
\setlength{\rightskip}{0.00mm}
\raggedright
\textbf{3.3. The Right to Development in other Legal Instruments}
\vspace{2.08mm}

\end{itemize}
\end{itemize}
\vspace{0.00mm}
\setlength{\parindent}{0.00mm}
\setlength{\leftskip}{0.00mm}
\setlength{\rightskip}{0.00mm}
\raggedright

\vspace{0.00mm}

\vspace{0.00mm}
\setlength{\parindent}{0.00mm}
\setlength{\leftskip}{0.00mm}
\setlength{\rightskip}{0.00mm}
\raggedright
Many commentators have suggested that TRTD derived from several key legal instruments -- including the UN Charter and the International Bill of Rights, amongst others $^{}$ -- and the legal obligation to realise the right stems from these. While there is some merit in this line of reasoning, such arguments have now lost their force$^{}$. 
\vspace{0.00mm}

\vspace{0.00mm}
\setlength{\parindent}{0.00mm}
\setlength{\leftskip}{0.00mm}
\setlength{\rightskip}{0.00mm}
\raggedright

\vspace{0.00mm}

\vspace{0.00mm}
\setlength{\parindent}{0.00mm}
\setlength{\leftskip}{0.00mm}
\setlength{\rightskip}{0.00mm}

At the time of its emergence, explicit recognition of the right, was very minimal. It seems that, to address this problem, some commentators have attempted to base the right in well established documents. These efforts have, in most cases, augmented the controversy and, after close scrutiny of TRTD, it would be redundant to repeat the exercise. Whilst it is evident that some of the principles of TRTD can be found in well endorsed legal instruments, the whole right itself simply cannot. Nevertheless, it may be worthwhile to briefly mention how this approach can be applied to the UN Charter.
\vspace{0.00mm}

\vspace{0.00mm}
\setlength{\parindent}{0.00mm}
\setlength{\leftskip}{0.00mm}
\setlength{\rightskip}{0.00mm}
\raggedright

\vspace{0.00mm}

\vspace{0.00mm}
\setlength{\parindent}{0.00mm}
\setlength{\leftskip}{0.00mm}
\setlength{\rightskip}{0.00mm}

The preamble of the charter asserts that the Peoples of the UN are determined ``t\textit{o promote social progress and better standards of life in larger freedom}''. This has led to the claim that ``[\textit{t]he relationship between the right to development and other human rights had already been stated in the preamble of the United Nations Charter''}$^{}$. M'Baye argued that TRTD ``\textit{is already enscribed (sic) in filigree in the Charter of the United Nation''}$^{}$. The core aspects of TRTD echo one of the main purposes of the UN. From Articles 1(3), Article 55 and 56, it can be implied that international cooperation is a legal obligation to realise human rights, but international cooperation as a legal obligation to realise TRTD cannot be deduced from these. A number of provisions in the International Bill of Rights have been used in these discussions as implicit evidence of TRTD$^{}$.  
\vspace{0.00mm}
\begin{itemize}

\item
\vspace{4.17mm}
\setlength{\parindent}{0.00mm}
\setlength{\leftskip}{0.00mm}
\setlength{\rightskip}{0.00mm}
\raggedright
\textbf{}
\vspace{2.08mm}

\end{itemize}
\vspace{0.00mm}
\setlength{\parindent}{0.00mm}
\setlength{\leftskip}{0.00mm}
\setlength{\rightskip}{0.00mm}
\raggedright
\newpage

\vspace{0.00mm}
\begin{itemize}

\item
\vspace{4.17mm}
\setlength{\parindent}{0.00mm}
\setlength{\leftskip}{0.00mm}
\setlength{\rightskip}{0.00mm}
\raggedright
\textbf{Conclusion }
\vspace{2.08mm}

\end{itemize}
\vspace{0.00mm}
\setlength{\parindent}{0.00mm}
\setlength{\leftskip}{0.00mm}
\setlength{\rightskip}{0.00mm}


\vspace{0.00mm}

\vspace{0.00mm}
\setlength{\parindent}{0.00mm}
\setlength{\leftskip}{0.00mm}
\setlength{\rightskip}{0.00mm}

It is impossible to answer the question ``\textit{Is there a Legal Obligation for the International Community to Realise the Right to Development?}'' in the affirmative. This is because there is no agreement on the definition of TRTD. There is no agreement by the international community on what kind of right TRTD is; whether it is a human right, legal right, collective right, right of states, synthesis of rights etc. This results in confusion on where the right is to be placed -- within the framework of human rights or within international law as a whole. As TRTD is unclear and ambiguous, states are able to interpret the right according to their own agenda. This results in TRTD meaning different things, to different states, at different times. 
\vspace{0.00mm}

\vspace{0.00mm}
\setlength{\parindent}{0.00mm}
\setlength{\leftskip}{0.00mm}
\setlength{\rightskip}{0.00mm}


\vspace{0.00mm}

\vspace{0.00mm}
\setlength{\parindent}{0.00mm}
\setlength{\leftskip}{0.00mm}
\setlength{\rightskip}{0.00mm}

While the OEWG and the IE have concentrated their work on implementing TRTD, the conceptual difficulties surrounding the right hinders its effective realisation. It then becomes clear that the OEWG and IE have steered the debate away from the conceptual difficulties towards that of implementation. However, without clarifying the nature, scope and definition of TRTD, it cannot be realised effectively. TRTD has been built upon incomplete foundations and when the layers are stripped off, it is evident that it is conceptually flawed, therefore the aim of realising TRTD is a fallacy. I recommend that the international community, the UN GA and the Commission on Human Rights,  focus on the conceptual problems surrounding TRTD once again, clarifying the definition and content of the right. Furthermore, the relationship between sustainable development and TRTD should be analysed. This could serve to clarify some aspects of TRTD.  
\vspace{0.00mm}

\vspace{0.00mm}
\setlength{\parindent}{0.00mm}
\setlength{\leftskip}{0.00mm}
\setlength{\rightskip}{0.00mm}


\vspace{0.00mm}

\vspace{0.00mm}
\setlength{\parindent}{0.00mm}
\setlength{\leftskip}{0.00mm}
\setlength{\rightskip}{0.00mm}
\raggedright
\newpage

\vspace{0.00mm}
\begin{itemize}

\item
\vspace{4.17mm}
\setlength{\parindent}{0.00mm}
\setlength{\leftskip}{0.00mm}
\setlength{\rightskip}{0.00mm}
\raggedright
\textbf{Bibliography}
\vspace{2.08mm}

\end{itemize}
\vspace{0.00mm}
\setlength{\parindent}{-4.91mm}
\setlength{\leftskip}{4.91mm}
\setlength{\rightskip}{0.00mm}


\vspace{0.00mm}

\vspace{0.00mm}
\setlength{\parindent}{0.00mm}
\setlength{\leftskip}{0.00mm}
\setlength{\rightskip}{0.00mm}

Abi- Saab G., \textit{'The Legal Formulation of the Right to Development}', in R. Dupuy (ed), The Right to Development at the International Level,  Netherlands: Sijthoff and Noordhoff, 1981, p163.
\vspace{0.00mm}

\vspace{0.00mm}
\setlength{\parindent}{0.00mm}
\setlength{\leftskip}{0.00mm}
\setlength{\rightskip}{0.00mm}


\vspace{0.00mm}

\vspace{0.00mm}
\setlength{\parindent}{0.00mm}
\setlength{\leftskip}{0.00mm}
\setlength{\rightskip}{0.00mm}

Aguda T., \textit{Human Rights and the Right to Development in Africa}, Lagos: Nigerian Institute of International Affairs, 1989. 
\vspace{0.00mm}

\vspace{0.00mm}
\setlength{\parindent}{0.00mm}
\setlength{\leftskip}{0.00mm}
\setlength{\rightskip}{0.00mm}


\vspace{0.00mm}

\vspace{0.00mm}
\setlength{\parindent}{0.00mm}
\setlength{\leftskip}{0.00mm}
\setlength{\rightskip}{0.00mm}

Alston P., and Quinn G., \textit{The Nature and Scope of States Parties Obligations under the ICESCR}, 9 Hum Rts, Q 122, 1987  
\vspace{0.00mm}

\vspace{0.00mm}
\setlength{\parindent}{0.00mm}
\setlength{\leftskip}{0.00mm}
\setlength{\rightskip}{0.00mm}


\vspace{0.00mm}

\vspace{0.00mm}
\setlength{\parindent}{0.00mm}
\setlength{\leftskip}{0.00mm}
\setlength{\rightskip}{0.00mm}

Alston P., \textit{A third generation of solidarity rights,: progressive development or obfuscation of international human rights law?} 29 Netherlands Int'l L Rev 307, 1982
\vspace{0.00mm}

\vspace{0.00mm}
\setlength{\parindent}{0.00mm}
\setlength{\leftskip}{0.00mm}
\setlength{\rightskip}{0.00mm}


\vspace{0.00mm}

\vspace{0.00mm}
\setlength{\parindent}{0.00mm}
\setlength{\leftskip}{0.00mm}
\setlength{\rightskip}{0.00mm}

Alston P., \textit{Making Space for New Human Rights : the case of the right to development,} Harvard Human Rights Yearbook; V.1; pp.3-40
\vspace{0.00mm}

\vspace{0.00mm}
\setlength{\parindent}{0.00mm}
\setlength{\leftskip}{0.00mm}
\setlength{\rightskip}{0.00mm}


\vspace{0.00mm}

\vspace{0.00mm}
\setlength{\parindent}{0.00mm}
\setlength{\leftskip}{0.00mm}
\setlength{\rightskip}{0.00mm}

Alston P.,\textit{ Revitalising United Nations Work on Human Rights and Development}, 18 MELB U. L. Rev. 216-219
\vspace{0.00mm}

\vspace{0.00mm}
\setlength{\parindent}{0.00mm}
\setlength{\leftskip}{0.00mm}
\setlength{\rightskip}{0.00mm}


\vspace{0.00mm}

\vspace{0.00mm}
\setlength{\parindent}{0.00mm}
\setlength{\leftskip}{0.00mm}
\setlength{\rightskip}{0.00mm}

Alston P., \textit{'The Right to Development at the International Level'} in R. Dupuy (ed), The Right to Development at the International Level,  Netherlands: Sijthoff and Noordhoff, 1981
\vspace{0.00mm}

\vspace{0.00mm}
\setlength{\parindent}{0.00mm}
\setlength{\leftskip}{0.00mm}
\setlength{\rightskip}{0.00mm}


\vspace{0.00mm}

\vspace{0.00mm}
\setlength{\parindent}{0.00mm}
\setlength{\leftskip}{0.00mm}
\setlength{\rightskip}{0.00mm}

Alston P., '\textit{The Shortcomings of a ``Garfield the Cat'' Approach to the Right to Development'}, California Western International Law Journal, Vol 15 (1985), 510.
\vspace{0.00mm}

\vspace{0.00mm}
\setlength{\parindent}{0.00mm}
\setlength{\leftskip}{0.00mm}
\setlength{\rightskip}{0.00mm}


\vspace{0.00mm}

\vspace{0.00mm}
\setlength{\parindent}{0.00mm}
\setlength{\leftskip}{0.00mm}
\setlength{\rightskip}{0.00mm}

Alston P., \textit{`What is in a Name: Does it really matter if development refers to goals, ideals of human rights'?} in H. Helmich (Ed.), Human Rights in Development Cooperation, SIM Special No 22 pp.95-106, Netherlands Institute for Human Rights, Utrecht, 1998. 
\vspace{0.00mm}

\vspace{0.00mm}
\setlength{\parindent}{0.00mm}
\setlength{\leftskip}{0.00mm}
\setlength{\rightskip}{0.00mm}


\vspace{0.00mm}

\vspace{0.00mm}
\setlength{\parindent}{0.00mm}
\setlength{\leftskip}{0.00mm}
\setlength{\rightskip}{0.00mm}

Alston P., '\textit{The Myopia of the Herdmaidens: International Lawyers and Globalisation'}, 8 European Journal of International Law (1997) 435 
\vspace{0.00mm}

\vspace{0.00mm}
\setlength{\parindent}{0.00mm}
\setlength{\leftskip}{0.00mm}
\setlength{\rightskip}{0.00mm}


\vspace{0.00mm}

\vspace{0.00mm}
\setlength{\parindent}{0.00mm}
\setlength{\leftskip}{0.00mm}
\setlength{\rightskip}{0.00mm}

Alston P., (ed), \textit{Peoples' Rights}, Oxford: Oxford University Press, 2001. 
\vspace{0.00mm}

\vspace{0.00mm}
\setlength{\parindent}{0.00mm}
\setlength{\leftskip}{0.00mm}
\setlength{\rightskip}{0.00mm}


\vspace{0.00mm}

\vspace{0.00mm}
\setlength{\parindent}{0.00mm}
\setlength{\leftskip}{0.00mm}
\setlength{\rightskip}{0.00mm}

Aristotle., \textit{The politics.} Rev ed (translated by TA Sinclair, revised and re-presented by TJ Saunders).  Harmondsworth: Penguin, 1992.
\vspace{0.00mm}

\vspace{0.00mm}
\setlength{\parindent}{0.00mm}
\setlength{\leftskip}{0.00mm}
\setlength{\rightskip}{0.00mm}


\vspace{0.00mm}

\vspace{0.00mm}
\setlength{\parindent}{0.00mm}
\setlength{\leftskip}{0.00mm}
\setlength{\rightskip}{0.00mm}

Baehr P R., \textit{Human rights universality in practice }.Basingstoke: Palgrave, 2001.
\vspace{0.00mm}

\vspace{0.00mm}
\setlength{\parindent}{0.00mm}
\setlength{\leftskip}{0.00mm}
\setlength{\rightskip}{0.00mm}


\vspace{0.00mm}

\vspace{0.00mm}
\setlength{\parindent}{0.00mm}
\setlength{\leftskip}{0.00mm}
\setlength{\rightskip}{0.00mm}

Barsh R., '\textit{The Right to Development as a Human Right: Results of the Global Consultation'}, Human \hfill{}Rights Quarterly, Vol 13 (1991), 321. 
\vspace{0.00mm}

\vspace{0.00mm}
\setlength{\parindent}{0.00mm}
\setlength{\leftskip}{0.00mm}
\setlength{\rightskip}{0.00mm}


\vspace{0.00mm}

\vspace{0.00mm}
\setlength{\parindent}{0.00mm}
\setlength{\leftskip}{0.00mm}
\setlength{\rightskip}{0.00mm}

Bentham J.,\textit{ Of laws in general }(edited by HLA Hart). Athlone Press, 1970
\vspace{0.00mm}

\vspace{0.00mm}
\setlength{\parindent}{0.00mm}
\setlength{\leftskip}{0.00mm}
\setlength{\rightskip}{0.00mm}


\vspace{0.00mm}

\vspace{0.00mm}
\setlength{\parindent}{0.00mm}
\setlength{\leftskip}{0.00mm}
\setlength{\rightskip}{0.00mm}

Bentham J., \textit{The principles of morals and legislation}. Prometheus, 1988
\vspace{0.00mm}

\vspace{0.00mm}
\setlength{\parindent}{0.00mm}
\setlength{\leftskip}{0.00mm}
\setlength{\rightskip}{0.00mm}


\vspace{0.00mm}

\vspace{0.00mm}
\setlength{\parindent}{0.00mm}
\setlength{\leftskip}{0.00mm}
\setlength{\rightskip}{0.00mm}

Birnie P., and Boyle A., \textit{International Law \& The Environment}, 2$^{nd}$ ed, Oxford: Oxford University Press, 2002.
\vspace{0.00mm}

\vspace{0.00mm}
\setlength{\parindent}{0.00mm}
\setlength{\leftskip}{0.00mm}
\setlength{\rightskip}{0.00mm}


\vspace{0.00mm}

\vspace{0.00mm}
\setlength{\parindent}{0.00mm}
\setlength{\leftskip}{0.00mm}
\setlength{\rightskip}{0.00mm}

Boven TV.,\textit{ Human Rights and Rights of Peoples}, 6 EJIL (1995) 1-476,
\vspace{0.00mm}

\vspace{0.00mm}
\setlength{\parindent}{0.00mm}
\setlength{\leftskip}{0.00mm}
\setlength{\rightskip}{0.00mm}


\vspace{0.00mm}

\vspace{0.00mm}
\setlength{\parindent}{0.00mm}
\setlength{\leftskip}{0.00mm}
\setlength{\rightskip}{0.00mm}

Brownlie I.,  \textit{Basic documents on human rights}. 4th ed, Oxford, Oxford UP, 2002.
\vspace{0.00mm}

\vspace{0.00mm}
\setlength{\parindent}{0.00mm}
\setlength{\leftskip}{0.00mm}
\setlength{\rightskip}{0.00mm}


\vspace{0.00mm}

\vspace{0.00mm}
\setlength{\parindent}{0.00mm}
\setlength{\leftskip}{0.00mm}
\setlength{\rightskip}{0.00mm}

Brownlie I., \textit{Principles of public international law}, 5th ed, Oxford, OUP, 1998.
\vspace{0.00mm}

\vspace{0.00mm}
\setlength{\parindent}{0.00mm}
\setlength{\leftskip}{0.00mm}
\setlength{\rightskip}{0.00mm}


\vspace{0.00mm}

\vspace{0.00mm}
\setlength{\parindent}{0.00mm}
\setlength{\leftskip}{0.00mm}
\setlength{\rightskip}{0.00mm}

Bunn I D., '\textit{The Right to Development: Implications for International Economic Law'}, 15 American University International Law Review 1425 (2000): 1425-67
\vspace{0.00mm}

\vspace{0.00mm}
\setlength{\parindent}{0.00mm}
\setlength{\leftskip}{0.00mm}
\setlength{\rightskip}{0.00mm}


\vspace{0.00mm}

\vspace{0.00mm}
\setlength{\parindent}{0.00mm}
\setlength{\leftskip}{0.00mm}
\setlength{\rightskip}{0.00mm}

Center for Development and Human Rights., \textit{The Right to Development: A Primer}, India: Sage Publishers, 2004
\vspace{0.00mm}

\vspace{0.00mm}
\setlength{\parindent}{0.00mm}
\setlength{\leftskip}{0.00mm}
\setlength{\rightskip}{0.00mm}


\vspace{0.00mm}

\vspace{0.00mm}
\setlength{\parindent}{0.00mm}
\setlength{\leftskip}{0.00mm}
\setlength{\rightskip}{0.00mm}

Chapman A., \textit{A Violations Approach for Monitoring the International Covenant on Economic Social \hfill{}and Cultural rights}, 18 Hum Rts Q 23, 1996 
\vspace{0.00mm}

\vspace{0.00mm}
\setlength{\parindent}{0.00mm}
\setlength{\leftskip}{0.00mm}
\setlength{\rightskip}{0.00mm}


\vspace{0.00mm}

\vspace{0.00mm}
\setlength{\parindent}{0.00mm}
\setlength{\leftskip}{0.00mm}
\setlength{\rightskip}{0.00mm}

Charlesworth H., \textit{'The Public/Private Distinction and the Right to Development in International Law', Australian Year Book of International Law}, Vol 12 (1992), 190-204.
\vspace{0.00mm}

\vspace{0.00mm}
\setlength{\parindent}{0.00mm}
\setlength{\leftskip}{0.00mm}
\setlength{\rightskip}{0.00mm}


\vspace{0.00mm}

\vspace{0.00mm}
\setlength{\parindent}{0.00mm}
\setlength{\leftskip}{0.00mm}
\setlength{\rightskip}{0.00mm}

Chowdhury R., Denters E. and de Waart P. (eds), \textit{The Right to Development in International Law}, Dordrecht: Martinus Nijhoff, 1992.
\vspace{0.00mm}

\vspace{0.00mm}
\setlength{\parindent}{0.00mm}
\setlength{\leftskip}{0.00mm}
\setlength{\rightskip}{0.00mm}


\vspace{0.00mm}

\vspace{0.00mm}
\setlength{\parindent}{0.00mm}
\setlength{\leftskip}{0.00mm}
\setlength{\rightskip}{0.00mm}

Craven M., \textit{The International Covenant on Economic, Social and Cultural Rights: a perspective on its development}. Oxford, Clarendon Press, 1995
\vspace{0.00mm}

\vspace{0.00mm}
\setlength{\parindent}{0.00mm}
\setlength{\leftskip}{0.00mm}
\setlength{\rightskip}{0.00mm}


\vspace{0.00mm}

\vspace{0.00mm}
\setlength{\parindent}{0.00mm}
\setlength{\leftskip}{0.00mm}
\setlength{\rightskip}{0.00mm}

De Feyter K., \textit{World development law : Sharing Responsibility for development}, Insentia, Belgium, 2001 
\vspace{0.00mm}

\vspace{0.00mm}
\setlength{\parindent}{0.00mm}
\setlength{\leftskip}{0.00mm}
\setlength{\rightskip}{0.00mm}


\vspace{0.00mm}

\vspace{0.00mm}
\setlength{\parindent}{0.00mm}
\setlength{\leftskip}{0.00mm}
\setlength{\rightskip}{0.00mm}

Dixon M., \textit{Cases and materials on international law}, 3$^{rd}$ ed, London, Blackstone, 2000.
\vspace{0.00mm}

\vspace{0.00mm}
\setlength{\parindent}{0.00mm}
\setlength{\leftskip}{0.00mm}
\setlength{\rightskip}{0.00mm}


\vspace{0.00mm}

\vspace{0.00mm}
\setlength{\parindent}{0.00mm}
\setlength{\leftskip}{0.00mm}
\setlength{\rightskip}{0.00mm}

Donnelly J., \textit{Human Rights as Natural Rights}, 4 Hum Rts Q 391, 1982 
\vspace{0.00mm}

\vspace{0.00mm}
\setlength{\parindent}{0.00mm}
\setlength{\leftskip}{0.00mm}
\setlength{\rightskip}{0.00mm}


\vspace{0.00mm}

\vspace{0.00mm}
\setlength{\parindent}{0.00mm}
\setlength{\leftskip}{0.00mm}
\setlength{\rightskip}{0.00mm}

Donnelly J., \textit{Human Rights, Democracy and Development}, Human Rights Quarterly; V.21(3); pp.608-\hfill{}633\\

\vspace{0.00mm}

\vspace{0.00mm}
\setlength{\parindent}{0.00mm}
\setlength{\leftskip}{0.00mm}
\setlength{\rightskip}{0.00mm}

Donnelly J., \textit{In Search of the Unicorn : The Jurisprudence and Politics of the Right to Development}, 15 Cal W Int'l L J 473, 1985
\vspace{0.00mm}

\vspace{0.00mm}
\setlength{\parindent}{0.00mm}
\setlength{\leftskip}{0.00mm}
\setlength{\rightskip}{0.00mm}


\vspace{0.00mm}

\vspace{0.00mm}
\setlength{\parindent}{0.00mm}
\setlength{\leftskip}{0.00mm}
\setlength{\rightskip}{0.00mm}

Donnelly J., \textit{Universal Human Rights in Theory and Practice}, Ithaca: Cornell University Press, 1989
\vspace{0.00mm}

\vspace{0.00mm}
\setlength{\parindent}{0.00mm}
\setlength{\leftskip}{0.00mm}
\setlength{\rightskip}{0.00mm}


\vspace{0.00mm}

\vspace{0.00mm}
\setlength{\parindent}{0.00mm}
\setlength{\leftskip}{0.00mm}
\setlength{\rightskip}{0.00mm}

Douzinas C., \textit{The end of human rights critical legal thought at the turn of the century}. Oxford: Hart, 2000
\vspace{0.00mm}

\vspace{0.00mm}
\setlength{\parindent}{0.00mm}
\setlength{\leftskip}{0.00mm}
\setlength{\rightskip}{0.00mm}


\vspace{0.00mm}

\vspace{0.00mm}
\setlength{\parindent}{0.00mm}
\setlength{\leftskip}{0.00mm}
\setlength{\rightskip}{0.00mm}

Dreze J., \textit{The political economy of hunger selected essays}, Oxford: Clarendon Press, 1994
\vspace{0.00mm}

\vspace{0.00mm}
\setlength{\parindent}{0.00mm}
\setlength{\leftskip}{0.00mm}
\setlength{\rightskip}{0.00mm}


\vspace{0.00mm}

\vspace{0.00mm}
\setlength{\parindent}{0.00mm}
\setlength{\leftskip}{0.00mm}
\setlength{\rightskip}{0.00mm}

Dupuy R., (ed), \textit{The Right to Development at the International Level},  Netherlands: Sijthoff and Noordhoff, 1981
\vspace{0.00mm}

\vspace{0.00mm}
\setlength{\parindent}{0.00mm}
\setlength{\leftskip}{0.00mm}
\setlength{\rightskip}{0.00mm}


\vspace{0.00mm}

\vspace{0.00mm}
\setlength{\parindent}{0.00mm}
\setlength{\leftskip}{0.00mm}
\setlength{\rightskip}{0.00mm}

Eide A., Krause C., Rosas A., \textit{'Economic, Social and Cultural Rights: A Textbook'},  The Netherlands: Martinus Nijhoff Publishers, 1995.
\vspace{0.00mm}

\vspace{0.00mm}
\setlength{\parindent}{0.00mm}
\setlength{\leftskip}{0.00mm}
\setlength{\rightskip}{0.00mm}


\vspace{0.00mm}

\vspace{0.00mm}
\setlength{\parindent}{0.00mm}
\setlength{\leftskip}{0.00mm}
\setlength{\rightskip}{0.00mm}

Frankovits A., '\textit{Rejoinder:The Rights Way to Development}', Food Policy 21 no 1, 123-28. 
\vspace{0.00mm}

\vspace{0.00mm}
\setlength{\parindent}{0.00mm}
\setlength{\leftskip}{0.00mm}
\setlength{\rightskip}{0.00mm}


\vspace{0.00mm}

\vspace{0.00mm}
\setlength{\parindent}{0.00mm}
\setlength{\leftskip}{0.00mm}
\setlength{\rightskip}{0.00mm}

Galenkamp M., \textit{Collective Rights, }Report to the Advisory committee on Human Rights and Foreign Policy of the Netherlands, Rotterdam, Netherlands: 1998 available at http://www.uu.nl/content/16-3.pdf
\vspace{0.00mm}

\vspace{0.00mm}
\setlength{\parindent}{0.00mm}
\setlength{\leftskip}{0.00mm}
\setlength{\rightskip}{0.00mm}


\vspace{0.00mm}

\vspace{0.00mm}
\setlength{\parindent}{0.00mm}
\setlength{\leftskip}{0.00mm}
\setlength{\rightskip}{0.00mm}

Ghai Y., '\textit{Human Rights and Governance: The Asia Debate'}, 15 Australian Year Book of International Law (1994) 1. 
\vspace{0.00mm}

\vspace{0.00mm}
\setlength{\parindent}{0.00mm}
\setlength{\leftskip}{0.00mm}
\setlength{\rightskip}{0.00mm}


\vspace{0.00mm}

\vspace{0.00mm}
\setlength{\parindent}{0.00mm}
\setlength{\leftskip}{0.00mm}
\setlength{\rightskip}{0.00mm}

Gibney M., Skogly S., \textit{Transnational Human Rights Obligations}, Human Rights Quarterly; V.24; pp.781 
\vspace{0.00mm}

\vspace{0.00mm}
\setlength{\parindent}{0.00mm}
\setlength{\leftskip}{0.00mm}
\setlength{\rightskip}{0.00mm}


\vspace{0.00mm}

\vspace{0.00mm}
\setlength{\parindent}{0.00mm}
\setlength{\leftskip}{0.00mm}
\setlength{\rightskip}{0.00mm}

Goodwin-Gill G S.,  \textit{The reality of international law Essays in honour of Ian Brownlie}, Oxford : Oxford UP, 1999.  
\vspace{0.00mm}

\vspace{0.00mm}
\setlength{\parindent}{-4.91mm}
\setlength{\leftskip}{4.91mm}
\setlength{\rightskip}{0.00mm}


\vspace{0.00mm}

\vspace{0.00mm}
\setlength{\parindent}{0.00mm}
\setlength{\leftskip}{0.00mm}
\setlength{\rightskip}{0.00mm}

Hamm B., \textit{A human rights approach to development}, Human Rights Quarterly; V.23(4); pp.1005-10\\

\vspace{0.00mm}

\vspace{0.00mm}
\setlength{\parindent}{0.00mm}
\setlength{\leftskip}{0.00mm}
\setlength{\rightskip}{0.00mm}

Harris DJ., \textit{Cases and materials on international law.} 5th ed, London: Sweet and Maxwell, 1998
\vspace{0.00mm}

\vspace{0.00mm}
\setlength{\parindent}{0.00mm}
\setlength{\leftskip}{0.00mm}
\setlength{\rightskip}{0.00mm}


\vspace{0.00mm}

\vspace{0.00mm}
\setlength{\parindent}{0.00mm}
\setlength{\leftskip}{0.00mm}
\setlength{\rightskip}{0.00mm}

Hausserman J., \textit{The Realization and Implementation of Economic Social and Cultural Rights}, London: Macmillan, 1992
\vspace{0.00mm}

\vspace{0.00mm}
\setlength{\parindent}{0.00mm}
\setlength{\leftskip}{0.00mm}
\setlength{\rightskip}{0.00mm}


\vspace{0.00mm}

\vspace{0.00mm}
\setlength{\parindent}{0.00mm}
\setlength{\leftskip}{0.00mm}
\setlength{\rightskip}{0.00mm}

Haussermann J., \textit{Rights and humanity: a human rights approach to development. }Department for International Development, London. 1998
\vspace{0.00mm}

\vspace{0.00mm}
\setlength{\parindent}{0.00mm}
\setlength{\leftskip}{0.00mm}
\setlength{\rightskip}{0.00mm}


\vspace{0.00mm}

\vspace{0.00mm}
\setlength{\parindent}{0.00mm}
\setlength{\leftskip}{0.00mm}
\setlength{\rightskip}{0.00mm}

Hobbes T., \textit{Leviathan} (edited by JCA Gaskin).Oxford: OUP, 1996.
\vspace{0.00mm}

\vspace{0.00mm}
\setlength{\parindent}{0.00mm}
\setlength{\leftskip}{0.00mm}
\setlength{\rightskip}{0.00mm}


\vspace{0.00mm}

\vspace{0.00mm}
\setlength{\parindent}{0.00mm}
\setlength{\leftskip}{0.00mm}
\setlength{\rightskip}{0.00mm}

Howse R., \textit{Maintreaming the Right to Development into International Trade Law and Policy at the World Trade Organization}, UN Doc. E/CN.4/Sub.2/2004/17.\\

\vspace{0.00mm}

\vspace{0.00mm}
\setlength{\parindent}{0.00mm}
\setlength{\leftskip}{0.00mm}
\setlength{\rightskip}{0.00mm}

Hurrell A, \textit{Inequality, globalization, and world politic}, Oxford : OUP, 1999. 
\vspace{0.00mm}

\vspace{0.00mm}
\setlength{\parindent}{0.00mm}
\setlength{\leftskip}{0.00mm}
\setlength{\rightskip}{0.00mm}


\vspace{0.00mm}

\vspace{0.00mm}
\setlength{\parindent}{0.00mm}
\setlength{\leftskip}{0.00mm}
\setlength{\rightskip}{0.00mm}

International Law Association, London Conference (2000),  \textit{Final Report of the Committee on Formation of Customary (General) International Law}, Available at \textcolor{blue}{{\uline{www.ila-hq.org}}}. 
\vspace{0.00mm}

\vspace{0.00mm}
\setlength{\parindent}{0.00mm}
\setlength{\leftskip}{0.00mm}
\setlength{\rightskip}{0.00mm}


\vspace{0.00mm}

\vspace{0.00mm}
\setlength{\parindent}{0.00mm}
\setlength{\leftskip}{0.00mm}
\setlength{\rightskip}{0.00mm}

Kant I., \textit{Groundwork of the metaphysics of morals} (translated and edited by M Gregor) Cambridge, CUP, 1998.
\vspace{0.00mm}

\vspace{0.00mm}
\setlength{\parindent}{0.00mm}
\setlength{\leftskip}{0.00mm}
\setlength{\rightskip}{0.00mm}


\vspace{0.00mm}

\vspace{0.00mm}
\setlength{\parindent}{0.00mm}
\setlength{\leftskip}{0.00mm}
\setlength{\rightskip}{0.00mm}

Locke  J.,  \textit{Two treatises of government} (edited by P Laslett. 3rd ed). Cambridge, CUP, 1988.
\vspace{0.00mm}

\vspace{0.00mm}
\setlength{\parindent}{0.00mm}
\setlength{\leftskip}{0.00mm}
\setlength{\rightskip}{0.00mm}


\vspace{0.00mm}

\vspace{0.00mm}
\setlength{\parindent}{0.00mm}
\setlength{\leftskip}{0.00mm}
\setlength{\rightskip}{0.00mm}

Marks S., \textit{`The Human Right to Development: Between Rhetoric and Reality'}, 17 Harvard Human Rights Journal 137, 2004
\vspace{0.00mm}

\vspace{0.00mm}
\setlength{\parindent}{0.00mm}
\setlength{\leftskip}{0.00mm}
\setlength{\rightskip}{0.00mm}


\vspace{0.00mm}

\vspace{0.00mm}
\setlength{\parindent}{0.00mm}
\setlength{\leftskip}{0.00mm}
\setlength{\rightskip}{0.00mm}

Marks S., \textit{Emerging human rights : a new generation fo the 1980's?} 33 Rutgers L Rev 435, 451, 1981 
\vspace{0.00mm}

\vspace{0.00mm}
\setlength{\parindent}{0.00mm}
\setlength{\leftskip}{0.00mm}
\setlength{\rightskip}{0.00mm}


\vspace{0.00mm}

\vspace{0.00mm}
\setlength{\parindent}{0.00mm}
\setlength{\leftskip}{0.00mm}
\setlength{\rightskip}{0.00mm}

Marks S., \textit{The Human Rights Approach to Development: 5 Approaches}, FXB Centre for Health and Human Rights, Working Paper 5 (2000). \textcolor{blue}{{\uline{http://www.hsph.harvard.edu/fxbcenter/FXBC\_WP6--}}}\hfill{}Marks.pdf.  Last accessed 11.10.04\hfill{}
\vspace{0.00mm}

\vspace{0.00mm}
\setlength{\parindent}{0.00mm}
\setlength{\leftskip}{0.00mm}
\setlength{\rightskip}{0.00mm}


\vspace{0.00mm}

\vspace{0.00mm}
\setlength{\parindent}{0.00mm}
\setlength{\leftskip}{0.00mm}
\setlength{\rightskip}{0.00mm}

Marx K., \textit{Capital a critique of political economy}. Volume 1, Book 1 - the process of production of capital. - Lawrence and Wishart, 1954
\vspace{0.00mm}

\vspace{0.00mm}
\setlength{\parindent}{0.00mm}
\setlength{\leftskip}{0.00mm}
\setlength{\rightskip}{0.00mm}


\vspace{0.00mm}

\vspace{0.00mm}
\setlength{\parindent}{0.00mm}
\setlength{\leftskip}{0.00mm}
\setlength{\rightskip}{0.00mm}

Marx K., \textit{The Communist manifesto }(by K Marx and F Engels with introduction and notes by AJP Taylor). - London : Penguin, 1967
\vspace{0.00mm}

\vspace{0.00mm}
\setlength{\parindent}{0.00mm}
\setlength{\leftskip}{0.00mm}
\setlength{\rightskip}{0.00mm}


\vspace{0.00mm}

\vspace{0.00mm}
\setlength{\parindent}{0.00mm}
\setlength{\leftskip}{0.00mm}
\setlength{\rightskip}{0.00mm}

Middleton N., \textit{Rio plus ten, poverty and environment}, London: Pluto Press, 2003
\vspace{0.00mm}

\vspace{0.00mm}
\setlength{\parindent}{0.00mm}
\setlength{\leftskip}{0.00mm}
\setlength{\rightskip}{0.00mm}


\vspace{0.00mm}

\vspace{0.00mm}
\setlength{\parindent}{0.00mm}
\setlength{\leftskip}{0.00mm}
\setlength{\rightskip}{0.00mm}

Mill JS., \textit{Utilitarianism} (edited by R Crisp). - Oxford : OUP, 1998.
\vspace{0.00mm}

\vspace{0.00mm}
\setlength{\parindent}{0.00mm}
\setlength{\leftskip}{0.00mm}
\setlength{\rightskip}{0.00mm}


\vspace{0.00mm}

\vspace{0.00mm}
\setlength{\parindent}{0.00mm}
\setlength{\leftskip}{0.00mm}
\setlength{\rightskip}{0.00mm}

Mill JS., \textit{Utilitarianism; On liberty; Essay on Bentham; together with selected writings of Jeremy Bentham and John Austin} (edited with an introduction by M Warnock).Fontana, 1962
\vspace{0.00mm}

\vspace{0.00mm}
\setlength{\parindent}{0.00mm}
\setlength{\leftskip}{0.00mm}
\setlength{\rightskip}{0.00mm}


\vspace{0.00mm}

\vspace{0.00mm}
\setlength{\parindent}{0.00mm}
\setlength{\leftskip}{0.00mm}
\setlength{\rightskip}{0.00mm}

Morel C., \textit{Defending Human Rights in Africa: The Case for Minority and Indigenous Rights}, Essex Human Rights Review. Vol 1. No.1 available at http://projects.essex.ac.uk/EHRR/archive/pdf/49.pdf
\vspace{0.00mm}

\vspace{0.00mm}
\setlength{\parindent}{0.00mm}
\setlength{\leftskip}{0.00mm}
\setlength{\rightskip}{0.00mm}


\vspace{0.00mm}

\vspace{0.00mm}
\setlength{\parindent}{0.00mm}
\setlength{\leftskip}{0.00mm}
\setlength{\rightskip}{0.00mm}

Mowbray A., \textit{Cases and Materials on the European Convention on Human Rights}, London: Butterworths, 2001. 
\vspace{0.00mm}

\vspace{0.00mm}
\setlength{\parindent}{0.00mm}
\setlength{\leftskip}{0.00mm}
\setlength{\rightskip}{0.00mm}


\vspace{0.00mm}

\vspace{0.00mm}
\setlength{\parindent}{0.00mm}
\setlength{\leftskip}{0.00mm}
\setlength{\rightskip}{0.00mm}

Nanda V., Shepherd G. and McCarthy-Arnolds E. (eds), \textit{World Debt and the Human Condition: Structural Adjustment and the Right to Development}, Westport, CT: Greenwood Press, 1993.
\vspace{0.00mm}

\vspace{0.00mm}
\setlength{\parindent}{0.00mm}
\setlength{\leftskip}{0.00mm}
\setlength{\rightskip}{0.00mm}


\vspace{0.00mm}

\vspace{0.00mm}
\setlength{\parindent}{0.00mm}
\setlength{\leftskip}{0.00mm}
\setlength{\rightskip}{0.00mm}

Nuscheler., \textit{The international debate on human rights and the right to development,} 1998, \textit{The right to development: advance or greek gift in the development of human rights?} \hfill{}\hfill{}\hfill{}\hfill{}\hfill{}\textcolor{blue}{{\uline{http://www.inwent.org/E+Z/content/archive-eng/01-2004/tribune\_art1.htm}}}. Last accessed 15.8.04
\vspace{0.00mm}

\vspace{0.00mm}
\setlength{\parindent}{0.00mm}
\setlength{\leftskip}{0.00mm}
\setlength{\rightskip}{0.00mm}


\vspace{0.00mm}

\vspace{0.00mm}
\setlength{\parindent}{0.00mm}
\setlength{\leftskip}{0.00mm}
\setlength{\rightskip}{0.00mm}

Obiora Amede L.,\textit{ Beyond the Rhetoric of a Right to Development}, Law and Policy; V.18(3/4); pp.355-418 
\vspace{0.00mm}

\vspace{0.00mm}
\setlength{\parindent}{0.00mm}
\setlength{\leftskip}{0.00mm}
\setlength{\rightskip}{0.00mm}


\vspace{0.00mm}

\vspace{0.00mm}
\setlength{\parindent}{0.00mm}
\setlength{\leftskip}{0.00mm}
\setlength{\rightskip}{0.00mm}

O'Brien J., \textit{International law}, London: Cavendish, 2001
\vspace{0.00mm}

\vspace{0.00mm}
\setlength{\parindent}{0.00mm}
\setlength{\leftskip}{0.00mm}
\setlength{\rightskip}{0.00mm}


\vspace{0.00mm}

\vspace{0.00mm}
\setlength{\parindent}{0.00mm}
\setlength{\leftskip}{0.00mm}
\setlength{\rightskip}{0.00mm}

OHCHR., \textit{Human Rights and Poverty Reduction: A Conceptual Framework}, United Nations, 2004. \hfill{}\textcolor{blue}{{\uline{http://www.ohchr.org/english/issues/poverty/}}}. Last accessed 25.10.04
\vspace{0.00mm}

\vspace{0.00mm}
\setlength{\parindent}{0.00mm}
\setlength{\leftskip}{0.00mm}
\setlength{\rightskip}{0.00mm}


\vspace{0.00mm}

\vspace{0.00mm}
\setlength{\parindent}{0.00mm}
\setlength{\leftskip}{0.00mm}
\setlength{\rightskip}{0.00mm}

Orford A., '\textit{Globalisation and the Right to Development}', in Alston P. (ed), Peoples' Rights, Oxford: Oxford University Press, 2001., 127-185
\vspace{0.00mm}

\vspace{0.00mm}
\setlength{\parindent}{0.00mm}
\setlength{\leftskip}{0.00mm}
\setlength{\rightskip}{0.00mm}


\vspace{0.00mm}

\vspace{0.00mm}
\setlength{\parindent}{0.00mm}
\setlength{\leftskip}{0.00mm}
\setlength{\rightskip}{0.00mm}

Overseas Development Institute., \textit{What Can We Do with a Rights-Based Approach to Development?} Overseas Development Institute Briefing Paper 1999\hfill{}\hfill{}\hfill{}\hfill{}\hfill{}\hfill{}\hfill{} \textcolor{blue}{{\uline{http://www.odi.org.uk/publications/briefing/3\_99.html}}} Last accessed 25.10.04
\vspace{0.00mm}

\vspace{0.00mm}
\setlength{\parindent}{0.00mm}
\setlength{\leftskip}{0.00mm}
\setlength{\rightskip}{0.00mm}


\vspace{0.00mm}

\vspace{0.00mm}
\setlength{\parindent}{0.00mm}
\setlength{\leftskip}{0.00mm}
\setlength{\rightskip}{0.00mm}

Paine T., \textit{Rights of man, Common sense, and other political writings} (edited with an introduction by Mark Philip). Oxford: OUP, 1995
\vspace{0.00mm}

\vspace{0.00mm}
\setlength{\parindent}{0.00mm}
\setlength{\leftskip}{0.00mm}
\setlength{\rightskip}{0.00mm}


\vspace{0.00mm}

\vspace{0.00mm}
\setlength{\parindent}{0.00mm}
\setlength{\leftskip}{0.00mm}
\setlength{\rightskip}{0.00mm}

Paul., \textit{The Human Right to Development its Meaning and Importance}, The John Marshall Law Review; V.25; pp.235
\vspace{0.00mm}

\vspace{0.00mm}
\setlength{\parindent}{0.00mm}
\setlength{\leftskip}{0.00mm}
\setlength{\rightskip}{0.00mm}


\vspace{0.00mm}

\vspace{0.00mm}
\setlength{\parindent}{0.00mm}
\setlength{\leftskip}{0.00mm}
\setlength{\rightskip}{0.00mm}

Piron L-H., \textit{The Right to Development: A Review of the Current State of the Debate}, (2002). Prepared for the UK Department for International Development. \textcolor{blue}{{\uline{http://www.odi.org.uk/pppg/publications/papers\_reports/dfid/issues/rights01/index.html}}}. Last accessed 26.10.04
\vspace{0.00mm}

\vspace{0.00mm}
\setlength{\parindent}{0.00mm}
\setlength{\leftskip}{0.00mm}
\setlength{\rightskip}{0.00mm}


\vspace{0.00mm}

\vspace{0.00mm}
\setlength{\parindent}{0.00mm}
\setlength{\leftskip}{0.00mm}
\setlength{\rightskip}{0.00mm}

Piron L-H., \textit{The Right to Development: Study on Existing Bilateral and Multilateral Programmes and Policies for Development Partnership}, UN Doc. E/CN.4/Sub.2/2004/15.
\vspace{0.00mm}

\vspace{0.00mm}
\setlength{\parindent}{0.00mm}
\setlength{\leftskip}{0.00mm}
\setlength{\rightskip}{0.00mm}


\vspace{0.00mm}

\vspace{0.00mm}
\setlength{\parindent}{0.00mm}
\setlength{\leftskip}{0.00mm}
\setlength{\rightskip}{0.00mm}

Plato., \textit{The republic}. 2nd ed (translated by D Lee). London: Penguin, 1970.
\vspace{0.00mm}

\vspace{0.00mm}
\setlength{\parindent}{0.00mm}
\setlength{\leftskip}{0.00mm}
\setlength{\rightskip}{0.00mm}


\vspace{0.00mm}

\vspace{0.00mm}
\setlength{\parindent}{0.00mm}
\setlength{\leftskip}{0.00mm}
\setlength{\rightskip}{0.00mm}

Rehman J., \textit{International human rights law a practical approach}, Harlow: Pearson Education, 2003
\vspace{0.00mm}

\vspace{0.00mm}
\setlength{\parindent}{0.00mm}
\setlength{\leftskip}{0.00mm}
\setlength{\rightskip}{0.00mm}


\vspace{0.00mm}

\vspace{0.00mm}
\setlength{\parindent}{0.00mm}
\setlength{\leftskip}{0.00mm}
\setlength{\rightskip}{0.00mm}

Rich R., '\textit{The Right to Development: A Right of Peoples?}', in J. Crawford (ed), The Rights of Peoples, Oxford: Clarendon Press, 1988, 39-54.
\vspace{0.00mm}

\vspace{0.00mm}
\setlength{\parindent}{0.00mm}
\setlength{\leftskip}{0.00mm}
\setlength{\rightskip}{0.00mm}


\vspace{0.00mm}

\vspace{0.00mm}
\setlength{\parindent}{0.00mm}
\setlength{\leftskip}{0.00mm}
\setlength{\rightskip}{0.00mm}

Rousseau J-J., \textit{The discourses and other early political writings} (edited by V Gourevitch).Cambridge:\hfill{}CUP, 1997.
\vspace{0.00mm}

\vspace{0.00mm}
\setlength{\parindent}{0.00mm}
\setlength{\leftskip}{0.00mm}
\setlength{\rightskip}{0.00mm}


\vspace{0.00mm}

\vspace{0.00mm}
\setlength{\parindent}{0.00mm}
\setlength{\leftskip}{0.00mm}
\setlength{\rightskip}{0.00mm}

Rousseau J-J., \textit{The social contract }(translated with introduction and notes by Christopher Betts) Oxford: OUP, 1994
\vspace{0.00mm}

\vspace{0.00mm}
\setlength{\parindent}{0.00mm}
\setlength{\leftskip}{0.00mm}
\setlength{\rightskip}{0.00mm}


\vspace{0.00mm}

\vspace{0.00mm}
\setlength{\parindent}{0.00mm}
\setlength{\leftskip}{0.00mm}
\setlength{\rightskip}{0.00mm}

Salomon M E., and Sengupta A., \textit{The Right to Development: Obligations of States and the Rights of Minorities and Indigenous Peoples}, Minority Rights Group International, 2003. 
\vspace{0.00mm}

\vspace{0.00mm}
\setlength{\parindent}{0.00mm}
\setlength{\leftskip}{0.00mm}
\setlength{\rightskip}{0.00mm}


\vspace{0.00mm}

\vspace{0.00mm}
\setlength{\parindent}{0.00mm}
\setlength{\leftskip}{0.00mm}
\setlength{\rightskip}{0.00mm}

Santos Pais M., \textit{A Human Rights Conceptual Framework for UNICEF}, UNICEF, Innocenti Essay No. 9, Florence, 1999.Avaliable at \textcolor{blue}{{\uline{http://www.unicef-icdc.org/publications/pdf/essay9.pdf}}}. Last accessed 17.10.04
\vspace{0.00mm}

\vspace{0.00mm}
\setlength{\parindent}{0.00mm}
\setlength{\leftskip}{0.00mm}
\setlength{\rightskip}{0.00mm}


\vspace{0.00mm}

\vspace{0.00mm}
\setlength{\parindent}{0.00mm}
\setlength{\leftskip}{0.00mm}
\setlength{\rightskip}{0.00mm}

Sen A., \textit{Democracy as a Universal Value}, Journal of Democracy; V.10(3); pp.3-17 \\

\vspace{0.00mm}

\vspace{0.00mm}
\setlength{\parindent}{0.00mm}
\setlength{\leftskip}{0.00mm}
\setlength{\rightskip}{0.00mm}

Sen A., '\textit{The Right not to be Hungry'}, in Alston P., and Tomasevski K., (eds), The Right to Food, Netherlands: SIM, 1984. 
\vspace{0.00mm}

\vspace{0.00mm}
\setlength{\parindent}{0.00mm}
\setlength{\leftskip}{0.00mm}
\setlength{\rightskip}{0.00mm}


\vspace{0.00mm}

\vspace{0.00mm}
\setlength{\parindent}{0.00mm}
\setlength{\leftskip}{0.00mm}
\setlength{\rightskip}{0.00mm}

Sen A., \textit{Development as Freedom}, Oxford University Press: Oxford, 1999
\vspace{0.00mm}

\vspace{0.00mm}
\setlength{\parindent}{0.00mm}
\setlength{\leftskip}{0.00mm}
\setlength{\rightskip}{0.00mm}


\vspace{0.00mm}

\vspace{0.00mm}
\setlength{\parindent}{0.00mm}
\setlength{\leftskip}{0.00mm}
\setlength{\rightskip}{0.00mm}

Sengupta A., \textit{Theory and Practice on the Right to Development}, Human Rights Quarterly; V.24(4); pp.837-889
\vspace{0.00mm}

\vspace{0.00mm}
\setlength{\parindent}{0.00mm}
\setlength{\leftskip}{0.00mm}
\setlength{\rightskip}{0.00mm}


\vspace{0.00mm}

\vspace{0.00mm}
\setlength{\parindent}{0.00mm}
\setlength{\leftskip}{0.00mm}
\setlength{\rightskip}{0.00mm}

Sengupta A., \textit{Towards Realizing the Right to Development}, Development and Change; June; V.31(3) \\

\vspace{0.00mm}

\vspace{0.00mm}
\setlength{\parindent}{0.00mm}
\setlength{\leftskip}{0.00mm}
\setlength{\rightskip}{0.00mm}

Sengupta A., \textit{The Right to Development as Human Right}, 2000, Harvard School of Public Health, available at www.hsph.harvard.edu/fxbcenter/FXBC\_WP7--Sengupta.pdf 
\vspace{0.00mm}

\vspace{0.00mm}
\setlength{\parindent}{0.00mm}
\setlength{\leftskip}{0.00mm}
\setlength{\rightskip}{0.00mm}


\vspace{0.00mm}

\vspace{0.00mm}
\setlength{\parindent}{0.00mm}
\setlength{\leftskip}{0.00mm}
\setlength{\rightskip}{0.00mm}

Sieghart P., \textit{The Lawful Rights of Mankind}, Oxford: Oxford University Press,1985. 
\vspace{0.00mm}

\vspace{0.00mm}
\setlength{\parindent}{0.00mm}
\setlength{\leftskip}{0.00mm}
\setlength{\rightskip}{0.00mm}


\vspace{0.00mm}

\vspace{0.00mm}
\setlength{\parindent}{0.00mm}
\setlength{\leftskip}{0.00mm}
\setlength{\rightskip}{0.00mm}

Sieghart P., \textit{The International Law of Human Rights}, Oxford: Oxford University Press, 1985
\vspace{0.00mm}

\vspace{0.00mm}
\setlength{\parindent}{0.00mm}
\setlength{\leftskip}{0.00mm}
\setlength{\rightskip}{0.00mm}


\vspace{0.00mm}

\vspace{0.00mm}
\setlength{\parindent}{0.00mm}
\setlength{\leftskip}{0.00mm}
\setlength{\rightskip}{0.00mm}

Shaw M N., \textit{International law}. 4$^{th}$ ed. Cambridge:  Cambridge UP, 1997
\vspace{0.00mm}

\vspace{0.00mm}
\setlength{\parindent}{0.00mm}
\setlength{\leftskip}{0.00mm}
\setlength{\rightskip}{0.00mm}


\vspace{0.00mm}

\vspace{0.00mm}
\setlength{\parindent}{0.00mm}
\setlength{\leftskip}{0.00mm}
\setlength{\rightskip}{0.00mm}

Shaw M N., \textit{International law}. 5$^{th}$ ed. Cambridge: Cambridge UP, 2003
\vspace{0.00mm}

\vspace{0.00mm}
\setlength{\parindent}{0.00mm}
\setlength{\leftskip}{0.00mm}
\setlength{\rightskip}{0.00mm}


\vspace{0.00mm}

\vspace{0.00mm}
\setlength{\parindent}{0.00mm}
\setlength{\leftskip}{0.00mm}
\setlength{\rightskip}{0.00mm}

Shelton D., 'Environmental Rights', in Alston P. (ed), Peoples' Rights, Oxford: Oxford University Press, 2001.189-256.
\vspace{0.00mm}

\vspace{0.00mm}
\setlength{\parindent}{0.00mm}
\setlength{\leftskip}{0.00mm}
\setlength{\rightskip}{0.00mm}


\vspace{0.00mm}

\vspace{0.00mm}
\setlength{\parindent}{0.00mm}
\setlength{\leftskip}{0.00mm}
\setlength{\rightskip}{0.00mm}

Slimane S., '\textit{Recognising Minorities in Africa}', Minority Rights Group International 2003, May 2003. available at www.minorityrights.org
\vspace{0.00mm}

\vspace{0.00mm}
\setlength{\parindent}{0.00mm}
\setlength{\leftskip}{0.00mm}
\setlength{\rightskip}{0.00mm}


\vspace{0.00mm}

\vspace{0.00mm}
\setlength{\parindent}{0.00mm}
\setlength{\leftskip}{0.00mm}
\setlength{\rightskip}{0.00mm}

Sohn., \textit{The international law of human rights: a reply to recent criticisms} 9 Hofstra L Rev 347, 1981 
\vspace{0.00mm}

\vspace{0.00mm}
\setlength{\parindent}{0.00mm}
\setlength{\leftskip}{0.00mm}
\setlength{\rightskip}{0.00mm}


\vspace{0.00mm}

\vspace{0.00mm}
\setlength{\parindent}{0.00mm}
\setlength{\leftskip}{0.00mm}
\setlength{\rightskip}{0.00mm}

Steiner H J.,  and Alston P., \textit{International human rights in context: law, politics, morals}, Oxford: Oxford UP, 1996
\vspace{0.00mm}

\vspace{0.00mm}
\setlength{\parindent}{0.00mm}
\setlength{\leftskip}{0.00mm}
\setlength{\rightskip}{0.00mm}


\vspace{0.00mm}

\vspace{0.00mm}
\setlength{\parindent}{0.00mm}
\setlength{\leftskip}{0.00mm}
\setlength{\rightskip}{0.00mm}

Steiner H J., and Alston P., \textit{International human rights in context: law, politics, morals}, 2$^{nd}$ ed, Oxford :Oxford UP, 2000 
\vspace{0.00mm}

\vspace{0.00mm}
\setlength{\parindent}{0.00mm}
\setlength{\leftskip}{0.00mm}
\setlength{\rightskip}{0.00mm}


\vspace{0.00mm}

\vspace{0.00mm}
\setlength{\parindent}{0.00mm}
\setlength{\leftskip}{0.00mm}
\setlength{\rightskip}{0.00mm}

Symonides J (ed)., \textit{Human rights concepts and standards}, Aldershot, Ashgate,  Paris: UNESCO, 2000.
\vspace{0.00mm}

\vspace{0.00mm}
\setlength{\parindent}{0.00mm}
\setlength{\leftskip}{0.00mm}
\setlength{\rightskip}{0.00mm}


\vspace{0.00mm}

\vspace{0.00mm}
\setlength{\parindent}{0.00mm}
\setlength{\leftskip}{0.00mm}
\setlength{\rightskip}{0.00mm}

'\textit{Symposium: Development as an Emerging Human Right}', California Western International Law Journal,  (1985), 429.
\vspace{0.00mm}

\vspace{0.00mm}
\setlength{\parindent}{0.00mm}
\setlength{\leftskip}{0.00mm}
\setlength{\rightskip}{0.00mm}


\vspace{0.00mm}

\vspace{0.00mm}
\setlength{\parindent}{0.00mm}
\setlength{\leftskip}{0.00mm}
\setlength{\rightskip}{0.00mm}

\textit{The Code of Hummurabi }(1792-1750 BC); A translation of the Code of Hammurabi can be found at: \hfill{}http://www.wsu.edu:8080/~dee/MESO/CODE.HTM  (last accessed 24.09.04)
\vspace{0.00mm}

\vspace{0.00mm}
\setlength{\parindent}{-4.91mm}
\setlength{\leftskip}{4.91mm}
\setlength{\rightskip}{0.00mm}


\vspace{0.00mm}

\vspace{0.00mm}
\setlength{\parindent}{0.00mm}
\setlength{\leftskip}{0.00mm}
\setlength{\rightskip}{0.00mm}

The Human Rights Council of Australia., \textit{The Rights Way to Development: A Human Rights Approach to Development Assistance}, The Human Rights Council of Australia, Australia, 2001,  \hfill{}\hfill{}\hfill{}\textcolor{blue}{{\uline{http://www.hrca.org.au/symposium.htm}}} and \textcolor{blue}{{\uline{http://www.hrca.org.au/The\_Rights\_Way\_to\_Development\_Manual.htm. }}} Last accessed 26.10.04
\vspace{0.00mm}

\vspace{0.00mm}
\setlength{\parindent}{0.00mm}
\setlength{\leftskip}{0.00mm}
\setlength{\rightskip}{0.00mm}


\vspace{0.00mm}

\vspace{0.00mm}
\setlength{\parindent}{0.00mm}
\setlength{\leftskip}{0.00mm}
\setlength{\rightskip}{0.00mm}

\textit{The Limburg Principles on the Implementation of the International Covenant on Economic Social and Cultural Rights}, 9 Hum Rts Q 122, 1987
\vspace{0.00mm}

\vspace{0.00mm}
\setlength{\parindent}{0.00mm}
\setlength{\leftskip}{0.00mm}
\setlength{\rightskip}{0.00mm}


\vspace{0.00mm}

\vspace{0.00mm}
\setlength{\parindent}{0.00mm}
\setlength{\leftskip}{0.00mm}
\setlength{\rightskip}{0.00mm}

\textit{The Maastrict Guidelines on Violations of Economic Social and Cultural Rights,} 20 Hum Rts Q 691, (1998).
\vspace{0.00mm}

\vspace{0.00mm}
\setlength{\parindent}{0.00mm}
\setlength{\leftskip}{0.00mm}
\setlength{\rightskip}{0.00mm}


\vspace{0.00mm}

\vspace{0.00mm}
\setlength{\parindent}{0.00mm}
\setlength{\leftskip}{0.00mm}
\setlength{\rightskip}{0.00mm}

Udombana N., '\textit{The Third World and the Right to Development: Agenda for the Next Millennium'}, Human Rights Quarterly, Vol 22(3) (2000), 753-87.    
\vspace{0.00mm}

\vspace{0.00mm}
\setlength{\parindent}{0.00mm}
\setlength{\leftskip}{0.00mm}
\setlength{\rightskip}{0.00mm}


\vspace{0.00mm}

\vspace{0.00mm}
\setlength{\parindent}{0.00mm}
\setlength{\leftskip}{0.00mm}
\setlength{\rightskip}{0.00mm}

Umozurike U O., \textit{The African Charter on Human and Peoples' Rights}, 77 Am. J. Int'l L. 902, (1983).
\vspace{0.00mm}

\vspace{0.00mm}
\setlength{\parindent}{0.00mm}
\setlength{\leftskip}{0.00mm}
\setlength{\rightskip}{0.00mm}


\vspace{0.00mm}

\vspace{0.00mm}
\setlength{\parindent}{0.00mm}
\setlength{\leftskip}{0.00mm}
\setlength{\rightskip}{0.00mm}

United Nations., \textit{The realization of the right to development: Global consultation on the right to development as a human right : report prepared by the Secretary-General pursuant to Commission on Human Rights resolution 1989/45,} United Nations, 1991.
\vspace{0.00mm}

\vspace{0.00mm}
\setlength{\parindent}{0.00mm}
\setlength{\leftskip}{0.00mm}
\setlength{\rightskip}{0.00mm}


\vspace{0.00mm}

\vspace{0.00mm}
\setlength{\parindent}{0.00mm}
\setlength{\leftskip}{0.00mm}
\setlength{\rightskip}{0.00mm}

Uvin P., \textit{On High Moral Ground: The Incorporation of Human Rights by the Development Enterprise}, Praxis- The Fletcher Journal of Development Studies, Volume XVII 2002,   available at \textcolor{blue}{{\uline{http://fletcher.tufts.edu/praxis/xvii/Uvin.pdf}}}
\vspace{0.00mm}

\vspace{0.00mm}
\setlength{\parindent}{0.00mm}
\setlength{\leftskip}{0.00mm}
\setlength{\rightskip}{0.00mm}


\vspace{0.00mm}

\vspace{0.00mm}
\setlength{\parindent}{0.00mm}
\setlength{\leftskip}{0.00mm}
\setlength{\rightskip}{0.00mm}

Uvin P., Human Rights and Development, USA: Kumarian Press Inc; 2004 
\vspace{0.00mm}

\vspace{0.00mm}
\setlength{\parindent}{0.00mm}
\setlength{\leftskip}{0.00mm}
\setlength{\rightskip}{0.00mm}


\vspace{0.00mm}

\vspace{0.00mm}
\setlength{\parindent}{0.00mm}
\setlength{\leftskip}{0.00mm}
\setlength{\rightskip}{0.00mm}

Vasak K., and Alston P (eds)., \textit{The International Dimensions of Human Rights}, Westport, CT: Greenwood Press, Volume 1, 1982.
\vspace{0.00mm}

\vspace{0.00mm}
\setlength{\parindent}{0.00mm}
\setlength{\leftskip}{0.00mm}
\setlength{\rightskip}{0.00mm}


\vspace{0.00mm}

\vspace{0.00mm}
\setlength{\parindent}{0.00mm}
\setlength{\leftskip}{0.00mm}
\setlength{\rightskip}{0.00mm}

Watson.,\textit{ The limited utility of international law in the protection of human rights,} 74 Am Soc'y Int'l L.Proc, 1981
\vspace{0.00mm}

\vspace{0.00mm}
\setlength{\parindent}{0.00mm}
\setlength{\leftskip}{0.00mm}
\setlength{\rightskip}{0.00mm}


\vspace{0.00mm}

\vspace{0.00mm}
\setlength{\parindent}{0.00mm}
\setlength{\leftskip}{0.00mm}
\setlength{\rightskip}{0.00mm}


\vspace{0.00mm}

\vspace{0.00mm}
\setlength{\parindent}{0.00mm}
\setlength{\leftskip}{0.00mm}
\setlength{\rightskip}{0.00mm}

Official Documents of the United Nations 
\vspace{0.00mm}

\vspace{0.00mm}
\setlength{\parindent}{0.00mm}
\setlength{\leftskip}{0.00mm}
\setlength{\rightskip}{0.00mm}

Available at: \textcolor{blue}{{\uline{http://ap.ohchr.org/documents/dpage\_e.aspx?s=103}}} 
\vspace{0.00mm}

\vspace{0.00mm}
\setlength{\parindent}{0.00mm}
\setlength{\leftskip}{0.00mm}
\setlength{\rightskip}{0.00mm}


\vspace{0.00mm}

\vspace{0.00mm}
\setlength{\parindent}{0.00mm}
\setlength{\leftskip}{0.00mm}
\setlength{\rightskip}{0.00mm}


\vspace{0.00mm}

\vspace{0.00mm}
\setlength{\parindent}{0.00mm}
\setlength{\leftskip}{0.00mm}
\setlength{\rightskip}{0.00mm}

UN CHR , \textit{Draft report of the Commission}, 60th session,13/04/2004, UN Doc.E/CN.4/2004/L.10/Add.7 
\vspace{0.00mm}

\vspace{0.00mm}
\setlength{\parindent}{0.00mm}
\setlength{\leftskip}{0.00mm}
\setlength{\rightskip}{0.00mm}


\vspace{0.00mm}

\vspace{0.00mm}
\setlength{\parindent}{0.00mm}
\setlength{\leftskip}{0.00mm}
\setlength{\rightskip}{0.00mm}

UN CHR , \textit{Summary Record of the 18th Meeting} 60$^{th}$ session, 30/03/2004, UN Doc. \hfill{}\hfill{}\hfill{}E/CN.4/2004/SR.18\hfill{}
\vspace{0.00mm}

\vspace{0.00mm}
\setlength{\parindent}{0.00mm}
\setlength{\leftskip}{0.00mm}
\setlength{\rightskip}{0.00mm}


\vspace{0.00mm}

\vspace{0.00mm}
\setlength{\parindent}{0.00mm}
\setlength{\leftskip}{0.00mm}
\setlength{\rightskip}{0.00mm}

UN CHR,  \textit{Report of the Working Group on the Right to Development on its fifth session}, 60$^{th}$ session, 22/04/2004 UN Doc. E/CN.4/2004/23/Corr.1
\vspace{0.00mm}

\vspace{0.00mm}
\setlength{\parindent}{0.00mm}
\setlength{\leftskip}{0.00mm}
\setlength{\rightskip}{0.00mm}


\vspace{0.00mm}

\vspace{0.00mm}
\setlength{\parindent}{0.00mm}
\setlength{\leftskip}{0.00mm}
\setlength{\rightskip}{0.00mm}

UN CHR,  \textit{Report of the Working Group on the Right to Development on its fifth session}, 60th session \hfill{}16/03/2004, UN Doc.E/CN.4/2004/23 \hfill{}\hfill{}
\vspace{0.00mm}

\vspace{0.00mm}
\setlength{\parindent}{0.00mm}
\setlength{\leftskip}{0.00mm}
\setlength{\rightskip}{0.00mm}


\vspace{0.00mm}

\vspace{0.00mm}
\setlength{\parindent}{0.00mm}
\setlength{\leftskip}{0.00mm}
\setlength{\rightskip}{0.00mm}

UN CHR,  \textit{Written statement submitted by the Hariri Foundation - The Islamic Foundation for Culture \hfill{}and Higher Education}, 60$^{th}$ session, 11/03/2004, UN Doc.E/CN.4/2004/NGO/226 \hfill{} \hfill{}
\vspace{0.00mm}

\vspace{0.00mm}
\setlength{\parindent}{0.00mm}
\setlength{\leftskip}{0.00mm}
\setlength{\rightskip}{0.00mm}


\vspace{0.00mm}

\vspace{0.00mm}
\setlength{\parindent}{0.00mm}
\setlength{\leftskip}{0.00mm}
\setlength{\rightskip}{0.00mm}

UN CHR, \textit{Country studies on the right to development - Argentina, Chile and Brazil}, 60$^{th}$ session, \hfill{}23/01/2004 UN Doc. E/CN.4/2004/WG.18/3  
\vspace{0.00mm}

\vspace{0.00mm}
\setlength{\parindent}{0.00mm}
\setlength{\leftskip}{0.00mm}
\setlength{\rightskip}{0.00mm}


\vspace{0.00mm}

\vspace{0.00mm}
\setlength{\parindent}{0.00mm}
\setlength{\leftskip}{0.00mm}
\setlength{\rightskip}{0.00mm}

UN CHR, \textit{Draft report of the Commission}, 59$^{th}$ session, 25/04/2003 UN Doc. E/CN.4/2003/L.10/Add.7 
\vspace{0.00mm}

\vspace{0.00mm}
\setlength{\parindent}{0.00mm}
\setlength{\leftskip}{0.00mm}
\setlength{\rightskip}{0.00mm}


\vspace{0.00mm}

\vspace{0.00mm}
\setlength{\parindent}{0.00mm}
\setlength{\leftskip}{0.00mm}
\setlength{\rightskip}{0.00mm}

UN CHR, \textit{Draft report of the Commission,} 59$^{th}$ session, 25/04/2003 UN Doc. E/CN.4/2003/L.11/Add.8
\vspace{0.00mm}

\vspace{0.00mm}
\setlength{\parindent}{0.00mm}
\setlength{\leftskip}{0.00mm}
\setlength{\rightskip}{0.00mm}


\vspace{0.00mm}

\vspace{0.00mm}
\setlength{\parindent}{0.00mm}
\setlength{\leftskip}{0.00mm}
\setlength{\rightskip}{0.00mm}

UN CHR, \textit{Exposici�n escrita presentada por el Movimiento Cubano por la Paz y la Soberan�a de los \hfill{}Pueblos}, 60$^{th}$ session,11/03/2004, UN Doc.E/CN.4/2004/NGO/222 \hfill{}\hfill{}
\vspace{0.00mm}

\vspace{0.00mm}
\setlength{\parindent}{0.00mm}
\setlength{\leftskip}{0.00mm}
\setlength{\rightskip}{0.00mm}


\vspace{0.00mm}

\vspace{0.00mm}
\setlength{\parindent}{0.00mm}
\setlength{\leftskip}{0.00mm}
\setlength{\rightskip}{0.00mm}

UN CHR, \textit{Exposici�n escrita presentada por el Movimiento Cubano por la Paz y la Soberan�a de los \hfill{}Pueblos}, 60$^{th}$ session, 11/03/2004, UN Doc.E/CN.4/2004/NGO/221 \hfill{}\hfill{}
\vspace{0.00mm}

\vspace{0.00mm}
\setlength{\parindent}{0.00mm}
\setlength{\leftskip}{0.00mm}
\setlength{\rightskip}{0.00mm}


\vspace{0.00mm}

\vspace{0.00mm}
\setlength{\parindent}{0.00mm}
\setlength{\leftskip}{0.00mm}
\setlength{\rightskip}{0.00mm}

UN CHR, \textit{Exposici�n escrita presentada por el Movimiento Indio "Tupaj Amaru"}, 60$^{th}$ session, \hfill{}\hfill{}11/03/2004, UN Doc.E/CN.4/2004/NGO/199 
\vspace{0.00mm}

\vspace{0.00mm}
\setlength{\parindent}{0.00mm}
\setlength{\leftskip}{0.00mm}
\setlength{\rightskip}{0.00mm}


\vspace{0.00mm}

\vspace{0.00mm}
\setlength{\parindent}{0.00mm}
\setlength{\leftskip}{0.00mm}
\setlength{\rightskip}{0.00mm}

UN CHR, \textit{Fifth report of the independent expert on the right to development}, 59$^{th}$ session, 16/09/2002 \hfill{}UN Doc. E/CN.4/2002/WG.18/6 
\vspace{0.00mm}

\vspace{0.00mm}
\setlength{\parindent}{0.00mm}
\setlength{\leftskip}{0.00mm}
\setlength{\rightskip}{0.00mm}


\vspace{0.00mm}

\vspace{0.00mm}
\setlength{\parindent}{0.00mm}
\setlength{\leftskip}{0.00mm}
\setlength{\rightskip}{0.00mm}

UN CHR, \textit{Fifth report of the independent expert on the right to development}, 59$^{th}$ session, 30/12/2002 \hfill{}UN Doc. E/CN.4/2002/WG.18/6/Add.1 
\vspace{0.00mm}

\vspace{0.00mm}
\setlength{\parindent}{0.00mm}
\setlength{\leftskip}{0.00mm}
\setlength{\rightskip}{0.00mm}


\vspace{0.00mm}

\vspace{0.00mm}
\setlength{\parindent}{0.00mm}
\setlength{\leftskip}{0.00mm}
\setlength{\rightskip}{0.00mm}

UN CHR, \textit{Forum on economic, social and cultural rights: the Social Forum}, 55$^{th}$ session, 27/04/1999 \hfill{}UN Doc. E/CN.4/RES/1999/53 
\vspace{0.00mm}

\vspace{0.00mm}
\setlength{\parindent}{0.00mm}
\setlength{\leftskip}{0.00mm}
\setlength{\rightskip}{0.00mm}


\vspace{0.00mm}

\vspace{0.00mm}
\setlength{\parindent}{0.00mm}
\setlength{\leftskip}{0.00mm}
\setlength{\rightskip}{0.00mm}

UN CHR, \textit{Fourth report of the independent expert on the right to development -- Mission}, 58$^{th}$ session, \hfill{}05/03/2002 UN Doc. E/CN.4/2002/WG.18/2/Add.1 
\vspace{0.00mm}

\vspace{0.00mm}
\setlength{\parindent}{0.00mm}
\setlength{\leftskip}{0.00mm}
\setlength{\rightskip}{0.00mm}


\vspace{0.00mm}

\vspace{0.00mm}
\setlength{\parindent}{0.00mm}
\setlength{\leftskip}{0.00mm}
\setlength{\rightskip}{0.00mm}

UN CHR, \textit{Fourth report of the independent expert on the right to development}, 58$^{th}$ session, 20/12/2001 UN Doc. E/CN.4/2002/WG.18/2 
\vspace{0.00mm}

\vspace{0.00mm}
\setlength{\parindent}{0.00mm}
\setlength{\leftskip}{0.00mm}
\setlength{\rightskip}{0.00mm}


\vspace{0.00mm}

\vspace{0.00mm}
\setlength{\parindent}{0.00mm}
\setlength{\leftskip}{0.00mm}
\setlength{\rightskip}{0.00mm}

UN CHR, \textit{Globalization and its impact on the full enjoyment of all human rights}, 57$^{th}$ session, \hfill{}\hfill{}20/04/2001 UN Doc. E/CN.4/RES/2001/32  
\vspace{0.00mm}

\vspace{0.00mm}
\setlength{\parindent}{0.00mm}
\setlength{\leftskip}{0.00mm}
\setlength{\rightskip}{0.00mm}


\vspace{0.00mm}

\vspace{0.00mm}
\setlength{\parindent}{0.00mm}
\setlength{\leftskip}{0.00mm}
\setlength{\rightskip}{0.00mm}

UN CHR, \textit{Globalization and its impact on the full enjoyment of human rights}, 59$^{th}$ session, 11/04/2003 \hfill{}UN Doc. E/CN.4/2003/L.25
\vspace{0.00mm}

\vspace{0.00mm}
\setlength{\parindent}{0.00mm}
\setlength{\leftskip}{0.00mm}
\setlength{\rightskip}{0.00mm}


\vspace{0.00mm}

\vspace{0.00mm}
\setlength{\parindent}{0.00mm}
\setlength{\leftskip}{0.00mm}
\setlength{\rightskip}{0.00mm}

UN CHR, \textit{High-level seminar on the right to development: Global partnership for development}, 60$^{th}$ \hfill{}session, 23/03/2004 UN Doc.E/CN.4/2004/23/Add.1\hfill{}
\vspace{0.00mm}

\vspace{0.00mm}
\setlength{\parindent}{0.00mm}
\setlength{\leftskip}{0.00mm}
\setlength{\rightskip}{0.00mm}


\vspace{0.00mm}

\vspace{0.00mm}
\setlength{\parindent}{0.00mm}
\setlength{\leftskip}{0.00mm}
\setlength{\rightskip}{0.00mm}

UN CHR, \textit{Human rights and unilateral coercive measures}, 55$^{th}$ session, 23/04/1999 UN Doc. E/CN.4/RES/1999/21 
\vspace{0.00mm}

\vspace{0.00mm}
\setlength{\parindent}{0.00mm}
\setlength{\leftskip}{0.00mm}
\setlength{\rightskip}{0.00mm}


\vspace{0.00mm}

\vspace{0.00mm}
\setlength{\parindent}{0.00mm}
\setlength{\leftskip}{0.00mm}
\setlength{\rightskip}{0.00mm}

UN CHR, \textit{Interim report of the High Commissioner for Human Rights}, 55$^{th}$ session, 20/08/1999 UN \hfill{}Doc. E/CN.4/1999/WG.18/3 
\vspace{0.00mm}

\vspace{0.00mm}
\setlength{\parindent}{0.00mm}
\setlength{\leftskip}{0.00mm}
\setlength{\rightskip}{0.00mm}


\vspace{0.00mm}

\vspace{0.00mm}
\setlength{\parindent}{0.00mm}
\setlength{\leftskip}{0.00mm}
\setlength{\rightskip}{0.00mm}

UN CHR, \textit{Note by the secretariat: open-ended working group on the right to development}, 59$^{th}$ session, \hfill{}17/12/2002 UN Doc. E/CN.4/2003/WG.18/3 
\vspace{0.00mm}

\vspace{0.00mm}
\setlength{\parindent}{0.00mm}
\setlength{\leftskip}{0.00mm}
\setlength{\rightskip}{0.00mm}


\vspace{0.00mm}

\vspace{0.00mm}
\setlength{\parindent}{0.00mm}
\setlength{\leftskip}{0.00mm}
\setlength{\rightskip}{0.00mm}

UN CHR, \textit{Preliminary study of the independent expert on the right to development on the impact of \hfill{}international economic and financial issues on the enjoyment of human rights}, 59$^{th}$ session, \hfill{}\hfill{}27/01/2003 UN Doc. E/CN.4/2003/WG.18/2 
\vspace{0.00mm}

\vspace{0.00mm}
\setlength{\parindent}{0.00mm}
\setlength{\leftskip}{0.00mm}
\setlength{\rightskip}{0.00mm}


\vspace{0.00mm}

\vspace{0.00mm}
\setlength{\parindent}{0.00mm}
\setlength{\leftskip}{0.00mm}
\setlength{\rightskip}{0.00mm}

UN CHR, \textit{Proceedings of the working group on the right to development}: Note by the secretariat, 56$^{th}$ \hfill{}session, 08/03/2000 UN Doc. E/CN.4/2000/21 
\vspace{0.00mm}

\vspace{0.00mm}
\setlength{\parindent}{0.00mm}
\setlength{\leftskip}{0.00mm}
\setlength{\rightskip}{0.00mm}


\vspace{0.00mm}

\vspace{0.00mm}
\setlength{\parindent}{0.00mm}
\setlength{\leftskip}{0.00mm}
\setlength{\rightskip}{0.00mm}

UN CHR, \textit{Provisional Agenda - Open-ended working group on the right to development,} 57$^{th}$ session, \hfill{}18/12/2000 UN Doc. E/CN.4/2001/WG.18/1 
\vspace{0.00mm}

\vspace{0.00mm}
\setlength{\parindent}{0.00mm}
\setlength{\leftskip}{0.00mm}
\setlength{\rightskip}{0.00mm}


\vspace{0.00mm}

\vspace{0.00mm}
\setlength{\parindent}{0.00mm}
\setlength{\leftskip}{0.00mm}
\setlength{\rightskip}{0.00mm}

UN CHR,\textit{ Provisional agenda of the Working Group on the Right to Development}, 56$^{th}$ session, \hfill{}\hfill{}15/02/2000 UN Doc. E/CN.4/2000/WG.18/1 
\vspace{0.00mm}

\vspace{0.00mm}
\setlength{\parindent}{0.00mm}
\setlength{\leftskip}{0.00mm}
\setlength{\rightskip}{0.00mm}


\vspace{0.00mm}

\vspace{0.00mm}
\setlength{\parindent}{0.00mm}
\setlength{\leftskip}{0.00mm}
\setlength{\rightskip}{0.00mm}

UN CHR,\textit{ Provisional agenda}, 60$^{th}$ session, 08/12/2003 UN Doc. E/CN.4/2004/WG.18/1 
\vspace{0.00mm}

\vspace{0.00mm}
\setlength{\parindent}{0.00mm}
\setlength{\leftskip}{0.00mm}
\setlength{\rightskip}{0.00mm}


\vspace{0.00mm}

\vspace{0.00mm}
\setlength{\parindent}{0.00mm}
\setlength{\leftskip}{0.00mm}
\setlength{\rightskip}{0.00mm}

UN CHR, \textit{Report of the High Commissioner for Human Rights: the Right to Development}, 59$^{th}$ session, \hfill{}12/09/2002 UN Doc. E/CN.4/2003/7 
\vspace{0.00mm}

\vspace{0.00mm}
\setlength{\parindent}{0.00mm}
\setlength{\leftskip}{0.00mm}
\setlength{\rightskip}{0.00mm}


\vspace{0.00mm}

\vspace{0.00mm}
\setlength{\parindent}{0.00mm}
\setlength{\leftskip}{0.00mm}
\setlength{\rightskip}{0.00mm}

UN CHR, \textit{Report of the Open-ended working group on the right to development on its fourth session}, \hfill{}59$^{th}$ session, 24/03/2003 UN Doc. E/CN.4/2003/26
\vspace{0.00mm}

\vspace{0.00mm}
\setlength{\parindent}{0.00mm}
\setlength{\leftskip}{0.00mm}
\setlength{\rightskip}{0.00mm}


\vspace{0.00mm}

\vspace{0.00mm}
\setlength{\parindent}{0.00mm}
\setlength{\leftskip}{0.00mm}
\setlength{\rightskip}{0.00mm}

UN CHR, \textit{Report of the Open-ended working group on the right to development on its fourth session}, \hfill{}59$^{th}$ session, 24/03/2003 UN Doc. E/CN.4/2003/26
\vspace{0.00mm}

\vspace{0.00mm}
\setlength{\parindent}{0.00mm}
\setlength{\leftskip}{0.00mm}
\setlength{\rightskip}{0.00mm}


\vspace{0.00mm}

\vspace{0.00mm}
\setlength{\parindent}{0.00mm}
\setlength{\leftskip}{0.00mm}
\setlength{\rightskip}{0.00mm}

UN CHR, \textit{Report of the open-ended Working Group on the Right to Development on its third session}, \hfill{}58$^{th}$ session, 11/04/2002 UN Doc. E/CN.4/2002/28/Rev.1
\vspace{0.00mm}

\vspace{0.00mm}
\setlength{\parindent}{0.00mm}
\setlength{\leftskip}{0.00mm}
\setlength{\rightskip}{0.00mm}


\vspace{0.00mm}

\vspace{0.00mm}
\setlength{\parindent}{0.00mm}
\setlength{\leftskip}{0.00mm}
\setlength{\rightskip}{0.00mm}

UN CHR, \textit{Report of the Working Group on the Right to Development on its fifth session }60$^{th}$ session\textit{,} \hfill{}
\vspace{0.00mm}

\vspace{0.00mm}
\setlength{\parindent}{0.00mm}
\setlength{\leftskip}{0.00mm}
\setlength{\rightskip}{0.00mm}

16/03/2004 UN Doc. E/CN.4/2004/23 
\vspace{0.00mm}

\vspace{0.00mm}
\setlength{\parindent}{0.00mm}
\setlength{\leftskip}{0.00mm}
\setlength{\rightskip}{0.00mm}


\vspace{0.00mm}

\vspace{0.00mm}
\setlength{\parindent}{0.00mm}
\setlength{\leftskip}{0.00mm}
\setlength{\rightskip}{0.00mm}

UN CHR, \textit{Report of the Working Group on the Right to Development on its fifth session,} 60$^{th}$ session, \hfill{}22/04/2004 UN Doc. E/CN.4/2004/23/Corr.1 
\vspace{0.00mm}

\vspace{0.00mm}
\setlength{\parindent}{0.00mm}
\setlength{\leftskip}{0.00mm}
\setlength{\rightskip}{0.00mm}


\vspace{0.00mm}

\vspace{0.00mm}
\setlength{\parindent}{0.00mm}
\setlength{\leftskip}{0.00mm}
\setlength{\rightskip}{0.00mm}

UN CHR,\textit{ Review of progress and obstacles in the promotion, implementation, operationalization, and \hfill{}enjoyment of the right to development}, 60$^{th}$ session, 17/02/2004, UN Doc. \hfill{}\hfill{}\hfill{}\hfill{}E/CN.4/2004/WG.18/2 
\vspace{0.00mm}

\vspace{0.00mm}
\setlength{\parindent}{0.00mm}
\setlength{\leftskip}{0.00mm}
\setlength{\rightskip}{0.00mm}


\vspace{0.00mm}

\vspace{0.00mm}
\setlength{\parindent}{0.00mm}
\setlength{\leftskip}{0.00mm}
\setlength{\rightskip}{0.00mm}

UN CHR, \textit{Right to development - Aide-m�moire prepared by the Secretariat}, 57$^{th}$ session, 18/12/2000 \hfill{}UN Doc. E/CN.4/2000/WG.18/CRP.5/Rev.1 
\vspace{0.00mm}

\vspace{0.00mm}
\setlength{\parindent}{0.00mm}
\setlength{\leftskip}{0.00mm}
\setlength{\rightskip}{0.00mm}


\vspace{0.00mm}

\vspace{0.00mm}
\setlength{\parindent}{0.00mm}
\setlength{\leftskip}{0.00mm}
\setlength{\rightskip}{0.00mm}

UN CHR, \textit{Right to Development - Report of the Open-Ended Working Group}, 57$^{th}$ session, 20/03/2001 \hfill{}UN Doc. E/CN.4/2001/26 
\vspace{0.00mm}

\vspace{0.00mm}
\setlength{\parindent}{0.00mm}
\setlength{\leftskip}{0.00mm}
\setlength{\rightskip}{0.00mm}


\vspace{0.00mm}

\vspace{0.00mm}
\setlength{\parindent}{0.00mm}
\setlength{\leftskip}{0.00mm}
\setlength{\rightskip}{0.00mm}

UN CHR, \textit{Study on the current state of progress in the implementation of the right to development}, 55$^{th}$ \hfill{}session, 27/07/1999 UN Doc. E/CN.4/1999/WG.18/2 
\vspace{0.00mm}

\vspace{0.00mm}
\setlength{\parindent}{0.00mm}
\setlength{\leftskip}{0.00mm}
\setlength{\rightskip}{0.00mm}


\vspace{0.00mm}

\vspace{0.00mm}
\setlength{\parindent}{0.00mm}
\setlength{\leftskip}{0.00mm}
\setlength{\rightskip}{0.00mm}

UN CHR, T\textit{he importance and application of the principle of equity, at both the national and \hfill{}\hfill{}international levels: Report submitted by the High Commissioner}, 59$^{th}$ session, 30/12/2002, UN \hfill{}Doc. E/CN.4/2003/25 
\vspace{0.00mm}

\vspace{0.00mm}
\setlength{\parindent}{0.00mm}
\setlength{\leftskip}{0.00mm}
\setlength{\rightskip}{0.00mm}


\vspace{0.00mm}

\vspace{0.00mm}
\setlength{\parindent}{0.00mm}
\setlength{\leftskip}{0.00mm}
\setlength{\rightskip}{0.00mm}

UN CHR, \textit{The Right to Development - Report of the High Commissioner for Human Rights}, 57$^{th}$ \hfill{}\hfill{}session, 12/09/2000 UN Doc. E/CN.4/2000/WG.18/CRP.2 
\vspace{0.00mm}

\vspace{0.00mm}
\setlength{\parindent}{0.00mm}
\setlength{\leftskip}{0.00mm}
\setlength{\rightskip}{0.00mm}


\vspace{0.00mm}

\vspace{0.00mm}
\setlength{\parindent}{0.00mm}
\setlength{\leftskip}{0.00mm}
\setlength{\rightskip}{0.00mm}

UN CHR, \textit{The right to development - Report of the High Commissioner}, 60$^{th}$ session, 08/01/2004 UN \hfill{}Doc. E/CN.4/2004/22 
\vspace{0.00mm}

\vspace{0.00mm}
\setlength{\parindent}{0.00mm}
\setlength{\leftskip}{0.00mm}
\setlength{\rightskip}{0.00mm}


\vspace{0.00mm}

\vspace{0.00mm}
\setlength{\parindent}{0.00mm}
\setlength{\leftskip}{0.00mm}
\setlength{\rightskip}{0.00mm}

UN CHR, \textit{The right to development - Third report of the independent exper}t, 57$^{th}$ session, 02/01/2001 \hfill{}UN Doc. E/CN.4/2001/WG.18/2
\vspace{0.00mm}

\vspace{0.00mm}
\setlength{\parindent}{0.00mm}
\setlength{\leftskip}{0.00mm}
\setlength{\rightskip}{0.00mm}


\vspace{0.00mm}

\vspace{0.00mm}
\setlength{\parindent}{0.00mm}
\setlength{\leftskip}{0.00mm}
\setlength{\rightskip}{0.00mm}

UN CHR, \textit{The right to development,} 54th session, 22/04/1998 UN Doc. E/CN.4/RES/1998/72
\vspace{0.00mm}

\vspace{0.00mm}
\setlength{\parindent}{0.00mm}
\setlength{\leftskip}{0.00mm}
\setlength{\rightskip}{0.00mm}


\vspace{0.00mm}

\vspace{0.00mm}
\setlength{\parindent}{0.00mm}
\setlength{\leftskip}{0.00mm}
\setlength{\rightskip}{0.00mm}

UN CHR, \textit{The right to development}, 55$^{th}$ session, 28/04/1999 UN Doc. E/CN.4/RES/1999/79
\vspace{0.00mm}

\vspace{0.00mm}
\setlength{\parindent}{0.00mm}
\setlength{\leftskip}{0.00mm}
\setlength{\rightskip}{0.00mm}


\vspace{0.00mm}

\vspace{0.00mm}
\setlength{\parindent}{0.00mm}
\setlength{\leftskip}{0.00mm}
\setlength{\rightskip}{0.00mm}

UN CHR, \textit{The right to development}, 56$^{th}$ session, 13/04/2000 UN Doc. E/CN.4/RES/2000/5 
\vspace{0.00mm}

\vspace{0.00mm}
\setlength{\parindent}{0.00mm}
\setlength{\leftskip}{0.00mm}
\setlength{\rightskip}{0.00mm}


\vspace{0.00mm}

\vspace{0.00mm}
\setlength{\parindent}{0.00mm}
\setlength{\leftskip}{0.00mm}
\setlength{\rightskip}{0.00mm}

UN CHR, \textit{The right to development,} 57$^{th}$ session, 18/04/2001 UN Doc. E/CN.4/RES/2001/9 
\vspace{0.00mm}

\vspace{0.00mm}
\setlength{\parindent}{0.00mm}
\setlength{\leftskip}{0.00mm}
\setlength{\rightskip}{0.00mm}


\vspace{0.00mm}

\vspace{0.00mm}
\setlength{\parindent}{0.00mm}
\setlength{\leftskip}{0.00mm}
\setlength{\rightskip}{0.00mm}

UN CHR, \textit{The right to development}, 58$^{th}$ session, 25/04/2002 UN Doc. E/CN.4/RES/2002/69 
\vspace{0.00mm}

\vspace{0.00mm}
\setlength{\parindent}{0.00mm}
\setlength{\leftskip}{0.00mm}
\setlength{\rightskip}{0.00mm}


\vspace{0.00mm}

\vspace{0.00mm}
\setlength{\parindent}{0.00mm}
\setlength{\leftskip}{0.00mm}
\setlength{\rightskip}{0.00mm}

UN CHR, \textit{The right to development}, 59$^{th}$ session, 08/04/2003 UN Doc. E/CN.4/2003/L.14 
\vspace{0.00mm}

\vspace{0.00mm}
\setlength{\parindent}{0.00mm}
\setlength{\leftskip}{0.00mm}
\setlength{\rightskip}{0.00mm}


\vspace{0.00mm}

\vspace{0.00mm}
\setlength{\parindent}{0.00mm}
\setlength{\leftskip}{0.00mm}
\setlength{\rightskip}{0.00mm}

UN CHR, \textit{The right to development,} 59$^{th}$ session, 08/04/2003 UN Doc. E/CN.4/2003/L.14 
\vspace{0.00mm}

\vspace{0.00mm}
\setlength{\parindent}{0.00mm}
\setlength{\leftskip}{0.00mm}
\setlength{\rightskip}{0.00mm}


\vspace{0.00mm}

\vspace{0.00mm}
\setlength{\parindent}{0.00mm}
\setlength{\leftskip}{0.00mm}
\setlength{\rightskip}{0.00mm}

UN CHR,\textit{ The right to development,} 59$^{th}$ session, 25/04/2003 UN Doc. E/CN.4/RES/2003/83 
\vspace{0.00mm}

\vspace{0.00mm}
\setlength{\parindent}{0.00mm}
\setlength{\leftskip}{0.00mm}
\setlength{\rightskip}{0.00mm}


\vspace{0.00mm}

\vspace{0.00mm}
\setlength{\parindent}{0.00mm}
\setlength{\leftskip}{0.00mm}
\setlength{\rightskip}{0.00mm}

UN CHR, \textit{The right to development}, 60$^{th}$ session, 13/04/2004, UN Doc. E/CN.4/RES/2004/7 
\vspace{0.00mm}

\vspace{0.00mm}
\setlength{\parindent}{0.00mm}
\setlength{\leftskip}{0.00mm}
\setlength{\rightskip}{0.00mm}


\vspace{0.00mm}

\vspace{0.00mm}
\setlength{\parindent}{0.00mm}
\setlength{\leftskip}{0.00mm}
\setlength{\rightskip}{0.00mm}

UN CHR, \textit{The right to development}, 60$^{th}$ session, 21/01/2004 UN Doc. E/CN.4/2004/116 
\vspace{0.00mm}

\vspace{0.00mm}
\setlength{\parindent}{0.00mm}
\setlength{\leftskip}{0.00mm}
\setlength{\rightskip}{0.00mm}


\vspace{0.00mm}

\vspace{0.00mm}
\setlength{\parindent}{0.00mm}
\setlength{\leftskip}{0.00mm}
\setlength{\rightskip}{0.00mm}

UN CHR, \textit{Working group on the right to development: Information submitted by the Centre Europe-\hfill{}}
\vspace{0.00mm}

\vspace{0.00mm}
\setlength{\parindent}{0.00mm}
\setlength{\leftskip}{0.00mm}
\setlength{\rightskip}{0.00mm}

\textit{Tiers Monde}, 56$^{th}$ session, 23/11/1999 UN Doc. E/CN.4/2000/WG.18/CRP.3 
\vspace{0.00mm}

\vspace{0.00mm}
\setlength{\parindent}{0.00mm}
\setlength{\leftskip}{0.00mm}
\setlength{\rightskip}{0.00mm}


\vspace{0.00mm}

\vspace{0.00mm}
\setlength{\parindent}{0.00mm}
\setlength{\leftskip}{0.00mm}
\setlength{\rightskip}{0.00mm}

UN CHR, \textit{Written statement submitted by International League for Human Rights}, 59$^{th}$ session \hfill{}\hfill{}15/03/2003 UN Doc. E/CN.4/2003/NGO/163 
\vspace{0.00mm}

\vspace{0.00mm}
\setlength{\parindent}{0.00mm}
\setlength{\leftskip}{0.00mm}
\setlength{\rightskip}{0.00mm}


\vspace{0.00mm}

\vspace{0.00mm}
\setlength{\parindent}{0.00mm}
\setlength{\leftskip}{0.00mm}
\setlength{\rightskip}{0.00mm}

UN CHR, \textit{Written statement submitted by the International Federation of Rural Adult Catholic }\hfill{}\hfill{}\textit{Movements (FIMARC)}, 60$^{th}$ session, 18/03/2004, UN Doc.E/CN.4/2004/NGO/257 
\vspace{0.00mm}

\vspace{0.00mm}
\setlength{\parindent}{0.00mm}
\setlength{\leftskip}{0.00mm}
\setlength{\rightskip}{0.00mm}


\vspace{0.00mm}

\vspace{0.00mm}
\setlength{\parindent}{0.00mm}
\setlength{\leftskip}{0.00mm}
\setlength{\rightskip}{0.00mm}

UN CHR, \textit{Written submission by the World Health Organization (WHO)}, 60$^{th}$ session,12/03/2004, UN \hfill{}Doc.E/CN.4/2004/120 \hfill{}\hfill{}
\vspace{0.00mm}

\vspace{0.00mm}
\setlength{\parindent}{0.00mm}
\setlength{\leftskip}{0.00mm}
\setlength{\rightskip}{0.00mm}


\vspace{0.00mm}

\vspace{0.00mm}
\setlength{\parindent}{0.00mm}
\setlength{\leftskip}{0.00mm}
\setlength{\rightskip}{0.00mm}

UN ECOSOC, \textit{The right to developmen}t, 1995, 25/07/1995 UN Doc. E/DEC/1995/258 
\vspace{0.00mm}

\vspace{0.00mm}
\setlength{\parindent}{0.00mm}
\setlength{\leftskip}{0.00mm}
\setlength{\rightskip}{0.00mm}


\vspace{0.00mm}

\vspace{0.00mm}
\setlength{\parindent}{0.00mm}
\setlength{\leftskip}{0.00mm}
\setlength{\rightskip}{0.00mm}

UN ECOSOC,\textit{ The right to development,} 1998, 30/07/1998 UN Doc. E/DEC/1998/269 
\vspace{0.00mm}

\vspace{0.00mm}
\setlength{\parindent}{0.00mm}
\setlength{\leftskip}{0.00mm}
\setlength{\rightskip}{0.00mm}


\vspace{0.00mm}

\vspace{0.00mm}
\setlength{\parindent}{0.00mm}
\setlength{\leftskip}{0.00mm}
\setlength{\rightskip}{0.00mm}

UN ECOSOC, \textit{The right to development}, 2000, 28/07/2000 UN Doc. E/DEC/2000/246 
\vspace{0.00mm}

\vspace{0.00mm}
\setlength{\parindent}{0.00mm}
\setlength{\leftskip}{0.00mm}
\setlength{\rightskip}{0.00mm}


\vspace{0.00mm}

\vspace{0.00mm}
\setlength{\parindent}{0.00mm}
\setlength{\leftskip}{0.00mm}
\setlength{\rightskip}{0.00mm}

UN ECOSOC, \textit{The right to development}, 2002, 25/07/2002 UN Doc. E/DEC/2002/271
\vspace{0.00mm}

\vspace{0.00mm}
\setlength{\parindent}{0.00mm}
\setlength{\leftskip}{0.00mm}
\setlength{\rightskip}{0.00mm}


\vspace{0.00mm}

\vspace{0.00mm}
\setlength{\parindent}{0.00mm}
\setlength{\leftskip}{0.00mm}
\setlength{\rightskip}{0.00mm}

UN GA, \textit{Globalization and its impact on the full enjoyment of all human rights}, 58$^{th}$ session, \hfill{}\hfill{}07/08/2003 UN Doc. A/58/257  
\vspace{0.00mm}

\vspace{0.00mm}
\setlength{\parindent}{0.00mm}
\setlength{\leftskip}{0.00mm}
\setlength{\rightskip}{0.00mm}


\vspace{0.00mm}

\vspace{0.00mm}
\setlength{\parindent}{0.00mm}
\setlength{\leftskip}{0.00mm}
\setlength{\rightskip}{0.00mm}

UN GA, \textit{The right to development}, 56$^{th}$ session, 08/02/2002 UN Doc. A/RES/56/150 
\vspace{0.00mm}

\vspace{0.00mm}
\setlength{\parindent}{0.00mm}
\setlength{\leftskip}{0.00mm}
\setlength{\rightskip}{0.00mm}


\vspace{0.00mm}

\vspace{0.00mm}
\setlength{\parindent}{0.00mm}
\setlength{\leftskip}{0.00mm}
\setlength{\rightskip}{0.00mm}

UN GA, \textit{United Nations Millennium Declaration}, 55$^{th}$ session, 23/09/2002 UN Doc. A/RES/55/2 
\vspace{0.00mm}

\vspace{0.00mm}
\setlength{\parindent}{0.00mm}
\setlength{\leftskip}{0.00mm}
\setlength{\rightskip}{0.00mm}


\vspace{0.00mm}

\vspace{0.00mm}
\setlength{\parindent}{0.00mm}
\setlength{\leftskip}{0.00mm}
\setlength{\rightskip}{0.00mm}

UN SUBCOM,  \textit{Promoting the right to development in the context of the United Nations Decade for \hfill{}the Elimination of Poverty (1997-2006)}, 56$^{th}$ session, 01/06/2004, UN Doc.E/CN.4/Sub.2/2004/13 
\vspace{0.00mm}

\vspace{0.00mm}
\setlength{\parindent}{0.00mm}
\setlength{\leftskip}{0.00mm}
\setlength{\rightskip}{0.00mm}


\vspace{0.00mm}

\vspace{0.00mm}
\setlength{\parindent}{0.00mm}
\setlength{\leftskip}{0.00mm}
\setlength{\rightskip}{0.00mm}

UN SUBCOM, \textit{Mainstreaming the right to development into international trade law and policy at the \hfill{}World Trade Organization}, 56$^{th}$ session, 09/06/2004, UN Doc.E/CN.4/Sub.2/2004/17 
\vspace{0.00mm}

\vspace{0.00mm}
\setlength{\parindent}{0.00mm}
\setlength{\leftskip}{0.00mm}
\setlength{\rightskip}{0.00mm}


\vspace{0.00mm}

\vspace{0.00mm}
\setlength{\parindent}{0.00mm}
\setlength{\leftskip}{0.00mm}
\setlength{\rightskip}{0.00mm}

UN SUBCOM, \textit{Promotion of the realization of the right to drinking water and sanitation}, 55$^{th}$ session, 07/08/2003 UN doc. E/CN.4/Sub.2/2003/L.17
\vspace{0.00mm}

\vspace{0.00mm}
\setlength{\parindent}{0.00mm}
\setlength{\leftskip}{0.00mm}
\setlength{\rightskip}{0.00mm}


\vspace{0.00mm}

\vspace{0.00mm}
\setlength{\parindent}{0.00mm}
\setlength{\leftskip}{0.00mm}
\setlength{\rightskip}{0.00mm}

UN SUBCOM, \textit{The legal nature of the right to development and enhancement of its binding nature}, 56$^{th}$ session, 01/06/2004,  UN Doc.E/CN.4/Sub.2/2004/16
\vspace{0.00mm}

\vspace{0.00mm}
\setlength{\parindent}{0.00mm}
\setlength{\leftskip}{0.00mm}
\setlength{\rightskip}{0.00mm}


\vspace{0.00mm}

\vspace{0.00mm}
\setlength{\parindent}{0.00mm}
\setlength{\leftskip}{0.00mm}
\setlength{\rightskip}{0.00mm}

UN SUBCOM, \textit{The right to development,} 55$^{th}$ session, 06/08/2003 UN Doc. E/CN.4/Sub.2/2003/L.7 
\vspace{0.00mm}

\vspace{0.00mm}
\setlength{\parindent}{0.00mm}
\setlength{\leftskip}{0.00mm}
\setlength{\rightskip}{0.00mm}


\vspace{0.00mm}

\vspace{0.00mm}
\setlength{\parindent}{0.00mm}
\setlength{\leftskip}{0.00mm}
\setlength{\rightskip}{0.00mm}

UN SUBCOM, \textit{The right to development}, 56$^{th}$ session, 09/08/2004, UN Doc.  \hfill{}\hfill{}\hfill{}\hfill{}E/CN.4/SUB.2/DEC/2004/104 
\vspace{0.00mm}

\vspace{0.00mm}
\setlength{\parindent}{0.00mm}
\setlength{\leftskip}{0.00mm}
\setlength{\rightskip}{0.00mm}


\vspace{0.00mm}

\vspace{0.00mm}
\setlength{\parindent}{0.00mm}
\setlength{\leftskip}{0.00mm}
\setlength{\rightskip}{0.00mm}

UN SUBCOM, \textit{The right to development}, 56$^{th}$ session, 26/07/2004, UN Doc.E/CN.4/Sub.2/2004/14 
\vspace{0.00mm}

\vspace{0.00mm}
\setlength{\parindent}{0.00mm}
\setlength{\leftskip}{0.00mm}
\setlength{\rightskip}{0.00mm}


\vspace{0.00mm}

\vspace{0.00mm}
\setlength{\parindent}{0.00mm}
\setlength{\leftskip}{0.00mm}
\setlength{\rightskip}{0.00mm}

UN SUBCOM, \textit{The right to development: study on existing bilateral and multilateral programmes and }\hfill{}\textit{policies for development partnership}, 03/08/2004, 56$^{th}$ session, UN Doc. E/CN.4/Sub.2/2004/15/Corr.1 
\vspace{0.00mm}

\vspace{0.00mm}
\setlength{\parindent}{0.00mm}
\setlength{\leftskip}{0.00mm}
\setlength{\rightskip}{0.00mm}


\vspace{0.00mm}

\vspace{0.00mm}
\setlength{\parindent}{0.00mm}
\setlength{\leftskip}{0.00mm}
\setlength{\rightskip}{0.00mm}

UN SUBCOM, \textit{The right to development: study on existing bilateral and multilateral programmes and \hfill{}policies for development partnership}, 56$^{th}$ session, 03/08/2004, UN Doc. E/CN.4/Sub.2/2004/15 
\vspace{0.00mm}

\vspace{0.00mm}
\setlength{\parindent}{0.00mm}
\setlength{\leftskip}{0.00mm}
\setlength{\rightskip}{0.00mm}


\vspace{0.00mm}

\vspace{0.00mm}
\setlength{\parindent}{0.00mm}
\setlength{\leftskip}{0.00mm}
\setlength{\rightskip}{0.00mm}

UNDP, \textit{Human Rights and Human Development,} Human Development Report 2000,  United Nations, \hfill{}2000 \textcolor{blue}{{\uline{http://hdr.undp.org/reports/global/2000/en/}}}. Last accessed 5.7.04
\vspace{0.00mm}

\vspace{0.00mm}
\setlength{\parindent}{0.00mm}
\setlength{\leftskip}{0.00mm}
\setlength{\rightskip}{0.00mm}


\vspace{0.00mm}

\vspace{0.00mm}
\setlength{\parindent}{0.00mm}
\setlength{\leftskip}{0.00mm}
\setlength{\rightskip}{0.00mm}

UNDP,\textit{ Integrating Human Rights with Sustainable Human Development:} A UNDP Policy Document, \hfill{}United  Nations,  January 1998. \textcolor{blue}{{\uline{http://magnet.undp.org/Docs/policy5.html}}}. Last accessed 5.7.04
\vspace{0.00mm}

\vspace{0.00mm}
\setlength{\parindent}{0.00mm}
\setlength{\leftskip}{0.00mm}
\setlength{\rightskip}{0.00mm}


\vspace{0.00mm}

\vspace{0.00mm}
\setlength{\parindent}{0.00mm}
\setlength{\leftskip}{0.00mm}
\setlength{\rightskip}{0.00mm}

UNDP, \textit{Report of Symposium on Human Development and Human Rights,} Oslo, 2-3 October 1998 
\vspace{0.00mm}

\vspace{0.00mm}
\setlength{\parindent}{0.00mm}
\setlength{\leftskip}{0.00mm}
\setlength{\rightskip}{0.00mm}

UNDP, United Nations, 1998. \textcolor{blue}{{\uline{http://www.undp.org/hdro/events/oslo/Oslorep1.html}}}. Last accessed 10.8.04
\vspace{0.00mm}
\begin{itemize}

\item
\vspace{4.17mm}
\setlength{\parindent}{0.00mm}
\setlength{\leftskip}{0.00mm}
\setlength{\rightskip}{0.00mm}
\raggedright
\textbf{Annex 1: Declaration on the Right to Development}
\vspace{2.08mm}

\end{itemize}
\vspace{0.00mm}
\setlength{\parindent}{0.00mm}
\setlength{\leftskip}{0.00mm}
\setlength{\rightskip}{0.00mm}

Adopted by General Assembly resolution 41/128 of 4 December 1986
\vspace{0.00mm}

\vspace{0.00mm}
\setlength{\parindent}{0.00mm}
\setlength{\leftskip}{0.00mm}
\setlength{\rightskip}{0.00mm}

\\
\textit{The General Assembly,}
\vspace{0.00mm}

\vspace{0.00mm}
\setlength{\parindent}{0.00mm}
\setlength{\leftskip}{0.00mm}
\setlength{\rightskip}{0.00mm}

\textit{}
\vspace{0.00mm}

\vspace{0.00mm}
\setlength{\parindent}{0.00mm}
\setlength{\leftskip}{0.00mm}
\setlength{\rightskip}{0.00mm}

\textit{Bearing in mind} the purposes and principles of the Charter of the United Nations relating to the achievement of international co-operation in solving international problems of an economic, social, cultural or humanitarian nature, and in promoting and encouraging respect for human rights and fundamental freedoms for all without distinction as to race, sex, language or religion, 
\vspace{0.00mm}

\vspace{0.00mm}
\setlength{\parindent}{0.00mm}
\setlength{\leftskip}{0.00mm}
\setlength{\rightskip}{0.00mm}


\vspace{0.00mm}

\vspace{0.00mm}
\setlength{\parindent}{0.00mm}
\setlength{\leftskip}{0.00mm}
\setlength{\rightskip}{0.00mm}

\textit{Recognizing} that development is a comprehensive economic, social, cultural and political process, which aims at the constant improvement of the well-being of the entire population and of all individuals on the basis of their active, free and meaningful participation in development and in the fair distribution of benefits resulting therefrom, 
\vspace{0.00mm}

\vspace{0.00mm}
\setlength{\parindent}{0.00mm}
\setlength{\leftskip}{0.00mm}
\setlength{\rightskip}{0.00mm}


\vspace{0.00mm}

\vspace{0.00mm}
\setlength{\parindent}{0.00mm}
\setlength{\leftskip}{0.00mm}
\setlength{\rightskip}{0.00mm}

\textit{Considering} that under the provisions of the Universal Declaration of Human Rights everyone is entitled to a social and international order in which the rights and freedoms set forth in that Declaration can be fully realized, 
\vspace{0.00mm}

\vspace{0.00mm}
\setlength{\parindent}{0.00mm}
\setlength{\leftskip}{0.00mm}
\setlength{\rightskip}{0.00mm}


\vspace{0.00mm}

\vspace{0.00mm}
\setlength{\parindent}{0.00mm}
\setlength{\leftskip}{0.00mm}
\setlength{\rightskip}{0.00mm}

\textit{Recalling }the provisions of the International Covenant on Economic, Social and Cultural Rights and of the International Covenant on Civil and Political Rights, 
\vspace{0.00mm}

\vspace{0.00mm}
\setlength{\parindent}{0.00mm}
\setlength{\leftskip}{0.00mm}
\setlength{\rightskip}{0.00mm}


\vspace{0.00mm}

\vspace{0.00mm}
\setlength{\parindent}{0.00mm}
\setlength{\leftskip}{0.00mm}
\setlength{\rightskip}{0.00mm}

\textit{Recalling further }the relevant agreements, conventions, resolutions, recommendations and other instruments of the United Nations and its specialized agencies concerning the integral development of the human being, economic and social progress and development of all peoples, including those instruments concerning decolonization, the prevention of discrimination, respect for and observance of, human rights and fundamental freedoms, the maintenance of international peace and security and the further promotion of friendly relations and co-operation among States in accordance with the Charter, 
\vspace{0.00mm}

\vspace{0.00mm}
\setlength{\parindent}{0.00mm}
\setlength{\leftskip}{0.00mm}
\setlength{\rightskip}{0.00mm}


\vspace{0.00mm}

\vspace{0.00mm}
\setlength{\parindent}{0.00mm}
\setlength{\leftskip}{0.00mm}
\setlength{\rightskip}{0.00mm}

\textit{Recalling} the right of peoples to self-determination, by virtue of which they have the right freely to determine their political status and to pursue their economic, social and cultural development, 
\vspace{0.00mm}

\vspace{0.00mm}
\setlength{\parindent}{0.00mm}
\setlength{\leftskip}{0.00mm}
\setlength{\rightskip}{0.00mm}


\vspace{0.00mm}

\vspace{0.00mm}
\setlength{\parindent}{0.00mm}
\setlength{\leftskip}{0.00mm}
\setlength{\rightskip}{0.00mm}

\textit{Recalling also} the right of peoples to exercise, subject to the relevant provisions of both International Covenants on Human Rights, full and complete sovereignty over all their natural wealth and resources, 
\vspace{0.00mm}

\vspace{0.00mm}
\setlength{\parindent}{0.00mm}
\setlength{\leftskip}{0.00mm}
\setlength{\rightskip}{0.00mm}


\vspace{0.00mm}

\vspace{0.00mm}
\setlength{\parindent}{0.00mm}
\setlength{\leftskip}{0.00mm}
\setlength{\rightskip}{0.00mm}

\textit{Mindful }of the obligation of States under the Charter to promote universal respect for and observance of human rights and fundamental freedoms for all without distinction of any kind such as race, colour, sex, language, religion, political or other opinion, national or social origin, property, birth or other status, 
\vspace{0.00mm}

\vspace{0.00mm}
\setlength{\parindent}{0.00mm}
\setlength{\leftskip}{0.00mm}
\setlength{\rightskip}{0.00mm}


\vspace{0.00mm}

\vspace{0.00mm}
\setlength{\parindent}{0.00mm}
\setlength{\leftskip}{0.00mm}
\setlength{\rightskip}{0.00mm}

\textit{Considering }that the elimination of the massive and flagrant violations of the human rights of the peoples and individuals affected by situations such as those resulting from colonialism, neo-colonialism, apartheid, all forms of racism and racial discrimination, foreign domination and occupation, aggression and threats against national sovereignty, national unity and territorial integrity and threats of war would contribute to the establishment of circumstances propitious to the development of a great part of mankind,
\vspace{0.00mm}

\vspace{0.00mm}
\setlength{\parindent}{0.00mm}
\setlength{\leftskip}{0.00mm}
\setlength{\rightskip}{0.00mm}

 
\vspace{0.00mm}

\vspace{0.00mm}
\setlength{\parindent}{0.00mm}
\setlength{\leftskip}{0.00mm}
\setlength{\rightskip}{0.00mm}

\textit{Concerned }at the existence of serious obstacles to development, as well as to the complete fulfilment of human beings and of peoples, constituted, inter alia, by the denial of civil, political, economic, social and cultural rights, and considering that all human rights and fundamental freedoms are indivisible and interdependent and that, in order to promote development, equal attention and urgent consideration should be given to the implementation, promotion and protection of civil, political, economic, social and cultural rights and that, accordingly, the promotion of, respect for and enjoyment of certain human rights and fundamental freedoms cannot justify the denial of other human rights and fundamental freedoms, 
\vspace{0.00mm}

\vspace{0.00mm}
\setlength{\parindent}{0.00mm}
\setlength{\leftskip}{0.00mm}
\setlength{\rightskip}{0.00mm}


\vspace{0.00mm}

\vspace{0.00mm}
\setlength{\parindent}{0.00mm}
\setlength{\leftskip}{0.00mm}
\setlength{\rightskip}{0.00mm}

\textit{Considering} that international peace and security are essential elements for the realization of the right to development, 
\vspace{0.00mm}

\vspace{0.00mm}
\setlength{\parindent}{0.00mm}
\setlength{\leftskip}{0.00mm}
\setlength{\rightskip}{0.00mm}


\vspace{0.00mm}

\vspace{0.00mm}
\setlength{\parindent}{0.00mm}
\setlength{\leftskip}{0.00mm}
\setlength{\rightskip}{0.00mm}

\textit{Reaffirming} that there is a close relationship between disarmament and development and that progress in the field of disarmament would considerably promote progress in the field of development and that resources released through disarmament measures should be devoted to the economic and social development and well-being of all peoples and, in particular, those of the developing countries, 
\vspace{0.00mm}

\vspace{0.00mm}
\setlength{\parindent}{0.00mm}
\setlength{\leftskip}{0.00mm}
\setlength{\rightskip}{0.00mm}


\vspace{0.00mm}

\vspace{0.00mm}
\setlength{\parindent}{0.00mm}
\setlength{\leftskip}{0.00mm}
\setlength{\rightskip}{0.00mm}

\textit{Recognizing} that the human person is the central subject of the development process and that development policy should therefore make the human being the main participant and beneficiary of development, 
\vspace{0.00mm}

\vspace{0.00mm}
\setlength{\parindent}{0.00mm}
\setlength{\leftskip}{0.00mm}
\setlength{\rightskip}{0.00mm}


\vspace{0.00mm}

\vspace{0.00mm}
\setlength{\parindent}{0.00mm}
\setlength{\leftskip}{0.00mm}
\setlength{\rightskip}{0.00mm}

\textit{Recognizing} that the creation of conditions favourable to the development of peoples and individuals is the primary responsibility of their States, 
\vspace{0.00mm}

\vspace{0.00mm}
\setlength{\parindent}{0.00mm}
\setlength{\leftskip}{0.00mm}
\setlength{\rightskip}{0.00mm}


\vspace{0.00mm}

\vspace{0.00mm}
\setlength{\parindent}{0.00mm}
\setlength{\leftskip}{0.00mm}
\setlength{\rightskip}{0.00mm}

\textit{Aware} that efforts at the international level to promote and protect human rights should be accompanied by efforts to establish a new international economic order, 
\vspace{0.00mm}

\vspace{0.00mm}
\setlength{\parindent}{0.00mm}
\setlength{\leftskip}{0.00mm}
\setlength{\rightskip}{0.00mm}


\vspace{0.00mm}

\vspace{0.00mm}
\setlength{\parindent}{0.00mm}
\setlength{\leftskip}{0.00mm}
\setlength{\rightskip}{0.00mm}

\textit{Confirming }that the right to development is an inalienable human right and that equality of opportunity for development is a prerogative both of nations and of individuals who make up nations, 
\vspace{0.00mm}

\vspace{0.00mm}
\setlength{\parindent}{0.00mm}
\setlength{\leftskip}{0.00mm}
\setlength{\rightskip}{0.00mm}


\vspace{0.00mm}

\vspace{0.00mm}
\setlength{\parindent}{0.00mm}
\setlength{\leftskip}{0.00mm}
\setlength{\rightskip}{0.00mm}

\textit{Proclaims} the following Declaration on the Right to Development: 
\vspace{0.00mm}

\vspace{0.00mm}
\setlength{\parindent}{0.00mm}
\setlength{\leftskip}{0.00mm}
\setlength{\rightskip}{0.00mm}


\vspace{0.00mm}

\vspace{0.00mm}
\setlength{\parindent}{0.00mm}
\setlength{\leftskip}{0.00mm}
\setlength{\rightskip}{0.00mm}

\textit{Article 1} 
\vspace{0.00mm}

\vspace{0.00mm}
\setlength{\parindent}{0.00mm}
\setlength{\leftskip}{0.00mm}
\setlength{\rightskip}{0.00mm}


\vspace{0.00mm}

\vspace{0.00mm}
\setlength{\parindent}{0.00mm}
\setlength{\leftskip}{0.00mm}
\setlength{\rightskip}{0.00mm}

1. The right to development is an inalienable human right by virtue of which every human person and all peoples are entitled to participate in, contribute to, and enjoy economic, social, cultural and political development, in which all human rights and fundamental freedoms can be fully realized. 
\vspace{0.00mm}

\vspace{0.00mm}
\setlength{\parindent}{0.00mm}
\setlength{\leftskip}{0.00mm}
\setlength{\rightskip}{0.00mm}

2. The human right to development also implies the full realization of the right of peoples to self-determination, which includes, subject to the relevant provisions of both International Covenants on Human Rights, the exercise of their inalienable right to full sovereignty over all their natural wealth and resources. 
\vspace{0.00mm}

\vspace{0.00mm}
\setlength{\parindent}{0.00mm}
\setlength{\leftskip}{0.00mm}
\setlength{\rightskip}{0.00mm}


\vspace{0.00mm}

\vspace{0.00mm}
\setlength{\parindent}{0.00mm}
\setlength{\leftskip}{0.00mm}
\setlength{\rightskip}{0.00mm}

\textit{Article 2} 
\vspace{0.00mm}

\vspace{0.00mm}
\setlength{\parindent}{0.00mm}
\setlength{\leftskip}{0.00mm}
\setlength{\rightskip}{0.00mm}


\vspace{0.00mm}

\vspace{0.00mm}
\setlength{\parindent}{0.00mm}
\setlength{\leftskip}{0.00mm}
\setlength{\rightskip}{0.00mm}

1. The human person is the central subject of development and should be the active participant and beneficiary of the right to development. 
\vspace{0.00mm}

\vspace{0.00mm}
\setlength{\parindent}{0.00mm}
\setlength{\leftskip}{0.00mm}
\setlength{\rightskip}{0.00mm}

2. All human beings have a responsibility for development, individually and collectively, taking into account the need for full respect for their human rights and fundamental freedoms as well as their duties to the community, which alone can ensure the free and complete fulfilment of the human being, and they should therefore promote and protect an appropriate political, social and economic order for development. 
\vspace{0.00mm}

\vspace{0.00mm}
\setlength{\parindent}{0.00mm}
\setlength{\leftskip}{0.00mm}
\setlength{\rightskip}{0.00mm}

3. States have the right and the duty to formulate appropriate national development policies that aim at the constant improvement of the well-being of the entire population and of all individuals, on the basis of their active, free and meaningful participation in development and in the fair distribution of the benefits resulting therefrom. 
\vspace{0.00mm}

\vspace{0.00mm}
\setlength{\parindent}{0.00mm}
\setlength{\leftskip}{0.00mm}
\setlength{\rightskip}{0.00mm}


\vspace{0.00mm}

\vspace{0.00mm}
\setlength{\parindent}{0.00mm}
\setlength{\leftskip}{0.00mm}
\setlength{\rightskip}{0.00mm}

\textit{Article 3} 
\vspace{0.00mm}

\vspace{0.00mm}
\setlength{\parindent}{0.00mm}
\setlength{\leftskip}{0.00mm}
\setlength{\rightskip}{0.00mm}


\vspace{0.00mm}

\vspace{0.00mm}
\setlength{\parindent}{0.00mm}
\setlength{\leftskip}{0.00mm}
\setlength{\rightskip}{0.00mm}

1. States have the primary responsibility for the creation of national and international conditions favourable to the realization of the right to development. 
\vspace{0.00mm}

\vspace{0.00mm}
\setlength{\parindent}{0.00mm}
\setlength{\leftskip}{0.00mm}
\setlength{\rightskip}{0.00mm}

2. The realization of the right to development requires full respect for the principles of international law concerning friendly relations and co-operation among States in accordance with the Charter of the United Nations. 
\vspace{0.00mm}

\vspace{0.00mm}
\setlength{\parindent}{0.00mm}
\setlength{\leftskip}{0.00mm}
\setlength{\rightskip}{0.00mm}

3. States have the duty to co-operate with each other in ensuring development and eliminating obstacles to development. States should realize their rights and fulfil their duties in such a manner as to promote a new international economic order based on sovereign equality, interdependence, mutual interest and co-operation among all States, as well as to encourage the observance and realization of human rights. 
\vspace{0.00mm}

\vspace{0.00mm}
\setlength{\parindent}{0.00mm}
\setlength{\leftskip}{0.00mm}
\setlength{\rightskip}{0.00mm}


\vspace{0.00mm}

\vspace{0.00mm}
\setlength{\parindent}{0.00mm}
\setlength{\leftskip}{0.00mm}
\setlength{\rightskip}{0.00mm}

\textit{Article 4} 
\vspace{0.00mm}

\vspace{0.00mm}
\setlength{\parindent}{0.00mm}
\setlength{\leftskip}{0.00mm}
\setlength{\rightskip}{0.00mm}


\vspace{0.00mm}

\vspace{0.00mm}
\setlength{\parindent}{0.00mm}
\setlength{\leftskip}{0.00mm}
\setlength{\rightskip}{0.00mm}

1. States have the duty to take steps, individually and collectively, to formulate international development policies with a view to facilitating the full realization of the right to development. 
\vspace{0.00mm}

\vspace{0.00mm}
\setlength{\parindent}{0.00mm}
\setlength{\leftskip}{0.00mm}
\setlength{\rightskip}{0.00mm}

2. Sustained action is required to promote more rapid development of developing countries. As a complement to the efforts of developing countries, effective international co-operation is essential in providing these countries with appropriate means and facilities to foster their comprehensive development. 
\vspace{0.00mm}

\vspace{0.00mm}
\setlength{\parindent}{0.00mm}
\setlength{\leftskip}{0.00mm}
\setlength{\rightskip}{0.00mm}


\vspace{0.00mm}

\vspace{0.00mm}
\setlength{\parindent}{0.00mm}
\setlength{\leftskip}{0.00mm}
\setlength{\rightskip}{0.00mm}

\textit{Article 5} 
\vspace{0.00mm}

\vspace{0.00mm}
\setlength{\parindent}{0.00mm}
\setlength{\leftskip}{0.00mm}
\setlength{\rightskip}{0.00mm}


\vspace{0.00mm}

\vspace{0.00mm}
\setlength{\parindent}{0.00mm}
\setlength{\leftskip}{0.00mm}
\setlength{\rightskip}{0.00mm}

States shall take resolute steps to eliminate the massive and flagrant violations of the human rights of peoples and human beings affected by situations such as those resulting from apartheid, all forms of racism and racial discrimination, colonialism, foreign domination and occupation, aggression, foreign interference and threats against national sovereignty, national unity and territorial integrity, threats of war and refusal to recognize the fundamental right of peoples to self-determination. 
\vspace{0.00mm}

\vspace{0.00mm}
\setlength{\parindent}{0.00mm}
\setlength{\leftskip}{0.00mm}
\setlength{\rightskip}{0.00mm}


\vspace{0.00mm}

\vspace{0.00mm}
\setlength{\parindent}{0.00mm}
\setlength{\leftskip}{0.00mm}
\setlength{\rightskip}{0.00mm}

\textit{Article 6} 
\vspace{0.00mm}

\vspace{0.00mm}
\setlength{\parindent}{0.00mm}
\setlength{\leftskip}{0.00mm}
\setlength{\rightskip}{0.00mm}


\vspace{0.00mm}

\vspace{0.00mm}
\setlength{\parindent}{0.00mm}
\setlength{\leftskip}{0.00mm}
\setlength{\rightskip}{0.00mm}

1. All States should co-operate with a view to promoting, encouraging and strengthening universal respect for and observance of all human rights and fundamental freedoms for all without any distinction as to race, sex, language or religion. 
\vspace{0.00mm}

\vspace{0.00mm}
\setlength{\parindent}{0.00mm}
\setlength{\leftskip}{0.00mm}
\setlength{\rightskip}{0.00mm}

2. All human rights and fundamental freedoms are indivisible and interdependent; equal attention and urgent consideration should be given to the implementation, promotion and protection of civil, political, economic, social and cultural rights. 
\vspace{0.00mm}

\vspace{0.00mm}
\setlength{\parindent}{0.00mm}
\setlength{\leftskip}{0.00mm}
\setlength{\rightskip}{0.00mm}

3. States should take steps to eliminate obstacles to development resulting from failure to observe civil and political rights, as well as economic social and cultural rights. 
\vspace{0.00mm}

\vspace{0.00mm}
\setlength{\parindent}{0.00mm}
\setlength{\leftskip}{0.00mm}
\setlength{\rightskip}{0.00mm}


\vspace{0.00mm}

\vspace{0.00mm}
\setlength{\parindent}{0.00mm}
\setlength{\leftskip}{0.00mm}
\setlength{\rightskip}{0.00mm}

\textit{Article 7} 
\vspace{0.00mm}

\vspace{0.00mm}
\setlength{\parindent}{0.00mm}
\setlength{\leftskip}{0.00mm}
\setlength{\rightskip}{0.00mm}


\vspace{0.00mm}

\vspace{0.00mm}
\setlength{\parindent}{0.00mm}
\setlength{\leftskip}{0.00mm}
\setlength{\rightskip}{0.00mm}

All States should promote the establishment, maintenance and strengthening of international peace and security and, to that end, should do their utmost to achieve general and complete disarmament under effective international control, as well as to ensure that the resources released by effective disarmament measures are used for comprehensive development, in particular that of the developing countries. 
\vspace{0.00mm}

\vspace{0.00mm}
\setlength{\parindent}{0.00mm}
\setlength{\leftskip}{0.00mm}
\setlength{\rightskip}{0.00mm}


\vspace{0.00mm}

\vspace{0.00mm}
\setlength{\parindent}{0.00mm}
\setlength{\leftskip}{0.00mm}
\setlength{\rightskip}{0.00mm}

\textit{Article 8} 
\vspace{0.00mm}

\vspace{0.00mm}
\setlength{\parindent}{0.00mm}
\setlength{\leftskip}{0.00mm}
\setlength{\rightskip}{0.00mm}


\vspace{0.00mm}

\vspace{0.00mm}
\setlength{\parindent}{0.00mm}
\setlength{\leftskip}{0.00mm}
\setlength{\rightskip}{0.00mm}

1. States should undertake, at the national level, all necessary measures for the realization of the right to development and shall ensure, inter alia, equality of opportunity for all in their access to basic resources, education, health services, food, housing, employment and the fair distribution of income. Effective measures should be undertaken to ensure that women have an active role in the development process. Appropriate economic and social reforms should be carried out with a view to eradicating all social injustices. 
\vspace{0.00mm}

\vspace{0.00mm}
\setlength{\parindent}{0.00mm}
\setlength{\leftskip}{0.00mm}
\setlength{\rightskip}{0.00mm}

2. States should encourage popular participation in all spheres as an important factor in development and in the full realization of all human rights. 
\vspace{0.00mm}

\vspace{0.00mm}
\setlength{\parindent}{0.00mm}
\setlength{\leftskip}{0.00mm}
\setlength{\rightskip}{0.00mm}


\vspace{0.00mm}

\vspace{0.00mm}
\setlength{\parindent}{0.00mm}
\setlength{\leftskip}{0.00mm}
\setlength{\rightskip}{0.00mm}

\textit{Article 9} 
\vspace{0.00mm}

\vspace{0.00mm}
\setlength{\parindent}{0.00mm}
\setlength{\leftskip}{0.00mm}
\setlength{\rightskip}{0.00mm}


\vspace{0.00mm}

\vspace{0.00mm}
\setlength{\parindent}{0.00mm}
\setlength{\leftskip}{0.00mm}
\setlength{\rightskip}{0.00mm}

1. All the aspects of the right to development set forth in the present Declaration are indivisible and interdependent and each of them should be considered in the context of the whole. 
\vspace{0.00mm}

\vspace{0.00mm}
\setlength{\parindent}{0.00mm}
\setlength{\leftskip}{0.00mm}
\setlength{\rightskip}{0.00mm}

2. Nothing in the present Declaration shall be construed as being contrary to the purposes and principles of the United Nations, or as implying that any State, group or person has a right to engage in any activity or to perform any act aimed at the violation of the rights set forth in the Universal Declaration of Human Rights and in the International Covenants on Human Rights.
\vspace{0.00mm}

\vspace{0.00mm}
\setlength{\parindent}{0.00mm}
\setlength{\leftskip}{0.00mm}
\setlength{\rightskip}{0.00mm}

 
\vspace{0.00mm}

\vspace{0.00mm}
\setlength{\parindent}{0.00mm}
\setlength{\leftskip}{0.00mm}
\setlength{\rightskip}{0.00mm}

\textit{Article 10} 
\vspace{0.00mm}

\vspace{0.00mm}
\setlength{\parindent}{0.00mm}
\setlength{\leftskip}{0.00mm}
\setlength{\rightskip}{0.00mm}


\vspace{0.00mm}

\vspace{0.00mm}
\setlength{\parindent}{0.00mm}
\setlength{\leftskip}{0.00mm}
\setlength{\rightskip}{0.00mm}

Steps should be taken to ensure the full exercise and progressive enhancement of the right to development, including the formulation, adoption and implementation of policy, legislative and other measures at the national and international levels. 
\vspace{0.00mm}

\vspace{0.00mm}
\setlength{\parindent}{0.00mm}
\setlength{\leftskip}{0.00mm}
\setlength{\rightskip}{0.00mm}


\vspace{0.00mm}

\vspace{0.00mm}
\setlength{\parindent}{-4.91mm}
\setlength{\leftskip}{4.91mm}
\setlength{\rightskip}{0.00mm}
\raggedright
$^{}$GA Resolution 4/128, December 4 1986, available at http://www.unhchr.ch/html/menu3/b/74.htm
\vspace{0.00mm}

\vspace{0.00mm}
\setlength{\parindent}{-4.91mm}
\setlength{\leftskip}{4.91mm}
\setlength{\rightskip}{0.00mm}

$^{}$The Charter of the United Nations 1945, Article 55 states: 'With a view to the creation of conditions of stability and well-being which are necessary for peaceful and friendly relations among nations based on respect for the principle of equal rights and self-determination of peoples, the United Nations shall promote: a. higher standards of living, full employment, and conditions of economic and social progress and development; b. solutions of international economic, social, health, and related problems; and international cultural and educational co-operation; and c. universal respect for, and observance of, human rights and fundamental freedoms for all without distinction as to race, sex, language, or religion'. Article 56: 'All Members pledge themselves to take joint and separate action in cooperation with the Organisation for the achievement of the purposes set forth in Article 55'.
\vspace{0.00mm}

\vspace{0.00mm}
\setlength{\parindent}{-4.91mm}
\setlength{\leftskip}{4.91mm}
\setlength{\rightskip}{0.00mm}

$^{}$Universal Declaration of Human Rights 1948, Adopted by GA Resolution 217A III, December 10, 1949
\vspace{0.00mm}

\vspace{0.00mm}
\setlength{\parindent}{-4.91mm}
\setlength{\leftskip}{4.91mm}
\setlength{\rightskip}{0.00mm}

$^{}$Ibid Articles 1-21
\vspace{0.00mm}

\vspace{0.00mm}
\setlength{\parindent}{-4.91mm}
\setlength{\leftskip}{4.91mm}
\setlength{\rightskip}{0.00mm}
\raggedright
$^{}$Ibid Articles 22-28
\vspace{0.00mm}

\vspace{0.00mm}
\setlength{\parindent}{-4.91mm}
\setlength{\leftskip}{4.91mm}
\setlength{\rightskip}{0.00mm}

$^{}$Eide A, Krause C, Rosas A, '\textit{Economic, social and cultural rights: A Textbook', }1995, Martinus Nijhoff Publishers, The Netherlands.
\vspace{0.00mm}

\vspace{0.00mm}
\setlength{\parindent}{-4.91mm}
\setlength{\leftskip}{4.91mm}
\setlength{\rightskip}{0.00mm}
\raggedright
$^{}$International Labor Organisation Philadelphia Declaration, Article 2(a)
\vspace{0.00mm}

\vspace{0.00mm}
\setlength{\parindent}{-4.91mm}
\setlength{\leftskip}{4.91mm}
\setlength{\rightskip}{0.00mm}
\raggedright
$^{}$The Charter of the United Nations 1945, Preamble
\vspace{0.00mm}

\vspace{0.00mm}
\setlength{\parindent}{-4.91mm}
\setlength{\leftskip}{4.91mm}
\setlength{\rightskip}{0.00mm}
\raggedright
$^{}$Ibid Article 55
\vspace{0.00mm}

\vspace{0.00mm}
\setlength{\parindent}{-4.91mm}
\setlength{\leftskip}{4.91mm}
\setlength{\rightskip}{0.00mm}
\raggedright
$^{}$Ibid Article 56
\vspace{0.00mm}

\vspace{0.00mm}
\setlength{\parindent}{-4.91mm}
\setlength{\leftskip}{4.91mm}
\setlength{\rightskip}{0.00mm}
\raggedright
$^{}$Ibid Article 68
\vspace{0.00mm}

\vspace{0.00mm}
\setlength{\parindent}{-4.91mm}
\setlength{\leftskip}{4.91mm}
\setlength{\rightskip}{0.00mm}
\raggedright
$^{}$Cited in O'Brien J, \textit{International law}, London: Cavendish, 2001, p478
\vspace{0.00mm}

\vspace{0.00mm}
\setlength{\parindent}{-4.91mm}
\setlength{\leftskip}{4.91mm}
\setlength{\rightskip}{0.00mm}
\raggedright
$^{}$Article 1 of the Universal Declaration of Human Rights 
\vspace{0.00mm}

\vspace{0.00mm}
\setlength{\parindent}{-4.91mm}
\setlength{\leftskip}{4.91mm}
\setlength{\rightskip}{0.00mm}
\raggedright
$^{}$Article 2 ibid
\vspace{0.00mm}

\vspace{0.00mm}
\setlength{\parindent}{-4.91mm}
\setlength{\leftskip}{4.91mm}
\setlength{\rightskip}{0.00mm}

$^{}$CPR's in the Universal Declaration of Human Rights included: Article 3, the right to life liberty and security; Article 4, freedom from slavery; Article 5, freedom from torture, or cruel, inhuman or degrading treatment; Article 6 and Article 7 recognition and equality before the law; Article 8, the right to an effective remedy; Article 9, freedom from arbitrary arrest, detention and exile; Articles 10 and 11, due process of law; Article 12, the right to privacy; Article 13, freedom of movement; Article 14, right to seek asylum; Article 15, right to a nationality; Article 16, right to marry; Article 17, to own property; Article18, 19 and 20, freedoms of thought, expression and peaceful assembly; Article 21, the right to participate in the government of ones country.  
\vspace{0.00mm}

\vspace{0.00mm}
\setlength{\parindent}{-4.91mm}
\setlength{\leftskip}{4.91mm}
\setlength{\rightskip}{0.00mm}

$^{}$ESCR's in the Universal Declaration of Human Rights included: Article 23, the right to work; Article 24, the right to leisure and rest; Article 25, the right to an adequate standard of living including food, clothing, housing and medical care and necessary social services; Article 26,  the right to education; Article 27, the right to participate in the community. 
\vspace{0.00mm}

\vspace{0.00mm}
\setlength{\parindent}{-4.91mm}
\setlength{\leftskip}{4.91mm}
\setlength{\rightskip}{0.00mm}

$^{}$See  O'Brien J, \textit{International law}, London: Cavendish, 2001; Steiner H J and Alston P, \textit{International human rights in context: law, politics, morals}, 2$^{nd}$ ed, Oxford :Oxford UP, 2000 ; Shaw M N, \textit{International law}. 5$^{th}$ ed. Cambridge: Cambridge UP, 2003.
\vspace{0.00mm}

\vspace{0.00mm}
\setlength{\parindent}{-4.91mm}
\setlength{\leftskip}{4.91mm}
\setlength{\rightskip}{0.00mm}
\raggedright
$^{}$GA Res 1161 (XII) 26$^{th}$ November 1957
\vspace{0.00mm}

\vspace{0.00mm}
\setlength{\parindent}{-4.91mm}
\setlength{\leftskip}{4.91mm}
\setlength{\rightskip}{0.00mm}
\raggedright
$^{}$UN Doc E/3347Rev. 1 1960
\vspace{0.00mm}

\vspace{0.00mm}
\setlength{\parindent}{-4.91mm}
\setlength{\leftskip}{4.91mm}
\setlength{\rightskip}{0.00mm}

$^{}$The International Covenant on Civil and Political Rights 1966,  includes from Article 1 which is a general acknowledgement of the right of peoples to self-determination to the right to life, Art 6; freedom from torture and inhuman treatment, Art 7; freedom from slavery and forced labour, Art 8; the right to liberty and security, Art 9; the right of detained persons to be treated with humanity, Art 10; freedom from imprisonment for debt, Art 11; freedom of movement and choice of residence, Art 12; freedom of aliens from arbitrary expulsion, Art 13; the right to a fair trial, Art 14;  prohibition of conviction under retroactive criminal laws, Art 15; the right to recognition as a person before the law, Art 16; the right to privacy, Art 17; freedom of thought, conscience and religion, Art 18; freedom of opinion and expression, Art 19; the prohibition of propaganda for war and of incitement to national, racial or religious hatred, Art 20; freedom of assembly, Art 21; freedom of association, Art 22; the right to marry and have a family, Art 23; the rights of a child, Art 24; the right to participation in political life, Art 25; the right to equality before the law, Art 26 and the rights of minorities, Art 27.       
\vspace{0.00mm}

\vspace{0.00mm}
\setlength{\parindent}{-4.91mm}
\setlength{\leftskip}{4.91mm}
\setlength{\rightskip}{0.00mm}

$^{}$The International Covenant on Civil and Political Rights 1966, Optional Protocol 1 of 1966, in effect 1976, Article 1 'A State Party to the Covenant that becomes a Party to the present Protocol recognises the competence of the Committee to receive and consider communications from individuals subject to its jurisdiction who claim to be victims of a violation by that State Party of any of the rights set forth in the Covenant. No communication shall be received by the Committee if it concerns a State Party to the Covenant which is not a Party to the present Protocol'
\vspace{0.00mm}

\vspace{0.00mm}
\setlength{\parindent}{-4.91mm}
\setlength{\leftskip}{4.91mm}
\setlength{\rightskip}{0.00mm}

$^{}$International Covenant on Economic, Social and Cultural Rights 1966: Art 6, the right to employment; Art 7, the right to just and favourable conditions of work, including fair wages, equal pay, and paid holidays; Art 8, the right to form and join a trade union and the right to strike; Art 9, the right to social security, Art 10, protection of the family including mothers and children; Art 11, The right to an adequate standard of living including food, clothing and housing; Art 12, the right to health, physical and mental; Art 13, the right to education, Art 14 permits states to progressively implement Art 13; Art 15 the right to participate in the cultural life of the community.  
\vspace{0.00mm}

\vspace{0.00mm}
\setlength{\parindent}{-4.91mm}
\setlength{\leftskip}{4.91mm}
\setlength{\rightskip}{0.00mm}
\raggedright
$^{}$Ibid Article 18
\vspace{0.00mm}

\vspace{0.00mm}
\setlength{\parindent}{-4.91mm}
\setlength{\leftskip}{4.91mm}
\setlength{\rightskip}{0.00mm}
\raggedright
$^{}$Ibid Article 19
\vspace{0.00mm}

\vspace{0.00mm}
\setlength{\parindent}{-4.91mm}
\setlength{\leftskip}{4.91mm}
\setlength{\rightskip}{0.00mm}

$^{}$See Baehr P R., \textit{Human rights universality in practice }. Basingstoke: Palgrave, 2001; O'Brien J, \textit{International law}, London: Cavendish, 2001;  Shaw M N, \textit{International law}. 5$^{th}$ ed. Cambridge: Cambridge UP, 2003
\vspace{0.00mm}

\vspace{0.00mm}
\setlength{\parindent}{-4.91mm}
\setlength{\leftskip}{4.91mm}
\setlength{\rightskip}{0.00mm}
\raggedright
$^{}$Regional protections include the European Social Charter 1961.
\vspace{0.00mm}

\vspace{0.00mm}
\setlength{\parindent}{-4.91mm}
\setlength{\leftskip}{4.91mm}
\setlength{\rightskip}{0.00mm}

$^{}$See Vasak K and Alston P (eds)., \textit{The International Dimensions of Human Rights}, Westport, CT: Greenwood Press, Volume 1, 1982.
\vspace{0.00mm}

\vspace{0.00mm}
\setlength{\parindent}{-4.91mm}
\setlength{\leftskip}{4.91mm}
\setlength{\rightskip}{0.00mm}
\raggedright
$^{}$Ibid 
\vspace{0.00mm}

\vspace{0.00mm}
\setlength{\parindent}{-4.91mm}
\setlength{\leftskip}{4.91mm}
\setlength{\rightskip}{0.00mm}

$^{}$See Alston P.,\textit{ Revitalising United Nations Work on Human Rights and Development}, 18 MELB U. L. Rev. 216-219
\vspace{0.00mm}

\vspace{0.00mm}
\setlength{\parindent}{-4.91mm}
\setlength{\leftskip}{4.91mm}
\setlength{\rightskip}{0.00mm}
\raggedright
$^{}$Ibid
\vspace{0.00mm}

\vspace{0.00mm}
\setlength{\parindent}{-4.91mm}
\setlength{\leftskip}{4.91mm}
\setlength{\rightskip}{0.00mm}
\raggedright
$^{}$Ibid
\vspace{0.00mm}

\vspace{0.00mm}
\setlength{\parindent}{-4.91mm}
\setlength{\leftskip}{4.91mm}
\setlength{\rightskip}{0.00mm}

$^{}$The Proclamation of Tehran, International Conference on Human Rights, Tehran 13 May 1968, GA Res 2442 (XXIII) 19 December 1968 
\vspace{0.00mm}

\vspace{0.00mm}
\setlength{\parindent}{-4.91mm}
\setlength{\leftskip}{4.91mm}
\setlength{\rightskip}{0.00mm}
\raggedright
$^{}$Declaration on Social Progress and Development, adopted 11 December 1969, GA Res 2542 (XXIV)
\vspace{0.00mm}

\vspace{0.00mm}
\setlength{\parindent}{-4.91mm}
\setlength{\leftskip}{4.91mm}
\setlength{\rightskip}{0.00mm}

$^{}$Cited in Alston P., \textit{'The Right to Development at the International Level'} in R. Dupuy (ed), The Right to Development at the International Level,  Netherlands: Sijthoff and Noordhoff, 1981 p 101
\vspace{0.00mm}

\vspace{0.00mm}
\setlength{\parindent}{-4.91mm}
\setlength{\leftskip}{4.91mm}
\setlength{\rightskip}{0.00mm}
\raggedright
$^{}$Article 3 of the Declaration on Race and Racial Prejudice 1978 (UNESCO) 
\vspace{0.00mm}

\vspace{0.00mm}
\setlength{\parindent}{-4.91mm}
\setlength{\leftskip}{4.91mm}
\setlength{\rightskip}{0.00mm}
\raggedright
$^{}$Declaration on the Preparation of Societies for Life in Peace 1978, Resolution 33/73 (1978)
\vspace{0.00mm}

\vspace{0.00mm}
\setlength{\parindent}{-4.91mm}
\setlength{\leftskip}{4.91mm}
\setlength{\rightskip}{0.00mm}
\raggedright
$^{}$Commission on Human Rights, Resolution 4, (XXXV) 1979
\vspace{0.00mm}

\vspace{0.00mm}
\setlength{\parindent}{-4.91mm}
\setlength{\leftskip}{4.91mm}
\setlength{\rightskip}{0.00mm}
\raggedright
$^{}$African Charter of Human and Peoples' Rights 1981
\vspace{0.00mm}

\vspace{0.00mm}
\setlength{\parindent}{-4.91mm}
\setlength{\leftskip}{4.91mm}
\setlength{\rightskip}{0.00mm}
\raggedright
$^{}$The Federal Republic of Germany
\vspace{0.00mm}

\vspace{0.00mm}
\setlength{\parindent}{-4.91mm}
\setlength{\leftskip}{4.91mm}
\setlength{\rightskip}{0.00mm}
\raggedright
$^{}$Japan
\vspace{0.00mm}

\vspace{0.00mm}
\setlength{\parindent}{-4.91mm}
\setlength{\leftskip}{4.91mm}
\setlength{\rightskip}{0.00mm}
\raggedright
$^{}$Australia, although they did not abstain from TDRTD, they still expressed this concern.
\vspace{0.00mm}

\vspace{0.00mm}
\setlength{\parindent}{-4.91mm}
\setlength{\leftskip}{4.91mm}
\setlength{\rightskip}{0.00mm}
\raggedright
$^{}$USA
\vspace{0.00mm}

\vspace{0.00mm}
\setlength{\parindent}{-4.91mm}
\setlength{\leftskip}{4.91mm}
\setlength{\rightskip}{0.00mm}
\raggedright
$^{}$UK
\vspace{0.00mm}

\vspace{0.00mm}
\setlength{\parindent}{-4.91mm}
\setlength{\leftskip}{4.91mm}
\setlength{\rightskip}{0.00mm}

$^{}$See Bunn I D., '\textit{The Right to Development: Implications for International Economic Law'}, 15 American University International Law Review 1425 (2000): 1425-67. 
\vspace{0.00mm}

\vspace{0.00mm}
\setlength{\parindent}{-4.91mm}
\setlength{\leftskip}{4.91mm}
\setlength{\rightskip}{0.00mm}

$^{}$The Working Group of Governmental Experts was established pursuant to UN Commission on Human Rights  Resolution 36 (XXXVII) of 11$^{th}$ March 1981.  
\vspace{0.00mm}

\vspace{0.00mm}
\setlength{\parindent}{-4.91mm}
\setlength{\leftskip}{4.91mm}
\setlength{\rightskip}{0.00mm}

$^{}$Upon the recommendation of the Working Group of Governmental Experts
\vspace{0.00mm}

\vspace{0.00mm}
\setlength{\parindent}{-4.91mm}
\setlength{\leftskip}{4.91mm}
\setlength{\rightskip}{0.00mm}
\raggedright
$^{}$UN Commission on Human Rights  Resolution 1989/45 of 6$^{th}$ March 1989
\vspace{0.00mm}

\vspace{0.00mm}
\setlength{\parindent}{-4.91mm}
\setlength{\leftskip}{4.91mm}
\setlength{\rightskip}{0.00mm}

$^{}$Established pursuant to UN Commission on Human Rights  Resolution 1993/22 (1993). The second working group consisted of 15 experts and had a mandate for three years, to identify the obstacles of implementation and realisation of TRTD and to make recommendations towards its realisation and implementation by all states . The third working group was established pursuant to UN Commission on Human Rights  Resolution 1996/15. This group consisted of 10 experts, who had a mandate for two years, to elaborate on a strategy for the implementation and promotion of TRTD, concerning its multidimensional and integrated aspects.   
\vspace{0.00mm}

\vspace{0.00mm}
\setlength{\parindent}{-4.91mm}
\setlength{\leftskip}{4.91mm}
\setlength{\rightskip}{0.00mm}

$^{}$General Assembly Resolution 52/156, 12$^{th}$ December 1997 UN Doc. A/RES/52/136 adopted by a vote with 129 States in favour, 12 against and 32 abstentions.
\vspace{0.00mm}

\vspace{0.00mm}
\setlength{\parindent}{-4.91mm}
\setlength{\leftskip}{4.91mm}
\setlength{\rightskip}{0.00mm}

$^{}$The Independent Expert on the Right to Development, Arjun Sengupta, has written extensively on TRTD and has produced a number of reports for the Commission on Human Rights and also several journal articles. 
\vspace{0.00mm}

\vspace{0.00mm}
\setlength{\parindent}{-4.91mm}
\setlength{\leftskip}{4.91mm}
\setlength{\rightskip}{0.00mm}
\raggedright
$^{}$Commission on Human Rights Res. 72 UN Doc. E/CN.4/1998/177 (1998)
\vspace{0.00mm}

\vspace{0.00mm}
\setlength{\parindent}{-4.91mm}
\setlength{\leftskip}{4.91mm}
\setlength{\rightskip}{0.00mm}

$^{}$See preamble and Article 6 of  TDRTD
\vspace{0.00mm}

\vspace{0.00mm}
\setlength{\parindent}{-4.91mm}
\setlength{\leftskip}{4.91mm}
\setlength{\rightskip}{0.00mm}

$^{}$See Article 6 of  TDRTD
\vspace{0.00mm}

\vspace{0.00mm}
\setlength{\parindent}{-4.91mm}
\setlength{\leftskip}{4.91mm}
\setlength{\rightskip}{0.00mm}

$^{}$This is beyond the scope of this paper, for further reading see Baehr P R., \textit{Human rights universality in practice}. Basingstoke: Palgrave, 2001; Donnelly J., \textit{Universal Human Rights in Theory and Practice}, Ithaca: Cornell University Press, 1989.
\vspace{0.00mm}

\vspace{0.00mm}
\setlength{\parindent}{-4.91mm}
\setlength{\leftskip}{4.91mm}
\setlength{\rightskip}{0.00mm}

$^{}$Arjun Sengupta, Fourth Report of the independent expert to the Commission on Human Rights , E/CN.4/2002/WG.18/4, 20 December 2001, para 25.
\vspace{0.00mm}

\vspace{0.00mm}
\setlength{\parindent}{-4.91mm}
\setlength{\leftskip}{4.91mm}
\setlength{\rightskip}{0.00mm}
\raggedright
$^{}$Arjun Sengupta, Fourth Report of the independent expert to the Commission on Human Rights , E/CN.4/2002/WG.18/4, 20 December 2001, para 25.
\vspace{0.00mm}

\vspace{0.00mm}
\setlength{\parindent}{-4.91mm}
\setlength{\leftskip}{4.91mm}
\setlength{\rightskip}{0.00mm}

$^{}$The IE defines non-discrimination as; ``Non-discrimination means that there will be no discrimination on the grounds of race, colour, sex, language, political or other opinion, religion, national or social origin, property, birth or other status, not only between the beneficiaries but also between agents''. See Arjun Sengupta, Fourth Report of the independent expert to the Commission on Human Rights , E/CN.4/2002/WG.18/4, 20 December 2001 paras 26-32. Most of what the IE states is fairly straightforward but the concept of non-discrimination ``between agents'' was not specifically mentioned in TDRTD.  This element is not clarified satisfactorily in the IE's reports nor was it found to be used outside of the before-mentioned documents. Thus an in-depth discussion on this matter this is considered to be outside of the scope of the paper. 
\vspace{0.00mm}

\vspace{0.00mm}
\setlength{\parindent}{-4.91mm}
\setlength{\leftskip}{4.91mm}
\setlength{\rightskip}{0.00mm}

$^{}$It has been noted above that the word `should' carries less force and thus perhaps states could interpret the provision as carrying less force than other documents have given it. For example the ICESCR non discrimination clause in Article 2(2) states; ``The State Parties to the present Covenant undertake to guarantee that the rights enunciated in the present Covenant will be exercised without discrimination of any kind as to race, colour, sex, language, religion, political or other opinion, national or social origin, property, birth or other status''. Article 2(1) of the ICCPR states; ``Each State Party to the present Covenant undertakes to respect and to ensure to all individuals within its territory and subject to its jurisdiction the rights recognised in the present Covenant, without distinction of any kind, such as to race, colour, sex, language, religion, political or other opinion, national or social origin, property, birth or other status''.  
\vspace{0.00mm}

\vspace{0.00mm}
\setlength{\parindent}{-4.91mm}
\setlength{\leftskip}{4.91mm}
\setlength{\rightskip}{0.00mm}
\raggedright
$^{}$Cited in Bunn I D., '\textit{The Right to Development: Implications for International Economic Law'}, 15 American University International Law Review 1425 (2000): 1425-67. p1447 Footnote 81.
\vspace{0.00mm}

\vspace{0.00mm}
\setlength{\parindent}{-4.91mm}
\setlength{\leftskip}{4.91mm}
\setlength{\rightskip}{0.00mm}
\raggedright
$^{}$United Nations, \textit{The realization of the right to development: Global consultation on the right to development as a human right : report prepared by the Secretary-General pursuant to Commission on Human Rights resolution 1989/45,} United Nations, 1991, at 48. 
\vspace{0.00mm}

\vspace{0.00mm}
\setlength{\parindent}{-4.91mm}
\setlength{\leftskip}{4.91mm}
\setlength{\rightskip}{0.00mm}
\raggedright
$^{}$Bunn I D., '\textit{The Right to Development: Implications for International Economic Law'}, 15 American University International Law Review 1425 (2000): 1425-67. p1445
\vspace{0.00mm}

\vspace{0.00mm}
\setlength{\parindent}{-4.91mm}
\setlength{\leftskip}{4.91mm}
\setlength{\rightskip}{0.00mm}
\raggedright
$^{}$ Human Rights Council of Australia Inc. \textit{The Rights Way to Development: A Human Rights Approach to Development Assistance}. (1995) 118-21
\vspace{0.00mm}

\vspace{0.00mm}
\setlength{\parindent}{-4.91mm}
\setlength{\leftskip}{4.91mm}
\setlength{\rightskip}{0.00mm}
\raggedright
$^{}$Arjun Sengupta, Fourth Report of the independent expert to the Commission on Human Rights , E/CN.4/2002/WG.18/4, 20 December 2001 paras 26-32
\vspace{0.00mm}

\vspace{0.00mm}
\setlength{\parindent}{-4.91mm}
\setlength{\leftskip}{4.91mm}
\setlength{\rightskip}{0.00mm}

$^{}$See Bunn I D., '\textit{The Right to Development: Implications for International Economic Law'}, 15 American University International Law Review 1425 (2000): 1425-67 p1451 (Report of the Secretary-General on the Regional and International Dimensions of the Right to Development, 1980). 
\vspace{0.00mm}

\vspace{0.00mm}
\setlength{\parindent}{-4.91mm}
\setlength{\leftskip}{4.91mm}
\setlength{\rightskip}{0.00mm}

$^{}$Arjun Sengupta, Fourth Report of the independent expert to the Commission on Human Rights , E/CN.4/2002/WG.18/4, 20 December 2001 paras 26-32
\vspace{0.00mm}

\vspace{0.00mm}
\setlength{\parindent}{-4.91mm}
\setlength{\leftskip}{4.91mm}
\setlength{\rightskip}{0.00mm}
\raggedright
$^{}$Article 2(3) TDRTD
\vspace{0.00mm}

\vspace{0.00mm}
\setlength{\parindent}{-4.91mm}
\setlength{\leftskip}{4.91mm}
\setlength{\rightskip}{0.00mm}
\raggedright
$^{}$Article 8 TDRTD
\vspace{0.00mm}

\vspace{0.00mm}
\setlength{\parindent}{-4.91mm}
\setlength{\leftskip}{4.91mm}
\setlength{\rightskip}{0.00mm}

$^{}$Arjun Sengupta, Fourth Report of the independent expert to the Commission on Human Rights , E/CN.4/2002/WG.18/4, 20 December 2001 para 26.
\vspace{0.00mm}

\vspace{0.00mm}
\setlength{\parindent}{-4.91mm}
\setlength{\leftskip}{4.91mm}
\setlength{\rightskip}{0.00mm}

$^{}$Bunn I D., '\textit{The Right to Development: Implications for International Economic Law'}, 15 American University International Law Review 1425 (2000): 1425-67 p1447
\vspace{0.00mm}

\vspace{0.00mm}
\setlength{\parindent}{-4.91mm}
\setlength{\leftskip}{4.91mm}
\setlength{\rightskip}{0.00mm}
\raggedright
$^{}$Article 1(1) TDRTD 
\vspace{0.00mm}

\vspace{0.00mm}
\setlength{\parindent}{-4.91mm}
\setlength{\leftskip}{4.91mm}
\setlength{\rightskip}{0.00mm}
\raggedright
$^{}$Article 1(1) TDRTD
\vspace{0.00mm}

\vspace{0.00mm}
\setlength{\parindent}{-4.91mm}
\setlength{\leftskip}{4.91mm}
\setlength{\rightskip}{0.00mm}
\raggedright
$^{}$Article 2(2) TDRTD
\vspace{0.00mm}

\vspace{0.00mm}
\setlength{\parindent}{-4.91mm}
\setlength{\leftskip}{4.91mm}
\setlength{\rightskip}{0.00mm}
\raggedright
$^{}$Article 5 TDRTD
\vspace{0.00mm}

\vspace{0.00mm}
\setlength{\parindent}{-4.91mm}
\setlength{\leftskip}{4.91mm}
\setlength{\rightskip}{0.00mm}
\raggedright
$^{}$Articles 6(1), 6(2) TDRTD
\vspace{0.00mm}

\vspace{0.00mm}
\setlength{\parindent}{-4.91mm}
\setlength{\leftskip}{4.91mm}
\setlength{\rightskip}{0.00mm}

$^{}$Article 6(3) TDRTD could imply this is less obligatory because of the term ``should'' and thus is open to interpretation.
\vspace{0.00mm}

\vspace{0.00mm}
\setlength{\parindent}{-4.91mm}
\setlength{\leftskip}{4.91mm}
\setlength{\rightskip}{0.00mm}

$^{}$Sengupta A., \textit{Theory and Practice on the Right to Development}, Human Rights Quarterly; V.24(4); pp.837-889, p 843
\vspace{0.00mm}

\vspace{0.00mm}
\setlength{\parindent}{-4.91mm}
\setlength{\leftskip}{4.91mm}
\setlength{\rightskip}{0.00mm}

$^{}$Sen A., '\textit{The Right not to be Hungry'}, in Alston P., and Tomasevski K., (eds), The Right to Food, Netherlands: SIM, 1984. 
\vspace{0.00mm}

\vspace{0.00mm}
\setlength{\parindent}{-4.91mm}
\setlength{\leftskip}{4.91mm}
\setlength{\rightskip}{0.00mm}

$^{}$See Donnelly J, \textit{In Search of the Unicorn : The Jurisprudence and Politics of the Right to Development}, 15 Cal W Int'l L J 473, 1985
\vspace{0.00mm}

\vspace{0.00mm}
\setlength{\parindent}{-4.91mm}
\setlength{\leftskip}{4.91mm}
\setlength{\rightskip}{0.00mm}

$^{}$Sengupta A, \textit{Theory and Practice on the Right to Development}, Human Rights Quarterly; V.24(4); pp.837-889.
\vspace{0.00mm}

\vspace{0.00mm}
\setlength{\parindent}{-4.91mm}
\setlength{\leftskip}{4.91mm}
\setlength{\rightskip}{0.00mm}

$^{}$Donnelly J, \textit{In Search of the Unicorn : The Jurisprudence and Politics of the Right to Development}, 15 Cal W Int'l L J 473, 1985
\vspace{0.00mm}

\vspace{0.00mm}
\setlength{\parindent}{-4.91mm}
\setlength{\leftskip}{4.91mm}
\setlength{\rightskip}{0.00mm}

$^{}$Alston P., \textit{Making Space for New Human Rights : the case of the right to development,} Harvard Human Rights Yearbook; V.1; pp.3-40 p 32 
\vspace{0.00mm}

\vspace{0.00mm}
\setlength{\parindent}{-4.91mm}
\setlength{\leftskip}{4.91mm}
\setlength{\rightskip}{0.00mm}
\raggedright
$^{}$Ibid p 29 
\vspace{0.00mm}

\vspace{0.00mm}
\setlength{\parindent}{-4.91mm}
\setlength{\leftskip}{4.91mm}
\setlength{\rightskip}{0.00mm}
\raggedright
$^{}$Ibid p 29 and also Footnote 94
\vspace{0.00mm}

\vspace{0.00mm}
\setlength{\parindent}{-4.91mm}
\setlength{\leftskip}{4.91mm}
\setlength{\rightskip}{0.00mm}

$^{}$Article 1(2) TDRTD recognises the right of peoples to self determination. This has been endorsed in numerous human rights instruments and has been the subject of intense debate. For this reason, this aspect of TRTD will not be discussed in this paper. For further reading see; Alston P. (ed), \textit{Peoples' Rights}, Oxford: Oxford University Press, 2001; Abi- Saab G., \textit{'The Legal Formulation of the Right to Development}', in R. Dupuy (ed), The Right to Development at the International Level,  Netherlands: Sijthoff and Noordhoff, 1981. O'Brien J, \textit{International law}, London: Cavendish, 2001; Steiner H J and Alston P, \textit{International human rights in context: law, politics, morals}, 2$^{nd}$ ed, Oxford :Oxford UP, 2000 ; Shaw M N, \textit{International law}. 5$^{th}$ ed. Cambridge: Cambridge UP, 2003;  
\vspace{0.00mm}

\vspace{0.00mm}
\setlength{\parindent}{-4.91mm}
\setlength{\leftskip}{4.91mm}
\setlength{\rightskip}{0.00mm}

$^{}$Rich R., '\textit{The Right to Development: A Right of Peoples?}', in J. Crawford (ed), The Rights of Peoples, Oxford: Clarendon Press, 1988, 39-54.
\vspace{0.00mm}

\vspace{0.00mm}
\setlength{\parindent}{-4.91mm}
\setlength{\leftskip}{4.91mm}
\setlength{\rightskip}{0.00mm}

$^{}$See Donnelly J., \textit{In Search of the Unicorn : The Jurisprudence and Politics of the Right to Development}, 15 Cal W Int'l L J 473, 1985; Donnelly J., \textit{Human Rights as Natural Rights}, 4 Hum Rts Q 391, 1982; Donnelly J., \textit{Universal Human Rights in Theory and Practice}, Ithaca: Cornell University Press, 1989; Ghai Y., '\textit{Human Rights and Governance: The Asia Debate'}, 15 Australian Year Book of International Law (1994) 1. 
\vspace{0.00mm}

\vspace{0.00mm}
\setlength{\parindent}{-4.91mm}
\setlength{\leftskip}{4.91mm}
\setlength{\rightskip}{0.00mm}

$^{}$See Alston P., \textit{Making Space for New Human Rights : the case of the right to development,} Harvard Human Rights Yearbook; V.1; pp.3-40; Alston P., '\textit{The Shortcomings of a ``Garfield the Cat'' Approach to the Right to Development'}, California Western International Law Journal, Vol 15 (1985), 510; Alston P., \textit{A third generation of solidarity rights,: progressive development or obfuscation of international human rights law?} 29 Netherlands Int'l L Rev 307, 1982; Abi- Saab G., \textit{'The Legal Formulation of the Right to Development}', in R. Dupuy (ed), The Right to Development at the International Level,  Netherlands: Sijthoff and Noordhoff, 1981, p163; Marks S., \textit{Emerging human rights : a new generation fo the 1980's?} 33 Rutgers L Rev 435, 451, 1981; Obiora Amede L,\textit{ Beyond the Rhetoric of a Right to Development}, Law and Policy; V.18(3/4); pp.355-418 
\vspace{0.00mm}

\vspace{0.00mm}
\setlength{\parindent}{-4.91mm}
\setlength{\leftskip}{4.91mm}
\setlength{\rightskip}{0.00mm}
\raggedright
 
\vspace{0.00mm}

\vspace{0.00mm}
\setlength{\parindent}{-4.91mm}
\setlength{\leftskip}{4.91mm}
\setlength{\rightskip}{0.00mm}
\raggedright

\vspace{0.00mm}

\vspace{0.00mm}
\setlength{\parindent}{-4.91mm}
\setlength{\leftskip}{4.91mm}
\setlength{\rightskip}{0.00mm}
\raggedright

\vspace{0.00mm}

\vspace{0.00mm}
\setlength{\parindent}{-4.91mm}
\setlength{\leftskip}{4.91mm}
\setlength{\rightskip}{0.00mm}

$^{}$For further reading, see Hausserman J, \textit{The Realisation and Implementation of Economic Social and Cultural Rights}, London: Macmillan, 1992; Eide A, Krause C, Rosas A, \textit{'Economic, Social and Cultural Rights: A Textbook'},  The Netherlands: Martinus Nijhoff Publishers, 1995.
\vspace{0.00mm}

\vspace{0.00mm}
\setlength{\parindent}{-4.91mm}
\setlength{\leftskip}{4.91mm}
\setlength{\rightskip}{0.00mm}

$^{}$See Rich R., '\textit{The Right to Development: A Right of Peoples?}', in J. Crawford (ed), The Rights of Peoples, Oxford: Clarendon Press, 1988, 39-54.
\vspace{0.00mm}

\vspace{0.00mm}
\setlength{\parindent}{-4.91mm}
\setlength{\leftskip}{4.91mm}
\setlength{\rightskip}{0.00mm}

$^{}$Prott L, \textit{Cultural Rights as Peoples' Rights}, in Crawford (ed) The Rights of Peoples', 1992, Oxford University Press, New York. 
\vspace{0.00mm}

\vspace{0.00mm}
\setlength{\parindent}{-4.91mm}
\setlength{\leftskip}{4.91mm}
\setlength{\rightskip}{0.00mm}

$^{}$Prott L, \textit{Cultural Rights as Peoples' Rights}, in Crawford (ed) The Rights of Peoples', 1992, Oxford University Press, New York. P 102-103
\vspace{0.00mm}

\vspace{0.00mm}
\setlength{\parindent}{-4.91mm}
\setlength{\leftskip}{4.91mm}
\setlength{\rightskip}{0.00mm}

$^{}$Sieghart's articulation of the incompatible nature of collective rights demonstrates how dangerous the concept can actually be. See Sieghart P., \textit{The International Law of Human Rights}, Oxford: Oxford University Press, 1985,  p163-164
\vspace{0.00mm}

\vspace{0.00mm}
\setlength{\parindent}{-4.91mm}
\setlength{\leftskip}{4.91mm}
\setlength{\rightskip}{0.00mm}

$^{}$Ghai Y., '\textit{Human Rights and Governance: The Asia Debate'}, 15 Australian Year Book of International Law (1994) 1, at 10 
\vspace{0.00mm}

\vspace{0.00mm}
\setlength{\parindent}{-4.91mm}
\setlength{\leftskip}{4.91mm}
\setlength{\rightskip}{0.00mm}
\raggedright
$^{}$Ibid
\vspace{0.00mm}

\vspace{0.00mm}
\setlength{\parindent}{-4.91mm}
\setlength{\leftskip}{4.91mm}
\setlength{\rightskip}{0.00mm}

$^{}$Donnelly J., \textit{In Search of the Unicorn : The Jurisprudence and Politics of the Right to Development}, 15 Cal W Int'l L J 473, 1985, p 499
\vspace{0.00mm}

\vspace{0.00mm}
\setlength{\parindent}{-4.91mm}
\setlength{\leftskip}{4.91mm}
\setlength{\rightskip}{0.00mm}
\raggedright
$^{}$Article 9 DRTD
\vspace{0.00mm}

\vspace{0.00mm}
\setlength{\parindent}{-4.91mm}
\setlength{\leftskip}{4.91mm}
\setlength{\rightskip}{0.00mm}
\raggedright
$^{}$Article 2 DRTD
\vspace{0.00mm}

\vspace{0.00mm}
\setlength{\parindent}{-4.91mm}
\setlength{\leftskip}{4.91mm}
\setlength{\rightskip}{0.00mm}
\raggedright
$^{}$To be discussed below.
\vspace{0.00mm}

\vspace{0.00mm}
\setlength{\parindent}{-4.91mm}
\setlength{\leftskip}{4.91mm}
\setlength{\rightskip}{0.00mm}

$^{}$Orford A., '\textit{Globalisation and the Right to Development}', in Alston P. (ed), Peoples' Rights, Oxford: Oxford University Press, 2001., 127-185, p137
\vspace{0.00mm}

\vspace{0.00mm}
\setlength{\parindent}{-4.91mm}
\setlength{\leftskip}{4.91mm}
\setlength{\rightskip}{0.00mm}

$^{}$Alston P., '\textit{The Shortcomings of a ``Garfield the Cat'' Approach to the Right to Development'}, California Western International Law Journal, Vol 15 (1985), 510, p 512
\vspace{0.00mm}

\vspace{0.00mm}
\setlength{\parindent}{-4.91mm}
\setlength{\leftskip}{4.91mm}
\setlength{\rightskip}{0.00mm}
\raggedright
$^{}$To be discussed below
\vspace{0.00mm}

\vspace{0.00mm}
\setlength{\parindent}{-4.91mm}
\setlength{\leftskip}{4.91mm}
\setlength{\rightskip}{0.00mm}

$^{}$Galenkamp M., \textit{Collective Rights, }Report to the Advisory committee on Human Rights and Foreign Policy of the Netherlands, Rotterdam, Netherlands: 1998 available at \textcolor{blue}{{\uline{http://www.uu.nl/content/16-3.pdf}}}  p22
\vspace{0.00mm}

\vspace{0.00mm}
\setlength{\parindent}{-4.91mm}
\setlength{\leftskip}{4.91mm}
\setlength{\rightskip}{0.00mm}

$^{}$Alston P., \textit{'The Right to Development at the International Level'} in R. Dupuy (ed), The Right to Development at the International Level,  Netherlands: Sijthoff and Noordhoff, 1981; Alston P., '\textit{The Shortcomings of a ``Garfield the Cat'' Approach to the Right to Development'}, California Western International Law Journal, Vol 15 (1985), 510.
\vspace{0.00mm}

\vspace{0.00mm}
\setlength{\parindent}{-4.91mm}
\setlength{\leftskip}{4.91mm}
\setlength{\rightskip}{0.00mm}

$^{}$Working Group of Government Experts on the Right to Development, UN Doc E/1981/25 at para 17
\vspace{0.00mm}

\vspace{0.00mm}
\setlength{\parindent}{-4.91mm}
\setlength{\leftskip}{4.91mm}
\setlength{\rightskip}{0.00mm}

$^{}$Working Group of Government Experts on the Right to Development UN Doc ST/HR/SER.A.A/A at para 72
\vspace{0.00mm}

\vspace{0.00mm}
\setlength{\parindent}{-4.91mm}
\setlength{\leftskip}{4.91mm}
\setlength{\rightskip}{0.00mm}

$^{}$\textit{The International Dimensions of the Right to Development as a Human Right in relation to Other Human Rights based on International Cooperation, including the Right to Peace, taking into account the Requirements of the New International Economic Order and the Fundamental Human Needs}, Report of the Secretary-General, UN Doc E/CN.4/1334 (1979),  Cited in Donnelly J., \textit{In Search of the Unicorn : The Jurisprudence and Politics of the Right to Development}, 15 Cal W Int'l L J 473, 1985, p 475
\vspace{0.00mm}

\vspace{0.00mm}
\setlength{\parindent}{-4.91mm}
\setlength{\leftskip}{4.91mm}
\setlength{\rightskip}{0.00mm}
\raggedright
$^{}$Ibid
\vspace{0.00mm}

\vspace{0.00mm}
\setlength{\parindent}{-4.91mm}
\setlength{\leftskip}{4.91mm}
\setlength{\rightskip}{0.00mm}

$^{}$Ghai Y., '\textit{Human Rights and Governance: The Asia Debate'}, 15 Australian Year Book of International Law (1994) 1, p10 
\vspace{0.00mm}

\vspace{0.00mm}
\setlength{\parindent}{-4.91mm}
\setlength{\leftskip}{4.91mm}
\setlength{\rightskip}{0.00mm}

$^{}$Donnelly J., \textit{In Search of the Unicorn : The Jurisprudence and Politics of the Right to Development}, 15 Cal W Int'l L J 473, 1985,  p 502
\vspace{0.00mm}

\vspace{0.00mm}
\setlength{\parindent}{-4.91mm}
\setlength{\leftskip}{4.91mm}
\setlength{\rightskip}{0.00mm}

$^{}$Orford A., '\textit{Globalisation and the Right to Development}', in Alston P. (ed), Peoples' Rights, Oxford: Oxford University Press, 2001., 127-185 at p140
\vspace{0.00mm}

\vspace{0.00mm}
\setlength{\parindent}{-4.91mm}
\setlength{\leftskip}{4.91mm}
\setlength{\rightskip}{0.00mm}

$^{}$Donnelly J., \textit{In Search of the Unicorn : The Jurisprudence and Politics of the Right to Development}, 15 Cal W Int'l L J 473, 1985,  p 502
\vspace{0.00mm}

\vspace{0.00mm}
\setlength{\parindent}{-4.91mm}
\setlength{\leftskip}{4.91mm}
\setlength{\rightskip}{0.00mm}

$^{}$As described in Donnelly J., \textit{In Search of the Unicorn : The Jurisprudence and Politics of the Right to Development}, 15 Cal W Int'l L J 473, 1985,  p 502; See Arjun Sengupta, Fifth Report of the Independent Expert on the Right to Development, UN Doc E/CN.4/2002/WG.18/6 (2002)
\vspace{0.00mm}

\vspace{0.00mm}
\setlength{\parindent}{-4.91mm}
\setlength{\leftskip}{4.91mm}
\setlength{\rightskip}{0.00mm}

$^{}$Sengupta A., \textit{Theory and Practice on the Right to Development}, Human Rights Quarterly; V.24(4); pp.837-889, p 868
\vspace{0.00mm}

\vspace{0.00mm}
\setlength{\parindent}{-4.91mm}
\setlength{\leftskip}{4.91mm}
\setlength{\rightskip}{0.00mm}

$^{}$See Orford A., '\textit{Globalisation and the Right to Development}', in Alston P. (ed), Peoples' Rights, Oxford: Oxford University Press, 2001., 127-185,  p144
\vspace{0.00mm}

\vspace{0.00mm}
\setlength{\parindent}{-4.91mm}
\setlength{\leftskip}{4.91mm}
\setlength{\rightskip}{0.00mm}

$^{}$See Udombana N., '\textit{The Third World and the Right to Development: Agenda for the Next Millennium'}, Human Rights Quarterly, Vol 22(3) (2000), 753-87.  p755, footnote 9. 
\vspace{0.00mm}

\vspace{0.00mm}
\setlength{\parindent}{-4.91mm}
\setlength{\leftskip}{4.91mm}
\setlength{\rightskip}{0.00mm}

$^{}$Rodney W., \textit{How Europe Underdeveloped Africa}, Howard University Press, (revised ed) 1982, London and Tanzanian Publishing House: Dar es Salaam. 
\vspace{0.00mm}

\vspace{0.00mm}
\setlength{\parindent}{-4.91mm}
\setlength{\leftskip}{4.91mm}
\setlength{\rightskip}{0.00mm}

$^{}$Ibid, Rodney analyses what economic development is and how people developed themselves throughout history by having the capacity, jointly, to deal with nature. For him, this means, people being able to exploit the natural resources available to them and also understanding the laws of science and technology in order to effectively do this. See p 1-31  
\vspace{0.00mm}

\vspace{0.00mm}
\setlength{\parindent}{-4.91mm}
\setlength{\leftskip}{4.91mm}
\setlength{\rightskip}{0.00mm}

$^{}$Uvin P., \textit{On High Moral Ground: The Incorporation of Human Rights by the Development Enterprise}, Praxis- The Fletcher Journal of Development Studies, Volume XVII 2002,   available at \uline{http://fletcher.tufts.edu/praxis/xvii/Uvin.pdf }p2
\vspace{0.00mm}

\vspace{0.00mm}
\setlength{\parindent}{-4.91mm}
\setlength{\leftskip}{4.91mm}
\setlength{\rightskip}{0.00mm}

$^{}$Frankovits stated that; ``With an increasing demand for economic and social rights to be a major factor in development assistance, donors have tended to reformulate their terminology. Beginning with the World Bank's statement at the 1993 Conference on Human Rights in Vienna, followed by frequently heard assertions by individual donor agencies, the claim is made that all development assistance contributes to economic and social rights. Thus agricultural projects -- whatever their nature -- are claimed to contribute directly to the fulfilment of the right to food.'' Frankovits A., '\textit{Rejoinder:The Rights Way to Development}', Food Policy 21 no 1, 123-28, p 26
\vspace{0.00mm}

\vspace{0.00mm}
\setlength{\parindent}{-4.91mm}
\setlength{\leftskip}{4.91mm}
\setlength{\rightskip}{0.00mm}

$^{}$Udombana N., '\textit{The Third World and the Right to Development: Agenda for the Next Millennium'}, Human Rights Quarterly, Vol 22(3) (2000), 753-87. p 757
\vspace{0.00mm}

\vspace{0.00mm}
\setlength{\parindent}{-4.91mm}
\setlength{\leftskip}{4.91mm}
\setlength{\rightskip}{0.00mm}
\raggedright
$^{}$See UNDP Human Development Reports available at http://hdr.undp.org/
\vspace{0.00mm}

\vspace{0.00mm}
\setlength{\parindent}{-4.91mm}
\setlength{\leftskip}{4.91mm}
\setlength{\rightskip}{0.00mm}

$^{}$Charlesworth stated that; ``While the formulation of the right to development does not rest on a simple economic model of development, and includes within it a synthesis of all recognised human rights, redress of economic inequality is at its heart. An assumption of the international law of development is that underdevelopment is caused by a failure to meet the model of a capitalist economy. Development means industrialisation and westernisation.'' Charlesworth H, `The Public/Private Distinction and the Right to Development in International Law, 12 Australian Year Book of International Law (1992) at 196-197
\vspace{0.00mm}

\vspace{0.00mm}
\setlength{\parindent}{-4.91mm}
\setlength{\leftskip}{4.91mm}
\setlength{\rightskip}{0.00mm}

$^{}$Sieghart P., \textit{The International Law of Human Rights}, Oxford: Oxford University Press, 1985,  p164-165
\vspace{0.00mm}

\vspace{0.00mm}
\setlength{\parindent}{-4.91mm}
\setlength{\leftskip}{4.91mm}
\setlength{\rightskip}{0.00mm}
\raggedright
$^{}$Sen A., \textit{Development as Freedom}, Oxford University Press: Oxford, 1999 
\vspace{0.00mm}

\vspace{0.00mm}
\setlength{\parindent}{-4.91mm}
\setlength{\leftskip}{4.91mm}
\setlength{\rightskip}{0.00mm}

$^{}$Sen A., '\textit{The Right not to be Hungry'}, in Alston P., and Tomasevski K., (eds), The Right to Food, Netherlands: SIM, 1984. 
\vspace{0.00mm}

\vspace{0.00mm}
\setlength{\parindent}{-4.91mm}
\setlength{\leftskip}{4.91mm}
\setlength{\rightskip}{0.00mm}

$^{}$Piron L-H, \textit{The Right to Development: A Review of the Current State of the Debate}, (2002). Prepared for the UK Department for International Development. \textcolor{blue}{{\uline{http://www.odi.org.uk/pppg/publications/papers\_reports/dfid/issues/rights01/index.html}}}. Last accessed 26.10.04 
\vspace{0.00mm}

\vspace{0.00mm}
\setlength{\parindent}{-4.91mm}
\setlength{\leftskip}{4.91mm}
\setlength{\rightskip}{0.00mm}
\raggedright
$^{}$Article 2(2) of TDRTD 
\vspace{0.00mm}

\vspace{0.00mm}
\setlength{\parindent}{-4.91mm}
\setlength{\leftskip}{4.91mm}
\setlength{\rightskip}{0.00mm}

$^{}$See  Arjun Sengupta, Fourth Report of the independent expert to the Commission on Human Rights , E/CN.4/2002/WG.18/4, 20 December 2001. 
\vspace{0.00mm}

\vspace{0.00mm}
\setlength{\parindent}{-4.91mm}
\setlength{\leftskip}{4.91mm}
\setlength{\rightskip}{0.00mm}

$^{}$Cited in Steiner and Alston, International Human Rights in Context: Law, Politics and Morals, (1996) at 488. Also see Orford A., '\textit{Globalisation and the Right to Development}', in Alston P. (ed), Peoples' Rights, Oxford: Oxford University Press, 2001., 127-185  at 143 
\vspace{0.00mm}

\vspace{0.00mm}
\setlength{\parindent}{-4.91mm}
\setlength{\leftskip}{4.91mm}
\setlength{\rightskip}{0.00mm}
\raggedright
$^{}$Article 3(1) of TDRTD
\vspace{0.00mm}

\vspace{0.00mm}
\setlength{\parindent}{-4.91mm}
\setlength{\leftskip}{4.91mm}
\setlength{\rightskip}{0.00mm}
\raggedright
$^{}$Report of the Working Group on the Right to Development, UN Doc E/CN.4/1995/27
\vspace{0.00mm}

\vspace{0.00mm}
\setlength{\parindent}{-4.91mm}
\setlength{\leftskip}{4.91mm}
\setlength{\rightskip}{0.00mm}

$^{}$Udombana N., '\textit{The Third World and the Right to Development: Agenda for the Next Millennium'}, Human Rights Quarterly, Vol 22(3) (2000), 753-87.   
\vspace{0.00mm}

\vspace{0.00mm}
\setlength{\parindent}{-4.91mm}
\setlength{\leftskip}{4.91mm}
\setlength{\rightskip}{0.00mm}

$^{}$See Odumbana for an interesting notion of how Third World countries need to define and implement development for themselves. He states Third World countries ``must devise internal strategies for economic growth. They must develop their own resources and technology. Their future will remain bleak as long as they continue to copy foreign patterns of development. They should search for means of development within their own resources. They must change their attitude of depending on the goodwill of others \ldots They must, consequently, begin to pay more attention to the traditional values and attitudes of their societies.'' Udombana N., '\textit{The Third World and the Right to Development: Agenda for the Next Millennium'}, Human Rights Quarterly, Vol 22(3) (2000), 753-87, at 763
\vspace{0.00mm}

\vspace{0.00mm}
\setlength{\parindent}{-4.91mm}
\setlength{\leftskip}{4.91mm}
\setlength{\rightskip}{0.00mm}

$^{}$See Reports of the Independent Expert on the Right to Development, Arjun Sengupta, UN Docs E/CN.4/2002/WG.18/6 ; E/CN.4/2002/WG.18/6/Add.1; E/CN.4/2002/WG.18/2; 
\vspace{0.00mm}

\vspace{0.00mm}
\setlength{\parindent}{-4.91mm}
\setlength{\leftskip}{4.91mm}
\setlength{\rightskip}{0.00mm}

$^{}$Steiner and Alston enumerate five different obligations which include; to respect the rights of others, create institutional machinery essential to the realisation of rights, protect rights and prevent violations, provide goods and services to satisfy rights and promote rights. See Steiner H J., and Alston P., \textit{International human rights in context: law, politics, morals}, 2$^{nd}$ ed, Oxford :Oxford UP, 2000, pp 182-184    
\vspace{0.00mm}

\vspace{0.00mm}
\setlength{\parindent}{-4.91mm}
\setlength{\leftskip}{4.91mm}
\setlength{\rightskip}{0.00mm}

$^{}$Marks S, \textit{The Human Rights Approach to Development: 5 Approaches}, FXB Centre for Health and Human Rights, Working Paper 5 (2000). \textcolor{blue}{{\uline{http://www.hsph.harvard.edu/fxbcenter/FXBC\_WP6--}}}\hfill{}Marks.pdf.  At 14 
\vspace{0.00mm}

\vspace{0.00mm}
\setlength{\parindent}{-4.91mm}
\setlength{\leftskip}{4.91mm}
\setlength{\rightskip}{0.00mm}
\raggedright
$^{}$Ibid 
\vspace{0.00mm}

\vspace{0.00mm}
\setlength{\parindent}{-4.91mm}
\setlength{\leftskip}{4.91mm}
\setlength{\rightskip}{0.00mm}
\raggedright
$^{}$Ibid
\vspace{0.00mm}

\vspace{0.00mm}
\setlength{\parindent}{-4.91mm}
\setlength{\leftskip}{4.91mm}
\setlength{\rightskip}{0.00mm}
\raggedright
$^{}$Ibid 
\vspace{0.00mm}

\vspace{0.00mm}
\setlength{\parindent}{-4.91mm}
\setlength{\leftskip}{4.91mm}
\setlength{\rightskip}{0.00mm}
\raggedright
$^{}$Article 3(3) of TDRTD
\vspace{0.00mm}

\vspace{0.00mm}
\setlength{\parindent}{-4.91mm}
\setlength{\leftskip}{4.91mm}
\setlength{\rightskip}{0.00mm}

$^{}$For further reading see Udombana N., '\textit{The Third World and the Right to Development: Agenda for the Next Millennium'}, Human Rights Quarterly, Vol 22(3) (2000), 753-87. 
\vspace{0.00mm}

\vspace{0.00mm}
\setlength{\parindent}{-4.91mm}
\setlength{\leftskip}{4.91mm}
\setlength{\rightskip}{0.00mm}
\raggedright
$^{}$Article 4 of TDRTD
\vspace{0.00mm}

\vspace{0.00mm}
\setlength{\parindent}{-4.91mm}
\setlength{\leftskip}{4.91mm}
\setlength{\rightskip}{0.00mm}

$^{}$The specific paragraph of the preamble of TDRTD states that ``\textit{Bearing in mind} the purposes and principles of the Charter of the  United Nations relating to the achievement of international cooperation in solving international problems of an economic, social, cultural and humanitarian nature, and in promoting and encouraging respect for human rights and fundamental freedoms.''    
\vspace{0.00mm}

\vspace{0.00mm}
\setlength{\parindent}{-4.91mm}
\setlength{\leftskip}{4.91mm}
\setlength{\rightskip}{0.00mm}
\raggedright
$^{}$Article 1 UN Charter 
\vspace{0.00mm}

\vspace{0.00mm}
\setlength{\parindent}{-4.91mm}
\setlength{\leftskip}{4.91mm}
\setlength{\rightskip}{0.00mm}
\raggedright
$^{}$Article 55 UN Charter 
\vspace{0.00mm}

\vspace{0.00mm}
\setlength{\parindent}{-4.91mm}
\setlength{\leftskip}{4.91mm}
\setlength{\rightskip}{0.00mm}
\raggedright
$^{}$Article 56 UN Charter 
\vspace{0.00mm}

\vspace{0.00mm}
\setlength{\parindent}{-4.91mm}
\setlength{\leftskip}{4.91mm}
\setlength{\rightskip}{0.00mm}

$^{}$See Sengupta A., \textit{Theory and Practice on the Right to Development}, Human Rights Quarterly; V.24(4); pp.837-889, p854
\vspace{0.00mm}

\vspace{0.00mm}
\setlength{\parindent}{-4.91mm}
\setlength{\leftskip}{4.91mm}
\setlength{\rightskip}{0.00mm}
\raggedright
$^{}$Ibid
\vspace{0.00mm}

\vspace{0.00mm}
\setlength{\parindent}{-4.91mm}
\setlength{\leftskip}{4.91mm}
\setlength{\rightskip}{0.00mm}

$^{}$Office of the High Commissioner on Human Rights, The Right to Development, \textcolor{blue}{{\uline{www.unhchr.ch/development/right}}} .Also see Sengupta A, \textit{Towards Realizing the Right to Development}, Development and Change; June; V.31(3) 
\vspace{0.00mm}

\vspace{0.00mm}
\setlength{\parindent}{-4.91mm}
\setlength{\leftskip}{4.91mm}
\setlength{\rightskip}{0.00mm}

$^{}$Donnelly J., \textit{In Search of the Unicorn : The Jurisprudence and Politics of the Right to Development}, 15 Cal W Int'l L J 473, 1985 at 505
\vspace{0.00mm}

\vspace{0.00mm}
\setlength{\parindent}{-4.91mm}
\setlength{\leftskip}{4.91mm}
\setlength{\rightskip}{0.00mm}
\raggedright
$^{}$Ibid at 506 
\vspace{0.00mm}

\vspace{0.00mm}
\setlength{\parindent}{-4.91mm}
\setlength{\leftskip}{4.91mm}
\setlength{\rightskip}{0.00mm}
\raggedright
$^{}$Ibid  
\vspace{0.00mm}

\vspace{0.00mm}
\setlength{\parindent}{-4.91mm}
\setlength{\leftskip}{4.91mm}
\setlength{\rightskip}{0.00mm}
\raggedright
$^{}$Ibid 
\vspace{0.00mm}

\vspace{0.00mm}
\setlength{\parindent}{-4.91mm}
\setlength{\leftskip}{4.91mm}
\setlength{\rightskip}{0.00mm}

$^{}$Ghai Y., '\textit{Human Rights and Governance: The Asia Debate'}, 15 Australian Year Book of International Law (1994) 1.  at 10
\vspace{0.00mm}

\vspace{0.00mm}
\setlength{\parindent}{-4.91mm}
\setlength{\leftskip}{4.91mm}
\setlength{\rightskip}{0.00mm}
\raggedright
$^{}$Ibid
\vspace{0.00mm}

\vspace{0.00mm}
\setlength{\parindent}{-4.91mm}
\setlength{\leftskip}{4.91mm}
\setlength{\rightskip}{0.00mm}

$^{}$Orford A., '\textit{Globalisation and the Right to Development}', in Alston P. (ed), Peoples' Rights, Oxford: Oxford University Press, 2001., 127-185 at p 144
\vspace{0.00mm}

\vspace{0.00mm}
\setlength{\parindent}{-4.91mm}
\setlength{\leftskip}{4.91mm}
\setlength{\rightskip}{0.00mm}
\raggedright
$^{}$Ibid 
\vspace{0.00mm}

\vspace{0.00mm}
\setlength{\parindent}{-4.91mm}
\setlength{\leftskip}{4.91mm}
\setlength{\rightskip}{0.00mm}

$^{}$See Alston P., '\textit{The Myopia of the Herdmaidens: International Lawyers and Globalisation'}, 8 European Journal of International Law (1997) 435 
\vspace{0.00mm}

\vspace{0.00mm}
\setlength{\parindent}{-4.91mm}
\setlength{\leftskip}{4.91mm}
\setlength{\rightskip}{0.00mm}

$^{}$Orford A., '\textit{Globalisation and the Right to Development}', in Alston P. (ed), Peoples' Rights, Oxford: Oxford University Press, 2001., 127-185 at p 144
\vspace{0.00mm}

\vspace{0.00mm}
\setlength{\parindent}{-4.91mm}
\setlength{\leftskip}{4.91mm}
\setlength{\rightskip}{0.00mm}
\raggedright
$^{}$Ibid
\vspace{0.00mm}

\vspace{0.00mm}
\setlength{\parindent}{-4.91mm}
\setlength{\leftskip}{4.91mm}
\setlength{\rightskip}{0.00mm}

$^{}$The link between globalisation and TRTD and the activities of the WB, IMF and WTO have been extensively discussed in  Orford A., '\textit{Globalisation and the Right to Development}', in Alston P. (ed), Peoples' Rights, Oxford: Oxford University Press, 2001., 127-185. It is deemed as beyond the scope of this paper to discuss this issue in any depth.    
\vspace{0.00mm}

\vspace{0.00mm}
\setlength{\parindent}{-4.91mm}
\setlength{\leftskip}{4.91mm}
\setlength{\rightskip}{0.00mm}
\raggedright
$^{}$Article 7 TDRTD
\vspace{0.00mm}

\vspace{0.00mm}
\setlength{\parindent}{-4.91mm}
\setlength{\leftskip}{4.91mm}
\setlength{\rightskip}{0.00mm}

$^{}$However, due to the nature of this paper and the strict restrictions and as further research is necessary to understand the links between security, peace and development, this area is deemed as beyond the scope of this paper. 
\vspace{0.00mm}

\vspace{0.00mm}
\setlength{\parindent}{-4.91mm}
\setlength{\leftskip}{4.91mm}
\setlength{\rightskip}{0.00mm}

$^{}$Article 38(1) of the Statute of the Court states; ``The Court, whose function is to decide in accordance with international law such disputes as are submitted to it, shall apply: (a) international conventions, whether general or particular, establishing rules expressly recognized by the contesting states; (b) international custom, as evidence of a general practice accepted as law; (c) the general principles of law recognized by civilized nations; (d) subject to the provisions of Article 59, judicial decisions and the teachings of the most highly qualified publicists of the various nations, as subsidiary means for the determination of rules of law'', Article 38 (2) states: ``This provision shall not prejudice the power of the Court to decide a case \textit{ex aequo et bono}, if the parties agree thereto''. Article 59: ``The decision of the Court has no binding force except between the parties and in respect of that particular case.''
\vspace{0.00mm}

\vspace{0.00mm}
\setlength{\parindent}{-4.91mm}
\setlength{\leftskip}{4.91mm}
\setlength{\rightskip}{0.00mm}

$^{}$See Sieghart P., \textit{The International Law of Human Rights}, Oxford: Oxford University Press, 1985
\vspace{0.00mm}

\vspace{0.00mm}
\setlength{\parindent}{-4.91mm}
\setlength{\leftskip}{4.91mm}
\setlength{\rightskip}{0.00mm}
\raggedright
$^{}$Ibid p92
\vspace{0.00mm}

\vspace{0.00mm}
\setlength{\parindent}{-4.91mm}
\setlength{\leftskip}{4.91mm}
\setlength{\rightskip}{0.00mm}

$^{}$There are other incentives to create human rights treaties, for further reading see Sieghart P., \textit{The International Law of Human Rights}, Oxford: Oxford University Press, 1985
\vspace{0.00mm}

\vspace{0.00mm}
\setlength{\parindent}{-4.91mm}
\setlength{\leftskip}{4.91mm}
\setlength{\rightskip}{0.00mm}

$^{}$Provisions contained in other international treaties which are linked to TRTD indirectly will be discussed in a separate section.  
\vspace{0.00mm}

\vspace{0.00mm}
\setlength{\parindent}{-4.91mm}
\setlength{\leftskip}{4.91mm}
\setlength{\rightskip}{0.00mm}

$^{}$Adopted June 27, 1981, OAU Doc. CAB/LEG/67/3 rev. 5, 21 I.L.M. 58 (1982), \textit{entered into force} Oct. 21, 1986
\vspace{0.00mm}

\vspace{0.00mm}
\setlength{\parindent}{-4.91mm}
\setlength{\leftskip}{4.91mm}
\setlength{\rightskip}{0.00mm}

$^{}$Umozurike U O., \textit{The African Charter on Human and Peoples' Rights}, 77 Am. J. Int'l L. 902, (1983). at 911
\vspace{0.00mm}

\vspace{0.00mm}
\setlength{\parindent}{-4.91mm}
\setlength{\leftskip}{4.91mm}
\setlength{\rightskip}{0.00mm}
\raggedright
$^{}$Ibid
\vspace{0.00mm}

\vspace{0.00mm}
\setlength{\parindent}{-4.91mm}
\setlength{\leftskip}{4.91mm}
\setlength{\rightskip}{0.00mm}

$^{}$Umozurike U O., \textit{The African Charter on Human and Peoples' Rights}, 77 Am. J. Int'l L. 902, (1983). at 911
\vspace{0.00mm}

\vspace{0.00mm}
\setlength{\parindent}{-4.91mm}
\setlength{\leftskip}{4.91mm}
\setlength{\rightskip}{0.00mm}

$^{}$Slimane S, '\textit{Recognising Minorities in Africa}', Minority Rights Group International 2003, May 2003. available at www.minorityrights.org
\vspace{0.00mm}

\vspace{0.00mm}
\setlength{\parindent}{-4.91mm}
\setlength{\leftskip}{4.91mm}
\setlength{\rightskip}{0.00mm}
\raggedright
$^{}$Ibid at p 3. 
\vspace{0.00mm}

\vspace{0.00mm}
\setlength{\parindent}{-4.91mm}
\setlength{\leftskip}{4.91mm}
\setlength{\rightskip}{0.00mm}

$^{}$Umozurike U O., \textit{The African Charter on Human and Peoples' Rights}, 77 Am. J. Int'l L. 902, (1983). at 911
\vspace{0.00mm}

\vspace{0.00mm}
\setlength{\parindent}{-4.91mm}
\setlength{\leftskip}{4.91mm}
\setlength{\rightskip}{0.00mm}
\raggedright
$^{}$Ibid
\vspace{0.00mm}

\vspace{0.00mm}
\setlength{\parindent}{-4.91mm}
\setlength{\leftskip}{4.91mm}
\setlength{\rightskip}{0.00mm}

$^{}$Slimane S, '\textit{Recognising Minorities in Africa}', Minority Rights Group International 2003, May 2003. available at \textcolor{blue}{{\uline{www.minorityrights.org}}} p 4
\vspace{0.00mm}

\vspace{0.00mm}
\setlength{\parindent}{-4.91mm}
\setlength{\leftskip}{4.91mm}
\setlength{\rightskip}{0.00mm}

$^{}$ The European Convention for the Protection of Human Rights and Fundamental Freedoms, entered into force on 3$^{rd}$ September 1953, as amended by Protocol II which entered into force on 1$^{st}$ November 1998. 
\vspace{0.00mm}

\vspace{0.00mm}
\setlength{\parindent}{-4.91mm}
\setlength{\leftskip}{4.91mm}
\setlength{\rightskip}{0.00mm}
\raggedright
$^{}$ The American Convention on Human Rights entered into force on 18$^{th}$ July 1978
\vspace{0.00mm}

\vspace{0.00mm}
\setlength{\parindent}{-4.91mm}
\setlength{\leftskip}{4.91mm}
\setlength{\rightskip}{0.00mm}

$^{}$ However, an additional Protocol to establish an African Court on Human Rights entered into force on 26$^{th}$ December 2003. See Protocol to the African Charter on Human and Peoples' Rights Establishing an African Court on Human and Peoples' Rights. 
\vspace{0.00mm}

\vspace{0.00mm}
\setlength{\parindent}{-4.91mm}
\setlength{\leftskip}{4.91mm}
\setlength{\rightskip}{0.00mm}
\raggedright
$^{}$As noted above
\vspace{0.00mm}

\vspace{0.00mm}
\setlength{\parindent}{-4.91mm}
\setlength{\leftskip}{4.91mm}
\setlength{\rightskip}{0.00mm}

$^{}$Morel C., \textit{Defending Human Rights in Africa: The Case for Minority and Indigenous Rights}, Essex Human Rights Review. Vol 1. No.1  p55
\vspace{0.00mm}

\vspace{0.00mm}
\setlength{\parindent}{-4.91mm}
\setlength{\leftskip}{4.91mm}
\setlength{\rightskip}{0.00mm}

$^{}$The African Commission on Human and People's Rights, Guidelines for Submission of Communications, published by the Secretariat of the African Commission on Human and People's Rights, available at: http://www.achpr.org/english/\_info/guidelines\_communications\_en.html
\vspace{0.00mm}

\vspace{0.00mm}
\setlength{\parindent}{-4.91mm}
\setlength{\leftskip}{4.91mm}
\setlength{\rightskip}{0.00mm}

$^{}$S\textit{ocial and Economic Rights Action Center and the Center for Economic and Social Rights v Nigeria }Communication 155/96, 27 October 2001
\vspace{0.00mm}

\vspace{0.00mm}
\setlength{\parindent}{-4.91mm}
\setlength{\leftskip}{4.91mm}
\setlength{\rightskip}{0.00mm}

$^{}$Article 24 of the Banjul Charter states; ``All peoples shall have the right to a general satisfactory environment favourable to their development.'' This principle will be discussed in the subsequent chapter.
\vspace{0.00mm}

\vspace{0.00mm}
\setlength{\parindent}{-4.91mm}
\setlength{\leftskip}{4.91mm}
\setlength{\rightskip}{0.00mm}

$^{}$\textit{Social and Economic Rights Action Center and the Center for Economic and Social Rights v Nigeria }Communication 155/96, 27 October 2001, para 55
\vspace{0.00mm}

\vspace{0.00mm}
\setlength{\parindent}{-4.91mm}
\setlength{\leftskip}{4.91mm}
\setlength{\rightskip}{0.00mm}

$^{}$\textit{Malawi African Association etc v Mauritania} (2000) AHRLR 149 (ACHPR 2000).
\vspace{0.00mm}

\vspace{0.00mm}
\setlength{\parindent}{-4.91mm}
\setlength{\leftskip}{4.91mm}
\setlength{\rightskip}{0.00mm}

$^{}$\textit{Social and Economic Rights Action Center and the Center for Economic and Social Rights v Nigeria }Communication 155/96, 27 October 2001
\vspace{0.00mm}

\vspace{0.00mm}
\setlength{\parindent}{-4.91mm}
\setlength{\leftskip}{4.91mm}
\setlength{\rightskip}{0.00mm}

$^{}$\textit{Social and Economic Rights Action Center and the Center for Economic and Social Rights v Nigeria }Communication 155/96, 27 October 2001. Paragraph 67 states ``the survival of the Ogonis depended on their land and farms \ldots These and similar brutalities not only persecuted individuals on Ogoniland, but also the Ogoni community as a whole.''
\vspace{0.00mm}

\vspace{0.00mm}
\setlength{\parindent}{-4.91mm}
\setlength{\leftskip}{4.91mm}
\setlength{\rightskip}{0.00mm}

$^{}$Morel C., \textit{Defending Human Rights in Africa: The Case for Minority and Indigenous Rights}, Essex Human Rights Review. Vol 1. No.1 available at \uline{http://projects.essex.ac.uk/EHRR/archive/pdf/49.pdf}  p 57
\vspace{0.00mm}

\vspace{0.00mm}
\setlength{\parindent}{-4.91mm}
\setlength{\leftskip}{4.91mm}
\setlength{\rightskip}{0.00mm}

$^{}$For further reading see; International Law Association, London Conference (2000),  \textit{Final Report of the Committee on Formation of Customary (General) International Law}, Available at \uline{www.ila-hq.org}.
\vspace{0.00mm}

\vspace{0.00mm}
\setlength{\parindent}{-4.91mm}
\setlength{\leftskip}{4.91mm}
\setlength{\rightskip}{0.00mm}
\raggedright
$^{}$Asylum Case: Columbia v Peru (1950) ICJ Rep 266
\vspace{0.00mm}

\vspace{0.00mm}
\setlength{\parindent}{-4.91mm}
\setlength{\leftskip}{4.91mm}
\setlength{\rightskip}{0.00mm}

$^{}$See Anglo-Norwegian Fisheries Case (United Kingdom v Norway), Judgment of 18$^{th}$ December 1951 (1951) ICJ Rep 116. North Sea Continental  Shelf Cases (Federal Republic of Germany v Denmark); Federal Republic of Germany v Netherlands) Judgment of 20$^{th}$ February 1969 (1969) ICJ Rep 3.  
\vspace{0.00mm}

\vspace{0.00mm}
\setlength{\parindent}{-4.91mm}
\setlength{\leftskip}{4.91mm}
\setlength{\rightskip}{0.00mm}

$^{}$Military and Paramilitary Activities in and Against Nicaragua (Nicaragua v United States of America), (Merits), Judgment of 27$^{th}$ June 1986 ICJ Rep 14
\vspace{0.00mm}

\vspace{0.00mm}
\setlength{\parindent}{-4.91mm}
\setlength{\leftskip}{4.91mm}
\setlength{\rightskip}{0.00mm}

$^{}$North Sea Continental  Shelf Cases (Federal Republic of Germany v Denmark); Federal Republic of Germany v Netherlands) Judgment of 20$^{th}$ February 1969 (1969) ICJ Rep 3.  
\vspace{0.00mm}

\vspace{0.00mm}
\setlength{\parindent}{-4.91mm}
\setlength{\leftskip}{4.91mm}
\setlength{\rightskip}{0.00mm}
\raggedright
$^{}$Ibid 
\vspace{0.00mm}

\vspace{0.00mm}
\setlength{\parindent}{-4.91mm}
\setlength{\leftskip}{4.91mm}
\setlength{\rightskip}{0.00mm}

$^{}$Anglo-Norwegian Fisheries Case (United Kingdom v Norway), Judgment of 18$^{th}$ December 1951 (1951) ICJ Rep 116
\vspace{0.00mm}

\vspace{0.00mm}
\setlength{\parindent}{-4.91mm}
\setlength{\leftskip}{4.91mm}
\setlength{\rightskip}{0.00mm}

$^{}$Ibid
\vspace{0.00mm}

\vspace{0.00mm}
\setlength{\parindent}{-4.91mm}
\setlength{\leftskip}{4.91mm}
\setlength{\rightskip}{0.00mm}
\raggedright
$^{}$Asylum Case: Columbia v Peru (1950) ICJ Rep 266
\vspace{0.00mm}

\vspace{0.00mm}
\setlength{\parindent}{-4.91mm}
\setlength{\leftskip}{4.91mm}
\setlength{\rightskip}{0.00mm}
\raggedright
$^{}$Asylum Case: Columbia v Peru (1950) ICJ Rep 266, at p 288
\vspace{0.00mm}

\vspace{0.00mm}
\setlength{\parindent}{-4.91mm}
\setlength{\leftskip}{4.91mm}
\setlength{\rightskip}{0.00mm}

$^{}$North Sea Continental  Shelf Cases (Federal Republic of Germany v Denmark); Federal Republic of Germany v Netherlands) Judgment of 20$^{th}$ February 1969 (1969) ICJ Rep 3, 
\vspace{0.00mm}

\vspace{0.00mm}
\setlength{\parindent}{-4.91mm}
\setlength{\leftskip}{4.91mm}
\setlength{\rightskip}{0.00mm}

$^{}$Military and Paramilitary Activities in and Against Nicaragua (Nicaragua v United States of America), (Merits), Judgment of 27$^{th}$ June 1986 ICJ Rep 14 pp 97-98
\vspace{0.00mm}

\vspace{0.00mm}
\setlength{\parindent}{-4.91mm}
\setlength{\leftskip}{4.91mm}
\setlength{\rightskip}{0.00mm}

$^{}$Military and Paramilitary Activities in and Against Nicaragua (Nicaragua v United States of America), (Merits), Judgment of 27$^{th}$ June 1986 ICJ Rep 14. This was decided in relation to where a customary rule existed alongside a treaty. `
\vspace{0.00mm}

\vspace{0.00mm}
\setlength{\parindent}{-4.91mm}
\setlength{\leftskip}{4.91mm}
\setlength{\rightskip}{0.00mm}

$^{}$North Sea Continental  Shelf Cases (Federal Republic of Germany v Denmark); Federal Republic of Germany v Netherlands) Judgment of 20$^{th}$ February 1969 (1969) ICJ Rep 3. at pp 43-44 
\vspace{0.00mm}

\vspace{0.00mm}
\setlength{\parindent}{-4.91mm}
\setlength{\leftskip}{4.91mm}
\setlength{\rightskip}{0.00mm}
\raggedright
$^{}$Ibid 
\vspace{0.00mm}

\vspace{0.00mm}
\setlength{\parindent}{-4.91mm}
\setlength{\leftskip}{4.91mm}
\setlength{\rightskip}{0.00mm}

$^{}$International Law Association, London Conference (2000),  \textit{Final Report of the Committee on Formation of Customary }\textit{(General) International Law}, Available at \uline{www.ila-hq.org}. at p 26
\vspace{0.00mm}

\vspace{0.00mm}
\setlength{\parindent}{-4.91mm}
\setlength{\leftskip}{4.91mm}
\setlength{\rightskip}{0.00mm}

$^{}$Anglo-Norwegian Fisheries Case (United Kingdom v Norway), Judgment of 18$^{th}$ December 1951 (1951) ICJ Rep 116
\vspace{0.00mm}

\vspace{0.00mm}
\setlength{\parindent}{-4.91mm}
\setlength{\leftskip}{4.91mm}
\setlength{\rightskip}{0.00mm}
\raggedright
$^{}$Ibid pp 116 at 136
\vspace{0.00mm}

\vspace{0.00mm}
\setlength{\parindent}{-4.91mm}
\setlength{\leftskip}{4.91mm}
\setlength{\rightskip}{0.00mm}

$^{}$For further reading see Shaw M N, \textit{International law}. 5$^{th}$ ed. Cambridge: Cambridge UP, 2003
\vspace{0.00mm}

\vspace{0.00mm}
\setlength{\parindent}{-4.91mm}
\setlength{\leftskip}{4.91mm}
\setlength{\rightskip}{0.00mm}

$^{}$Declaration on Principles of International Law Concerning Friendly Relations and Co-operation among States in Accordance with the Charter of the United Nations 1970, Resolution 2625 (XXV). 
\vspace{0.00mm}

\vspace{0.00mm}
\setlength{\parindent}{-4.91mm}
\setlength{\leftskip}{4.91mm}
\setlength{\rightskip}{0.00mm}

$^{}$Military and Paramilitary Activities in and Against Nicaragua (Nicaragua v United States of America), (Merits), Judgment of 27$^{th}$ June 1986 ICJ Rep 14 pp 101
\vspace{0.00mm}

\vspace{0.00mm}
\setlength{\parindent}{-4.91mm}
\setlength{\leftskip}{4.91mm}
\setlength{\rightskip}{0.00mm}

$^{}$Legality of the Threat or Use of Nuclear Weapons (Request for Advisory Opinion by the World Health Organisation) (1996) ICJ Rep 90
\vspace{0.00mm}

\vspace{0.00mm}
\setlength{\parindent}{-4.91mm}
\setlength{\leftskip}{4.91mm}
\setlength{\rightskip}{0.00mm}
\raggedright
$^{}$Ibid pp 213
\vspace{0.00mm}

\vspace{0.00mm}
\setlength{\parindent}{-4.91mm}
\setlength{\leftskip}{4.91mm}
\setlength{\rightskip}{0.00mm}

$^{}$See Brownlie I., \textit{Principles of public international law}, 5th ed, Oxford, OUP, 1998.
\vspace{0.00mm}

\vspace{0.00mm}
\setlength{\parindent}{-4.91mm}
\setlength{\leftskip}{4.91mm}
\setlength{\rightskip}{0.00mm}

$^{}$It has been argued that Development Assistance could be used as evidence of state practice of TRTD. However, the \textit{opinio juris} of states who provide such assistance, is questionable. For an in depth study on why states do not believe they are legally obligated to give development assistance see; Piron L-H, \textit{The Right to Development: A Review of the Current State of the Debate}, (2002). Prepared for the UK Department for International Development. \textcolor{blue}{{\uline{http://www.odi.org.uk/pppg/publications/papers\_reports/dfid/issues/rights01/index.html}}}. Last accessed 26.10.04 
\vspace{0.00mm}

\vspace{0.00mm}
\setlength{\parindent}{-4.91mm}
\setlength{\leftskip}{4.91mm}
\setlength{\rightskip}{0.00mm}

$^{}$McDougal, Panel Discussions, The Legal Status of General Assembly Resolutions, Proceedings of the American Society of International Law. 73$^{rd}$ Annual Meeting (1979), 324, 328-29. Cited in International Law Association, London Conference (2000),  \textit{Final Report of the Committee on Formation of Customary (General) International Law}, Available at \textcolor{blue}{{\uline{www.ila-hq.org}}}. p59
\vspace{0.00mm}

\vspace{0.00mm}
\setlength{\parindent}{-4.91mm}
\setlength{\leftskip}{4.91mm}
\setlength{\rightskip}{0.00mm}
\raggedright
$^{}$Commission of Human Rights Resolution 4 (XXXIII) 1977
\vspace{0.00mm}

\vspace{0.00mm}
\setlength{\parindent}{-4.91mm}
\setlength{\leftskip}{4.91mm}
\setlength{\rightskip}{0.00mm}

$^{}$General Assembly Resolution 1161 (XII) 1957 states for example, ``a balanced and integrated economic and social development would contribute towards the promotion and maintenance of peace and security, social progress and better standards of living, and the observance of and respect for human rights and fundamental freedoms''. Compare with TDRTD Articles, for example, Article 1, 8 and the preamble of TDRTD confirm that economic, social and cultural development would contribute to the well-being of the entire population.  Articles 1(1), 2(2), 5 and 6, underline the need for respect for human rights.  Article 7 linking development and international peace and security. 
\vspace{0.00mm}

\vspace{0.00mm}
\setlength{\parindent}{-4.91mm}
\setlength{\leftskip}{4.91mm}
\setlength{\rightskip}{0.00mm}

$^{}$International Conference on Human Rights, Tehran 1969, Proclamation of Tehran paragraph 12 
\vspace{0.00mm}

\vspace{0.00mm}
\setlength{\parindent}{-4.91mm}
\setlength{\leftskip}{4.91mm}
\setlength{\rightskip}{0.00mm}
\raggedright
$^{}$Ibid para 13 
\vspace{0.00mm}

\vspace{0.00mm}
\setlength{\parindent}{-4.91mm}
\setlength{\leftskip}{4.91mm}
\setlength{\rightskip}{0.00mm}

$^{}$Declaration on Social Progress and Development 1969, Resolution 2542 (XXIV) stated ``[T]he right and responsibility of each State and, as far as they are concerned, each nation and people to determine freely its own objectives of social development, to set its own priorities and to decide in conformity with the principles of the Charter of the United Nations the means and methods of their achievement without any external interference''
\vspace{0.00mm}

\vspace{0.00mm}
\setlength{\parindent}{-4.91mm}
\setlength{\leftskip}{4.91mm}
\setlength{\rightskip}{0.00mm}
\raggedright
$^{}$Charter of Economic Rights and Duties of States 1974, Resolution 3281 (XXIX).
\vspace{0.00mm}

\vspace{0.00mm}
\setlength{\parindent}{-4.91mm}
\setlength{\leftskip}{4.91mm}
\setlength{\rightskip}{0.00mm}
\raggedright
$^{}$Declaration on Race and Racial Prejudice adopted 1978 (UNESCO)  Article 3
\vspace{0.00mm}

\vspace{0.00mm}
\setlength{\parindent}{-4.91mm}
\setlength{\leftskip}{4.91mm}
\setlength{\rightskip}{0.00mm}
\raggedright
$^{}$Declaration on the Preparation of Societies for Life in Peace 1978, Resolution 33/73 (1978)
\vspace{0.00mm}

\vspace{0.00mm}
\setlength{\parindent}{-4.91mm}
\setlength{\leftskip}{4.91mm}
\setlength{\rightskip}{0.00mm}
\raggedright
$^{}$Commission on Human Rights, Resolution 4, (XXXV) 1979
\vspace{0.00mm}

\vspace{0.00mm}
\setlength{\parindent}{-4.91mm}
\setlength{\leftskip}{4.91mm}
\setlength{\rightskip}{0.00mm}

$^{}$The Working Group of Governmental Experts was established pursuant to UN Commission on Human Rights  Resolution 36 (XXXVII) of 11$^{th}$ March 1981. With a mandate to clarify the nature and content of the right to development including its national and international aspects and also to prepare a legal document of some kind on TRTD, (a resolution, convention or declaration, see Commission on Human Rights, thirty-eighth session, 1 February-12 March 1982, Report of the Working Group of governmental experts on the right to development. UN Doc. E/CN.4/1489. 11$^{th}$ February 1982, paragraph 45)
\vspace{0.00mm}

\vspace{0.00mm}
\setlength{\parindent}{-4.91mm}
\setlength{\leftskip}{4.91mm}
\setlength{\rightskip}{0.00mm}
\raggedright
$^{}$African Charter of Human and Peoples' Rights 1981 (also known as the Banjul Charter) 
\vspace{0.00mm}

\vspace{0.00mm}
\setlength{\parindent}{-4.91mm}
\setlength{\leftskip}{4.91mm}
\setlength{\rightskip}{0.00mm}

$^{}$ If, on the other hand it was not taken for granted that the USA was a state whose interests were specially affected by the rule, then TDRTD could be used as evidence of uniform practice and the USA could be regarded as a persistent objector who would not be bound by the rule if it was finally established as customary law. However, this is merely speculative and it would not be likely that anyone could argue that the USA was not a state whose interests were specially affected.
\vspace{0.00mm}

\vspace{0.00mm}
\setlength{\parindent}{-4.91mm}
\setlength{\leftskip}{4.91mm}
\setlength{\rightskip}{0.00mm}

$^{}$The eight abstentions to the declaration were made by Denmark, Finland, The Federal Republic of Germany, Iceland, Israel, Japan, Sweden and the United Kingdom. Objections to TDRTD included; the possibility that the declaration would lead to an erosion of individual rights (The Federal Republic of Germany); TRTD could be used to legitimise violations of the rights of citizens (Japan).  TDRTD did not make a distinction between individual rights and group rights (Australia, although they did not abstain from TDRTD, they still expressed this concern.); the human rights discourse tended to become diluted and confused with the addition of TRTD (USA); the declaration provided a mistaken link between human rights and a NIEO, furthermore it did not take into account the complex relationship between development, security and disarmament but rather provided a very simplistic view of the relationship (UK).     
\vspace{0.00mm}

\vspace{0.00mm}
\setlength{\parindent}{-4.91mm}
\setlength{\leftskip}{4.91mm}
\setlength{\rightskip}{0.00mm}
\raggedright
$^{}$Birnie P., and Boyle A., \textit{International Law \& The Environment}, 2$^{nd}$ ed, Oxford: Oxford University Press, 2002. p87 
\vspace{0.00mm}

\vspace{0.00mm}
\setlength{\parindent}{-4.91mm}
\setlength{\leftskip}{4.91mm}
\setlength{\rightskip}{0.00mm}

$^{}$For example, Principle 4 (integration of environmental protection and development), Principle 10 (public participation), Principle 15 (precautionary approach) and Principle 17 (environmental impact assessment) were all strongly supported by developed states. 
\vspace{0.00mm}

\vspace{0.00mm}
\setlength{\parindent}{-4.91mm}
\setlength{\leftskip}{4.91mm}
\setlength{\rightskip}{0.00mm}

$^{}$Principle 3 (right to development), Principles 6 \& 7 (special needs of developing countries and principle of common but differentiated responsibilities), Principles 5 \& 9 (poverty and capacity building), were strongly supported by developing countries. 
\vspace{0.00mm}

\vspace{0.00mm}
\setlength{\parindent}{-4.91mm}
\setlength{\leftskip}{4.91mm}
\setlength{\rightskip}{0.00mm}

$^{}$Principle 4 ``In order to achieve sustainable development, environmental protection shall constitute an integral part of the development process and cannot be considered in isolation from it''.
\vspace{0.00mm}

\vspace{0.00mm}
\setlength{\parindent}{-4.91mm}
\setlength{\leftskip}{4.91mm}
\setlength{\rightskip}{0.00mm}
\raggedright
$^{}$Birnie P., and Boyle A., \textit{International Law \& The Environment}, 2$^{nd}$ ed, Oxford: Oxford University Press, 2002. p87 
\vspace{0.00mm}

\vspace{0.00mm}
\setlength{\parindent}{-4.91mm}
\setlength{\leftskip}{4.91mm}
\setlength{\rightskip}{0.00mm}
\raggedright
$^{}$Birnie P., and Boyle A., \textit{International Law \& The Environment}, 2$^{nd}$ ed, Oxford: Oxford University Press, 2002. p156 footnote 71 
\vspace{0.00mm}

\vspace{0.00mm}
\setlength{\parindent}{-4.91mm}
\setlength{\leftskip}{4.91mm}
\setlength{\rightskip}{0.00mm}
\raggedright
$^{}$Report of the Rio Conference on Environment and Development UN Doc A/CONF.151/26/Rev.1 (Vol II) (1993) Para 16
\vspace{0.00mm}

\vspace{0.00mm}
\setlength{\parindent}{-4.91mm}
\setlength{\leftskip}{4.91mm}
\setlength{\rightskip}{0.00mm}

$^{}$Birnie P., and Boyle A., \textit{International Law \& The Environment}, 2$^{nd}$ ed, Oxford: Oxford University Press, 2002. p87
\vspace{0.00mm}

\vspace{0.00mm}
\setlength{\parindent}{-4.91mm}
\setlength{\leftskip}{4.91mm}
\setlength{\rightskip}{0.00mm}
\raggedright
$^{}$GA Resoulution UN Doc A/RES/47/ 123 of 18 December 1992
\vspace{0.00mm}

\vspace{0.00mm}
\setlength{\parindent}{-4.91mm}
\setlength{\leftskip}{4.91mm}
\setlength{\rightskip}{0.00mm}

$^{}$See paragraph 10, ``The World Conference on Human Rights reaffirms the right to development, as established in the Declaration on the Right to Development, as a universal and inalienable right and an integral part of fundamental human rights. As stated in the Declaration on the Right to Development, the human person is the central subject of development. While development facilitates the enjoyment of all human rights, the lack of development may not be invoked to justify the abridgement of internationally recognized human rights. States should cooperate with each other in ensuring development and eliminating obstacles to development. The international community should promote an effective international cooperation for the realization of the right to development and the elimination of obstacles to development. Lasting progress towards the implementation of the right to development requires effective development policies at the national level, as well as equitable economic relations and a favourable economic environment at the international level.'' 
\vspace{0.00mm}

\vspace{0.00mm}
\setlength{\parindent}{-4.91mm}
\setlength{\leftskip}{4.91mm}
\setlength{\rightskip}{0.00mm}

$^{}$Sengupta A., \textit{Theory and Practice on the Right to Development}, Human Rights Quarterly; V.24(4); pp.837-889 p 841 states; ``There has been considerable debate as to whether the right to development can be regarded as a human right. This issue can now be taken as settled, after the achievement of the consensus for the Vienna Declaration and Programme of Action  in 1993''
\vspace{0.00mm}

\vspace{0.00mm}
\setlength{\parindent}{-4.91mm}
\setlength{\leftskip}{4.91mm}
\setlength{\rightskip}{0.00mm}
\raggedright
$^{}$Boven TV.,\textit{ Human Rights and Rights of Peoples}, 6 EJIL (1995) 1-476
\vspace{0.00mm}

\vspace{0.00mm}
\setlength{\parindent}{-4.91mm}
\setlength{\leftskip}{4.91mm}
\setlength{\rightskip}{0.00mm}
\raggedright
$^{}$Boven TV.,\textit{ Human Rights and Rights of Peoples}, 6 EJIL (1995) 1-476, p9
\vspace{0.00mm}

\vspace{0.00mm}
\setlength{\parindent}{-4.91mm}
\setlength{\leftskip}{4.91mm}
\setlength{\rightskip}{0.00mm}
\raggedright
$^{}$Boven TV.,\textit{ Human Rights and Rights of Peoples}, 6 EJIL (1995) 1-476, p9
\vspace{0.00mm}

\vspace{0.00mm}
\setlength{\parindent}{-4.91mm}
\setlength{\leftskip}{4.91mm}
\setlength{\rightskip}{0.00mm}

$^{}$World Conference on Human Rights, Vienna, Vienna Declaration and Programme of Action 1993, Paragraph 15
\vspace{0.00mm}

\vspace{0.00mm}
\setlength{\parindent}{-4.91mm}
\setlength{\leftskip}{4.91mm}
\setlength{\rightskip}{0.00mm}

$^{}$See Shaw M N, \textit{International law}. 5$^{th}$ ed. Cambridge: Cambridge UP, 2003.
\vspace{0.00mm}

\vspace{0.00mm}
\setlength{\parindent}{-4.91mm}
\setlength{\leftskip}{4.91mm}
\setlength{\rightskip}{0.00mm}

$^{}$World Conference on Human Rights, Vienna, Vienna Declaration and Programme of Action 1993, paragraph 11 states ``11. The right to development should be fulfilled so as to meet equitably the developmental and environmental needs of present and future generations. The World Conference on Human Rights recognizes that illicit dumping of toxic and dangerous substances and waste potentially constitutes a serious threat to the human rights to life and health of everyone. Consequently, the World Conference on Human Rights calls on all States to adopt and vigorously implement existing conventions relating to the dumping of toxic and dangerous products and waste and to cooperate in the prevention of illicit dumping. Everyone has the right to enjoy the benefits of scientific progress and its applications. The World Conference on Human Rights notes that certain advances, notably in the biomedical and life sciences as well as in information technology, may have potentially adverse consequences for the integrity, dignity and human rights of the individual, and calls for international cooperation to ensure that human rights and dignity are fully respected in this area of universal concern''
\vspace{0.00mm}

\vspace{0.00mm}
\setlength{\parindent}{-4.91mm}
\setlength{\leftskip}{4.91mm}
\setlength{\rightskip}{0.00mm}
\raggedright
$^{}$Ibid Paragraph 12
\vspace{0.00mm}

\vspace{0.00mm}
\setlength{\parindent}{-4.91mm}
\setlength{\leftskip}{4.91mm}
\setlength{\rightskip}{0.00mm}

$^{}$Ibid Paragraph 13; ``There is a need for States and international organizations, in cooperation with non-governmental organizations, to create favourable conditions at the national, regional and international levels to ensure the full and effective enjoyment of human rights. States should eliminate all violations of human rights and their causes, as well as obstacles to the enjoyment of these rights.''
\vspace{0.00mm}

\vspace{0.00mm}
\setlength{\parindent}{-4.91mm}
\setlength{\leftskip}{4.91mm}
\setlength{\rightskip}{0.00mm}

$^{}$See General Assembly Resolutions UN Docs: A/RES/50/184 of 22 December 1995; A/RES/49/183 of 23 Decemeber 1994; A/RES/ 48/130 of 20 December 1993 
\vspace{0.00mm}

\vspace{0.00mm}
\setlength{\parindent}{-4.91mm}
\setlength{\leftskip}{4.91mm}
\setlength{\rightskip}{0.00mm}

$^{}$For example Principle 3 ``The right to development is a universal and inalienable right and an integral part of fundamental human rights, and the human person is the central subject of development.  While development facilitates the enjoyment of all human rights, the lack of development may not be invoked to justify the abridgement of internationally recognized human rights.  The right to development must be fulfilled so as to equitably meet the population, development and environment needs of present and future generations''. Paragraph 9.2 states  ``The objectives are:
\vspace{0.00mm}

\vspace{0.00mm}
\setlength{\parindent}{-4.91mm}
\setlength{\leftskip}{4.91mm}
\setlength{\rightskip}{0.00mm}
\raggedright
 (a)  To foster a more balanced spatial distribution of the population by promoting in an integrated manner the equitable and ecologically sustainable development of major sending and receiving areas, with particular emphasis on the promotion of economic, social and gender equity based on respect for human rights, especially the right to development;'' Also see paragraphs 3.16 and 9.2
\vspace{0.00mm}

\vspace{0.00mm}
\setlength{\parindent}{-4.91mm}
\setlength{\leftskip}{4.91mm}
\setlength{\rightskip}{0.00mm}

$^{}$For example Commitment 1 states ``We commit ourselves to creating an economic, political, social, cultural and legal environment that will enable people to achieve social development.
\vspace{0.00mm}

\vspace{0.00mm}
\setlength{\parindent}{-4.91mm}
\setlength{\leftskip}{4.91mm}
\setlength{\rightskip}{0.00mm}

To this end, at the national level, we will:
\vspace{0.00mm}

\vspace{0.00mm}
\setlength{\parindent}{0.00mm}
\setlength{\leftskip}{0.00mm}
\setlength{\rightskip}{0.00mm}

(f) Reaffirm, promote and strive to ensure the realization of the rights set out in relevant international instruments and declarations, such as the Universal Declaration of Human Rights, 6/ the Covenant on Economic, social and cultural rights 7/ and the Declaration on the Right to Development, 8/ including those relating to education, food, shelter, employment, health and information, particularly in order to assist people living in poverty;
\vspace{0.00mm}

\vspace{0.00mm}
\setlength{\parindent}{0.00mm}
\setlength{\leftskip}{0.00mm}
\setlength{\rightskip}{0.00mm}

At the international level, we will:
\vspace{0.00mm}

\vspace{0.00mm}
\setlength{\parindent}{0.00mm}
\setlength{\leftskip}{0.00mm}
\setlength{\rightskip}{0.00mm}

(n) Reaffirm and promote all human rights, which are universal, indivisible, interdependent and interrelated, including the right to development as a universal and inalienable right and an integral part of fundamental human rights, and strive to ensure that they are respected, protected and observed.'' Also see Programme of Action, Chapter 1, Paragraphs 15, 17 which also directly refer to TRTD.   
\vspace{0.00mm}

\vspace{0.00mm}
\setlength{\parindent}{-4.91mm}
\setlength{\leftskip}{4.91mm}
\setlength{\rightskip}{0.00mm}

$^{}$For example states ``The Platform for Action 1995 Fourth World Conference on Women: ``reaffirm that all human rights -- civil, cultural, economic, political and social, including the right to development -- are universal, indivisible, interdependant and interrelated, as expressed in the Vienna Declaration and Programme of Action adopted by the World Conference on Human Rights.''
\vspace{0.00mm}

\vspace{0.00mm}
\setlength{\parindent}{-4.91mm}
\setlength{\leftskip}{4.91mm}
\setlength{\rightskip}{0.00mm}

$^{}$The World Food Summit 1996: ``emphasise the need for democracy and the promotion and protection of all human rights and freedoms, the right to development and the full and equal participation of men and women as essential determinants of success in achieving sustainable food security for all.''
\vspace{0.00mm}

\vspace{0.00mm}
\setlength{\parindent}{-4.91mm}
\setlength{\leftskip}{4.91mm}
\setlength{\rightskip}{0.00mm}

$^{}$For example states ``While the significance of national and regional particularities and various historical, cultural and religious backgrounds must be borne in mind, it is the duty of all States to promote and protect all human rights and fundamental freedoms, including the right to development.''
\vspace{0.00mm}

\vspace{0.00mm}
\setlength{\parindent}{-4.91mm}
\setlength{\leftskip}{4.91mm}
\setlength{\rightskip}{0.00mm}

$^{}$General Assembly Resolution UN Doc: A/RES/50/184 of 22 December 1995
\vspace{0.00mm}

\vspace{0.00mm}
\setlength{\parindent}{-4.91mm}
\setlength{\leftskip}{4.91mm}
\setlength{\rightskip}{0.00mm}

$^{}$OHCHR Human Rights in Development Latest news: www.unhchr.ch/development
\vspace{0.00mm}

\vspace{0.00mm}
\setlength{\parindent}{-4.91mm}
\setlength{\leftskip}{4.91mm}
\setlength{\rightskip}{0.00mm}
\raggedright
$^{}$General Assembly Resolution UN Doc: A/RES/53/155 of 9 December 1998
\vspace{0.00mm}

\vspace{0.00mm}
\setlength{\parindent}{-4.91mm}
\setlength{\leftskip}{4.91mm}
\setlength{\rightskip}{0.00mm}

$^{}$United Nations Commission on Human Rights, Fifty-fourth session Question of the realisation of the right to development, Report of the Secretary-General in accordance with Commission resolution 1997/72. UN Doc. E/CN1998/28. 16$^{th}$ February 1998. Para 19
\vspace{0.00mm}

\vspace{0.00mm}
\setlength{\parindent}{-4.91mm}
\setlength{\leftskip}{4.91mm}
\setlength{\rightskip}{0.00mm}

$^{}$See United Nations General Assembly, Fifty-third session, Human rights questions, Right to development, Report of the Secretary-General. UN Doc. A/53/268. 17$^{th}$ August 1998. Para 11. Ghana states that ``In its preliminary views and observations, the Government of Ghana, while recognising that the Declaration on the Right to Development and the International Bill of Human Rights constitute a milestone in the construction of a universally accepted human rights architecture, believes that these instruments need to be revisited for the following reasons:
\vspace{0.00mm}

\vspace{0.00mm}
\setlength{\parindent}{-4.91mm}
\setlength{\leftskip}{4.91mm}
\setlength{\rightskip}{0.00mm}
\raggedright
(a) The Definition of ``development'' is ambiguous and therefore needs more clarity and precision;''
\vspace{0.00mm}

\vspace{0.00mm}
\setlength{\parindent}{-4.91mm}
\setlength{\leftskip}{4.91mm}
\setlength{\rightskip}{0.00mm}

$^{}$United Nations General Assembly, Fifty-third session, Human rights questions, Right to development, Report of the Secretary-General. UN Doc. A/53/268. 17$^{th}$ August 1998. Para 11
\vspace{0.00mm}

\vspace{0.00mm}
\setlength{\parindent}{-4.91mm}
\setlength{\leftskip}{4.91mm}
\setlength{\rightskip}{0.00mm}

$^{}$United Nations Commission on Human Rights, Fifty-fourth session, Question of the realisation of the right to development, Report of the Secretary-General in accordance with Commission resolution 1997/72. UN Doc. E/CN1998/28. 16$^{th}$ February 1998. Para 8
\vspace{0.00mm}

\vspace{0.00mm}
\setlength{\parindent}{-4.91mm}
\setlength{\leftskip}{4.91mm}
\setlength{\rightskip}{0.00mm}

$^{}$Obiora Amede L,\textit{ Beyond the Rhetoric of a Right to Development}, Law and Policy; V.18(3/4); pp.355-418  p 379
\vspace{0.00mm}

\vspace{0.00mm}
\setlength{\parindent}{-4.91mm}
\setlength{\leftskip}{4.91mm}
\setlength{\rightskip}{0.00mm}

$^{}$UN SUBCOM, \textit{The legal nature of the right to development and enhancement of its binding nature}, 56$^{th}$ session, 01/06/2004,  UN Doc.E/CN.4/Sub.2/2004/16, para 40
\vspace{0.00mm}

\vspace{0.00mm}
\setlength{\parindent}{-4.91mm}
\setlength{\leftskip}{4.91mm}
\setlength{\rightskip}{0.00mm}

$^{}$United Nations Millennium Declaration, General Assembly Resolution 55/2 UN Doc. A/RES/55/2 of 8 September 2002 stated in paragraph 11 that  ``We will spare no effort to free our fellow men, women and children from the abject and dehumanizing conditions of extreme poverty, to which more than a billion of them are currently subjected. We are committed to making the right to development a reality for everyone and to freeing the entire human race from want.'' See further paragraphs 12-14 and 24 and 25 which link good governance and TRTD, references are also made to international financial institutions and cooperation. 
\vspace{0.00mm}

\vspace{0.00mm}
\setlength{\parindent}{-4.91mm}
\setlength{\leftskip}{4.91mm}
\setlength{\rightskip}{0.00mm}

$^{}$World Conference against Racism, Racial Discrimination, Xenophobia and Related Intolerance 2001, Report: ``the solemn commitment of all States to promote universal respect for, and observance and protection of, all human rights, economic, social, cultural, civil and political, including the right to development, as a fundamental factor in the prevention and elimination of racism, racial discrimination, xenophobia and related intolerance.''
\vspace{0.00mm}

\vspace{0.00mm}
\setlength{\parindent}{-4.91mm}
\setlength{\leftskip}{4.91mm}
\setlength{\rightskip}{0.00mm}

$^{}$See General Assembly Resolutions on the Right to development:Resolution 50/184 (1996) Right to development; Resolution 51/99 (1997) Right to development; Resolution 52/136 (1998) Right to development; Resolution 53/155 (1999) Right to development; Resolution 54/175 (2000) The right to development; Resolution 55/108 (2001) The right to development; Resolution 56/150 (2002) The right to development; Resolution 58/172 (2003) The right to development 
\vspace{0.00mm}

\vspace{0.00mm}
\setlength{\parindent}{-4.91mm}
\setlength{\leftskip}{4.91mm}
\setlength{\rightskip}{0.00mm}

$^{}$For example in supporting resolution 58/172,  The Syrian Arab Republic ``emphasised its support to the resolution entitled ``The Right to Development'' because it believed that the right to development was as inalienable human right, that the equality of opportunity for development was a prerogative both of nations and of individuals and the human person was the main benificiary and central subject of development.'' United Nations General Assembly, Fifty-eighth session, Human rights questions, The right to development, Report of the Secretary-General. UN Doc. A/58/276. 12$^{th}$ August 2003. Paragraph 35
\vspace{0.00mm}

\vspace{0.00mm}
\setlength{\parindent}{-4.91mm}
\setlength{\leftskip}{4.91mm}
\setlength{\rightskip}{0.00mm}

$^{}$United Nations General Assembly, Fifty-eighth session, Human rights questions, The right to development, Report of the Secretary-General. UN Doc. A/58/276. 12$^{th}$ August 2003. Paragraph 11.
\vspace{0.00mm}

\vspace{0.00mm}
\setlength{\parindent}{-4.91mm}
\setlength{\leftskip}{4.91mm}
\setlength{\rightskip}{0.00mm}

$^{}$ Japan stated ``The Government of Japan expressed its conviction that the promotion and realisation of the right to development was relevant in the context of development, but stressed its view that, at present, the concept of the right to development was not yet clear or distinct and further discussion was therefore necessary. It also expressed the belief that the right to development should be ensured for individuals in the territory of each country and that each Government should have the primary responsibility for the protection and promotion of that right. While international cooperation was important in the area of development, the Government of Japan disagreed with the legal obligation of developed countries to render assistance to developing countries. United Nations General Assembly, Fifty-eighth session, Human rights questions, The right to development, Report of the Secretary-General. UN Doc. A/58/276. 12$^{th}$ August 2003. Paragraph 11.
\vspace{0.00mm}

\vspace{0.00mm}
\setlength{\parindent}{-4.91mm}
\setlength{\leftskip}{4.91mm}
\setlength{\rightskip}{0.00mm}

$^{}$Explanation of Vote (prior to vote) Item 7: Right to Development\\
Delivered by Joel Danies on April 25, 2003 available at \uline{http://www.humanrights-usa.net/2003/statements/0425RtoD.htm}. Last accessed 11.11.04
\vspace{0.00mm}

\vspace{0.00mm}
\setlength{\parindent}{-4.91mm}
\setlength{\leftskip}{4.91mm}
\setlength{\rightskip}{0.00mm}
\raggedright
$^{}$World Conference on Human Rights, Vienna, Vienna Declaration and Programme of Action 1993
\vspace{0.00mm}

\vspace{0.00mm}
\setlength{\parindent}{-4.91mm}
\setlength{\leftskip}{4.91mm}
\setlength{\rightskip}{0.00mm}
\raggedright
$^{}$See Boven TV.,\textit{ Human Rights and Rights of Peoples}, 6 EJIL (1995) 1-476
\vspace{0.00mm}

\vspace{0.00mm}
\setlength{\parindent}{-4.91mm}
\setlength{\leftskip}{4.91mm}
\setlength{\rightskip}{0.00mm}
\raggedright
$^{}$Explanation of Vote (prior to vote) Item 7: Right to Development\\
Delivered by Joel Danies on April 25, 2003 available at \uline{http://www.humanrights-usa.net/2003/statements/0425RtoD.htm}. 
\vspace{0.00mm}

\vspace{0.00mm}
\setlength{\parindent}{-4.91mm}
\setlength{\leftskip}{4.91mm}
\setlength{\rightskip}{0.00mm}
\raggedright
$^{}$Ibid 
\vspace{0.00mm}

\vspace{0.00mm}
\setlength{\parindent}{-4.91mm}
\setlength{\leftskip}{4.91mm}
\setlength{\rightskip}{0.00mm}
\raggedright
$^{}$Ibid
\vspace{0.00mm}

\vspace{0.00mm}
\setlength{\parindent}{-4.91mm}
\setlength{\leftskip}{4.91mm}
\setlength{\rightskip}{0.00mm}

$^{}$This discussion in no way attempts to be exhaustive as, for the purpose of this paper, it is deemed inappropriate to attempt to discuss all the possible treaties and instruments which may contain references to the principles contained in TRTD. Other instruments which have been suggested to include indirect references of principles of TRTD include; (International); Convention on the Elimination of All Forms of Racial Discrimination 1965, See Articles 1, 2, 5(e); Convention on the Elimination of All Forms of Discrimination Against Women 1979, See Articles 1, 10-14; Convention on the Rights of the Child 1989, See Articles 23, 24, 26-29, 31; International Convention on the Protection of the Rights of All Migrant Workers and Members of their Families 1990; (Regional) European Social Charter 1961; Revised European Social Charter 1996; Additional Protocol to the European Social Charter Providing for a System of Collective Complaints 1995; American Declaration of Human Rights and Duties of Man 1948 Art VII, IX, XI-XVI; American Convention on Human Rights 1969, See Article 26; Additional Protocol to the American Convention on Human Rights in the Area of Economic, social and cultural rights Protocol of San Salvador 1988; African Charter on the Rights and Welfare of the Child 1990, See Articles 4, 11-16, 23; Draft Protocol to the African Charter on Human and Peoples' Rights on the Rights of Women;  Arab Charter on Human Rights 1994, See Articles 29-34, 36-39.
\vspace{0.00mm}

\vspace{0.00mm}
\setlength{\parindent}{-4.91mm}
\setlength{\leftskip}{4.91mm}
\setlength{\rightskip}{0.00mm}

$^{}$For an in depth discussion on the provisions which have been suggested to relate to TRTD see Donnelly J., \textit{In Search of the Unicorn : The Jurisprudence and Politics of the Right to Development}, 15 Cal W Int'l L J 473, 1985; Sengupta A., \textit{Theory and Practice on the Right to Development}, Human Rights Quarterly; V.24(4); pp.837-889 
\vspace{0.00mm}

\vspace{0.00mm}
\setlength{\parindent}{-4.91mm}
\setlength{\leftskip}{4.91mm}
\setlength{\rightskip}{0.00mm}

$^{}$Cited in Donnelly J., \textit{In Search of the Unicorn : The Jurisprudence and Politics of the Right to Development}, 15 Cal W Int'l L J 473, 1985 p 483; UN Doc E/CN.4/1982/SR.3 at para 2 (China)
\vspace{0.00mm}

\vspace{0.00mm}
\setlength{\parindent}{-4.91mm}
\setlength{\leftskip}{4.91mm}
\setlength{\rightskip}{0.00mm}
\raggedright
$^{}$Ibid
\vspace{0.00mm}

\vspace{0.00mm}
\setlength{\parindent}{-4.91mm}
\setlength{\leftskip}{4.91mm}
\setlength{\rightskip}{0.00mm}

$^{}$The UDHR provisions which have been cited include Articles 22-28. Donnelly discusses the value of using Article 28 to establish TRTD and his argument is impeccable. Furthermore, he explains how the right to self-determination contained in both Covenants cannot be used to imply TRTD. He also dismisses the use of the separate rights and the right to an adequate standard of living as evidence of TRTD. See Donnelly J., \textit{In Search of the Unicorn : The Jurisprudence and Politics of the Right to Development}, 15 Cal W Int'l L J 473, 1985. However, for a contrasting view on the legal obligation for international cooperation of ESC rights, see Hamm B, \textit{A human rights approach to development}, Human Rights Quarterly; V.23(4); pp.1005-10
\vspace{0.00mm}

\vspace{0.00mm}
\setlength{\parindent}{0.00mm}
\setlength{\leftskip}{0.00mm}
\setlength{\rightskip}{0.00mm}
\raggedright

\vspace{0.00mm}

\vspace{0.00mm}
\setlength{\parindent}{0.00mm}
\setlength{\leftskip}{0.00mm}
\setlength{\rightskip}{0.00mm}
\raggedleft
3
\vspace{0.00mm}

\vspace{0.00mm}
\setlength{\parindent}{0.00mm}
\setlength{\leftskip}{0.00mm}
\setlength{\rightskip}{0.00mm}
\raggedright

\vspace{0.00mm}

\vspace{0.00mm}
\setlength{\parindent}{0.00mm}
\setlength{\leftskip}{0.00mm}
\setlength{\rightskip}{0.00mm}
\raggedright

\vspace{0.00mm}

\vspace{0.00mm}
\setlength{\parindent}{0.00mm}
\setlength{\leftskip}{0.00mm}
\setlength{\rightskip}{0.00mm}
\raggedright

\vspace{0.00mm}


\end{document}
